\documentclass[12pt]{article}

% report, book
%  Русский язык

%\usepackage{bookmark}

\usepackage[T2A]{fontenc}			% кодировка
\usepackage[utf8]{inputenc}			% кодировка исходного текста
\usepackage[english,russian]{babel}	% локализация и переносы
\usepackage[title,toc,page,header]{appendix}
\usepackage{amsfonts}
\usepackage{hyperref,bookmark}
\usepackage{xcolor} %цвет

\usepackage{ulem}

\usepackage{tikz-feynman}
\usepackage{simpler-wick}

% Математика
\usepackage{amsmath,amsfonts,amssymb,amsthm,mathtools,bm} 
%%% Дополнительная работа с математикой
%\usepackage{amsmath,amsfonts,amssymb,amsthm,mathtools} % AMS
%\usepackage{icomma} % "Умная" запятая: $0,2$ --- число, $0, 2$ --- перечисление

\usepackage{cancel}%зачёркивание
\usepackage{braket}
%% Шрифты
\usepackage{euscript}	 % Шрифт Евклид
\usepackage{mathrsfs} % Красивый матшрифт


\usepackage[left=2cm,right=2cm,top=1cm,bottom=2cm,bindingoffset=0cm]{geometry}
\usepackage{wasysym}

%размеры
\renewcommand{\appendixtocname}{Приложения}
\renewcommand{\appendixpagename}{Приложения}
\renewcommand{\appendixname}{Приложение}
\makeatletter
\let\oriAlph\Alph
\let\orialph\alph
\renewcommand{\@resets@pp}{\par
  \@ppsavesec
  \stepcounter{@pps}
  \setcounter{subsection}{0}%
  \if@chapter@pp
    \setcounter{chapter}{0}%
    \renewcommand\@chapapp{\appendixname}%
    \renewcommand\thechapter{\@Alph\c@chapter}%
  \else
    \setcounter{subsubsection}{0}%
    \renewcommand\thesubsection{\@Alph\c@subsection}%
  \fi
  \if@pphyper
    \if@chapter@pp
      \renewcommand{\theHchapter}{\theH@pps.\oriAlph{chapter}}%
    \else
      \renewcommand{\theHsubsection}{\theH@pps.\oriAlph{subsection}}%
    \fi
    \def\Hy@chapapp{appendix}%
  \fi
  \restoreapp
}
\makeatother
\newtheorem{theorem}{Теорема}[section]
\newtheorem{predl}[theorem]{Предложение}
\newtheorem{sled}[theorem]{Следствие}

\theoremstyle{definition}
\newtheorem{zad}{Задача}[section]
\newtheorem{upr}[zad]{Упражнение}
\newtheorem{vopr}[zad]{Вопрос}
\newtheorem{defin}{Определение}[section]

\title{Весенняя школа по математике и физике
\\
Теория поля Изинга.\\
Тема 4. Интегрируемое направление $T=T_c$, $h\neq 0$}
\author{Андрей Коцевич, 4 курс ЛФИ МФТИ}
\date{}

\begin{document}
\maketitle
\section*{План доклада}
\begin{enumerate}
    \item Получение уравнений бутстрапа (в зависимости от того, до чего дойдут предыдущие докладчики):
    \begin{equation}
        S_{cd}(\theta)=S_{bd}(\theta-iu_{bc}^a)S_{ad}(\theta+i\hat{u}_{ac}^b)
    \end{equation}
    Они ограничивают возможную форму двухчастичной $S$-матрицы.
    \item А. Замолодчиков обнаружил минимальное решение уравнений бустрапа, удовлетворяющих соотношению
    \begin{equation}
        \kappa^a_sm^s_ae^{-is\bar{u}_{ac}^b}+\kappa^b_sm^s_be^{is\bar{u}_{ac}^b}=\kappa_s^cm_c^s
    \end{equation}
    со спинами интегралов движения
    \begin{equation}
        s=1,7,11,13,17,19,23,29\; (\text{mod } 30)
    \end{equation}
    Отметим, что числа совпадают с размерностями казимиров алгебры Ли $E_8$.\\
    Замолодчиков обнаружил, что это требование приводит к замыканию системы уравнений из бустрапа на 8 частицах $A_a$ ($a = 1,..., 8$) с массами
    \begin{equation}
        m_2=2m_1\cos\frac{\pi}{5}\approx1,618m_1,\quad m_3=2m_1\cos\frac{\pi}{30}\approx1,989m_1
    \end{equation}
    \begin{equation}
        m_4=4m_1\cos\frac{\pi}{5}\cos\frac{7\pi}{30}\approx2,405m_1,\quad 
    \end{equation}
    \begin{equation}
        m_5=4m_1\cos\frac{\pi}{5}\cos\frac{2\pi}{15}\approx2,956m_1,\quad m_6=4m_1\cos\frac{\pi}{5}\cos\frac{\pi}{30}\approx3,218m_1
    \end{equation}
    \begin{equation}
        m_7=8m_1\cos^2\frac{\pi}{5}\cos\frac{7\pi}{30}\approx3,891m_1,\quad m_8=8m_1\cos^2\frac{\pi}{5}\cos\frac{2\pi}{15}\approx4,783m_1
    \end{equation}
    Частицы $A_1,A_2,A_3$ являются связанными состояниями в канале рассеяния 1+1, частицы $A_1,A_2,A_3,A_4$ -- в канале 1+2. Частицы $A_5$ и $A_6$ появляются в канале 2+2, частица $A_7$ -- в канале 3+3 и $A_8$ -- в канале 4+4. 
    \item Применяя подход усечённого свободнофермионного пространства, можно численно получить эти массы, анализируя спектр возмущённого гамильтониана.
\end{enumerate}
\section{Введение}
Структура CFT определяется алгеброй Вирасоро, которая порождается некоторой компонентой $T(z)$ ($\bar{T}(z)$ для антиголоморфной части) тензора энергии-импульса. Составные поля, состоящие из $T(z)$ (типа $:T^2:$, $:T^3:$, $:(\partial_zT)^2:$ и т.д.), порождают бесконечное множество локальных интегралов движения (IM) в CFT. Среди них есть подмножество коммутирующих интегралов движения. IM в CFT проанализированы в разделе 2. В разделах 3, 4 показано, что некоторые из этих коммутирующих операторов могут сохраняться как IM в возмущенной теории поля
\begin{equation}\label{eq1}
    H=H_{CFT}+\lambda\int\Phi(x)d^2x,
\end{equation}
где $H_{CFT}$ -- действие исходной CFT с конкретным релевантным полем $\Phi$, взятым в качестве возмущения. IM возмущенной теории поля являются интегралы локальных плотностей
\begin{equation}\label{eq2}
    P_s=\oint(T_{s+1}dz+\Theta_{s-1}d\bar{z}),
\end{equation}
где $(z,\bar{z})$ -- стандартные комплексные координаты на $\mathbb{R}^2$, $T_{s+1}$ и $\Theta_{s-1}$ -- некоторые локальные поля, удовлетворяющие
\begin{equation}
    \partial_{\bar{z}}T_{s+1}=\partial_z\Theta_{s-1}
\end{equation}
Целозначный индекс $s$, помечающий IM для данного теории поля (\ref{eq1}), указывает спин этого оператора
\begin{equation}\label{eq3}
    [M,P_s]=sP_s,
\end{equation}
где $M$ -- евклидов оператор поворота.
\section{Подсчёт числа интегралов движения из соображений размерностей пространств}
Найдём число потомков на каждом уровне, претендующих на то, чтобы давать ток. Пусть $\mathcal{H}_{mn}$ -- неприводимое представление алгебры Вирасоро со старшим весом $\Delta_{mn}$, а $(\mathcal{H}_{mn})_s$ -- подпространство уровня $s$. Размерности этих подпространств даются характерами
\begin{equation}
    \chi_{mn}(q)=\sum\limits_{s=0}q^s\text{dim}(\mathcal{H}_{mn})_s,
\end{equation}
вид которых известен из конформной теории поля. Тогда размерность таких пространств подходящих операторов, из которых мы можем построить $T_{s+1}$ есть
\begin{equation}
    k_s=\text{dim}(\mathcal{H}_{11})_{s+1}-\text{dim}(\mathcal{H}_{11})_s+\delta_{s0}
\end{equation}
Вычитание $\text{dim}(\mathcal{H}_{11})_s$ нужно для того, чтобы исключить операторы вида $L_{-1}(...)$, а прибавление символа Кронекера -- чтобы скомпенсировать тот факт, что производная от единичного оператора равна нулю (т.е. $k_0=0$). Учитывая, что в $\mathcal{H}_{11}$ существует нулевой вектор $\mathcal{L}_{-1}$ на уровне 1, то
\begin{equation}
    \chi_{11}(q)=\prod\limits_{k=1}^\infty\frac{1}{1-q^k}(1-q)=\prod\limits_{k=2}^\infty\frac{1}{1-q^k}=1+q^2+q^3+2q^4+2q^5+4q^6+4q^7+\mathcal{O}(q^8)
\end{equation}
Для величин $k_s$ имеем характер
\begin{multline}
    \chi_0(q)\equiv\sum\limits_{s=0}^\infty k_sq^s=\sum\limits_{s=0}^\infty(\text{dim}(\mathcal{H}_{11})_{s+1}-\text{dim}(\mathcal{H}_{11})_s+\delta_{s0})q^s=q^{-1}\sum\limits_{s=1}^\infty q^s\text{dim}(\mathcal{H}_{11})_s-\\-\sum\limits_{s=0}^\infty q^s\text{dim}(\mathcal{H}_{11})_s+1=(q^{-1}-1)(\chi_{11}(q)-1)=q+q^3+2q^5+3q^7+q^8+4q^9+\mathcal{O}(q^{10})
\end{multline}
Теперь найдем количество уравнений, которые накладываются на коэффициенты в операторе $T_s$:
\begin{equation}
    l_s=\text{dim}(\mathcal{H}_\Delta)_s-\text{dim}(\mathcal{H}_\Delta)_{s-1}
\end{equation}
Здесь мы снова вычли из количества всех линейно-независимых операторов количество операторов вида $L_{-1}(...)$. Для соответствующего характера имеем
\begin{equation}
    \chi_{1,\Delta}(q)=\sum\limits_{s=1}^\infty l_sq^s=\sum\limits_{s=1}^\infty(\text{dim}(\mathcal{H}_\Delta)_s-\text{dim}(\mathcal{H}_\Delta)_{s-1})q^s=(1-q)\chi_\Delta(q)
\end{equation}
Вычитая одно из другого, получаем нижнюю границу $\delta_s = k_s-l_s$ количества решений на каждом уровне. В частности, в точках общего положения по $c$ имеем
\begin{equation}
    \chi_{mn}(q)=(1-q^{mn})\prod\limits_{k=1}^\infty\frac{1}{1-q^k}
\end{equation}
Для $\Phi_\Delta =\Phi_{13}$ получаем
\begin{equation}
    \chi_0(q)-\chi_{1,13}(q)=-1+q-q^2+q^3-2q^4+q^5-3q^6+q^7-4q^8-6q^{10}-10q^{12}-2q^{13}+\mathcal{O}(q^{14})
\end{equation}
Мы видим, что $\delta_1 = \delta_3 = \delta_5 = \delta_7 = 1$, а все остальные $\delta_s \leq 0$. Это значит, что, по крайней мере, для спинов $\pm1$, $\pm3$, $\pm5$, $\pm7$ имеются решения. На более высоких уровнях на самом деле уравнения избыточны.\\
Для случаев $\Phi_\Delta = \Phi_{12}, \Phi_{21}$ имеем
\begin{equation}
    \chi_0(q)-\chi_{1,13}(q)=-1+q-q^4+q^5-2q^6+q^7-2q^8-3q^{10}+q^{11}-6q^{12}-6q^{14}-3q^{15}+\mathcal{O}(q^{16})
\end{equation}
Мы получаем $\delta_1=\delta_5=\delta_7=\delta_{11}=1$, т.е., по крайней мере, имеется 8 интегралов движения.
\section{Асимптотические состояния и свойства $S$-матрицы в теории рассеяния}
Рассмотрим релятивистскую теорию рассеяния, содержащую $n$ сортов частиц $A_a$, $a = 1, 2, ..., n$ с массами $m_a$. Будем использовать символ $A_a(p)$ для обозначения частицы $A_a$, имеющей 2-импульс $p^\mu$. 2-импульс удовлетворяет условию массовой поверхности:
\begin{equation}
    p_\mu p^\mu=p\Bar{p}=m^2,
\end{equation}
где
\begin{equation}\label{eq5}
    p=p^0+p^1,\quad\Bar{p}=p^0-p^1
\end{equation}
компоненты светового конуса $p^\mu$.\\
Хорошо известно, что в квантовой механике эволюция системы во времени может быть определена через унитарный оператор $U(t,t_0)$, который создаёт состояние $\ket{\psi(t)}$, действуя на состояние $\ket{\psi(t_0)}$
\begin{equation}
    \ket{\psi(t)}=U(t,t_0)\ket{\psi(t_0)}
\end{equation}
Для изучения процесса рассеяния фактически не обязательно знать $U(t,t_0)$ при любых значениях $t$, $t_0$, достаточно знать его при $t_0\rightarrow-\infty$ и $t\rightarrow\infty$. Действительно, если мы предположим, что взаимодействия между частицами происходят в очень малой области пространства-времени, то, очень далеко от области взаимодействия, мы можем рассматривать их как свободные частицы. Таким образом, нам необходимо формально определить эти квантовые состояния свободных возбуждений, вводя так называемые \textit{асимптотические состояния}
\begin{equation}\label{eq4}
    \ket{A_{a_1}(p_1)A_{a_2}(p_2)...A_{a_N}(p_N)}_{\text{in}/\text{out}}
\end{equation}
По существу, асимптотические состояния описывают волновые пакеты с приблизительными положениями в заданные моменты времени: в частности, $N$ свободных частиц в момент времени $t\rightarrow-\infty$ для in-состояний и при $t\rightarrow+\infty$ для out-состояний. Выберем порядок импульсов $p_1>...>p_n$. Любое промежуточное состояние может быть разложено по in/out-базисам.\\
$S$-матрица определяется как линейный оператор, переводящий конечные асимптотические состояния с $m$ частицами в начальные асимптотические состояния с $n$ частицами (или наоборот, в зависимости от принятого соглашения, связанного с инверсией такого оператора):
\begin{equation}\label{eq8}
    \ket{A_{a_1}(p_1)...A_{a_n}(p_n)}_{\text{in}}=\sum\limits_{m=2}^\infty\sum\limits_{p'_1>...>p'_m} S^{b_1...b_m}_{a_1...a_n}(p_1,...,p_n;p'_1,...,p'_m)\ket{A_{b_1}(p_1)...A_{b_m}(p_m)}_{\text{out}}
\end{equation}
где сумма по импульсам $p'_1,...,p'_m$ является интегралом.\\
Следовательно, $S$ -- оператор эволюции по времени от $t_0 = -\infty$ до $t =\infty$:
\begin{equation}
    S=\lim\limits_{\overset{t_0\rightarrow-\infty}{t\rightarrow\infty}}U(t,t_0)
\end{equation}
Если система имеет гамильтониан
\begin{equation}
    H(t)=H_0+H_I(t)
\end{equation}
где $H_0$ -- гамильтониан свободной системы, а $H_I=H_I(t)$ -- гамильтониан взаимодействия в картине взаимодействия Дирака. $S$ можно выразить как
\begin{equation}
    S=\mathcal{T}\exp\left(-i\int\limits_{-\infty}^\infty dtH_I(t)\right),
\end{equation}
где $\mathcal{T}$ обозначает временное упорядочивание разложения в ряд экспоненты.\\
В этом разделе мы обсудим некоторые общие предположения, мотивированные физическими свойствами КТП. Предполагаем, что взаимодействия между частицами происходят только на коротких расстояниях. Другим очевидным предположением является справедливость принципа суперпозиции, означающего, что асимптотические состояния образуют полный базис для начальных и конечных состояний, и любое in-состояние может быть выражено через базис out-состояний и наоборот с помощью линейного оператора временной эволюции $S$. Более того, из сохранения вероятности следует, что
\begin{equation}
    1=\sum\limits_m|\braket{m|S|\psi}|^2,
\end{equation}
где $\ket{\psi}=\sum\limits_na_n\ket{n}$ и $\ket{m}$, $\ket{n}$ -- ортогональные базисные векторы, порождающие гильбертово пространство асимптотических состояний. Тогда
\begin{equation}
    1=\sum\limits_m|\braket{m|S|\psi}|^2=\sum\limits_m\braket{\psi|S^\dagger|m}\braket{m|S|\psi}=\braket{\psi|S^\dagger S|\psi}=\sum\limits_{n,m}a^*_na_m\braket{n|S^\dagger S|m}
\end{equation}
Это означает, что $S$-матрица должна быть унитарной:
\begin{equation}
    S^\dagger S = 1
\end{equation}
Работая в основном с релятивистскими теориями, нас будут интересовать следствия лоренц-инвариантности. В частности, для общего преобразования Лоренца
\begin{equation}
    L\ket{m}=\ket{m'},
\end{equation}
Требование инвариантности при таком преобразовании на уровне $S$-матрицы эквивалентно
\begin{equation}
    \braket{m'|S|n'}=\braket{m|S|n}
\end{equation}
Для того, чтобы объяснить последствия этого предположения, рассмотрим процесс рассеяния $n=2$ частиц на 2, где входящие (выходящие) частицы имеют импульсы $p^\mu_1,p^\mu_2$ ($p^\mu_3,p^\mu_4$). В релятивистской $(1+1)$-мерной теории энергии и импульсы частиц, участвующих в таком процессе рассеяния, могут быть удобно параметризованы в наборе релятивистских инвариантов, называемых \textit{переменными Мандельштама}:
\begin{equation}
    s=(p^\mu_1+p^\mu_2)^2,\quad t=(p^\mu_1-p^\mu_3)^2,\quad u=(p^\mu_1-p^\mu_4)^2,
\end{equation}
где $p^\mu_i=(p^0_i=E_i,p^1_i)$. Из закона сохранения 
\begin{equation}
    p^\mu_1+p^\mu_2=p^\mu_3+p^\mu_4
\end{equation}
и уравнения массовой поверхности $p^\mu_ip_{\mu i}=m_i^2$ следует
\begin{equation}
    s+t+u=\sum\limits_{i=1}^4m_i^2
\end{equation}
Следовательно, амплитуда зависит только от этих лоренц-инвариантных комбинаций импульсов, На самом деле, только от двух переменных Мандельштама, поскольку они не являются независимыми.\\
Теперь импульсы и энергии можно параметризовать соответственно как $p_i=m_i\sinh\theta_i$ и $E_i=m_i\cosh\theta_i$ через переменную быстроты $\theta$, а переменные Мандельштама можно записать как
\begin{equation}
    s=m_1^2+m_2^2+2m_1m_2\cosh\theta_{12},
\end{equation}
\begin{equation}
    t=m_1^2+m_3^2-2m_1m_3\cosh\theta_{13},
\end{equation}
\begin{equation}
    u=m_1^2+m_4^2-2m_1m_4\cosh\theta_{14},
\end{equation}
где введено обозначение $\theta_{ij}=\theta_i-\theta_j$. Тогда лоренц-инвариантность подразумевает, что фазы рассеяния зависят только от разности быстрот.\\
Другим фундаментальным предположением является так называемая \textit{макропричинность}, которая играет фундаментальную роль в свойстве факторизации, обсуждаемом в следующем разделе. Грубо говоря, макрокаузальность говорит нам, что выходящие частицы могут распространяться только после того, как произойдет взаимодействие между входящими, где <<макро>> означает, что это свойство может быть нарушено на микроскопических временных масштабах. Наконец, мы будем предполагать аналитичность $S$-матриц, то есть предполагать, что они являются аналитическими функциями в $\theta$-плоскости с минимальным числом сингулярностей, диктуемых конкретными физическими процессами.
\section{Сохраняющиеся токи и факторизация $S$-матрицы}
В КТП понятие интегрируемости связано с существованием бесконечного числа независимых, сохраняющихся и взаимно коммутирующих зарядов. Предположим, что лежащая в основе теория поля обладает рядом нетривиальных локальных коммутирующих интегралов движения $P_s$; $s = s_1,s_2,...$, где индекс $s$ обозначает спин $P_s$ из (\ref{eq3}), которые являются интегралами локальных плотностей (\ref{eq2}). Тогда ператоры $P_s$ диагонализируются асимптотическими состояниями (\ref{eq4}) одновременно. Для одночастичных состояний:
\begin{equation}
    P_s\ket{A_a(p)}=\omega^a_s(p)\ket{A_a(p)}
\end{equation}
Поскольку они локальны, то аддитивны
If they are local, i.e. they can be expressed as integrals of local densities, then they are
additive:
\begin{equation}\label{eq19}
    P_s\ket{A_{a_1}(p_1)...A_{a_N}(p_N)}_{\text{in}/\text{out}}=(\omega_s^{a_1}(p_1)+...+\omega_s^{a_N}(p_N))\ket{A_{a_1}(p_1)...A_{a_N}(p_N)}_{\text{in}/\text{out}}
\end{equation}
Можно показать, что эти функции обладают симметрией
\begin{equation}\label{eq6}
    -\omega_s^a(p)=\pm\omega_s^a(-p),
\end{equation}
где знак в правой части соответствует $\mathcal{C}$-чётности $P_s$ ($\mathcal{C}$ -- зарядовое сопряжение). Свойства преобразования Лоренца (\ref{eq3}) определяют следующий общий вид
\begin{equation}\label{eq7}
    \omega_s^a(p)=\kappa_s^ap^s,
\end{equation}
где $\kappa_s^a$ -- константы, а $p$ определён в (\ref{eq5}). Оператор $P_1$ предполагается как левоконусная компонента импульса
\begin{equation}
    \kappa_1^a=1
\end{equation}
Сравнивая (\ref{eq7}) с (\ref{eq6}), можно заключить, что $\mathcal{C}$-чётные ($\mathcal{C}$-нечётные) IM $P_s$ имеют нечётные (чётные) спины $s$. Будучи интегралами движения, операторы $P_s$ коммутируют с $S$-матрицей. Можно показать, что при этих предположениях теория рассеяния является "чисто упругой", т.е. число частиц и набор их индивидуальных моментов сохраняются асимптотически. Кроме того, $S$-матрица $N$ частиц факторизуется в терминах амплитуды двухчастичного рассеяния. Двухчастичная $S$-матрица должна удовлетворять <<уравнению факторизации>> (\textit{уравнению Янга-Бакстера}).\\
Здесь мы также предполагаем, что все частицы $A_a$ имеют разные массы, т.е. $m_a\neq m_b$, если $a\neq b$. Это предположение, в частности, подразумевает, что все частицы $A_a$ нейтральны: $\bar{A}_a = A_a$, где $\bar{A}$ обозначает соответствующую античастицу. В этом случае $S$-матрица диагональна в базисе (\ref{eq4})
\begin{equation}\label{eq8}
    \ket{A_{a_1}(p_1)...A_{a_N}(p_N)}_{\text{in}}=S_{a_1...a_N}(p_1,...,p_N)\ket{A_{a_1}(p_1)...A_{a_N}(p_N)}_{\text{out}}
\end{equation}
тогда как элемент $S$-матрицы для $N$ частиц в (\ref{eq8}) является просто произведением $\frac{N(N - 1)}{2}$ двухчастичных элементов
\begin{equation}
    S_{a_1...a_N}(p_1,...,p_N)=\prod\limits_{i<j}S_{a_ia_j}(p_i,p_j)
\end{equation}
Двухточечная $S$-матрица определяется как
\begin{equation}\label{eq17}
    \ket{A_a(p_1)A_b(p_2)}_{\text{in}}=S_{ab}(p_1,p_2)\ket{A_a(p_1)A_b(p_2)}_{\text{out}}
\end{equation}
Уравнение Янга-Бакстера тривиально выполняется в этом случае. В теории, обладающей пространственно-отражательной симметрией, амплитуды $S_{ab}(p_1,p_2)$ удовлетворяют
\begin{equation}\label{eq9}
    S_{ab}(p_1,p_2)=S_{ba}(p_1,p_2)
\end{equation}
Амплитуды $S_{ab}(p_1,p_2)$ -- функции 1 переменной $\theta=\theta_{12}=\theta_1-\theta_2$, где $\theta_k$ -- быстроты
\begin{equation}
    p_k=m_ke^{\theta_k},\quad\bar{p}_k=m_ke^{-\theta_k},
\end{equation}
которые могут быть использованы вместо импульсов $p_k^\mu$ для характеристики асимптотических частиц. Будем использовать символ $A_a(\theta)$ для частицы $A_a$, имеющей быстроту $\theta$. Двухточечная функция удовлетворяет соотношению:
\begin{equation}\label{eq10}
    \ket{A_a(\theta_1)A_b(\theta_2)}=S_{ab}(\theta_1-\theta_2)\ket{A_b(\theta_2)A_a(\theta_1)}   
\end{equation}
Применим (\ref{eq10}) 2 раза:
\begin{equation}
    \ket{A_a(\theta_1)A_b(\theta_2)}=S_{ab}(\theta_1-\theta_2)\ket{A_b(\theta_2)A_a(\theta_1)}=S_{ab}(\theta_1-\theta_2)S_{ba}(\theta_2-\theta_1)\ket{A_a(\theta_1)A_b(\theta_2)}
\end{equation}
Пользуясь симметрией  (\ref{eq9}), получим \textit{условие унитарности}:
\begin{equation}
    S_{ab}(\theta)S_{ab}(-\theta)=1
\end{equation}
Поскольку мы предполагали частицы нейтральными, амплитуды $S_{ab}$ удовлетворяют \textit{условию кросс-симметрии} (пользуемся (\ref{eq9}))
\begin{equation}
    S_{ab}(\theta)=S_{ab}(i\pi-\theta)
\end{equation}
Таким образом, двухчастичная $S$-матрица удовлетворяет следующим условиям:
\begin{enumerate}
    \item Уравнение Янга-Бакстера:
    \begin{equation}\label{eq11}
        S_{ab}(\theta_1-\theta_2)S_{ac}(\theta_1-\theta_3)S_{bc}(\theta_2-\theta_3)=S_{bc}(\theta_2-\theta_3)S_{ac}(\theta_1-\theta_3)S_{ab}(\theta_1-\theta_2)
    \end{equation}
    \item Унитарность:
    \begin{equation}\label{eq12}
        S_{ab}(\theta)S_{ba}(-\theta)=1
    \end{equation}
    \item Кроссинг-симметрия:
    \begin{equation}\label{eq13}
        S_{ab}(\theta)=S_{ba}(i\pi-\theta)
    \end{equation}
\end{enumerate}
Условия (\ref{eq11}–\ref{eq13}) чрезвычайно ограничительны. Вместе с симметрией модели и условиями аналитичности они позволяют найти точное выражение для $S$-матрицы.\\
Рассмотрим условия аналитичности. Двухчастичные амплитуды $S_{ab}(\theta)$ являются мероморфными функциями $\theta$. Они являются вещественными при $\text{Re}\;\theta = 0$: $S_{ab}(i\theta)\in\mathbb{R}$. В силу (\ref{eq12}) и (\ref{eq13}) амплитуды $S_{ab}(\theta)$ являются $2\pi i$-периодическими функциями, полностью определяемыми положением своих нулей и полюсов в <<физической полосе>> $0\leq\text{Im}\;\theta\leq\pi$. Полюса в этой полосе расположены на мнимой оси $\text{Re}\;\theta = 0$. Части этих полюсов отвечают связанные состояния, однако чтобы установить, отвечает ли данный полюс связанному состоянию, обычно требуется дополнительное исследование.\\
Простые полюса соответствуют частицам в <<связанном состоянии>> либо прямого канала рассеяния $A_aA_b$, либо поперечного канала (в зависимости от знака вычета). Амплитуды также могут иметь двойные или множественные полюса, связанные с многократными процессами перерассеяния. Основное требование бутстрапа состоит в том, что эти <<связанные состояния>> должны принадлежать одному и тому же набору частиц $A_1, A_2, ..., A_n$. Полюс $S_{ab}(\theta)$, расположенный в $\theta = iu_{ab}^c$ и имеющий положительный вычет, представляет собой частицу $A_c$ при условии, что $u_{ab}^c$ связана с массами частиц уравнением
\begin{equation}\label{eq16}
    m_c^2-m_a^2-m_b^2=2m_am_b\cos u_{ab}^c
\end{equation}
В этом случае, согласно (\ref{eq13}), существует полюс $S_{ab}(\theta)$ в $\theta=i\Bar{u}_{ab}^c$, где $\bar{u}^c_{ab}\equiv\pi-u^c_{ab}$ с отрицательным вычетом, которые представляют ту же частицу $A_c$, что и <<связанное состояние>> поперечного канала.\\
Сингулярность
\begin{equation}\label{eq14}
    S_{ab}(\theta)\sim\frac{R^c_{ab}}{\theta-iu_{ab}^c},\quad\theta\rightarrow iu_{ab}^c
\end{equation}
соответствует диаграмме на рис. \ref{gr1}, из которой видно, что вычет $R^c_{ab}$ можно представить в виде квадрата
\begin{equation}\label{eq15}
    R^c_{ab}=f_{abc}f_{abc}
\end{equation}
от «константы слияния» $f_{abc}$, которая должна быть полностью симметричной функцией индексов $a, b, c$. Другими словами, если $S_{ab}(\theta)$ обладает полюсом (\ref{eq14}), то амплитуда $S_{ac}(\theta)$ должна иметь полюс в точке $\theta = iu_{ca}^b$ с точно таким же вычетом (\ref{eq15}). Константу $f_{abc}$ можно связать с трехногой диаграммой на рис. \ref{gr2}, где углы между ногами представлены как $u_{ab}^c$.
\begin{figure}
    \centering
    \includegraphics[scale=0.6]{gr1.png}
    \caption{Диаграмма, связанная с полюсом (\ref{eq14}) в $S_{ab}(\theta)$}
    \label{gr1}
\end{figure}
\begin{figure}
    \centering
    \includegraphics[scale=0.6]{gr2.png}
    \caption{<<Трёхчастичная вершина>>, связанная с $f_{abc}$·}
    \label{gr2}
\end{figure}
Обратим внимание, что отношение
\begin{equation}
    u_{ab}^c+u_{ca}^b+u_{bc}^a=2\pi
\end{equation}
является простым следствием (\ref{eq16}).\\
Полюса <<связанного состояния>>, разумеется, появляются и в многочастичных элементах $S$-матрицы. Соответствующие вычеты должны сводиться к амплитудам рассеяния с участием частицы в <<связанном состоянии>>. В частности, вычет при $\theta_{12}=iu_{ab}^c$ трёхчастичной амплитуды
\begin{equation}
    S_{abd}(\theta_1,\theta_2,\theta_3)=S_{ab}(\theta_{12})S_{ad}(\theta_{13})S_{bd}(\theta_{23})
\end{equation}
где $\theta_{ij}=\theta_i-\theta_j$, должен воспроизводить двухточечную амплитуду $S_{cd}(\theta)$. Это требование приводит к уравнениям бутстрапа
\begin{equation}\label{eq23}
    S_{cd}(\theta)=S_{bd}(\theta-i\bar{u}^a_{bc})S_{ad}(\theta+i\Bar{u}^b_{ac}),
\end{equation}
которые ограничивают возможную форму двухчастичной $S$-матрицы.\\
Описанная выше <<структура слияния>> $S$-матрицы должна быть совместима с IM. Двухчастичное состояние (\ref{eq17}) аналитически продолжается до комплексных значений $p_1$, $p_2$. Вблизи полюса связанного состояния (\ref{eq14}) ($\epsilon\rightarrow0$)
\begin{equation}
    \ket{A_a(\theta+i\bar{u}^b_{ca}-\epsilon)A_b(\theta-i\bar{u}^a_{bc}+\epsilon)}_\text{in}\sim\frac{1}{\epsilon}\ket{A_c(\theta)}
\end{equation}
Поэтому уравнение
\begin{equation}
    P_s\ket{A_a(\theta_1)A_b(\theta_2)}_\text{in}=(\kappa^a_sm^s_ae^{s\theta_1}+\kappa_s^bm^s_be^{s\theta_2})\ket{A_a(\theta_1)A_b(\theta_2)}_\text{in}
\end{equation}
при аналитическом продолжении в окрестности этого полюса приводит к соотношению
\begin{equation}\label{eq18}
    \kappa^a_sm^s_ae^{is\bar{u}^b_{ca}}+\kappa_s^bm_b^se^{-is\bar{u}_{bc}^a}=\kappa_s^cm_c^s
\end{equation}
Это можно рассматривать как (обычно переопределенную) систему линейных уравнений для констант $\kappa_s^a$. Конечно, уравнение (\ref{eq18}) всегда допускает тривиальное решение $\kappa^a_s=0$. Подчеркнём, однако, что $\kappa_s^a = 0$ при $a=1,2,...,n$ влечет $P_s = 0$, как это следует из (\ref{eq19}), (\ref{eq7}) и условие асимптотической полноты. Чтобы увидеть, что (\ref{eq18}) действительно даёт существенные ограничения, рассмотрим некоторые модели.\\
Предположим, что в теории есть частица $A_1$, которую можно считать <<фундаментальной>>, т.е. все другие частицы $A_a$ могут быть получены как <<связанные состояния>> некоторого числа $A_1$. Это свойство предполагается здесь только для того, чтобы констатировать, что $\kappa_s^1\neq0$ для всех IM $P_s$, которыми обладает теория. Предположим дополнительно, что частица $A_1$ выступает как <<связанное состояние>> в $A_1A_1$-рассеянии (<<$\varphi^3$-свойство>>), т.е. $f_{111}\neq0$. Тогда, полагая $a = b = c = 1$ в (\ref{eq18}) и учитывая
\begin{equation}
    u_{11}^1=\frac{2\pi}{3},
\end{equation}
мы получим
\begin{equation}
    e^{-\frac{i\pi s}{3}}+e^{\frac{i\pi s}{3}}=1
\end{equation}
Это уравнение выполняется тогда и только тогда, когда целое число $s$ не имеет общего делителя с 6. Итак, в любой теории поля, обладающей хотя бы одним IM (\ref{eq2}) с $s = 0\; (\text{mod}\;3)$, <<фундаментальная частица>> с <<$\varphi^3$-свойством>> запрещена.\\
Другая интересная модель появляется, если мы предположим существование двух частиц $A_1$ и $A_2$ таких, что $f_{112}\neq0$ и $f_{122}\neq0$, т.е. $A_2$ можно интерпретировать как <<связанное состояние>> $A_1A_1$, а $A_1$ появляется как полюс связанного состояния $A_2A_2$.\\
Полагая $a=b=1$, $c=2$ в (\ref{eq18}), получим
\begin{equation}
    e^{is\bar{u}^1_{21}}+e^{-is\bar{u}_{12}^1}=\frac{\kappa_s^2}{\kappa_s^1}\frac{m_2^s}{m_1^s}
\end{equation}
Полагая $a=b=2$, $c=1$ в (\ref{eq18}), получим
\begin{equation}
    e^{is\bar{u}^2_{12}}+e^{-is\bar{u}_{21}^2}=\frac{\kappa_s^1}{\kappa_s^2}\frac{m_1^s}{m_2^s}
\end{equation}
%Используя переменные
%\begin{equation}
    %x_1=\exp\left(\frac{iu^2_{11}}{2}\right),\quad x_2=\exp\left(\frac{iu^1_{22}}{2}\right),
%\end{equation}
%можно переписать (\ref{eq18}) в форме
%\begin{equation}
    %x_1^s+x_1^{-s}=\left(\frac{m_2}{m_1}\right)^s\frac{\kappa_s^2}{\kappa_s^1},\quad x_2^s+x_2^{-s}=\left(\frac{m_1}{m_2}\right)^s\frac{\kappa_s^1}{\kappa_s^2}
%\end{equation}
Исключая неизвестные константы в правой части, получаем систему уравнений (учитывая $\bar{u}_{12}^1=\bar{u}_{21}^1$ и $\bar{u}^2_{12}=\bar{u}^2_{21}$)
\begin{equation}\label{eq20}
    (e^{is\bar{u}^1_{21}}+e^{-is\bar{u}_{21}^1})(e^{is\bar{u}^2_{12}}+e^{-is\bar{u}_{12}^2})=1
\end{equation}
\begin{equation}
    4\cos s\bar{u}^1_{21}\cos s\bar{u}^2_{12}=1,
\end{equation}
где $s$ в (\ref{eq20}) пробегает множество спинов нетривиальных IM $P_s$:
\begin{equation}\label{eq25}
    s=1,7,11,13,17,19,23,29\;(\text{mod}\;30)
\end{equation}
Если в теории содержится более одного нетривиального IM, то алгебраическая система оказывается переопределенной. Тем не менее, при условии $2\cos\bar{u}^1_{21}=\frac{m_2}{m_1}>1$ и $s\neq 0\;(\text{mod}\;2,5)$ эта система уравнений имеет решение
\begin{equation}
    \bar{u}^1_{21}=\frac{\pi}{5},\quad \bar{u}^2_{12}=\frac{2\pi}{5}
\end{equation}
Конечно, можно поменять $\bar{u}^1_{21}\leftrightarrow \bar{u}^2_{12}$. Это решение соответствует <<золотому>> соотношению масс
\begin{equation}\label{eq22}
    \frac{m_2}{m_1}=2\cos\frac{\pi}{5}=\frac{\sqrt{5}+1}{2}\approx 1.618
\end{equation}
Предположим дополнительно, что одна из этих частиц, скажем $A_1$, обладает <<$\varphi^3$-свойством>>, т.е. $f_{111}\neq0$. Согласуемыми с этой теорией рассеяния являются IM (\ref{eq2}), у которых спин $s$ не имеет общего делителя с 30. В следующем разделе мы разработаем эту модель теории рассеяния более подробно.
\section{Интегралы движения и $S$-матрица в критической модели Изинга с магнитным полем}
Рассмотрим минимальную модель Изинга $\mathcal{M}_{3,4}$. Центральный заряд теории:
\begin{equation}
    c=\frac{1}{2}
\end{equation}
Конформные размерности примарных полей:
\begin{equation}
    \Delta_{m,n}=\frac{(3m-4n)^2-1}{48}
\end{equation}
Примарные поля модели $\mathcal{M}_{3,4}$:
\begin{equation}
    I=\Phi_{1,1}=\Phi_{3,2},\quad \Delta_{1,1}=0
\end{equation}
\begin{equation}
    \epsilon=\Phi_{3,1}=\Phi_{1,2},\quad \Delta_{1,3}=\frac{1}{2}
\end{equation}
\begin{equation}
    \sigma=\Phi_{2,1}=\Phi_{2,2},\quad\Delta_{2,1}=\frac{1}{16}
\end{equation}
Рассмотрим возмущённую теорию поля с гамильтонианом ($s=3$):
\begin{equation}
    H^{(1,2)}_3=H_3+\lambda\int\Phi_{2,1}d^2x
\end{equation}
Характеры неприводимых представлений:
\begin{equation}
    \chi^{(3,4)}_{1,1}(q)=\frac{1}{2}\left(\prod\limits_{n=0}^\infty(1+q^{n+\frac{1}{2}})+\prod\limits_{n=0}^\infty(1-q^{n+\frac{1}{2}})\right)=1+x^2+x^3+2x^4+2x^5+3x^6+3x^7+\mathcal{O}(x^8)
\end{equation}
\begin{equation}
    \chi^{(3,4)}_{2,1}(q)=q^{\frac{1}{16}}\prod\limits_{n=1}^\infty(1+q^n)=q^{\frac{1}{16}}\prod\limits_{n=0}^\infty\frac{1}{1-q^{2n+1}}
\end{equation}
\begin{equation}
    \chi^{(3,4)}_0(x)=(x^{-1}-1)(\chi^{(3,4)}_{11}(x)-1)=x+x^3+x^5+2x^7+2x^9+x^{10}+3x^{11}+x^{12}+4x^{13}+\mathcal{O}(x^{14})
\end{equation}
\begin{equation}
    \chi^{(3,4)}_{1,\Delta}(x)=(1-x)\chi^{(3,4)}_\Delta(x)
\end{equation}
Для $\Phi_\Delta =\Phi_{2,1}=\Phi_{2,2}$ (возмущение оператором $\sigma$) получаем
\begin{equation}
    \chi^{(3,4)}_{21}(x)=x^{-\frac{1}{16}}\chi(x)\sum\limits_{k\in\mathbb{Z}}(x^{\Delta_{2+8k,1}}-x^{\Delta_{-2+8k,1}})=1+x+x^2+2x^3+2x^4+3x^5+\mathcal{O}(x^6)
\end{equation}
\begin{equation}
    \chi^{(3,4)}_{1,21}(x)=1+x^3+x^5+x^6+x^7+x^8+2x^9+2x^{10}+2x^{11}+3x^{12}+3x^{13}+\mathcal{O}(x^{14})
\end{equation}
\begin{equation}
    \chi^{(3,4)}_0(x)-\chi^{(3,4)}_{1,21}(x)=-1+x-x^6+x^7-x^8-x^{10}+x^{11}-2x^{12}+x^{13}+\mathcal{O}(x^{14})
\end{equation}
Мы получаем $\delta_1=\delta_7=\delta_{11}=\delta_{13}=\delta_{17}=\delta_{19}=1$. Т.е., по крайней мере, имеется 6 интегралов движения со спинами
\begin{equation}\label{eq21}
    s=1,7,11,13,17,19
\end{equation}
Коэффициенты $k_s$ и $l_s$ перечислены в табл. 1 для нечётных $s$, поскольку для чётных $s$: $k_s<l_s$.
\begin{table}[h!]
    \centering
    \begin{tabular}{|c|c|c|c|c|c|c|c|c|c|c|c|}
    \hline
        $s$ & 1 & 3 & 5 & 7 & 9 & 11 & 13 & 15 & 17 & 19 & 21\\\hline
        $k_s$ & 1 & 1 & 1 & 2 & 2 & 3 & 4 & 5 & 7 & 9 & 11\\\hline
        $l_s$ & 0 & 1 & 1 & 1 & 2 & 2 & 3 & 5 & 6 & 8 & 12\\\hline
    \end{tabular}
    \caption{Коэффициенты $k_s$ и $l_s$ при нечётных $s$ для $\Phi_{2,1}$}
    \label{tab:my_label}
\end{table}
Гамильтониан $H^{(2,1)}_3$ описывает теорию поля модели Изинга при $T=T_c$, но с ненулевым магнитным полем $h\sim\lambda$, которое нарушает $\mathbb{Z}_2$-симметрию. Эта система имеет конечную корреляционную длину и, следовательно, теория поля $H^{(2,1)}_3$ должна быть массивной. Приведенные выше результаты доказывают, что соответствующая теория рассеяния является чисто упругой.\\
Качественные особенности спектра частиц в модели Изинга при $T-T_c\rightarrow 0$ и $\eta = h^{-8}(T-T_c)^{15}\sim1$ можно предсказать из физических соображений. В общем случае $h\neq0$ фурье-образ двухточечной функции
\begin{equation}
    G(p^2)=\int d^2xe^{ipx}\braket{\sigma(x)\sigma(0)}
\end{equation}
будет обладать сингулярностями, показанными на рис. \ref{gr3}.
\begin{figure}[h!]
    \centering
    \includegraphics[scale=0.5]{gr3.png}
    \caption{Сингулярности коррелятора $G(p^2)$. Полюса изображены точками, сплошная линия представляет собой разрез.}
    \label{gr3}
\end{figure}\\
Существует несколько полюсов, соответствующих стабильным частицам, расположенным ниже разреза. Самая лёгкая частица (назовем её $A_1$), как ожидается, будет <<фундаментальной>>. Кроме того, поскольку симметрия нарушается при $h\neq0$, она проявляет <<$\varphi^3$-свойство>>. В конкретном случае $\eta=0$ (т.е. $T=T_c$, $h \neq 0$) существует ровно три полюса $m_1^2$, $m_2^2$, $m_3^2$ ниже пороговой массы $4m_1^2$ на рис. \ref{gr3}. Фактически, теория содержит 8 стабильных частиц, но 5 из них лежат выше порога $A_1A_1$. Физическая причина их стабильности не очень ясна.\\
Набор спинов IM (\ref{eq21}), найденный нами в теории поля $H^{(2,1)}_3$, убедительно свидетельствует о наличии в теории двух частиц $A_1$ и $A_2$ с отношением масс (\ref{eq22}). Предположим, что это самые лёгкие частицы в теории.\\
Рассмотрим двухчастичную амплитуду $S_{11}(\theta)$, удовлетворяющую условиям $S_{11}(\theta)=S_{11}(i\pi-\theta)=S_{11}^{-1}(\theta)$ и уравнению бутстрапа (\ref{eq23}) при $a = b = c = d = 1$:
\begin{equation}\label{eq24}
    S_{11}(\theta)=S_{11}\left(\theta-\frac{i\pi}{3}\right)S_{11}\left(\theta+\frac{i\pi}{3}\right)
\end{equation}
Согласно нашему предположению, она имеет полюса при $\theta=\frac{2i\pi}{3}$ и $\theta=\frac{2i\pi}{5}$ с положительными вычетами, которые соответствуют частицам $A_1$ и $A_2$ соответственно. В силу кроссинг-симметрии (\ref{eq13}) имеются также полюсы $\theta = \frac{i\pi}{3}$ и $\theta=\frac{3i\pi}{5}$ с отрицательными вычетами. 
Это уравнение не может быть удовлетворено без дополнительных полюсов в $S_{11}(\theta)$. Простейший способ удовлетворить (\ref{eq24}) без нарушения вышеуказанных требований состоит в том, чтобы добавить полюсы в точках $\theta=\frac{i\pi}{15}$ и $\theta=\frac{14i\pi}{15}$. Второй (возникающий в силу кроссинг-симметрии) полюс должен иметь отрицательный вычет, чтобы не было частиц меньшей массы. Возможные полюсы $\theta=\frac{11i\pi}{15}$ и $\theta=\frac{4i\pi}{15}$ от второго сомножителя сокращаются нулями от первого сомножителя:
\begin{equation}
    S_{11}\left(\frac{11i\pi}{15}+\frac{i\pi}{3}\right)=S_{11}\left(i\pi+\frac{i\pi}{15}\right)=S_{11}\left(-\frac{i\pi}{15}\right)=S^{-1}_{11}\left(\frac{i\pi}{15}\right)=0
\end{equation}
<<Минимальное>> решение уравнения (\ref{eq24}) принимает вид
\begin{equation}
    S_{11}(\theta)=\frac{\th\frac{\theta+2i\pi/3}{2}\th\frac{\theta+2i\pi/5}{2}\th\frac{\theta+i\pi/15}{2}}{\th\frac{\theta-2i\pi/3}{2}\th\frac{\theta-2i\pi/5}{2}\th\frac{\theta-i\pi/15}{2}}
\end{equation}
Расположение полюсов и нулей этой амплитуды показано на рис. \ref{gr4}(a). Дополнительный полюс $\theta=\frac{i\pi}{15}$ представляет новую частицу $A_3$ с массой
\begin{equation}
    \bar{u}^1_{31}=\frac{\pi}{30},\quad\frac{m_3}{m_1}=2\cos\frac{\pi}{30}\approx1.989
\end{equation}
\begin{figure}[h!]
    \centering
    \includegraphics[scale=0.5]{gr4.png}
    \caption{Расположение нулей (крестики) и полюсов (точки) двухчастичных амплитуд $S_{11}(\theta)$ (a), $S_{12}(\theta)$ (b) и $S_{22}(\theta)$ (c). В последнем случае имеются двойные полюса (нули), которые показаны двойными точками (крестиками). Стрелки показывают полюса "<<связанного состояния>> с положительными вычетами и указывают на соответствующие частицы.}
    \label{gr4}
\end{figure}\\
Рассмотрим двухчастичную амплитуду $S_{12}(\theta)$, удовлетворяющую уравнению бутстрапа (\ref{eq23}) при $a = b = d = 1$ и $c=2$:
\begin{equation}
    S_{12}(\theta)=S_{11}\left(\theta-\frac{i\pi}{5}\right)S_{11}\left(\theta+\frac{i\pi}{5}\right)
\end{equation}
$S_{12}(\theta)$ имеет полюсы в точках $\theta=\frac{4i\pi}{5},\frac{3i\pi}{5},\frac{7i\pi}{15},\frac{4i\pi}{15}$. Нули и полюсы $S_{12}(\theta)$ изображены на рис. \ref{gr4}(b).
\begin{equation}
    S_{12}(\theta)=\frac{\th\frac{\theta+i\pi/5}{2}\th\frac{\theta+8i\pi/15}{2}\th\frac{\theta+4i\pi/15}{2}\th\frac{\theta+3i\pi/5}{2}}{\th\frac{\theta-i\pi/5}{2}\th\frac{\theta-8i\pi/15}{2}\th\frac{\theta-4i\pi/15}{2}\th\frac{\theta-3i\pi/5}{2}}
\end{equation}
Первые 3 полюса отождествляются с $u^a_{12},a=1,2,3$. Последний соответствует новой частице $A_4$ с $u^4_{12}=\frac{4\pi}{15}$. Запишем (\ref{eq16}) для $a=1,b=2,c=4$:
\begin{equation}
    m_4^2-m_1^2-m_2^2=2m_1m_2\cos u_{12}^4
\end{equation}
\begin{equation}
    \frac{m_4}{m_1}=\sqrt{1+4\cos^2\frac{\pi}{5}+4\cos\frac{\pi}{5}\cos\frac{4\pi}{15}}=4\cos\frac{\pi}{5}\cos\frac{7\pi}{30}\approx2.405
\end{equation}
Амплитуда $S_{22}(\theta)$, удовлетворяющую уравнению бутстрапа (\ref{eq23}) при $a=b=1$ и $c=d=2$:
\begin{equation}
    S_{22}(\theta)=S_{12}\left(\theta-\frac{i\pi}{5}\right)S_{12}\left(\theta+\frac{i\pi}{5}\right)
\end{equation}
Нули и полюсы $S_{22}(\theta)$ изображены на рис. \ref{gr4}(c). Аналитическое выражение для $S_{22}(\theta)$:
\begin{equation}\label{eq26}
    S_{22}(\theta)=\frac{\th\frac{\theta+i\pi/3}{2}\th\frac{\theta+4i\pi/15}{2}\th\frac{\theta+7i\pi/15}{2}\th\frac{\theta+11i\pi/15}{2}\th\frac{\theta+i\pi/15}{2}\left(\th\frac{\theta+2i\pi/5}{2}\right)^2}{\th\frac{\theta-i\pi/3}{2}\th\frac{\theta-4i\pi/15}{2}\th\frac{\theta-7i\pi/15}{2}\th\frac{\theta-11i\pi/15}{2}\th\frac{\theta-i\pi/15}{2}\left(\th\frac{\theta-2i\pi/5}{2}\right)^2}
\end{equation}
Кроме известных частиц $A_1$, $A_2$, $A_4$, которые представлены полюсами $\theta=\frac{4i\pi}{5}$, $\theta=\frac{2i\pi}{3}$, $\theta=\frac{7i\pi}{15}$ в (\ref{eq26}), амплитуда $S_{22}(\theta)$ имеет полюса при $\theta=\frac{4i\pi}{15}$ и $\theta=\frac{i\pi}{15}$, которые также имеют положительные вычеты и, следовательно, подразумевают существование 2 частиц $A_5$ и $A_6$ с массами
\begin{equation}
    \frac{m_5}{m_1}=4\cos\frac{\pi}{5}\cos\frac{2\pi}{15}\approx2.956
\end{equation}
\begin{equation}
    \frac{m_6}{m_1}=4\cos\frac{\pi}{5}\cos\frac{\pi}{30}\approx3.218
\end{equation}
Продолжая сливать все возможные частицы друг с другом, можно получать новые. Оказывается, уравнения бутстрапа замыкаются на 8 массах:
\begin{equation}
    m_2=2m_1\cos\frac{\pi}{5}\approx1,618m_1,\quad m_3=2m_1\cos\frac{\pi}{30}\approx1,989m_1
\end{equation}
\begin{equation}
    m_4=4m_1\cos\frac{\pi}{5}\cos\frac{7\pi}{30}\approx2,405m_1,\quad 
\end{equation}
\begin{equation}
    m_5=4m_1\cos\frac{\pi}{5}\cos\frac{2\pi}{15}\approx2,956m_1,\quad m_6=4m_1\cos\frac{\pi}{5}\cos\frac{\pi}{30}\approx3,218m_1
\end{equation}
\begin{equation}
    m_7=8m_1\cos^2\frac{\pi}{5}\cos\frac{7\pi}{30}\approx3,891m_1,\quad m_8=8m_1\cos^2\frac{\pi}{5}\cos\frac{2\pi}{15}\approx4,783m_1
\end{equation}
Двухчастичная амплитуда в общем случае
\begin{equation}
    S_{ab}(\theta)=\prod\limits_\gamma(t_{\frac{\gamma}{30}}(\theta))^{p_\gamma},
\end{equation}
где список индексов $\gamma$ и степени $p_\gamma$ перечислен в табл. 2 и
\begin{equation}
    t_\alpha=\frac{\th\frac{1}{2}(\theta+i\pi\alpha)}{\th\frac{1}{2}(\theta-i\pi\alpha)}
\end{equation}
Например, амплитуды $A_1A_1$ и $A_1A_2$:
\begin{equation}
    S_{11}(\theta)=t_{2/3}(\theta)t_{2/5}(\theta)t_{1/15}(\theta)
\end{equation}
\begin{equation}
    S_{12}(\theta)=t_{4/5}(\theta)t_{3/5}(\theta)t_{7/15}(\theta)t_{4/15}(\theta)
\end{equation}
Заметим, что числа (\ref{eq25}) -- это показатели алгебры Ли $E_8$, взятые по модулю 30. Кроме того, число стабильных частиц в вышеприведенной теории рассеяния связано с рангом $E_8$. Эта скрытая структура $E_8$ теории поля $H_3^{(1,2)}$ (а также скрытые структуры $E_7$ и $E_8$ в $H^{(1,2)}_4$ ($s=1,5,7,9,11,13,17(\text{mod\;18})$) и $H^{(1,2)}_5$ ($s=1,4,5,7,8,11(\text{mod\;12})$)) была предсказана В.А. Фатеевым на на основе его исследования $W$-алгебр, связанных с исключительными алгебрами Ли.
\begin{table}[]
    \centering
    \begin{tabular}{|c|c|c|c|c|c|}
    \hline
        $a$ & $b$ & $\gamma$, $p_\gamma$ & $a$ & $b$ & $\gamma$, $p_\gamma$ \\\hline
        1 & 1 & $\overset{\bm{1}}{20},\overset{\bm{2}}{12},\overset{\bm{3}}{2}$ & 3 & 6 & $\overset{\bm{2}}{26},\overset{\bm{3}}{24^3},\overset{\bm{6}}{18^3},\overset{\bm{8}}{8^3},10^2,16^4$ \\\hline
        1 & 2 & $\overset{\bm{1}}{24},\overset{\bm{2}}{18},\overset{\bm{3}}{14},\overset{\bm{4}}{8}$ & 3 & 7 & $\overset{\bm{3}}{28},\overset{\bm{5}}{22^3},4^2,6^2,10^4,12^4,16^4$ \\\hline
        1 & 3 & $\overset{\bm{1}}{29},\overset{\bm{2}}{21},\overset{\bm{4}}{13},\overset{\bm{5}}{3},11^2$ & 3 & 8 & $\overset{\bm{5}}{27},\overset{\bm{6}}{25^3},\overset{\bm{8}}{17^5},7^4,9^4,11^2,15^3$ \\\hline
        1 & 4 & $\overset{\bm{2}}{25},\overset{\bm{3}}{21},\overset{\bm{4}}{17},\overset{\bm{5}}{11},\overset{\bm{6}}{7},15$ & 4 & 4 & $\overset{\bm{1}}{26},\overset{\bm{4}}{20^3},\overset{\bm{6}}{16^3},\overset{\bm{7}}{12^3},\overset{\bm{8}}{2},6^2,8^2$ \\\hline
        1 & 5 & $\overset{\bm{3}}{28},\overset{\bm{4}}{22},\overset{\bm{6}}{14},\overset{\bm{7}}{4},10^2,12^2$ & 4 & 5 & $\overset{\bm{1}}{27},\overset{\bm{3}}{23^3},\overset{\bm{5}}{19^3},\overset{\bm{8}}{9^3},5^2,13^4,15^2$ \\\hline
        1 & 6 & $\overset{\bm{4}}{25},\overset{\bm{5}}{19},\overset{\bm{7}}{9},7^2,13^2,15$ & 4 & 6 & $\overset{\bm{1}}{28},\overset{\bm{4}}{22^3},4^2,6^2,10^4,12^4,16^4$ \\\hline
        1 & 7 & $\overset{\bm{5}}{27},\overset{\bm{6}}{23},\overset{\bm{8}}{5},9^2,11^2,13^2,15$ & 4 & 7 & $\overset{\bm{2}}{28},\overset{\bm{4}}{24^3},\overset{\bm{7}}{18^5},\overset{\bm{8}}{14^5},4^2,8^4,10^4$ \\\hline
        1 & 8 & $\overset{\bm{7}}{26},\overset{\bm{8}}{16^3},6^2,8^2,10^2,12^2$ & 4 & 8 & $\overset{\bm{4}}{29},\overset{\bm{5}}{25^3},\overset{\bm{7}}{21^5},3^2,7^4,11^6,13^6,15^3$ \\\hline
        2 & 2 & $\overset{\bm{1}}{24},\overset{\bm{2}}{20},\overset{\bm{4}}{14},\overset{\bm{5}}{8},\overset{\bm{6}}{2},12^2$ & 5 & 5 & $\overset{\bm{4}}{22^3},\overset{\bm{5}}{20^5},\overset{\bm{8}}{12^5},2^2,4^2,6^2,16^4$ \\\hline
        2 & 3 & $\overset{\bm{1}}{25},\overset{\bm{3}}{19},\overset{\bm{6}}{9},7^2,13^2,15$ & 5 & 6 & $\overset{\bm{1}}{27},\overset{\bm{2}}{25^3},\overset{\bm{7}}{17^5},7^4,9^4,11^4,15^3$ \\\hline
        2 & 4 & $\overset{\bm{1}}{27},\overset{\bm{2}}{23},\overset{7}{5},9^2,11^2,13^2,15$ & 5 & 7 & $\overset{\bm{1}}{29},\overset{\bm{3}}{25^3},\overset{\bm{6}}{21^5},3^2,7^4,11^6,13^6,15^3$ \\\hline
        2 & 5 & $\overset{\bm{2}}{26},\overset{\bm{6}}{16^3},6^2,8^2,10^2,12^2$ & 5 & 8 & $\overset{\bm{3}}{28},\overset{\bm{4}}{26^3},\overset{\bm{5}}{24^5},\overset{\bm{8}}{18^7},8^6,10^6,16^8$ \\\hline
        2 & 6 & $\overset{\bm{2}}{29},\overset{\bm{3}}{25},\overset{\bm{5}}{19^3},\overset{\bm{7}}{13^3},\overset{\bm{8}}{3},7^2,9^2,15$ & 6 & 6 & $\overset{\bm{3}}{24^3},\overset{\bm{6}}{20^5},\overset{\bm{8}}{14^5},2^2,4^2,8^4,12^6$ \\\hline
        2 & 7 & $\overset{\bm{4}}{27},\overset{\bm{6}}{21^3},\overset{\bm{7}}{17^3},\overset{\bm{8}}{11^3},5^2,7^2,15^2$ & 6 & 7 & $\overset{\bm{1}}{28},\overset{\bm{2}}{26^3},\overset{\bm{5}}{22^5},\overset{\bm{8}}{16^7},6^4,10^6,12^6$ \\\hline
        2 & 8 & $\overset{\bm{6}}{28},\overset{\bm{7}}{22^3},4^2,6^2,10^4,12^4,16^4$ & 6 & 8 & $\overset{\bm{2}}{29},\overset{\bm{3}}{27^3},\overset{\bm{6}}{23^5},\overset{\bm{7}}{21^7},5^4,11^8,13^8,15^4$ \\\hline
        3 & 3 & $\overset{\bm{2}}{22},\overset{\bm{3}}{20^3},\overset{\bm{5}}{14},\overset{\bm{6}}{12^3},\overset{\bm{7}}{4},2^2$ & 7 & 7 & $\overset{\bm{2}}{26^3},\overset{\bm{4}}{24^5},\overset{\bm{7}}{20^7},2^2,8^6,12^8,16^8$ \\\hline
        3 & 4 & $\overset{\bm{1}}{26},\overset{\bm{5}}{16^3},6^2,8^2,10^2,12^2$ & 7 & 8 & $\overset{\bm{1}}{29},\overset{\bm{2}}{27^3},\overset{\bm{4}}{25^5},\overset{\bm{6}}{23^7},\overset{\bm{8}}{19^9},9^8,13^{10},15^5$ \\\hline
        3 & 5 & $\overset{\bm{1}}{29},\overset{\bm{3}}{23},\overset{\bm{4}}{21^3},\overset{\bm{7}}{13^3},\overset{\bm{8}}{5},3^2,11^4,15$ & 8 & 8 & $\overset{\bm{1}}{28^3},\overset{\bm{3}}{26^5},\overset{\bm{5}}{24^7},\overset{\bm{7}}{22^9},\overset{\bm{8}}{20^{11}},12^{12},16^{12}$ \\\hline
    \end{tabular}
    \caption{Caption}
    \label{tab:my_label}
\end{table}
\end{document}
