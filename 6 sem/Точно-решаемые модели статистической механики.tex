\documentclass[12pt]{article}

% report, book
%  Русский язык

\usepackage{hyperref,bookmark}
\usepackage[warn]{mathtext} %русский язык в формулах
\usepackage[T2A]{fontenc}			% кодировка
\usepackage[utf8]{inputenc}			% кодировка исходного текста
\usepackage[english,russian]{babel}	% локализация и переносы
\usepackage[title,toc,page,header]{appendix}
\usepackage{amsfonts}


% Математика
\usepackage{amsmath,amsfonts,amssymb,amsthm,mathtools} 
%%% Дополнительная работа с математикой
%\usepackage{amsmath,amsfonts,amssymb,amsthm,mathtools} % AMS
%\usepackage{icomma} % "Умная" запятая: $0,2$ --- число, $0, 2$ --- перечисление

\usepackage{cancel}%зачёркивание
\usepackage{braket}
%% Шрифты
\usepackage{euscript}	 % Шрифт Евклид
\usepackage{mathrsfs} % Красивый матшрифт


\usepackage[left=2cm,right=2cm,top=1cm,bottom=2cm,bindingoffset=0cm]{geometry}
\usepackage{wasysym}

%размеры
\renewcommand{\appendixtocname}{Приложения}
\renewcommand{\appendixpagename}{Приложения}
\renewcommand{\appendixname}{Приложение}
\makeatletter
\let\oriAlph\Alph
\let\orialph\alph
\renewcommand{\@resets@pp}{\par
  \@ppsavesec
  \stepcounter{@pps}
  \setcounter{subsection}{0}%
  \if@chapter@pp
    \setcounter{chapter}{0}%
    \renewcommand\@chapapp{\appendixname}%
    \renewcommand\thechapter{\@Alph\c@chapter}%
  \else
    \setcounter{subsubsection}{0}%
    \renewcommand\thesubsection{\@Alph\c@subsection}%
  \fi
  \if@pphyper
    \if@chapter@pp
      \renewcommand{\theHchapter}{\theH@pps.\oriAlph{chapter}}%
    \else
      \renewcommand{\theHsubsection}{\theH@pps.\oriAlph{subsection}}%
    \fi
    \def\Hy@chapapp{appendix}%
  \fi
  \restoreapp
}
\makeatother
\newtheorem{resh}{Решение}
\newtheorem{theorem}{Теорема}
\newtheorem{predl}[theorem]{Предложение}
\newtheorem{sled}[theorem]{Следствие}

\theoremstyle{definition}
\newtheorem{zad}{Задача}[section]
\newtheorem{upr}[zad]{Упражнение}
\newtheorem{defin}[theorem]{Определение}

\title{Решение заданий\\ ОП "Квантовая теория поля, теория струн и математическая физика"\\[2cm]
Точно-решаемые модели статистической механики\\ (Я.П. Пугай)}
\author{Коцевич Андрей Витальевич, группа Б02-920с}
\date{6 семестр, 2022}

\begin{document}

\maketitle
\newpage
\tableofcontents{}
\newpage
\section{Основные определения.}
\textbf{Обобщёнаня модель Изинга.}\\
Рассмотрим модель Изинга общего вида. Пусть на решётке из $N$ узлов переменные ассоциированы с узлами и принимают значения $\sigma_i=\pm1$. Состояние системы определяется набором $\sigma=\{\sigma_1,...,\sigma_N\}$, а энергия состояния задаётся как $E(\sigma)$. Определим статистическую сумму $Z_N$, свободную энергию $F_N$ и ожидаемое значение оператора $X$ (со значением $X(\sigma)$ в состоянии $\sigma$) как
\begin{equation}
    Z_N=\sum\limits_{\{\sigma_i=\pm1\}}\exp\left(-\frac{E(\sigma)}{kT}\right),\quad F_N=-kT\log Z_N,\quad \braket{X}=\frac{1}{Z_N}\sum\limits_{\{\sigma_i=\pm1\}}X(\sigma)\exp\left(-\frac{E(\sigma)}{kT}\right)
\end{equation}
Пусть энергия состояния записывается в следующем виде
\begin{equation}
    E(\sigma)=E_\text{int}(\sigma)-H\sum\limits_i\sigma_i
\end{equation}
где $H$ -- магнитное поле и предполагается, что вещественная функция $E_\text{int}(\sigma)$, описывающая взаимодействие между узлами, является чётной функцией $E_\text{int}(\sigma)=E_\text{int}(-\sigma)$.
\begin{enumerate}
    \item Намагниченность на узел $M(H,T)$ выражается как
    \begin{equation}
        M(H,T)=-\frac{\partial f(H,T)}{\partial H}=\frac{\braket{\mathcal{M}}}{N}
    \end{equation}
    Здесь $f(H,T)=\frac{F_N(H,T)}{N}$ -- свободная энергия на узел и $\mathcal{M}=\sum\limits_i\sigma_i$. Проверьте, что при фиксированной температуре $T$ намагниченность является ограниченной, нечётной и неубывающей функцией от магнитного поля $H$
    \begin{equation}
        -1\leq M(H,T)\leq 1,\quad M(H.T)=-M(-H,T),\quad\chi(H,T)=\frac{\partial M(H,T)}{\partial H}\geq 0
    \end{equation}
    В частности, найдите, что восприимчивость $\chi$ выражается через среднее от оператора $\mathcal{M}$ как
    \begin{equation}
        \chi(H,T)=\frac{\partial M(H,T)}{\partial H}=\frac{\braket{(\mathcal{M}-\braket{\mathcal{M}})^2}}{NkT},
    \end{equation}
    а значит выражается через двухточечные корреляционные функции. Нарисуйте схематически поведение восприимчивости $\chi$ как функцию от $H$ для различных температур; $\chi$ как функцию от $T$ для различных значений поля.\\
    \textbf{Решение.}\\
    Для проверки ограниченности воспользуемся определением:
    \begin{equation}
        M(H,T)=\frac{\braket{\mathcal{M}}}{N}=\frac{\braket{\sum\limits_i\sigma_i}}{N}=\frac{\sum\limits_{\{\sigma_i=\pm1\}}\sum\limits_i\sigma_i\exp\left(-\frac{E(\sigma)}{kT}\right)}{NZ_N}
    \end{equation}
    Среднее при любых весах ограничено минимальным и максимальным элементом.
    \begin{equation}
        \frac{(\sum\limits_i\sigma_i)_\text{min}}{N}\leq M(H,T)\leq\frac{(\sum\limits_i\sigma_i)_\text{max}}{N}
    \end{equation}
    \begin{equation}
        (\sum\limits_i\sigma_i)_\text{max}=N,\quad (\sum\limits_i\sigma_i)_\text{min}=-N
    \end{equation}
    \begin{equation}
        \boxed{-1\leq M(H,T)\leq 1}
    \end{equation}
    Для проверки нечётности воспользуемся определением:
    \begin{equation}
        M(H,T)=-\frac{\partial f(H,T)}{\partial H}
    \end{equation}
    Статистическая сумма $Z_N$ является чётной функцией по $H$, значит $f$ тоже и $M(H,T)$ -- нечётная функция по $H$.\\
    Найдём восприимчивость:
    \begin{multline}
        \chi(H,T)=\frac{\partial M(H,T)}{\partial H}=\frac{1}{N}\frac{\partial\braket{\mathcal{M}}}{\partial H}=\frac{\braket{\sum\limits_i\sigma_i}'Z_N-\braket{\sum\limits_i\sigma_i}Z'_N}{NZ^2_N}=\\=\frac{\braket{(\sum\limits_i\sigma_i)^2}-\braket{\sum\limits_i\sigma_i}^2}{NkT}=\frac{\braket{\mathcal{M}^2}-\braket{\mathcal{M}}^2}{NkT}
    \end{multline}
    \begin{equation}
        \boxed{\chi(H,T)=\frac{\braket{(\mathcal{M}-\braket{\mathcal{M}})^2}}{NkT}}
    \end{equation}
    Поскольку $\chi(H,T)$ -- среднее от квадрата величины, то
    \begin{equation}
        \chi(H,T)\geq0
    \end{equation}
    Hамагниченность на узел $M(H,T)$ -- монотонно возрастающая функция.
    \item Рассмотрим простую модель бинарного сплава. Пусть имеются равные количества атомов цинка и меди, расположенные плотно (нет дырок) на кубической объёмно центрированной решётке. При высоких температурах все узлы заселены хаотично. При <<низких>> температурах ($T_c$ порядка 600 К) атомы меди (цинка) располагаются преимущественно на выделенных кубических подрешётках. Т.е. в узлах одной кубической подрешётки -- атомы одного типа, а в центрах этих кубов, т.е. соответственно, на 2 кубической подрешётке -- атомы другого типа. В замороженном состоянии атомы будут сидеть на разных подрешётках.\\
    Простейшая модель, описывающая эту ситуацию, задаётся так. Рассмотрим заселённость одной кубической подрешётки. Пусть $\sigma_i=1$, если заданный узел кубической решётки занимает атом меди, и $\sigma_i=-1$, если узел занят атомом цинка. Будем считать, что взаимодействуют только ближайшие соседи на этой кубической подрешётке и модельный гамильтониан определяется константами взаимодействия $J_{CuCu}$, $J_{ZnZn}$ и $J_{CuZn}$.
    \begin{multline}
        E(\sigma)=-J_{CuCu}\sum\limits_{<ij>}(1+\sigma_i)(1+\sigma_j)-J_{ZnZn}\sum\limits_{<ij>}(1-\sigma_i)(1-\sigma_j)-\\-J_{CuZn}\sum\limits_{<ij>}((1+\sigma_i)(1-\sigma_j)+(1-\sigma_i)(1+\sigma_j))
    \end{multline}
    где сумма берётся по ближайшим соседям на кубической подрешётке.\\
    Параметром порядка для нас будет плотность атомов, скажем, меди, на этой кубической подрешётке. Тогда в низкотемпературном режиме вакуумной конфигурацией будет состояние с либо всеми спинами $-1$, либо $+1$.\\
    Покажите, что такая модель бинарного сплава -- это модель Изинга на трёхмерной решётке. Т.е. модель на трёхмерной решётке, где спины принимают значения $\pm1$, а взаимодействие только между ближайшими соседями.\\
    \textbf{Решение.}
    \begin{multline}
        E(\sigma)=-J_{CuCu}\sum\limits_{<ij>}(1+\sigma_i+\sigma_j+\sigma_i\sigma_j)-J_{ZnZn}\sum\limits_{<ij>}(1-\sigma_i-\sigma_j+\sigma_i\sigma_j)-\\-2J_{CuZn}\sum\limits_{<ij>}(1-\sigma_i\sigma_j)=-(J_{CuCu}+J_{ZnZn}+2J_{CuZn})N_0-2(J_{CuCu}-J_{ZnZn})\sum\limits_{<ij>}\sigma_i-\\-(J_{CuCu}+J_{ZnZn}-2J_{CuZn})\sum\limits_{<ij>}\sigma_i\sigma_j,\quad N_0=\sum\limits_{<ij>}1
    \end{multline}
    Как видно, при $J_{CuCu}+J_{ZnZn}-2J_{CuZn}\neq0$ получилась модель Изинга.
    \item Критические показатели. Предположим, что была найдена следующая форма зависимости <<параметра порядка>> $M(t)$ как функция от обезразмеренной температуры $t=\frac{T-T_c}{T_c}\rightarrow0$.
    \begin{enumerate}
        \item $M(t)=at+bt^\frac{1}{5}+ct^3$;
        \item $M(t)=at^3e^{-t}$;
        \item $M(t)=ate^\frac{1}{t}$;
        \item $M(t)=a\log\left(\exp\frac{1}{t^2}-1\right)$;
        \item $M(t)=a\log t$;
        \item $M(t)=at\log t$.
    \end{enumerate}
    где $a$, $b$ и $c$ -- независящие от температуры константы. Чему равны <<показатели>> в каждом из случаев?\\
    \textbf{Решение.}
    \begin{enumerate}
        \item 
    \begin{equation}
        \lim\limits_{t\rightarrow0}\frac{M(t)}{t^\frac{1}{5}}=b\rightarrow \boxed{n=\frac{1}{5}}
    \end{equation}
    \item \begin{equation}
        \lim\limits_{t\rightarrow0}\frac{M(t)}{t^3}=a\rightarrow \boxed{n=3}
    \end{equation}
    \item \begin{equation}
        \lim\limits_{t\rightarrow0}\frac{M(t)}{t^n}=0\;\forall n\rightarrow \boxed{n=\infty}
    \end{equation}
    \item \begin{equation}
        \lim\limits_{t\rightarrow0}t^2M(t)=a\rightarrow \boxed{n=-2}
        \end{equation}
    \item \begin{equation}
        \lim\limits_{t\rightarrow0}t^\alpha M(t)=0\;\forall\alpha>0\rightarrow \boxed{n=0}
        \end{equation}
    \item \begin{equation}
        \lim\limits_{t\rightarrow0}t^\alpha M(t)=0\;\forall\alpha>-1\rightarrow \boxed{n=1}
        \end{equation}
    \end{enumerate}
    \textbf{Одномерная модель Изинга}\\
    Рассмотрим одномерную модель Изинга
    \begin{equation}
        Z_N=\sum\limits_{\{\sigma_i=\pm1\}}\exp\left(K\sum\limits_{i=1}^N\sigma_i\sigma_{i+1}+h\sum\limits_{i=1}^N\sigma_i\right)
    \end{equation}
    на окружности $\sigma_{N+1}=\sigma_1$
    \item$^*$ $D=1$ модель Изинга и трансфер матрица.
    \begin{itemize}
        \item[i)] Трансфер матрица. Запишите статистическую сумму в виде
        \begin{equation}
            Z_N=\sum\limits_{\{\sigma_i=\pm1\}}T_{\sigma_1,\sigma_2}T_{\sigma_2,\sigma_3}...T_{\sigma_N,\sigma_1}
        \end{equation}
        где $T_{\sigma,\sigma'}=T_{\sigma',\sigma}$. Введите матрицу $\textbf{T}$
        \begin{equation}
            \textbf{T}=\begin{pmatrix}
            T_{++} & T_{+-}\\
            T_{-+} & T_{--}
            \end{pmatrix}
        \end{equation}
        и покажите, что статистическая сумма в одномерном случае может быть записана в виде
        \begin{equation}
            Z_N=\text{Tr}\;\textbf{T}^N
        \end{equation}
        \item[ii)] Точная статистическая сумма $D=1$ модели Изинга. Диагонализуйте трансфер матрицу $\textbf{T}$. Найдите явно след для статистической суммы. Найдите в термодинамическом пределе явное выражение для свободной энергии на узел решётки.
        \item[iii)] Найдите явный вид матрицы \textbf{U}, которая диагонализует \textbf{T}
        \begin{equation}
            \textbf{U}^{-1}\textbf{T}\textbf{U}=\begin{pmatrix}
            \lambda_1 & 0\\
            0 & \lambda_2\\
            \end{pmatrix}
        \end{equation}
        \item[iv)] Намагниченность. Запишите выражение для намагниченности на узел решётки. Покажите, что выражение для одноточечной корреляционной функции, среднего значения спина, можно представить в виде следа
        \begin{equation}
            M=\braket{\sigma_j}=\frac{1}{Z}\text{Tr}\;\textbf{S}\textbf{T}^N
        \end{equation}
        где матрица \textbf{S} имеет вид
        \begin{equation}
            \textbf{S}=\begin{pmatrix}
            1 & 0\\
            0 & -1\\
            \end{pmatrix}
        \end{equation}
        Вычислите явно намагниченность в термодинамическом пределе.
    \end{itemize}
    \textbf{Решение.}
    \begin{itemize}
        \item[i)] Перепишем статистическую сумму одномерной модели Изинга в виде
        \begin{equation}
            Z_N=\sum\limits_{\{\sigma_i=\pm1\}}T_{\sigma_1,\sigma_2}T_{\sigma_2,\sigma_3}...T_{\sigma_N,\sigma_1}
        \end{equation}
        где
        \begin{equation}
            T_{\sigma,\sigma'}=\exp\left(K\sigma\sigma'+\frac{h(\sigma+\sigma')}{2}\right),\quad T_{\sigma,\sigma'}=T_{\sigma',\sigma}
        \end{equation}
        Введём матрицу
        \begin{equation}
            \textbf{T}=\begin{pmatrix}
        T_{++} & T_{+-}\\
        T_{-+} & T_{--}
        \end{pmatrix}=\begin{pmatrix}
        e^{K+h} & e^{-K}\\
        e^{-K} & e^{K-h}
        \end{pmatrix}
        \end{equation}
        При помощи её запишем $T_{\sigma,\sigma'}$:
        \begin{equation}
            T_{\sigma,\sigma'}=\begin{pmatrix}
            \delta_{\sigma,1} & \delta_{\sigma,-1}
            \end{pmatrix}\textbf{T}\begin{pmatrix}
            \delta_{\sigma',1}\\\delta_{\sigma',-1}
            \end{pmatrix}
        \end{equation}
        \begin{equation}
            T_{\sigma_1,\sigma_2}T_{\sigma_2,\sigma_3}...T_{\sigma_N,\sigma_1}=\begin{pmatrix}
            \delta_{\sigma_1,1} & \delta_{\sigma_1,-1}
            \end{pmatrix}\textbf{T}^N\begin{pmatrix}
            \delta_{\sigma_1,1}\\\delta_{\sigma_1,-1}
            \end{pmatrix}
        \end{equation}
        \begin{equation}
            Z_N=\sum\limits_{\{\sigma_i=\pm1\}}T_{\sigma_1,\sigma_2}T_{\sigma_2,\sigma_3}...T_{\sigma_N,\sigma_1}=\begin{pmatrix}
            1 & 0
            \end{pmatrix}\textbf{T}^N\begin{pmatrix}
            1\\0
            \end{pmatrix}+\begin{pmatrix}
            0 & 1
            \end{pmatrix}\textbf{T}^N\begin{pmatrix}
            0\\1
            \end{pmatrix}
        \end{equation}
        \begin{equation}
            \boxed{Z_N=\text{Tr}\;\textbf{T}^N}
        \end{equation}
    \item[ii)] Диагонализуем матрицу $\textbf{T}$:
    \begin{equation}
        \textbf{U}^{-1}\textbf{T}\textbf{U}=\begin{pmatrix}
            \lambda_1 & 0\\
            0 & \lambda_2\\
        \end{pmatrix}
    \end{equation}
    \begin{equation}
        \lambda_{1,2}=e^K\ch h\pm\sqrt{e^{2K}\sh^2h+e^{-2K}}
    \end{equation}
    \begin{equation}
        (\textbf{U}^{-1}\textbf{T}\textbf{U})^N=\textbf{U}^{-1}\textbf{T}^N\textbf{U}=\begin{pmatrix}
            \lambda^N_1 & 0\\
            0 & \lambda^N_2\\
        \end{pmatrix}
    \end{equation}
    \begin{equation}
        \text{Tr}\;\textbf{T}^N=\text{Tr}\;\textbf{U}^{-1}\textbf{T}^N\textbf{U}=\lambda^N_1+\lambda^N_2
    \end{equation}
    \begin{equation}
        \boxed{Z_N=\left(e^K\ch h+\sqrt{e^{2K}\sh^2h+e^{-2K}}\right)^N+\left(e^K\ch h-\sqrt{e^{2K}\sh^2h+e^{-2K}}\right)^N}
    \end{equation}
    Термодинамический предел $N\rightarrow\infty$:
    \begin{equation}
        Z_N=\text{Tr}\;\textbf{T}^N=\lambda^N_1\left(1+\frac{\lambda^N_2}{\lambda^N_1}\right)\rightarrow\lambda_1^N
    \end{equation}
    \begin{equation}
        f=\frac{F_N}{N}=-\frac{kT}{N}\log Z_N=-kT\log\lambda_1
    \end{equation}
    \begin{equation}
        \boxed{f=-kT\log\left(e^K\ch h+\sqrt{e^{2K}\sh^2h+e^{-2K}}\right)}
    \end{equation}
    \item[iii)] Собственные векторы \textbf{T}:
    \begin{equation}
        \Vec{v}_1=\begin{pmatrix}
        e^{2K}\ch h+\sqrt{e^{2K}\sh^2h+1}\\
        1
        \end{pmatrix},\quad\Vec{v}_2=\begin{pmatrix}
        e^{2K}\ch h-\sqrt{e^{2K}\sh^2h+1}\\
        1
        \end{pmatrix}
    \end{equation}
    Определим переменную $\phi$:
    \begin{equation}
        \ctg2\phi=e^{2K}\sh h,\quad0<\phi<\frac{\pi}{2}
    \end{equation}
    \begin{equation}
        \boxed{\textbf{U}=\begin{pmatrix}
        \cos\phi & -\sin\phi\\
        \sin\phi & \cos\phi
        \end{pmatrix}}
    \end{equation}
    \item[iv)] Намагниченность на узел решётки:
    \begin{equation}
        \boxed{M(H,T)=-\frac{\partial f(H,T)}{\partial H}=\frac{e^K\sinh h}{\sqrt{e^{2K}\sinh^2h+e^{-2K}}}}
    \end{equation}
    \begin{equation}
        \braket{\sigma_j}=\frac{1}{Z_N}\sum\limits_{\{\sigma_i=\pm1\}}\sigma_j\exp\left(-\frac{E(\sigma)}{kT}\right)=\frac{1}{Z_N}\sum\limits_{\{\sigma_i=\pm1\}}\sigma_jT_{\sigma_1,\sigma_2}T_{\sigma_2,\sigma_3}...T_{\sigma_N,\sigma_1}
    \end{equation}
    Введём матрицу:
    \begin{equation}
    \textbf{S}=\begin{pmatrix}
        1 & 0\\
        0 & -1\\
    \end{pmatrix}
    \end{equation}
    Введём по аналогии с $T_{\sigma,\sigma'}$:
    \begin{equation}
        S_{\sigma,\sigma'}=\begin{pmatrix}
        \delta_{\sigma,1} & \delta_{\sigma,-1}
        \end{pmatrix}\textbf{S}\begin{pmatrix}
        \delta_{\sigma',1}\\\delta_{\sigma',-1}
        \end{pmatrix}=\sigma\delta(\sigma,\sigma')
    \end{equation}
    \begin{equation}
        \braket{\sigma_j}=\frac{1}{Z_N}\text{Tr}\;\textbf{T}^{j-1}\textbf{S}\textbf{T}^{N+1-j}
    \end{equation}
    \begin{equation}
        \boxed{\braket{\sigma_j}=\frac{1}{Z_N}\text{Tr}\;\textbf{S}\textbf{T}^{N}}
    \end{equation}
    \begin{multline}
    \textbf{U}^{-1}\textbf{S}\textbf{U}\textbf{U}^{-1}\textbf{T}\textbf{U}...\textbf{U}^{-1}\textbf{T}\textbf{U}=\textbf{U}^{-1}\textbf{S}\textbf{U}\begin{pmatrix}
            \lambda^N_1 & 0\\
            0 & \lambda^N_2\\
        \end{pmatrix}=\\=\begin{pmatrix}
            \cos2\phi & -\sin2\phi\\
            -\sin2\phi & -\cos2\phi\\
        \end{pmatrix}\begin{pmatrix}
            \lambda^N_1 & 0\\
            0 & \lambda^N_2\\
        \end{pmatrix}=\begin{pmatrix}
            \lambda^N_1\cos2\phi & -\lambda_2^N\sin2\phi\\
            -\lambda_1^N\sin2\phi & -\lambda^N_2\cos2\phi\\
        \end{pmatrix}
    \end{multline}
    \begin{equation}
        \boxed{\braket{\sigma_j}=\frac{\lambda^N_1-\lambda^N_2}{\lambda^N_1+\lambda^N_2}\cos2\phi}
    \end{equation}
    Термодинамический предел $N\rightarrow\infty$:
    \begin{equation}
        \boxed{\braket{\sigma_j}=\cos2\phi}
    \end{equation}
    \end{itemize}
    \item Корреляционные функции
    \begin{itemize}
        \item[i)] Двухточечная корреляционная функция в $D=1$ модели Изинга. Запишите двухточечный коррелятор $\braket{\sigma_i\sigma_j}$, используя матрицы \textbf{S} и \textbf{T}. Покажите, что связная корреляционная функция $\braket{\sigma_i\sigma_j}-\braket{\sigma_i}\braket{\sigma_j}$ зависит от расстояния $i-j$ между узлами решётки, но не от индивидуальных $i$ и $j$. В пределе $N\rightarrow\infty$ (сохраняя расстояние $i-j$ фиксированным), найдите эту корреляционную функцию.
        \item[ii)] Проанализируйте свойства связного коррелятора как функции от расстояния для положительных температур $T$ и вещественного магнитного поля $H$ (здесь $K=\frac{J}{kT}$, $h=\frac{H}{kT}$). В общей модели Изинга ожидается, что коррелятор
        \begin{equation}
            g_{ij}=\braket{\sigma_i\sigma_j}-\braket{\sigma_i}\braket{\sigma_j}
        \end{equation}
        спадает экспоненциально с расстоянием $x=|i-j|$. Так ли это в данном случае? Что является аналогом корреляционной длины $\xi$, определяемой в общем случае для коррелятора $g_{ij}=g(r_{ij})$ как
        \begin{equation}
            g(x\Vec{k})\sim x^{-d+2-\eta}\exp(-x/\xi)
        \end{equation}
        где $d$ -- размерность, $\eta$ -- некоторое число, $\Vec{k}$ -- единичный вектор, определяющий направление от точки $i$ к точке $j$.\\
        Одним из признаков критической точки являются бесконечные корреляторы, т.е. расходимость корреляционной длины. Есть ли в одномерной модели Изинга такая расходимость?\\
        Как ведёт себя спиновый коррелятор и чему равно значение $\eta$ в такой точке?
    \end{itemize}
    \textbf{Решение.}
    \begin{itemize}
    \item[i)] По аналогии запишем двухточечный коррелятор $\braket{\sigma_i\sigma_j}$:
        \begin{equation}
            \braket{\sigma_i\sigma_j}=\frac{1}{Z_N}\text{Tr}\;\textbf{S}\textbf{T}^{j-i}\textbf{S}\textbf{T}^{N+i-j}
        \end{equation}
        \begin{multline}
            \textbf{U}^{-1}\textbf{S}\textbf{U}\textbf{U}^{-1}\textbf{T}\textbf{U}...\textbf{U}^{-1}\textbf{T}\textbf{U}=\begin{pmatrix}
        \cos2\phi & -\sin2\phi\\
        -\sin2\phi & -\cos2\phi\\
    \end{pmatrix}\begin{pmatrix}
        \lambda^{j-i}_1 & 0\\
        0 & \lambda^{j-i}_2\\
    \end{pmatrix}\times\\\times\begin{pmatrix}
        \cos2\phi & -\sin2\phi\\
        -\sin2\phi & -\cos2\phi\\
    \end{pmatrix}\begin{pmatrix}
        \lambda^{N+i-j}_1 & 0\\
        0 & \lambda^{N+i-j}_2\\
    \end{pmatrix}
    \end{multline}
    \begin{equation}
        \braket{\sigma_i\sigma_j}=\cos^22\phi+\frac{\lambda_1^{ij+N}\lambda_2^{j-i}+\lambda_1^{j-i}\lambda_2^{i-j+N}}{\lambda_1^N+\lambda_2^N}\sin^22\phi
    \end{equation}
    Связная корреляционная функция:
    \begin{equation}
        \boxed{g_{ij}=\cos^22\phi\left(1-\left(\frac{\lambda^N_1\lambda^N_2}{\lambda^N_1+\lambda^N_2}\right)^2\right)+\frac{\lambda_1^{i-j+N}\lambda_2^{j-i}+\lambda_1^{j-i}\lambda_2^{i-j+N}}{\lambda_1^N+\lambda_2^N}\sin^22\phi}
    \end{equation}
    Как видно, связная корреляционная функция зависит отрасстояния $i-j$ между узлами решётки, но не от индивидуальных $i$ и $j$. Термодинамический предел$N\rightarrow\infty$:
    \begin{equation}
        \boxed{g_{ij}=\sin^22\phi\left(\frac{\lambda_2}{\lambda_1}\right)^{j-i}}
    \end{equation}
    \item[ii)] Коррелятор $g_{ij}$ экспоненциально стремится к 0 при стремлении $j-i\rightarrow\infty$. Аналог корреляционной длины:
    \begin{equation}
        \boxed{\xi=\frac{1}{\ln\frac{\lambda_1}{\lambda_2}}}
    \end{equation}
    Заметим, что при $H=0$:
    \begin{equation}
        \lim\limits_{T\rightarrow0}\frac{\lambda_2}{\lambda_1}=1\rightarrow\xi\rightarrow\infty
    \end{equation}
    В одномерной модели Изинга существует расходимость $\xi$ при $H=0$, $T=0$. В критической точке $g_{ij}$ от расстояние не зависит, поэтому
    \begin{equation}
        -1+2-\eta=0\rightarrow\boxed{\eta=1}
    \end{equation}
    \end{itemize}
    \item Одномерная модель Изинга спина 1. По аналогии с предыдущими упражнениями, рассмотрите пример $D=1$ модели Изинга с периодическими граничными условиями, в которой спины принимают значения $\sigma\in\{-1,0,1\}$
    \begin{equation}
        Z_N=\sum\limits_{\{\sigma_i=\pm1,0\}}\exp\left(K\sum\limits_{i=1}^N\sigma_i\sigma_{i+1}+h\sum\limits_{i=1}^N\sigma_i+D\sum_{i=1}^N\sigma_i^2\right)
    \end{equation}
    \begin{itemize}
        \item[i)] Напишите соответствующую трансфер матрицу.
        \item[ii)] Найдите свободную энергию на узел в термодинамическом пределе ($h=D=0$).
        \item[iii)] Напишите матрицу, вставляющую спины для анализа корреляторов и выпишите двухточечный коррелятор в терминах следа трансфер матрицы.
    \end{itemize}
    \textbf{Решение.}
    \begin{itemize}
        \item[i)] Перепишем статистическую сумму в виде
        \begin{equation}
            Z_N=\sum\limits_{\{\sigma_i=\pm1,0\}}T_{\sigma_1,\sigma_2}T_{\sigma_2,\sigma_3}...T_{\sigma_N,\sigma_1}
        \end{equation}
        где
        \begin{equation}
            T_{\sigma,\sigma'}=\exp\left(K\sigma\sigma'+\frac{h(\sigma+\sigma')}{2}+\frac{D(\sigma^2+\sigma'^2)}{2}\right),\quad T_{\sigma,\sigma'}=T_{\sigma',\sigma}
        \end{equation}
        Введём матрицу
        \begin{equation}
            \textbf{T}=\begin{pmatrix}
        T_{11} & T_{10} & T_{1-1}\\
        T_{01} & T_{00} & T_{0-1}\\
        T_{-11} & T_{-10} & T_{-1-1}\\
        \end{pmatrix}=\begin{pmatrix}
        e^{K+h+D} & e^{\frac{h+D}{2}} & e^{-K+D}\\
        e^{\frac{h+D}{2}} & 1 & e^{\frac{-h+D}{2}}\\
        e^{-K+D} & e^{\frac{-h+D}{2}} & e^{K-h+D}
        \end{pmatrix}
        \end{equation}
        При помощи её запишем $T_{\sigma,\sigma'}$:
        \begin{equation}
            T_{\sigma,\sigma'}=\begin{pmatrix}
            \delta_{\sigma,1} & \delta_{\sigma,0} & \delta_{\sigma,-1}
            \end{pmatrix}\textbf{T}\begin{pmatrix}
            \delta_{\sigma',1}\\\delta_{\sigma',0}\\\delta_{\sigma',-1}
            \end{pmatrix}
        \end{equation}
        \begin{equation}
            T_{\sigma_1,\sigma_2}T_{\sigma_2,\sigma_3}...T_{\sigma_N,\sigma_1}=\begin{pmatrix}
            \delta_{\sigma,1} & \delta_{\sigma,0} & \delta_{\sigma,-1}
            \end{pmatrix}\textbf{T}^N\begin{pmatrix}
            \delta_{\sigma',1}\\\delta_{\sigma',0}\\\delta_{\sigma',-1}
            \end{pmatrix}
        \end{equation}
        \begin{equation}
            Z_N=\sum\limits_{\{\sigma_i=\pm1,0\}}T_{\sigma_1,\sigma_2}T_{\sigma_2,\sigma_3}...T_{\sigma_N,\sigma_1}=\text{Tr}\;\textbf{T}^N
        \end{equation}
        \item[ii)] Диагонализуем матрицу $\textbf{T}$:
    \begin{equation}
        \textbf{U}^{-1}\textbf{T}\textbf{U}=\begin{pmatrix}
            \lambda_1 & 0 & 0\\
            0 & \lambda_2 & 0\\
            0 & 0 & \lambda_3\\
        \end{pmatrix}
    \end{equation}
    \begin{equation}
        \lambda_{1,2}=\frac{e^{-K}}{2}\left(1+\sh K+\ch K+\sh 2K+\ch 2K\pm e^K\sqrt{2\ch2K-4\ch K+11}\right) 
    \end{equation}
    \begin{equation}
        \lambda_3=2\sh K
    \end{equation}
    \begin{equation}
        (\textbf{U}^{-1}\textbf{T}\textbf{U})^N=\textbf{U}^{-1}\textbf{T}^N\textbf{U}=\begin{pmatrix}
            \lambda_1^N & 0 & 0\\
            0 & \lambda_2^N & 0\\
            0 & 0 & \lambda_3^N\\
        \end{pmatrix}
    \end{equation}
    \begin{equation}
        \text{Tr}\;\textbf{T}^N=\text{Tr}\;\textbf{U}^{-1}\textbf{T}^N\textbf{U}=\lambda^N_1+\lambda^N_2+\lambda^N_3
    \end{equation}
    Термодинамический предел $N\rightarrow\infty$:
    \begin{equation}
        Z_N=\text{Tr}\;\textbf{T}^N\rightarrow\lambda_1^N
    \end{equation}
    \begin{equation}
        f=\frac{F_N}{N}=-\frac{kT}{N}\log Z_N=-kT\log\lambda_1
    \end{equation}
    \begin{equation*}
        \boxed{f=-kT\log\left(\frac{e^{-K}}{2}\left(1+\sh K+\ch K+\sh 2K+\ch 2K+e^K\sqrt{2\ch2K-4\ch K+11}\right) \right)}
    \end{equation*}
    \item[iii)] Матрица, вставляющая спины:
    \begin{equation}
        \textbf{S}=\begin{pmatrix}
        1 & 0 & 0\\
        0 & 0 & 0\\
        0 & 0 & -1\\
    \end{pmatrix}
    \end{equation}
    \begin{equation}
        S_{\sigma,\sigma'}=\begin{pmatrix}
        \delta_{\sigma,1} & \delta_{\sigma,0} & \delta_{\sigma,-1}
        \end{pmatrix}\textbf{S}\begin{pmatrix}
        \delta_{\sigma',1}\\\delta_{\sigma',0}\\\delta_{\sigma',-1}
        \end{pmatrix}=\sigma\delta(\sigma,\sigma')
    \end{equation}
    Двухточечный коррелятор $\braket{\sigma_i\sigma_j}$:
        \begin{equation}
            \boxed{\braket{\sigma_i\sigma_j}=\frac{1}{Z_N}\text{Tr}\;\textbf{S}\textbf{T}^{j-i}\textbf{S}\textbf{T}^{N+i-j}}
        \end{equation}
    \end{itemize}
\end{enumerate}
\section{Дуальность Краммерса-Ванье}
\textbf{Дуальность KW.}
\begin{enumerate}
    \item При сравнении высоко и низко температурных разложений мы нашли, что параметры моделей связаны как
    \begin{equation}
        \tanh K=e^{-2L^*},\quad \tanh L=e^{-2K^*}
    \end{equation}
    Утверждение дуальности звучит как связь между статсуммой от параметров $K$, $L$ и $K^*$, $L^*$.
    \begin{equation}
        Z[K,L]=\{known\; function\}\cdot Z[K^*,L^*]
    \end{equation}
    \begin{itemize}
        \item[i)] Проверьте, что верна следующая форма записи соотношения
        \begin{equation}
            \sinh2L\sinh2K^*=1,\quad\sinh2L^*\sinh2K=1
        \end{equation}
        \item[ii)] Покажите, что свободная энергия на узел
        \begin{equation}
            f[K,L]=-\lim\limits_{N\rightarrow\infty}\frac{1}{N}\log Z
        \end{equation}
        удовлетворяет соотношению
        \begin{equation}
            f[K^*,L^*]=f[K,L]+\frac{1}{2}\log\sinh2K\sinh2L
        \end{equation}
    \end{itemize}
    \textbf{Решение.}
    \begin{itemize}
        \item[i)] Воспользуемся формулой для двойного гиперболического синуса:
        \begin{equation}
            \sinh2L\sinh2K^*=\frac{2\tanh L}{1-\tanh^2L}\frac{e^{2K^*}-e^{-2K^*}}{2}=\frac{2e^{-2K^*}}{1-e^{-4K^*}}\frac{e^{2K^*}-e^{-2K^*}}{2}=1
        \end{equation}
        \begin{equation}
            \sinh2L^*\sinh2K=\frac{e^{2L^*}-e^{-2L^*}}{2}\frac{2\tanh K}{1-\tanh^2K}=\frac{e^{2L^*}-e^{-2L^*}}{2}\frac{2e^{-2L^*}}{1-e^{-4L^*}}=1
        \end{equation}
        \begin{equation}
            \boxed{\sinh2L\sinh2K^*=1,\quad\sinh2L^*\sinh2K=1}
        \end{equation}
        \item[ii)] Высокотемпературное разложение:
        \begin{equation}
            Z[K,L]=(2\cosh K\cosh L)^N\sum\limits_{P(s,r)}(\tanh K)^r(\tanh L)^s
        \end{equation}
        Низкотемпературное разложение:
        \begin{equation}
            Z[K^*,L^*]=2e^{N(K^*+L^*)}\sum\limits_{P(s,r)}(e^{-2L^*})^r(e^{-2K^*})^s
        \end{equation}
        Свободная энергия на узел
        \begin{multline}
            f[K,L]=-\lim\limits_{N\rightarrow\infty}\frac{1}{N}\log\left((2\cosh K\cosh L)^N\sum\limits_{P(s,r)}(\tanh K)^r(\tanh L)^s\right)=\\=-\log(2\cosh K\cosh L)-\lim\limits_{N\rightarrow\infty}\frac{1}{N}\log\left(\sum\limits_{P(s,r)}(\tanh K)^r(\tanh L)^s\right)
        \end{multline}
        \begin{multline}
            f[K^*,L^*]=-\lim\limits_{N\rightarrow\infty}\frac{1}{N}\log\left(2e^{N(K^*+L^*)}\sum\limits_{P(s,r)}(e^{-2L^*})^r(e^{-2K^*})^s\right)=\\=-K^*-L^*-\lim\limits_{N\rightarrow\infty}\frac{1}{N}\log\left(\sum\limits_{P(s,r)}(e^{-2L^*})^r(e^{-2K^*})^s\right)
        \end{multline}
        Подставим $L^*=-\frac{1}{2}\log\tanh K$, $K^*=-\frac{1}{2}\log\tanh L$ в $f[K^*,L^*]$:
        \begin{multline}
            f[K^*,L^*]=f[K,L]+\log\left(2\cosh K\cosh L\right)-K^*-L^*=\\=f[K,L]+\log\left(2\cosh K\cosh L\right)+\frac{1}{2}\log(\tanh K\tanh L)=f[K,L]+\\+\frac{1}{2}\log(4\cosh^2 K\cosh^2 L\tanh K\tanh L)
        \end{multline}
        \begin{equation}
            \boxed{f[K^*,L^*]=f[K,L]+\frac{1}{2}\log(\sinh2K\sinh2L)}
        \end{equation}
    \end{itemize}
    \textbf{Критическая точка.}
    \item $^*$ Модель Изинга -- простейший представитель $q$-позиционных моделей Поттса. В моделях Поттса спины принимают значения $\sigma_i\in\{1,2,...,q\}$. Пусть на квадратной решётке взаимодействуют только ближайшие соседи (символ $\braket{ij}$ для соседних по вертикали и горизонтали спинов $\sigma_i$, $\sigma_j$). Зададим статистическую суммму как
    \begin{equation}
        Z=\sum\limits_{\sigma_1,...,\sigma_N}\prod\limits_{\braket{ij}}e^{K\delta_{\sigma_i,\sigma_j}}
    \end{equation}
    Рассматривая пределы высоких и низких температур, попытайтесь вывести уравнение дуальности
    \begin{equation}
        Z=qe^{2NK}F(e^{-K})=q^{-N}(e^K+q-1)^{2N}F\left(\frac{e^K-1}{e^K+q-1}\right)
    \end{equation}
    Предполагая единственность критической точки, найдите критическую точку.\\
    \textbf{Решение.}\\
    Рассмотрим низкотемпературное разложение.\\
    Пусть среди соседних спинов по горизонтали ровно $s$ рёбер, соединяющих не равные друг другу спины. Тогда вклад от горизонтальных рёбер
    \begin{equation}
        \exp\left(K\sum\limits_{<ij>hor}\delta_{\sigma_i,\sigma_j}\right)=\exp((N-s)K)
    \end{equation}
    Пусть $r$ -- число неравных соседних спинов по вертикальному направлению.
    \begin{equation}
        \exp\left(K\sum\limits_{<ij>vert}\delta_{\sigma_i,\sigma_j}\right)=\exp((N-r)K)
    \end{equation}
    Вклад от конфигурации $\{\sigma_1,...,\sigma_N\}$ (с фиксированными значениями $(s,r)$) в статистическую сумму
    \begin{equation}
        \exp(2NK)\exp(-Ks-Kr)
    \end{equation}
    Рассмотрим дуальную решётку. Будем искать пары различных спинов, которые связаны ребром. Если такая пара найдётся, нарисуем на дуальной решётке ребро, проходящее между такими спинами. Если спины одинаковые, то ребро на дуальной решётке проводить не нужно. Таким образом, решётка спинов для данной конфигурации разбилась линиями на кластеры, внутри которых спины одинаковы. Нарисованные линии образуют замкнутые многоугольники (в каждом узле дуальной решётки сходится лишь чётной число линий).\\
    Обратно, нарисуем на дуальной решётке всевозможные замкнутые конфигурации, так что общее число вертикальных рёбер равно $s$, а число горизонтальных $r$. Каждая такая конфигурация $P(s,r)$ даёт вклад в статистическую сумму $\exp(N(K+L))\exp(-Ks-Lr)$. Для получения $Z$ остаётся только взять сумму по всем возможным конфигурациям и умножить результат на $q(q-1)^N$ (с учётом возможности выбора во внешнем домене из $q$ спинов, а во остальных $D$ доменах -- из $q-1$).
    \begin{equation}
        Z=q(q-1)^De^{2NK}\sum\limits_{P(s,r)}(e^{-K})^{s+r}
    \end{equation}
    Таким образом,
    \begin{equation}
        \boxed{F(x)=(q-1)^D\sum\limits_{P(s,r)}x^{s+r}}
    \end{equation}
    Рассмотрим высокотемпературное разложение. Запишем больцмановский вес соседних спинов в виде
    \begin{equation}
        e^{K\delta_{\sigma,\sigma'}}=C(K)(1+T(K)g(\sigma,\sigma')),\quad g(\sigma,\sigma')=q\delta_{\sigma\sigma'}-1
    \end{equation}
    Найдём $C(K)$ и $T(K)$.
    \begin{equation}
        \begin{cases}
            e^K=C(1+T(q-1)),\quad\sigma=\sigma'\\
            1=C(1-T),\quad\quad\quad\quad\;\;\sigma\neq\sigma'
        \end{cases}
    \end{equation}
    \begin{equation}
        T(K)=\frac{e^K-1}{e^K+q-1},\quad C(K)=\frac{e^K+q-1}{q}
    \end{equation}
    Статистическая сумма имеет вид
    \begin{equation}
        Z=C(K)^{2N}\sum\limits_{\sigma_1,...,\sigma_N}\prod\limits_{\braket{ij}}(1+T(K)g(\sigma,\sigma'))
    \end{equation}
    Раскроем скобки и заметим, что общий член будет иметь вид
    \begin{equation}
        T(K)^{s+r}\prod g(\sigma,\sigma')
    \end{equation}
    Будем на прямой решётке рисовать линию, если соответствующее ребро содержит спины, входящие в множитель произведения. Для вычисления $\sum\limits_{\sigma_1,...,\sigma_N}\prod g(\sigma,\sigma')$ выведем некоторые свойства $g(\sigma,\sigma')$.
    \begin{equation}
        \sum\limits_{\sigma=1}^qg(\sigma,\sigma')=q-1-(q-1)=0
    \end{equation}
    \begin{multline}
        \sum\limits_{\sigma=1}^qg(\sigma_1,\sigma)g(\sigma,\sigma_2)=\sum\limits_{\sigma=1}^q(q^2\delta_{\sigma_1,\sigma}\delta_{\sigma_2,\sigma}-q(\delta_{\sigma_1,\sigma}+\delta_{\sigma_2,\sigma})+1)=q^2\delta_{\sigma_1,\sigma_2}-2q+q=\\=qg(\sigma_1,\sigma_2)
    \end{multline}
    \begin{equation}
        \sum\limits_{\sigma,\sigma'=1}^qg(\sigma,\sigma')g(\sigma,\sigma')=\sum\limits_{\sigma,\sigma'=1}^q(q^2\delta_{\sigma,\sigma'}^2-2q\delta_{\sigma,\sigma'}+1)=q^3-2q^2+q^2=q^2(q-1)
    \end{equation}
    Из этих соотношений следует, что ненулевым слагаемым будут соответствовать замкнутые многоугольники. $r$ имеет смысл числа горизонтальных линий, а $s$ -- вертикальных.
    \begin{equation}
        \sum\limits_{\sigma_1,...,\sigma_N}\prod g(\sigma,\sigma')=q^N(q-1)^D
    \end{equation}
    где $D$ -- число замкнутых доменов (кроме внешнего).
    \begin{equation}
        Z=({e^K+q-1})^{2N}q^{-N}(q-1)^D\sum\limits_{P(s,r)}\left(\frac{e^K-1}{e^K+q-1}\right)^{r+s}
    \end{equation}
    Сравним разложения:
    \begin{equation}
        \boxed{Z=qe^{2NK}F(e^{-K})=q^{-N}(e^K+q-1)^{2N}F\left(\frac{e^K-1}{e^K+q-1}\right)}
    \end{equation}
    Условие критической температуры:
    \begin{equation}
        e^{-K_c}=\frac{e^{K_c}-1}{e^{K_c}+q-1}
    \end{equation}
    Решая квадратное уравнение, получим
    \begin{equation}
        \boxed{K_c=\log(\sqrt{q}+1)}
    \end{equation}
    \item Дуальные спины:
        \begin{equation}
            n_{i_*}=\frac{1-\mu_{i_*}}{2}
        \end{equation}
        В терминах дуальных спинов 
        \begin{equation}
            Z=2^{-N}\frac{e^{2NK}}{\cosh^{2N}K^*}\sum\limits_{\mu_1=\pm1}...\sum\limits_{\mu_N=\pm1}\exp\left(\sum\limits_{ij}K^*\mu_i\mu_j\right),\quad e^{-2K}=\tanh K^*
        \end{equation}
    \item Вывод KW для корреляторов.
    \begin{itemize}
        \item[i)] Произведите аналогичное преобразование дуальности для двухточечного коррелятора спинов и операторов беспорядка для ближайших соседей $\braket{\sigma_1\sigma_2}$ $\braket{\mu_1\mu_2}$. Т.е. произведите пересуммирование рёберных переменных и сделайте переход к суммированию по дуальным спинам. Чему равны ответы для корреляторов в теории с параметром $K^*$?
        \item[ii)] Повторите упражнение для корреляторов спинов и беспорядка более сложного типа -- например, для 4-точечной корреляционной функции.\\
        Напишите, в какие объекты в дуальной теории переходят общие двухточечные корреляторы $\braket{\sigma_i\sigma_j}|_K\rightarrow?|_{K^*}$ и
        $\braket{\mu_i\mu_j}|_K\rightarrow?|_{K^*}$. Возьмите любую нетривиальную конструкцию с $i$ и $j$, расположенными не на одной горизонтальной или вертикальной линии и выпишите явный ответ для теории со взаимодействием $K^*$.
    \end{itemize}
    \textbf{Решение.}
    \begin{itemize}
        \item[i)]
        \begin{multline}
            \braket{\sigma_1\sigma_2}=\frac{1}{Z}\sum\limits_{\sigma_1,...,\sigma_N}\prod_{\braket{ij}}\sigma_1\sigma_2e^{K\sigma_i\sigma_j}=\sum\limits_{\mu^*_1=\pm1}...\sum\limits_{\mu^*_N=\pm1}\prod_{\braket{ij}}(1-\mu^*_1\mu^*_2e^{-2K})(1+\mu^*_i\mu^*_je^{-2K})=\\=\frac{2^{-N}e^{2NK}}{Z\cosh^{2N}K^*}\sum\limits_{\mu^*_1=\pm1}...\sum\limits_{\mu^*_N=\pm1}\exp\left(\sum\limits_{ij}K^*\mu^*_i\mu^*_j-K^*\mu^*_1\mu^*_2\right)
        \end{multline}
        Сумму в экспоненте можно переписать в виде
        \begin{equation}
            \sum\limits_{ij}K^*\mu^*_i\mu^*_jg_{ij},\quad g_{ij}=\begin{cases}
                1,\quad\quad i,j\neq1,2\\
                -1,\quad i,j=1,2
            \end{cases}
        \end{equation}
        \begin{equation}
            \boxed{\braket{\sigma_1,\sigma_2}=\frac{2^{-N}e^{2NK}}{\cosh^{2N}K^*}\frac{Z(\tilde K,\tilde L)}{Z(K,L)}}
        \end{equation}
        \begin{equation}
            \boxed{\braket{\mu_1,\mu_2}=\frac{Z(\tilde K,\tilde L)}{Z(K,L)},\quad\braket{\mu^*_1,\mu^*_2}=\frac{Z(\tilde K^*,\tilde L^*)}{Z(K^*,L^*)}}
        \end{equation}
        \item[ii)]
        \begin{equation}
            \braket{\sigma_k\sigma_l}=\braket{\sigma_k\sigma_{k+1}...\sigma_{l-1}\sigma_l}
        \end{equation}
        \begin{equation}
            \braket{\sigma_k\sigma_l}=\frac{e^{2NK}2^{-N}}{Z}\sum\limits_{\mu^*_1=\pm1}...\sum\limits_{\mu^*_N=\pm1}\prod_{\braket{ij}}^{\in\Gamma}(1-\mu^*_1\mu^*_2e^{-2K})\prod_{\braket{ij}}^{\notin\Gamma}(1+\mu^*_i\mu^*_je^{-2K})
        \end{equation}
        \begin{equation}
            g_{ij}=\begin{cases}
                1,\quad\quad i,j\in\Gamma\\
                -1,\quad i,j\notin\Gamma
            \end{cases}
        \end{equation}
        \begin{equation}
            \braket{\sigma_k\sigma_l}=\frac{2^{-N}e^{2NK}}{Z\cosh^{2N}K^*}\sum\limits_{\mu^*_1=\pm1}...\sum\limits_{\mu^*_N=\pm1}\exp\left(\sum\limits_{ij}K^*g_{ij}\mu^*_i\mu^*_j\right)
        \end{equation}
        \begin{equation}
            \braket{\sigma_k\sigma_l}=\frac{2^{-N}e^{2NK}}{\cosh^{2N}K^*}\frac{Z(\tilde K^*,\tilde L^*)}{Z(K,L)}=\frac{Z(\tilde K^*,\tilde L^*)}{Z(K^*,L^*)}
        \end{equation}
        \begin{equation}
            \boxed{\braket{\sigma_k\sigma_l}|_K=\braket{\mu_k\mu_l}|_{K^*}}
        \end{equation}
        Аналогично,
        \begin{equation}
            \boxed{\braket{\mu_k\mu_l}|_K=\braket{\sigma_k\sigma_l}|_{K^*}}
        \end{equation}
    \end{itemize}
    \textbf{Оператор беспорядка}
    \item \begin{itemize}
        \item[i)] Фиксируйте координаты параметров беспорядка $j_1$, $j_2$ на двумерной плоскости. Проведите другой путь $\Gamma'$ между этими двумя точками. Покажите, что различие между двумя прескрипциями на соответствующих неоднородных решётках сводится к вычислению статистической суммы, в которой контур замкнут.
        \item[ii)] Докажите, что двухточечная корреляционная функция не изменяется при деформации контура $\Gamma\rightarrow\Gamma'$.
    \end{itemize}
    \textbf{Решение.}
    \begin{itemize}
        \item[i)] Разрушим связи (поменяем знаки констант связи), которые пересекаются путём $\Gamma$. Далее разрушим связи, которые пересекаются путём $\Gamma+\Gamma'$. Связи, которые пересекаются путём $\Gamma$, будут разрушены дважды, а значит их константы не поменяются, связи, пересекаемые путём $\Gamma'$, будут разрушены. А значит различие в подсчёте статистической суммы по $\Gamma+\Gamma'$.
        \item[ii)] Заметим интересный факт. Статистическая сумма вдоль пути $\Gamma+\Gamma'$:
        \begin{equation*}
            Z(\{\tilde K,\tilde L\})_{\Gamma+\Gamma'}=\sum\limits_{\{\sigma_i=\pm1\}}\exp\left(K\sum\limits_{<ij>\text{hor}}^{\text{out}\;\Gamma+\Gamma'}\sigma_i\sigma_j+L\sum\limits_{<ij>\text{vert}}^{\text{out}\;\Gamma+\Gamma'}\sigma_i\sigma_j-K\sum\limits_{<ij>\text{hor}}^{\text{in}\;\Gamma+\Gamma'}\sigma_i\sigma_j-L\sum\limits_{<ij>\text{vert}}^{\text{in}\;\Gamma+\Gamma'}\sigma_i\sigma_j\right)
        \end{equation*}
        Для подсчёта этой статистической суммы поменяем знаки всех спинов $\vec{\sigma}_\text{in}$, находящихся внутри контура $\Gamma+\Gamma'$. Тогда
        \begin{equation*}
            Z(\{\tilde K,\tilde L\})_{\Gamma+\Gamma'}=\sum\limits_{\{\sigma_i=\pm1\},\vec{\sigma}_\text{in}\rightarrow-\vec{\sigma}_\text{in}}\exp\left(K\sum\limits_{<ij>\text{hor}}\sigma_i\sigma_j+L\sum\limits_{<ij>\text{vert}}\sigma_i\sigma_j\right)
        \end{equation*}
        Но для любого слагаемого с некоторым $\vec{\sigma}_\text{in}$, в сумме существует слагаемое с $-\vec{\sigma}_\text{in}$, поэтому
        \begin{equation}
            Z(\{\tilde K,\tilde L\})_{\Gamma+\Gamma'}=Z(\{K,L\})
        \end{equation}
        Сравним $Z(\{\tilde K,\tilde L\})_{\Gamma}$ и $Z(\{\tilde K,\tilde L\})_{\Gamma'}$.
        \begin{multline}
            Z(\{\tilde K,\tilde L\})_{\Gamma}=\sum\limits_{\{\sigma_i=\pm1\}}\exp(K\sum\limits_{<ij>\text{hor}}^{\text{out}\;\Gamma'}\sigma_i\sigma_j+L\sum\limits_{<ij>\text{vert}}^{\text{out}\;\Gamma'}\sigma_i\sigma_j-K\sum\limits_{<ij>\text{hor}}^{\text{in}\;\Gamma}\sigma_i\sigma_j-L\sum\limits_{<ij>\text{vert}}^{\text{in}\;\Gamma}\sigma_i\sigma_j+\\+\sum\limits_{<ij>\text{vert}}^{\text{out}\;\Gamma+\Gamma'}...)
        \end{multline}
        \begin{multline}
            Z(\{\tilde K,\tilde L\})_{\Gamma'}=\sum\limits_{\{\sigma_i=\pm1\}}\exp(K\sum\limits_{<ij>\text{hor}}^{\text{out}\;\Gamma}\sigma_i\sigma_j+L\sum\limits_{<ij>\text{vert}}^{\text{out}\;\Gamma}\sigma_i\sigma_j-K\sum\limits_{<ij>\text{hor}}^{\text{in}\;\Gamma'}\sigma_i\sigma_j-L\sum\limits_{<ij>\text{vert}}^{\text{in}\;\Gamma'}\sigma_i\sigma_j+\\+\sum\limits_{<ij>\text{vert}}^{\text{out}\;\Gamma+\Gamma'}...)
        \end{multline}
        Поменяем знаки всех спинов $\vec{\sigma}_\text{in}$, находящихся внутри контура $\Gamma+\Gamma'$. Тогда $Z(\{\tilde K,\tilde L\})_{\Gamma}$ перейдёт в $Z(\{\tilde K,\tilde L\})_{\Gamma'}$.
        Таким образом, двухточечная корреляционная функция
        \begin{equation}
            \braket{\mu_{j_1}\mu_{j_2}}_\Gamma=\frac{Z(\{\tilde K,\tilde L\})_\Gamma}{Z(\{K,L\})}
        \end{equation}
        не зависит от контура $\Gamma$.
    \end{itemize}
\end{enumerate}
\section{Дуальность Краммерса-Ванье II}
\textbf{Параметр беспорядка.}
\begin{enumerate}
    \item Оператор беспорядка.
    \begin{itemize}
        \item[i)] Фиксируйте координаты параметров беспорядка $j_1$, $j_2$ на двумерной плоскости. Проведите другой путь $\Gamma'$ между этими двумя точками. Покажите, что различие между двумя прескрипциями на соответствующих неоднородных решётках сводится к вычислению статистической суммы, в которой контур замкнут.
        \item[ii)] Докажите, что двухточечная корреляционная функция не изменяется при деформации контура $\Gamma\rightarrow\Gamma'$.   
        \item[iii)] Рассмотрите коррелятор, включающий операторы беспорядка $\mu$ и операторы порядка $\sigma$. Одно из важных понятий, который появится в скейлинговом пределе -- в соответствующей двумерной конформной теории поля -- взаимная локальность разных квантовых полей типа $\Phi_i(x)$:
        \begin{equation}
            \Phi_1(x)\Phi_2(x)|_\Gamma=e^{2\pi i\gamma_{12}}\Phi_1(x)\Phi_2(x)
        \end{equation}
        Здесь $\gamma_{12}$ -- показатель взаимной локальности, левая часть вычисляется путём аналитического продолжения вдоль контура, обходящего точку $x$ по окружности.\\
        В решёточной теории примером некоммутативности операторов является появление нетривиального фазового сдвига, возникающего при обходе оператором беспорядка вокруг оператора порядка (или наоборот). Вычислите преобразование корреляционной функции $\braket{\mu_{j_1}\sigma_{j_2}...}$ при обходе оператором $\mu$ вокруг спинового оператора (или наоборот). Чему равна экспонента взаимной локальности для $\mu$, $\sigma$?
        \item[iv)] Коррелятор спина и беспорядка обладает не совсем обычным свойством. Продолжение по координате $j_1$ вокруг координаты $j_2$ даёт знак минус.\\
        Рассмотрите аналог такого поведения корреляционных функций операторов порядка и беспорядка. Пусть мы параметризуем точки плоскости комплексными числами $z$, $\bar z$.\\
        Пусть координата $j$ оператора беспорядка $\mu_j$ соответствует комплексной переменной $z$, а координаты $i_1$, $i_2$, ... операторов спина $\sigma_i$ соответствуют переменным $w_i$ (такая ситуация возникнет, например, в критической точке). Рассмотрим аналог функции $\braket{\mu_j\sigma_{i_1}...\sigma_{i_n}}\rightarrow F(z|w_1,...,w_n,...)$. Что можно предположить о виде зависимости функции $F$ от переменной $z$, исходя из взаимной нелокальности спина и беспорядка (вопрос о структуре сингулярности по $z-w_i$).
        \item[v)] Манипуляции с контурами на примере анализа. Рассмотрите представление логарифма в виде
        \begin{equation}
            \log z=\int\limits_1^z\frac{dw}{w}
        \end{equation}
        вдоль контура, соединяющего точки $1, z$. Сделайте в интеграле замену $z\rightarrow ze^{2\pi i}$. Как изменится контур и как изменится значение интеграла?
    \end{itemize}
    \textbf{Решение.}\\
    Пункты i) и ii) выполнены в предыдущей неделе.
    \begin{itemize}
        \item[iv)] Преобразование корреляционной функции при обходе оператором $\mu$ вокруг спинового оператора
        \begin{multline}
            \braket{\mu_{j_1}\sigma_{j_2}...}_\Gamma=\frac{1}{Z}\sum\limits_{\vec{\sigma}}\mu_{j_1}\sigma_{j_2}\exp\left(-K(\sigma_j\sigma_{j+\vec{1}}+\sigma_j\sigma_{j-\vec{1}})-L(\sigma_j\sigma_{j+\vec{2}}+\sigma_j\sigma_{j-\vec{2}})\right)\times\\\times\exp\left(-K\sum\limits_{\text{on}\;\Gamma}\sigma\sigma'-L\sum\limits_{\text{on}\;\Gamma}\sigma\sigma'\right)\exp\left(K\sum\limits_{\text{out}\;\Gamma}\sigma\sigma'+L\sum\limits_{\text{out}\;\Gamma}\sigma\sigma'\right)=\\=\frac{1}{Z}\sum\limits_{\sigma_{j_2}\rightarrow-\sigma_{j_2}}\mu_{j_1}(-\sigma_{j_2})\exp(K(\sigma_j\sigma_{j+\vec{1}}+\sigma_j\sigma_{j-\vec{1}})+L(\sigma_j\sigma_{j+\vec{2}}+\sigma_j\sigma_{j-\vec{2}}))\times\\\times\exp\left(-K\sum\limits_{\text{on}\;\Gamma}\sigma\sigma'-L\sum\limits_{\text{on}\;\Gamma}\sigma\sigma'\right)\exp\left(K\sum\limits_{\text{out}\;\Gamma}\sigma\sigma'+L\sum\limits_{\text{out}\;\Gamma}\sigma\sigma'\right)=-\braket{\mu_{j_1}\sigma_{j_2}...}
        \end{multline}
        Показатель взаимной локальности $\sigma$ и $\mu$:
        \begin{equation}
            \boxed{\gamma_{12}=\frac{1}{2}}
        \end{equation}
        \item[iv)] При обходе оператором $\mu_j$ спина $\sigma_{i_k}$ происходит смена знака. Сравним это со свойством корня $\sqrt{e^{2\pi i}}=e^{\pi i}=-1$. Таким образом, виды зависимостей совпадают
        \begin{equation}
            \boxed{F(z|w_1,...,w_n)=\prod\limits_{i=1}^n\sqrt{z-w_i}}
        \end{equation}
        \item[v)] При замене $z\rightarrow ze^{2\pi i}$ контур сделает полный оборот вокруг особенности $0$, интеграл изменится на $2\pi i\text{res}_{w=0}\frac{1}{w}=2\pi i$.
    \end{itemize}
\textbf{Преобразование KW для операторов}
\item
\begin{itemize}    
    \item[i)] Представление корреляторов спиновых операторов через пути на неоднородной решётке. Рассмотрим путь $\Gamma$, лежащий на прямой решётке и содиняющий точки $j_1$ и $j_2$ на ней же. Пусть на рёбрах вне пути все горизонтальные и вертикальные константы связи равны $K$, $L$ соотвественно. А вдоль пути пусть имеется сдвиг на $\frac{i\pi}{2}$:
    \begin{equation}
        \{K'\}=\begin{cases}
            K \quad\quad\quad\notin\Gamma\\
            K+\frac{i\pi}{2}\quad\in\Gamma
        \end{cases}
    \end{equation}
    Проверьте, что связь между двухточечной корреляционной функцией спиновых операторов и величиной $Z\{K',L'\}/Z\{K,L\}$ задаётся как
    \begin{equation}
        \braket{\sigma_{j_1}\sigma_{j_2}}_\Gamma=\frac{1}{i^n}\frac{Z(\{K',L'\})}{Z(\{K,L\})}
    \end{equation}
    Здесь $n$ -- длина пути.
    \item[ii)] Преобразование KW. Для неоднородной решётки определим преобразование KW общего вида как
    \begin{equation}
        K^*_{j+\frac{1}{2},k}=\frac{1}{2}\text{arcsinh}(\sinh^{-1}2L_{j+1,k+\frac{1}{2}}),\quad L^*_{j+\frac{1}{2},k}=\frac{1}{2}\text{arcsinh}(\sinh^{-1}2K_{j+1,k+\frac{1}{2}})
    \end{equation}
    Образуем функцию, которая совпадает со статистической суммой неоднородной модели, с точностью до простого множителя
    \begin{equation}
        Y(\{K,L\})=Z(\{K,L\})\left(2^N\prod\limits_{j,k}\cosh2K_{j+\frac{1}{2},k}\cosh2L_{j,k+\frac{1}{2}}\right)^{-\frac{1}{2}}
    \end{equation}
    Проверьте в ведущиих порядках (скажем, первые два члена) дуальность Краммерса-Ваннье
    \begin{equation}
        Y(\{K^*,L^*\})=Y(\{K,L\})
    \end{equation}
\end{itemize}
\textbf{Решение.}
\begin{itemize}
    \item[i)] Двухточечная корреляционная функция
    \begin{equation}
        \braket{\sigma_{j_1}\sigma_{j_2}}_\Gamma=\frac{1}{Z}\sum\limits_{\vec{\sigma}}\sigma_{j_1}\sigma_{j_2}\prod\limits_{<i,j>}e^{K\sigma_i\sigma_j+L\sigma_i\sigma_j}
    \end{equation}
    Воспользуемся соотношениями:
    \begin{equation}
        e^{K'\sigma\sigma'}=e^{K\sigma\sigma'}i\sigma\sigma',\quad e^{L'\sigma\sigma'}=e^{L\sigma\sigma'}i\sigma\sigma'
    \end{equation}
    \begin{multline}
        \braket{\sigma_{j_1}\sigma_{j_2}}_\Gamma=\frac{1}{Z}\sum\limits_{\vec{\sigma}}\sigma_{j_1}\sigma_{j_2}\prod\limits_{<i,j>}^{\text{on}\;\Gamma}e^{K\sigma_i\sigma_j+L\sigma_i\sigma_j}\prod\limits_{<i,j>}^{\text{out}\;\Gamma}e^{K\sigma_i\sigma_j+L\sigma_i\sigma_j}=\\=\frac{1}{Zi^n}\sum\limits_{\vec{\sigma}}(i\sigma_{j_1}\sigma_{i_1})(i\sigma_{i_1}\sigma_{i_2})...(i\sigma_{i_{n-2}}\sigma_{j_2})\prod\limits_{<i,j>}^{\text{on}\;\Gamma}e^{K\sigma_i\sigma_j+L\sigma_i\sigma_j}\prod\limits_{<i,j>}^{\text{out}\;\Gamma}e^{K\sigma_i\sigma_j+L\sigma_i\sigma_j}=\\=\frac{1}{Zi^n}\sum\limits_{\vec{\sigma}}\prod\limits_{<i,j>}^{\text{on}\;\Gamma}e^{K'\sigma_i\sigma_j+L'\sigma_i\sigma_j}\prod\limits_{<i,j>}^{\text{out}\;\Gamma}e^{K\sigma_i\sigma_j+L\sigma_i\sigma_j}=\frac{Z(\{K',L'\})}{Z(\{K,L\})}
    \end{multline}
    \begin{equation}
         \boxed{\braket{\sigma_{j_1}\sigma_{j_2}}_\Gamma=\frac{1}{i^n}\frac{Z(\{K',L'\})}{Z(\{K,L\})}}
    \end{equation}
    \item[ii)] 
    \begin{equation}
        Z(\{K^*,L^*\})=\sum\limits_{\{\sigma_i=\pm1\}}\exp\left(\sum\limits_{<jk>\text{hor}}K^*_{j+\frac{1}{2},k}\sigma_i\sigma_j+\sum\limits_{<jk>\text{vert}}L^*_{j+\frac{1}{2},k}\sigma_i\sigma_j\right)
    \end{equation}
    Первый член $Y_1$: все $\sigma_i=1$, второй член $Y_2$: все $\sigma_i=1$ при $i=j$, $\sigma_j=-1$.
    \begin{multline}
        \exp\left(\sum\limits_{<jk>\text{hor}}K^*_{j+\frac{1}{2},k}+\sum\limits_{<jk>\text{vert}}L^*_{j+\frac{1}{2},k}\right)=\exp\left(\sum\limits_{<jk>\text{hor}}\frac{1}{2}\text{arcsinh}(\sinh^{-1}2L_{j+1,k+\frac{1}{2}})\right.+\\+\left.\sum\limits_{<jk>\text{vert}}\frac{1}{2}\text{arcsinh}(\sinh^{-1}2K_{j+1,k+\frac{1}{2}})\right)
    \end{multline}
    \begin{equation}
        j\rightarrow j+\frac{1}{2},\quad k\rightarrow k+\frac{1}{2}
    \end{equation}
    \begin{multline}
        \exp\left(\sum\limits_{<jk>\text{hor}}\frac{1}{2}\ln\left(\frac{\exp2L_{j+1,k+\frac{1}{2}}+1}{\exp2L_{j+1,k+\frac{1}{2}}-1}\right)+\sum\limits_{<jk>\text{vert}}\frac{1}{2}\ln\left(\frac{\exp2K_{j+\frac12,k+1}+1}{\exp2K_{j+\frac12,k+1}-1}\right)\right)=\\=\left(\prod\limits_{<jk>\text{hor}}\frac{\exp2L_{j+1,k+\frac{1}{2}}+1}{\exp2L_{j+1,k+\frac{1}{2}}-1}\right)^\frac{1}{2}\left(\prod\limits_{<jk>\text{vert}}\frac{\exp2K_{j+\frac12,k+1}+1}{\exp2K_{j+\frac12,k+1}-1}\right)^\frac{1}{2}
    \end{multline}
    \begin{multline}
        Y_1(\{K^*,L^*\})=\frac{\left(\prod\limits_{<jk>\text{hor}}\frac{\exp2L_{j+1,k+\frac{1}{2}}+1}{\exp2L_{j+1,k+\frac{1}{2}}-1}\right)^\frac{1}{2}\left(\prod\limits_{<jk>\text{vert}}\frac{\exp2K_{j+\frac12,k+1}+1}{\exp2K_{j+\frac12,k+1}-1}\right)^\frac{1}{2}}{\left(2^N\prod\limits_{j,k}\cosh2K_{j+\frac{1}{2},k}\cosh2L_{j,k+\frac{1}{2}}\right)^{\frac{1}{2}}}=\\=\frac{\prod\limits_{\braket{jk}}(\exp(2L_{j+1,k+\frac{1}{2}}))^\frac{1}{2}(\exp(2K_{j+\frac{1}{2},k+1}))^\frac{1}{2}}{\left(2^N\prod\limits_{j,k}\cosh2K_{j+\frac{1}{2},k}\cosh2L_{j,k+\frac{1}{2}}\right)^{\frac{1}{2}}}
    \end{multline}
    \begin{equation}
        Y_1(\{K,L\})=\frac{\prod\limits_{\braket{jk}}(\exp(2L_{j+1,k+\frac{1}{2}}))^\frac{1}{2}(\exp(2K_{j+\frac{1}{2},k+1}))^\frac{1}{2}}{\left(2^N\prod\limits_{j,k}\cosh2K_{j+\frac{1}{2},k}\cosh2L_{j,k+\frac{1}{2}}\right)^{\frac{1}{2}}}
    \end{equation}
    Как видно, $Y_1(\{K^*,L^*\})=Y_1(\{K,L\})$. Докажем точное соотношение.
    \begin{equation}
        Y(\{K^*,L^*\})=Z(\{K^*,L^*\})\left(2^N\prod\limits_{j,k}\cosh2K^*_{j+\frac{1}{2},k}\cosh2L^*_{j,k+\frac{1}{2}}\right)^{-\frac{1}{2}}
    \end{equation}
    Высокотемпературное разложение:
    \begin{equation}
        Z[K,L]=\prod_{j,k}(2\cosh K_{j+\frac{1}{2},k}\cosh L_{j+\frac{1}{2},k})\sum\limits_{P(s,r)}(\tanh K)^r(\tanh L)^s
    \end{equation}
    Низкотемпературное разложение:
    \begin{equation}
        Z[K^*,L^*]=2\prod_{j,k}\exp(K^*_{j+\frac{1}{2},k}+L^*_{j+\frac{1}{2},k})\sum\limits_{P(s,r)}(e^{-2L^*})^r(e^{-2K^*})^s
    \end{equation}
    Проверим, что выполняется равенство
    \begin{equation}
        Y(\{K,L\})=Y(\{K^*,L^*\})
    \end{equation}
    \begin{multline}
        \left(2^N\prod\limits_{j,k}\cosh2K_{j+\frac{1}{2},k}\cosh2L_{j,k+\frac{1}{2}}\right)^{-\frac{1}{2}}\prod\limits_{j,k}(2\cosh2K_{j+\frac{1}{2},k}\cosh2L_{j,k+\frac{1}{2}})=\\=\left(2^N\prod\limits_{j,k}\cosh2K^*_{j+\frac{1}{2},k}\cosh2L^*_{j,k+\frac{1}{2}}\right)^{-\frac{1}{2}}2\prod\limits_{j,k}\exp(K^*_{j+\frac{1}{2},k}+L^*_{j+\frac{1}{2},k})
    \end{multline}
    \begin{equation}
        \frac{\prod\limits_{j,k}(4\cosh^22K_{j+\frac{1}{2},k}\cosh^22L_{j,k+\frac{1}{2}})}{\prod\limits_{j,k}\cosh2K_{j+\frac{1}{2},k}\cosh2L_{j,k+\frac{1}{2}}}=\frac{4\prod\limits_{j,k}\exp(2K^*_{j+\frac{1}{2},k}+2L^*_{j+\frac{1}{2},k})}{\prod\limits_{j,k}\cosh2K^*_{j+\frac{1}{2},k}\cosh2L^*_{j,k+\frac{1}{2}}}
    \end{equation}
    \begin{equation}
        \frac{\prod\limits_{j,k}(4\cosh^22K_{j+\frac{1}{2},k}\cosh^22L_{j,k+\frac{1}{2}})}{\prod\limits_{j,k}\cosh2K_{j+\frac{1}{2},k}\cosh2L_{j,k+\frac{1}{2}}}=\frac{4\prod\limits_{j,k}4}{\prod\limits_{j,k}(1+e^{-4K^*_{j+\frac{1}{2},k}})(1+e^{-4L^*_{j+\frac{1}{2},k}})}
    \end{equation}
    Соотношения дуальности:
    \begin{equation}
        (e^{2K^*}-e^{-2K^*})(e^{2L}-e^{-2L})=4,\quad (e^{2K}-e^{-2K})(e^{2L^*}-e^{-2L^*})=4
    \end{equation}
    \begin{equation}
        \exp(-4K^*_{j+\frac{1}{2},k})=\tanh^2L_{j+\frac{1}{2},k},\quad\exp(-4L^*_{j+\frac{1}{2},k})=\tanh^2K_{j+\frac{1}{2},k}
    \end{equation}
    \begin{equation}
        \frac{\prod\limits_{j,k}(\cosh^22K_{j+\frac{1}{2},k}\cosh^22L_{j,k+\frac{1}{2}})}{\prod\limits_{j,k}\cosh2K_{j+\frac{1}{2},k}\cosh2L_{j,k+\frac{1}{2}}}=4\frac{\prod\limits_{j,k}(\cosh^22K_{j+\frac{1}{2},k}\cosh^22L_{j,k+\frac{1}{2}})}{\prod\limits_{j,k}\cosh2K_{j+\frac{1}{2},k}\cosh2L_{j,k+\frac{1}{2}}}
    \end{equation}
    С точностью до численного коэффициента 2 выполняется $Y(\{K,L\})=Y(\{K^*,L^*\})$.
\end{itemize}
\textbf{Фермион}\\
Пусть для простоты $K=L$. Рассмотрите фермионы на решётке $\psi^{(a)}_j$:
\begin{equation}
    \psi^{(a)}_j=\sigma_j\mu_{j+\vec{e}_a}
\end{equation}
$j$ -- координата спина, параметр $a$ принимает значения 1,2,3,4. Обозначение $e_a$ стоит для векторов вдоль диагонали длиной $\frac{1}{\sqrt{2}}$ от спина к беспорядку.
\item
\begin{itemize}
    \item[i)] Рассмотрите коррелятор, в котором вставлен оператор беспорядка, характеризуемый контуром $\Gamma$. Предлагаемое ранее определение оператора беспорядка было сформулировано в терминах замены констант связи вдоль пути. Напишите альтернативное определение в терминах оригинальных спиновых операторов $\braket{\mu_j...}_\Gamma=\braket{X(\{\sigma\})...}_\Gamma$. 
    \item[ii)] Нарисуйте оператор $\psi_j^{(a)}$, зафиксировав контур. Совершите вращение этого оператора на 360$^\circ$. Докажите, что после вращения возникает знак
    \begin{equation}
        \psi_j^{(a+4)}=-\psi_j^{(a)}
    \end{equation}
    Какому спину соотвествует такое поведение?
    \item[iii)] Параметр беспорядка $\mu_j$ в комбинации $\psi_{1,j}$ и $\psi_{2,j}$ отличаются контурами, пересекающими одно ребро. Проверьте, что
    \begin{equation}
        \psi_{1,j}=\psi_{2,j}e^{2K\sigma_j\sigma_{j'}}
    \end{equation}
    где $j'$ сдвинут от $j$ на единицу по вертикали. Используя результат выше, попробуйте записать систему уравнений, связывающие $\psi_{a,j}$ с $\psi_{a+1,j}$, $\psi_{a+2,j'}$. Для этого разложите экспоненту по синусам и косинусам и примените определения $\psi_{a,j}$. Вы должны получить систему уравнений
    \begin{equation}
        \psi^{(a)}_j=\cosh2K\psi^{(a+1)}_j-\sinh2K\psi^{(a+2)}_{j+\vec{(a+1)}}
    \end{equation}
    Здесь $\vec{1}$ и $\vec{2}$ -- базисные единичные векторы вдоль осей $x$, $y$, а $\vec{3}=-\vec{1}$, $\vec{4}=-\vec{2}$.
\end{itemize}
\textbf{Решение.}
\begin{itemize}
    \item[i)]
    \item[ii)] При обходе замкнутого контура в оператором беспорядка возникает знак. Это соответствует спину $\frac{1}{2}$.
    \item[iii)] При переходе от $\psi_{1,j}$ к $\psi_{2,j}$ соответствующее ребру, соединяющему узлы $j$ и $j'$, слагаемое в экспоненте меняет знак, поэтому
    \begin{equation}
        \boxed{\psi_{1,j}=\psi_{2,j}e^{2K\sigma_j\sigma_{j'}}}
    \end{equation}
    где $j'$ сдвинут от $j$ на единицу по вертикали.
    \begin{multline}
        \psi_{1,j}=\sigma_j\mu_{j+\vec{e}_1}=\psi_{2,j}e^{-2K\sigma_j\sigma_{j+\vec{2}}}=\sigma_j\mu_{j+\vec{e}_2}(\cosh2K-\sigma_j\sigma_{j+2}\sinh2K)=\\=\cosh2K\psi_{2,j}-\sinh2K\psi_{3,j+\vec{2}}
    \end{multline}
    По аналогии,
    \begin{equation}
        \boxed{\psi^{(a)}_j=\cosh2K\psi^{(a+1)}_j-\sinh2K\psi^{(a+2)}_{j+(a+1)}}
    \end{equation}
\end{itemize}
\textbf{Спектр}
\item
\begin{itemize}
    \item[i)] В уравнении на решёточный фермион перейдите с помощью преобразования Фурье к импульсным переменным
    \begin{equation}
        \psi^{(a)}_j=\sum\limits_p\tilde\psi_p^{(a)}e^{ipj}
    \end{equation}
    Здесь индекс $p=(p_1,p_2)$, как и индекс $j=(j_1,j_2)$, двумерный. Выпишите систему линейных уравнений и найдите условия существования решений. Для компонент импульсов $p_1$, $p_2$ вы должны получить выражение
    \begin{equation}
        \cos\varepsilon p_1+\cos\varepsilon p_2-\frac{\cosh^22K}{\sinh2K}=0,
    \end{equation}
    где $\varepsilon$ -- это длина решётки. Интерпретируя $p=(p_1,p_2)$ как импульс некоторой частицы в соответствующей скейлинговой теории, найдите критическую точку, которая бы соответствовала нулевой массе. Вы должны получить результат, выведенный ранее из KW:
    \begin{equation}
        \cosh2K_c=\sqrt{2},\quad\sinh2K_c=1
    \end{equation}
    Чтобы фиксировать массу, разложитесь вблизи этой критической точки, вводя малый параметр $t=\frac{T-T_c}{T_c}$. Напомним, что $K=\frac{J}{kT}$. Наряду с этим малым параметром имеется малый параметр $\varepsilon$ -- шаг решётки. Выпишите соответствующие уравнения для импульсов (как мы предполагаем свободной частицы) и придумайте, как в этой ситуации можно определить массу частицы, а также как нужно определять непрерывный предел (чтобы частица имела нулевую массу).
    \item[ii)] Альтернативный способ найти критическую температуру в данном вычислении -- просто посмотреть на независящие от $j$ решения системы уравнений. Проверьте, что система допускает решение
    \begin{equation}
        \psi^{(a)}=\text{const}e^{\pm\frac{i\pi a}{4}},
    \end{equation}
    ведущее к критическому значению параметра $K_c$, предложенному выше. А также возможно и решение
    \begin{equation}
        \psi^{(a)}=\text{const}e^{\pm\frac{3i\pi a}{4}},
    \end{equation}
    дающее нефизическое значение для критической температуры. Данное решёточное решение исчезает в пределе к критической точке.
\end{itemize}
\textbf{Решение.}
\begin{itemize}
    \item[i)] После перехода с помощью преобразования Фурье получим систему уравнений
    \begin{equation}\label{eq1}
        \begin{cases}
            \tilde\psi_p^{(1)}-\cosh2K\tilde\psi_p^{(2)}+\sinh2Ke^{ip_2\varepsilon}\tilde\psi_p^{(3)}=0,\\
            \tilde\psi_p^{(2)}-\cosh2K\tilde\psi_p^{(3)}+\sinh2Ke^{-ip_1\varepsilon}\tilde\psi_p^{(4)}=0,\\
            \tilde\psi_p^{(3)}-\cosh2K\tilde\psi_p^{(4)}-\sinh2Ke^{-ip_2\varepsilon}\tilde\psi_p^{(3)}=0,\\
            \tilde\psi_p^{(4)}+\cosh2K\tilde\psi_p^{(1)}-\sinh2Ke^{ip_1\varepsilon}\tilde\psi_p^{(2)}=0,
        \end{cases}
    \end{equation}
    Запишем систему в матричной форме:
    \begin{equation*}
        \begin{pmatrix}
            1 & -\cosh2K & \sinh2Ke^{ip_2\varepsilon} & 0\\
            0 & 1 & -\cosh2K & \sinh2Ke^{-ip_1\varepsilon}\\
            -\sinh2Ke^{-ip_2\varepsilon} & 0 & 1 & -\cosh2K\\
            \cosh2K & -\sinh2Ke^{ip_1\varepsilon} & 0 & 1
        \end{pmatrix}\begin{pmatrix}
            \psi_p^{(1)}\\
            \psi_p^{(2)}\\
            \psi_p^{(3)}\\
            \psi_p^{(4)}
        \end{pmatrix}=\begin{pmatrix}
            0\\
            0\\
            0\\
            0
        \end{pmatrix}
    \end{equation*}
    Условие существования решений:
    \begin{equation}
        \text{det}\begin{pmatrix}
            1 & -\cosh2K & \sinh2Ke^{ip_2\varepsilon} & 0\\
            0 & 1 & -\cosh2K & \sinh2Ke^{-ip_1\varepsilon}\\
            -\sinh2Ke^{-ip_2\varepsilon} & 0 & 1 & -\cosh2K\\
            \cosh2K & -\sinh2Ke^{ip_1\varepsilon} & 0 & 1
        \end{pmatrix}=0
    \end{equation}
    \begin{equation}
        \cos\varepsilon p_1+\cos\varepsilon p_2-\frac{\cosh^22K}{\sinh2K}=0
    \end{equation}
    При малых $\varepsilon$ разложим до 1 порядка:
    \begin{equation}
        1-\frac{\varepsilon^2p_1^2}{2}+1-\frac{\varepsilon^2p_2^2}{2}-\frac{\cosh^22K}{\sinh2K}=0
    \end{equation}
    \begin{equation}
        \frac{p_1^2}{2}+\frac{p_2^2}{2}=\frac{2\sinh2K-\cosh^22K}{\varepsilon^2\sinh2K}
    \end{equation}
    В критической точке $m=0$:
    \begin{equation}
        2\sinh2K_c=\cosh^22K_c\rightarrow1+\sinh^22K_c-2\sinh2K_c=0
    \end{equation}
    \begin{equation}
        \boxed{\sinh2K_c=1,\quad \cosh K_c=\sqrt{2}}
    \end{equation}
    Введём массу, разложив $\sinh2K$ по малому параметру $t=\frac{T-T_c}{T_c}$ вблизи $K=K_c$:
    \begin{equation}
        K=\frac{J}{KT}=\frac{J}{KT_c(1+t)}=K_c(1-t)
    \end{equation}
    \begin{equation}
        \sinh2K=\sinh2K_c-2K_c\cosh2K_ct+O(t^2)=1-2\sqrt{2}K_ct+O(t^2)=1-m\varepsilon
    \end{equation}
    \begin{equation}
        \boxed{m=\frac{2\sqrt{2}K_ct}{\varepsilon}}
    \end{equation}
    Предельный переход $m\rightarrow0$ проходит при устремлении $t\rightarrow0$.
    \item[ii)] Подставим в систему (\ref{eq1}) решение $\psi^{(a)}=\text{const}e^{\pm\frac{i\pi a}{4}}$:
    \begin{equation}
        \begin{cases}
            e^\frac{i\pi}{4}-\cosh2Ke^\frac{i\pi}{2}+\sinh2Ke^{ip_2\varepsilon}e^\frac{i3\pi}{4}=0,\\
            e^\frac{i\pi}{2}-\cosh2Ke^{\frac{i3\pi}{4}}+\sinh2Ke^{-ip_2\varepsilon}e^{i\pi}=0,\\
            e^\frac{i3\pi}{4}-\cosh2Ke^{i\pi}-\sinh2Ke^{-ip_2\varepsilon}e^{\frac{i3\pi}{4}}=0,\\
            e^{i\pi}+\cosh2Ke^{\frac{i\pi}{4}}-\sinh2Ke^{ip_1\varepsilon}e^{\frac{i\pi}{2}}=0
        \end{cases}
    \end{equation}
    Из 4 уравнений только 1 является линейно независимым от остальных:
    \begin{equation}
        1=\cosh2Ke^{\frac{i\pi}{4}}-\sinh2Ke^{ip_1\varepsilon}e^{\frac{i\pi}{2}}
    \end{equation}
    \begin{equation}
        1=i\cosh^22K-\sinh^22Ke^{ip_1\varepsilon}-2\cosh2K\sinh2Ke^{\frac{i3\pi}{4}}
    \end{equation}
    \begin{equation}
        \cosh^22K(1-i)=-2\cosh2K\sinh2Ke^{\frac{i3\pi}{4}}
    \end{equation}
    \begin{equation}
        2\sinh2K_c=\cosh^22K_c
    \end{equation}
    \begin{equation}
        \boxed{\sinh2K_c=1,\quad \cosh K_c=\sqrt{2}}
    \end{equation}
\end{itemize}
\item Скейлинг: уравнение на свободный фермион.
\begin{itemize}
    \item[i)] Рассмотрите уравнение на решёточный фермион в окрестности критической точки. Предполагая, что поля изменяются медленно
    \begin{equation}
        \psi_j^{(a)}\rightarrow\psi^{(a)}(x),
    \end{equation}
    \begin{equation}
        \psi_{j+e_1}^{(a)}\rightarrow\psi^{(a)}(x)+\varepsilon\partial_1\psi^{(a)}(x),
    \end{equation}
    \begin{equation}
        \psi_{j+e_2}^{(a)}\rightarrow\psi^{(a)}(x)+\varepsilon\partial_2\psi^{(a)}(x)
    \end{equation}
    попробуйте получить, что в окрестности критической точки уравнение переходит в уравнение на майорановский фермион, которое выглядит так
    \begin{equation}
        (\partial_1+i\partial_2)u_+=imu_-,
    \end{equation}
    \begin{equation}
        (\partial_1-i\partial_2)u_+=-imu_+
    \end{equation}
\end{itemize}
\textbf{Решение.}
\begin{itemize}
    \item[i)] 
    \begin{equation}
        \begin{cases}
            \psi^{(1)}(x)=\cosh2K\psi^{(2)}(x)-\sinh2K(\psi^{(3)}(x)+\varepsilon\partial_2\psi^{(3)}(x)),\\
            \psi^{(2)}(x)=\cosh2K\psi^{(3)}(x)-\sinh2K(\psi^{(4)}(x)-\varepsilon\partial_1\psi^{(4)}(x)),\\
            \psi^{(3)}(x)=\cosh2K\psi^{(4)}(x)+\sinh2K(\psi^{(1)}(x)-\varepsilon\partial_2\psi^{(1)}(x)),\\
            \psi^{(4)}(x)=-\cosh2K\psi^{(1)}(x)+\sinh2K(\psi^{(2)}(x)+\varepsilon\partial_1\psi^{(2)}(x)).
        \end{cases}
    \end{equation}
    Воспользуемся разложениями:
    \begin{equation}
        \cosh2K=\sqrt{2}-2t,\quad \sinh2K=1-2\sqrt{2}t
    \end{equation}
    \begin{equation}
        \begin{cases}
            \varepsilon\partial_2\psi^{(3)}(x))=(\sqrt{2}+2t)\psi^{(2)}(x)-(1+2\sqrt{2}t)\psi^{(1)}(x)-\psi^{(3)}(x),\\
            \varepsilon\partial_2\psi^{(1)}(x))=(\sqrt{2}+2t)\psi^{(4)}(x)-(1+2\sqrt{2}t)\psi^{(3)}(x)+\psi^{(1)}(x),\\
            \varepsilon\partial_1\psi^{(4)}(x))=-(\sqrt{2}+2t)\psi^{(3)}(x)+(1+2\sqrt{2}t)\psi^{(2)}(x)+\psi^{(4)}(x),\\
            \varepsilon\partial_1\psi^{(2)}(x))=(\sqrt{2}+2t)\psi^{(1)}(x)-(1+2\sqrt{2}t)\psi^{(4)}(x)-\psi^{(2)}(x).
        \end{cases}
    \end{equation}
    Объединим 4 вещественные функции в 2 комплексные:
    \begin{equation}
        \chi_1=\psi^{(1)}+i\psi^{(3)},\quad\chi_2=\psi^{(2)}+i\psi^{(4)}
    \end{equation}
    \begin{equation}
        \begin{cases}
            \varepsilon\partial_2\chi_1(x))=-(\sqrt{2}+2t)\bar\chi_2(x)+(1+2\sqrt{2}t)\bar\chi_1(x)-\bar\chi_1(x),\\
            \varepsilon\partial_1\chi_2(x))=-(\sqrt{2}+2t)\bar\chi_1(x)+(1+2\sqrt{2}t)i\chi_2(x)-\bar\chi_2(x).
        \end{cases}
    \end{equation}
\end{itemize}
\item Аналог алгебры слияний. В теории поля важнуб роль играют алгебры разложений операторных произведений (operator product expansions), введённые Паташинским-Покровским и Вильсоном. Т.е. существует базис локальных полей $A_j(x)$, по которому раскладывается произведения всех полей.
\begin{equation}
    \mathcal{O}_1(x)\mathcal{O}_2(0)=\sum\limits_jC^j_{1,2}(x)A_j(x)
\end{equation}
Одно из требований -- ассоциативность такой бесконечномерной алгебры, что накладывает на структурные функции $C^j_{i,k}$ ограничения.\\
В двумерной конформной теории такие алгебры (базис полей и все структурные функции) были построены точно.\\
В решёточной модели Изинга уже ввели операторы спина, беспорядка, фермион. Введём плотность энергии.
\begin{itemize}
    \item[i)] Проверьте, что оператор энергии задаётся в виде
    \begin{equation}
        \varepsilon_j=\frac{1}{2}\sum\limits_{a=1}^4\sigma_j\sigma_{j+\vec{a}}
    \end{equation}
    Сумма по всей решётке даст энергию состояния (после умножения на $-\frac{J}{kT}$). Рассмотрим часть суммы -- оператор $\sigma_j\sigma_{j+\vec{1}}$. Покажите, что этот оператор представим в виде произведения двух решёточных фермионов.\\
    Перечисленные операторы -- лишь малая часть возможных локальных операторов. Нужно добавить и другие комбинации, в частности, их производные. Например, оператор $\sigma_j-\sigma_{j'}$ в скейлинговом пределе, когда $|j-j'|$ мало по сравнению с корреляционной длиной, будет выражаться через производные от спинового оператора. Кроме того, нужно добавить все другие возможные комбинации вставок разных операторов в разные точки на решётке -- в окрестности, много меньшей, чем корреляционная длина. Таких комбинаций (новых составных операторов) в небольшой окрестности много, а в скейлинговом пределе -- бесконечно много. Однако, в скейлинговом пределе изученные нами простые операторы окажутся наиболее важными (их скейлинговые размерности наименьшие): остальные поля опишутся как их произведения.\\
    Мы ранее определили фермион как
    \begin{equation}
        [\Psi]=[\sigma][\mu]
    \end{equation}
    \item[ii)] Напишите другие возможные соотношения между произведениями двух операторов (спина, беспорядка, фермион и плотность энергии).
    \item[iii)] Предположите, какие операторы можно было бы назвать базисными, в том смысле, что остальные операторы строятся из них в виде произведений.
\end{itemize}
\textbf{Решение.}
\begin{itemize}
    \item[i)] Поскольку при суммировании $\varepsilon_j$ по всем $j$ получается энергия, то оператор $\varepsilon_j=\frac{1}{2}\sum\limits_{a=1}^4\sigma_j\sigma_{j+\vec{a}}$ -- плотность энергии.\\
    Рассмотрим произведение фермионов:
    \begin{equation}
        \psi^{(1)}_j\psi^{(2)}_{j+\vec{1}}=\sigma_j\mu_{j+\vec{e}_1}\sigma_{j+1}\mu_{j+1+\vec{e}_2}
    \end{equation}
    Поскольку $\mu_{j+\vec{e}_1}=\mu_{j+1+\vec{e}_2}$, то $\mu_{j+\vec{e}_1}\mu_{j+1+\vec{e}_2}=1$.
    \begin{equation}
        \boxed{\sigma_j\sigma_{j+\vec{1}}=\psi^{(1)}_j\psi^{(2)}_{j+1}}
    \end{equation}
    \item[ii)] Пользуясь $[\psi][\psi]=1$, $[\sigma][\sigma]=1$, получим
    \begin{equation}
        \boxed{[\sigma][\mu][\psi]=1,\quad[\mu][\psi]=[\sigma],\quad[\psi][\sigma]=[\mu]}
    \end{equation}
    \item[iii)] Можно выбрать любые из 3, например $\sigma$ и $\mu$.
\end{itemize}
\end{enumerate}
\section{Свободные фермионы и преобразование Jordan-Wigner}
Продолжаем рассматривать двумерную модель Изинга в нулевом магнитном поле с параметрами взаимодействия $K$, $L$. Простой метод диагонализации ряд-в-ряд трансфер матрицы -- переход к свободным фермионам с помощью преобразования Йордана-Вигнера.\\
В случае $K\ll1$, $L\gg1$ ряд-в-ряд трансфер матрица определяется простым одномерным квантовым гамильтонианом (модель Изинга в поперечном магнитном поле)
\begin{equation}
    H=K\sum\sigma_j^z\sigma_{j+1}^z+K^*\sum\sigma_j^x
\end{equation}
\begin{equation}
    \sigma^x=\begin{pmatrix}
        0 & 1\\
        1 & 0
    \end{pmatrix},\quad \sigma^y=\begin{pmatrix}
        0 & i\\
        -i & 0
    \end{pmatrix},\quad\sigma^z=\begin{pmatrix}
        1 & 0\\
        0 & -1
    \end{pmatrix}
\end{equation}
Попробуем переписать гамильтониан в терминах алгебры свободных фермионов $\{\psi_1,...,\psi_n\}$, удовлетворяющих каноническим антикоммутационным соотношениям вида
\begin{equation}
    \{\psi^\dagger_j,\psi_k\}=\delta_{j,k},\quad \{\psi_j,\psi_k\}=0,\quad \{\psi^\dagger_j,\psi^\dagger_k\}=0
\end{equation}
Заметим, что
\begin{equation}
    \psi_j^2=0,\quad\psi_j^{\dagger^2}=0
\end{equation}
Определим представление старшего веса, задав вакуумный вектор условием
\begin{equation}
    \psi_j\ket{vac}=0,\quad j\in\{1,...,n\}
\end{equation}
с нормировкой
\begin{equation}
    \braket{vac|vac}=1
\end{equation}
Тогда пространство представлений с данным старшим вектором порождается действием операторов $\psi^\dagger$.
\begin{enumerate}
    \item 
    \item Преобразование Йордана-Вигнера
    \begin{itemize}
        \item[i)] Рассмотрим операторы вида
        \begin{equation}
            \sigma^+=\frac{\sigma^x+i\sigma^y}{2}=\begin{pmatrix}
                0 & 1\\
                0 & 0
            \end{pmatrix},\quad\sigma^-=\frac{\sigma^x-i\sigma^y}{2}=\begin{pmatrix}
                0 & 0\\
                1 & 0
            \end{pmatrix}
        \end{equation}
        Найдите алгебру (коммутационные соотношения) операторов $\sigma^\pm$, $\sigma^z$ и покажите, что данные матрицы являются двумерным представлением алгебры $\text{su}(2)$. Как реализованы вектора такого представления? Вычислите явно соотношение
        \begin{equation}
            \exp(i\pi\sigma^+\sigma^-)=-\sigma^z
        \end{equation}
        Рассмотрите операторы, действующие нетривиально на $i$-ую компоненту в тензорном произведении двумерных представлений
        \begin{equation}
            \sigma^\pm_i=1\otimes1\otimes...\otimes\sigma^\pm\otimes...\otimes1,\quad\sigma^z_i=1\otimes1\otimes...\otimes\sigma^z\otimes...\otimes1
        \end{equation}
        Используя предыдущие вычисления, напишите соотношения в алгебре таких операторов ((анти)коммутационные соотношения, выражения для квадратов операторов и т.д.).
        \item[ii)] Рассмотрите преобразование Йордана-Вигнера
        \begin{equation}
            C_j=\left(e^{i\pi\sum\limits_{i<j}\sigma^+_i\sigma^-_i}\right)\sigma_j^-,\quad C^\dagger_j=\left(e^{-i\pi\sum\limits_{i<j}\sigma^+_i\sigma^-_i}\right)\sigma_j^+
        \end{equation}
        Покажите явным вычислением, что операторы $C_j$, $C_j^\dagger$ удовлетворяют каноническим коммутационным соотношением алгебры свободных фермионов.
    \end{itemize}
    \textbf{Решение.}
    \begin{itemize}
        \item[i)] Коммутационные соотношения:
        \begin{equation}
            [\sigma^+,\sigma^-]=\begin{pmatrix}
                0 & 1\\
                0 & 0
            \end{pmatrix}\begin{pmatrix}
                0 & 0\\
                1 & 0
            \end{pmatrix}-\begin{pmatrix}
                0 & 0\\
                1 & 0
            \end{pmatrix}\begin{pmatrix}
                0 & 1\\
                0 & 0
            \end{pmatrix}=\begin{pmatrix}
                1 & 0\\
                0 & -1
            \end{pmatrix}=\sigma^z
        \end{equation}
        \begin{equation}
            [\sigma^z,\sigma^+]=\begin{pmatrix}
                1 & 0\\
                0 & -1
            \end{pmatrix}\begin{pmatrix}
                0 & 1\\
                0 & 0
            \end{pmatrix}-\begin{pmatrix}
                0 & 1\\
                0 & 0
            \end{pmatrix}\begin{pmatrix}
                1 & 0\\
                0 & -1
            \end{pmatrix}=\begin{pmatrix}
                0 & 2\\
                0 & 0
            \end{pmatrix}=2\sigma^+
        \end{equation}
        \begin{equation}
            [\sigma^z,\sigma^-]=\begin{pmatrix}
                1 & 0\\
                0 & -1
            \end{pmatrix}\begin{pmatrix}
                0 & 0\\
                1 & 0
            \end{pmatrix}-\begin{pmatrix}
                0 & 0\\
                1 & 0
            \end{pmatrix}\begin{pmatrix}
                1 & 0\\
                0 & -1
            \end{pmatrix}=\begin{pmatrix}
                0 & 0\\
                -2 & 0
            \end{pmatrix}=-2\sigma^-
        \end{equation}
        \begin{equation}
            \boxed{[\sigma^+,\sigma^-]=\sigma^z,\quad [\sigma^z,\sigma^\pm]=\pm2\sigma^\pm}
        \end{equation}
        Действия на векторы (двумерного представления $\text{su}(2)$).
        \begin{equation}
            \sigma^z\begin{pmatrix}
                1\\
                0
            \end{pmatrix}=\begin{pmatrix}
                1\\
                0
            \end{pmatrix},\quad\sigma^z\begin{pmatrix}
                0\\
                1
            \end{pmatrix}=-\begin{pmatrix}
                0\\
                1
            \end{pmatrix}
        \end{equation}
        \begin{equation}
            \sigma^+\begin{pmatrix}
                1\\
                0
            \end{pmatrix}=0,\quad\sigma^+\begin{pmatrix}
                0\\
                1
            \end{pmatrix}=\begin{pmatrix}
                1\\
                0
            \end{pmatrix},\quad\sigma^-\begin{pmatrix}
                1\\
                0
            \end{pmatrix}=\begin{pmatrix}
                0\\
                1
            \end{pmatrix},\quad\sigma^-\begin{pmatrix}
                0\\
                1
            \end{pmatrix}=0
        \end{equation}
        \begin{equation}
            \exp(i\pi\sigma^+\sigma^-)=\exp\left(i\pi\begin{pmatrix}
                1 & 0\\
                0 & 0
            \end{pmatrix}\right)=\begin{pmatrix}
                -1 & 0\\
                0 & 1
            \end{pmatrix}=-\sigma^z
        \end{equation}
        Соотношения в алгебре операторов $\sigma_i$:
        \begin{equation}
            \boxed{[\sigma_i^+,\sigma_j^-]=\sigma_i^z\delta_{ij},\quad [\sigma_i^z,\sigma_j^\pm]=\pm2\sigma_i^\pm\delta_{ij}}
        \end{equation}
        \begin{equation}
            \boxed{(\sigma_j^\pm)^2=0,\quad(\sigma^z_j)^2=\mathbb{I}}
        \end{equation}
        \item[ii)] Преобразование Йордана-Вигнера:
        \begin{equation}
            C_j=\left(e^{i\pi\sum\limits_{i<j}\sigma^+_i\sigma^-_i}\right)\sigma_j^-=\prod\limits_{i<j}(-\sigma_i^z)\sigma_j^-,\quad C^\dagger_j=\left(e^{-i\pi\sum\limits_{i<j}\sigma^+_i\sigma^-_i}\right)\sigma_j^+=\prod\limits_{i<j}(-\sigma_i^z)\sigma_j^+
        \end{equation}
        \begin{multline}
            \{C_j,C^\dagger_k\}=\prod\limits_{i<j}(-\sigma_i^z)\sigma_j^-\prod\limits_{i<k}(-\sigma_i^z)\sigma_k^++\prod\limits_{i<k}(-\sigma_i^z)\sigma_k^+\prod\limits_{i<j}(-\sigma_i^z)\sigma_j^-=\\=\sigma^-_j\sigma^+_k\prod\limits_{j\leq i<k}(-\sigma_i^z)+\prod\limits_{j\leq i<k}(-\sigma_i^z)\sigma^+_k\sigma^-_j
        \end{multline}
        Рассмотрим случаи:
        \begin{itemize}
            \item[а.] $j=k$:
            \begin{equation}
                \{C_j,C^\dagger_j\}=(\sigma^-_j\sigma^+_j+\sigma^+_j\sigma^-_j)=1
            \end{equation}
            \item[б.] $j\neq k$:
            \begin{equation}
                \{C_j,C^\dagger_k\}=\left(\sigma^-_j\prod\limits_{j\leq i<k}(-\sigma_i^z)+\prod\limits_{j\leq i<k}(-\sigma_i^z)\sigma^-_j\right)\sigma^+_k=0
            \end{equation}
        \end{itemize}
        \begin{equation}
            \boxed{\{C_j,C^\dagger_k\}=\delta_{j,k}}
        \end{equation}
        \begin{multline}
            \{C_j,C_k\}=\prod\limits_{i<j}(-\sigma_i^z)\sigma_j^-\prod\limits_{i<k}(-\sigma_i^z)\sigma_k^-+\prod\limits_{i<k}(-\sigma_i^z)\sigma_k^-\prod\limits_{i<j}(-\sigma_i^z)\sigma_j^-=\\=\sigma^-_j\sigma^-_k\prod\limits_{j\leq i<k}(-\sigma_i^z)+\prod\limits_{j\leq i<k}(-\sigma_i^z)\sigma^-_k\sigma^-_j=\left(\sigma^-_j\prod\limits_{j\leq i<k}(-\sigma_i^z)+\prod\limits_{j\leq i<k}(-\sigma_i^z)\sigma^-_j\right)\sigma^-_k
        \end{multline}
        \begin{equation}
            \boxed{\{C_j,C_k\}=0}
        \end{equation}
        \begin{multline}
            \{C^\dagger_j,C^\dagger_k\}=\prod\limits_{i<j}(-\sigma_i^z)\sigma_j^+\prod\limits_{i<k}(-\sigma_i^z)\sigma_k^++\prod\limits_{i<k}(-\sigma_i^z)\sigma_k^+\prod\limits_{i<j}(-\sigma_i^z)\sigma_j^+=\\=\sigma^+_j\sigma^+_k\prod\limits_{j\leq i<k}(-\sigma_i^z)+\prod\limits_{j\leq i<k}(-\sigma_i^z)\sigma^+_k\sigma^+_j=\left(\sigma^+_j\prod\limits_{j\leq i<k}(-\sigma_i^z)+\prod\limits_{j\leq i<k}(-\sigma_i^z)\sigma^+_j\right)\sigma^+_k
        \end{multline}
        \begin{equation}
            \boxed{\{C^\dagger_j,C^\dagger_k\}=0}
        \end{equation}
    \end{itemize}
    \item Вычислите обратное преобразование между матрицами Паули и свободными фермионами. Покажите, что операторы в левой и правой частях удовлетворяют одинаковым коммутационным соотношениям и действуют на соответствующие вектора (своих) представлений одинаковым образом.
    \begin{equation}
        \sigma^+_j=\left(e^{i\pi\sum\limits_{i<j}C^+_iC_i}\right)C_j^+,\quad \sigma^-_j=\left(e^{-i\pi\sum\limits_{i<j}C^+_iC_i}\right)C_j,\quad-\sigma_j^z=1-2C^+_jC_j
    \end{equation}
    Из явного вида преобразования Йордана-Вигнера покажите, что
    \begin{equation}
        \sigma^+_j\sigma^-_{j+1}=C^+_jC_{j+1},\quad\sigma^+_j\sigma^+_{j+1}=C^+_jC^+_{j+1},\quad\sigma^-_j\sigma^-_{j+1}=-C_jC_{j+1}
    \end{equation}
    \textbf{Решение.}\\
    Проверим, что $\left\{1-2C^+_jC_j,\left(e^{i\pi\sum\limits_{i<j}C^+_iC_i}\right)C^+_j\right\}=\mathbb{I}$.
    \begin{multline*}
        (1-2C^+_jC_j)\left(e^{i\pi\sum\limits_{i<j}C^+_iC_i}\right)C^+_j+\left(e^{i\pi\sum\limits_{i<j}C^+_iC_i}\right)C^+_j(1-2C^+_jC_j)=2\left(e^{i\pi\sum\limits_{i<j}C^+_iC_i}\right)C^+_j-\\-2\left(e^{i\pi\sum\limits_{i<j}C^+_iC_i}\right)C^+_jC_jC^+_j=2\left(e^{i\pi\sum\limits_{i<j}C^+_iC_i}\right)C^+_j-2\left(e^{i\pi\sum\limits_{i<j}C^+_iC_i}\right)(\mathbb{I}-C_jC^+_j)C^+_j=0
    \end{multline*}
    Проверим, что $\left\{1-2C^+_jC_j,\left(e^{-i\pi\sum\limits_{i<j}C^+_iC_i}\right)C_j\right\}=\mathbb{I}$.
    \begin{multline*}
        (1-2C^+_jC_j)\left(e^{-i\pi\sum\limits_{i<j}C^+_iC_i}\right)C_j+\left(e^{-i\pi\sum\limits_{i<j}C^+_iC_i}\right)C_j(1-2C^+_jC_j)=2\left(e^{-i\pi\sum\limits_{i<j}C^+_iC_i}\right)C_j-\\-2\left(e^{-i\pi\sum\limits_{i<j}C^+_iC_i}\right)C_jC^+_jC_j=2\left(e^{-i\pi\sum\limits_{i<j}C^+_iC_i}\right)C_j-2\left(e^{-i\pi\sum\limits_{i<j}C^+_iC_i}\right)(\mathbb{I}-C^+_jC_j)C_j=0
    \end{multline*}
    Проверим, что $\left\{\left(e^{i\pi\sum\limits_{i<j}C^+_iC_i}\right)C_j^+,\left(e^{-i\pi\sum\limits_{i<j}C^+_iC_i}\right)C_j\right\}=\mathbb{I}$.
    \begin{equation*}
        \left(e^{i\pi\sum\limits_{i<j}C^+_iC_i}\right)C_j^+\left(e^{-i\pi\sum\limits_{i<j}C^+_iC_i}\right)C_j+\left(e^{-i\pi\sum\limits_{i<j}C^+_iC_i}\right)C_j\left(e^{i\pi\sum\limits_{i<j}C^+_iC_i}\right)C_j^+=\{C_j^+,C_j\}=\mathbb{I}
    \end{equation*}
    Проверим действия на векторах.
    \begin{equation}
        \left(e^{i\pi\sum\limits_{i<j}C^+_iC_i}\right)C_j^+\begin{pmatrix}
            1\\0
        \end{pmatrix}_j=\left(e^{i\pi\sum\limits_{i<j}C^+_iC_i}\right)\left(e^{-i\pi\sum\limits_{i<j}\sigma^+_i\sigma^-_i}\right)\sigma^+_j\begin{pmatrix}
            1\\0
        \end{pmatrix}_j=0_j
    \end{equation}
    \begin{equation}
        \left(e^{i\pi\sum\limits_{i<j}C^+_iC_i}\right)C_j^+\begin{pmatrix}
            0\\1
        \end{pmatrix}_j=\left(e^{i\pi\sum\limits_{i<j}C^+_iC_i}\right)\left(e^{-i\pi\sum\limits_{i<j}\sigma^+_i\sigma^-_i}\right)\sigma^+_j\begin{pmatrix}
            0\\1
        \end{pmatrix}_j=\begin{pmatrix}
            1\\0
        \end{pmatrix}_j
    \end{equation}
    \begin{equation}
        \left(e^{-i\pi\sum\limits_{i<j}C^+_iC_i}\right)C_j\begin{pmatrix}
            1\\0
        \end{pmatrix}_j=\left(e^{-i\pi\sum\limits_{i<j}C^+_iC_i}\right)\left(e^{i\pi\sum\limits_{i<j}\sigma^+_i\sigma^-_i}\right)\sigma^-_j\begin{pmatrix}
            1\\0
        \end{pmatrix}_j=\begin{pmatrix}
            0\\1
        \end{pmatrix}_j
    \end{equation}
    \begin{equation}
        \left(e^{-i\pi\sum\limits_{i<j}C^+_iC_i}\right)C_j\begin{pmatrix}
            0\\1
        \end{pmatrix}_j=\left(e^{-i\pi\sum\limits_{i<j}C^+_iC_i}\right)\left(e^{i\pi\sum\limits_{i<j}\sigma^+_i\sigma^-_i}\right)\sigma^-_j\begin{pmatrix}
            0\\1
        \end{pmatrix}_j=0_j
    \end{equation}
    \begin{equation}
        2C^+_jC_j-1=2\left(e^{-i\pi\sum\limits_{i<j}\sigma^+_i\sigma^-_i}\right)\sigma^+_j\left(e^{-i\pi\sum\limits_{i<j}\sigma^+_i\sigma^-_i}\right)\sigma^-_j-1=2\sigma^+_j
        \sigma^-_j-1=\sigma^z_j
    \end{equation}
    \item $^*$
    \begin{itemize}
        \item[i)] Рассмотрите гамильтониан квантовой цепочки Изинга
        \begin{equation}
            -H=K\sum\sigma^x_j\sigma^x_{j+1}+K^*\sum\sigma^z_j
        \end{equation}
        и проверьте, что в терминах свободных фермионов $C_j$ он представляется в виде квадратичных комбинаций (игнорируем граничные условия).
        \item[ii)] Используя предыдущее вычисление, сделаейте преобразование Фурье
        \begin{equation}
            \eta_q=\frac{1}{\sqrt{n}}e^{-\frac{i\pi}{4}}\sum C_je^{iqj},\quad q=\frac{2\pi k}{n}
        \end{equation}
        предполагая, что $n$ чётное. Выпишите гамильтониан через новые переменные $\eta_q$. Для простоты опять игнорируйте граничные условия.
        \item[iii)] Проверьте, что получившийся гамильтониан диагонализуется, используя преобразование Боголюбова
        \begin{equation}
            \psi_q=\cos\phi_q\eta_q-\sin\psi_q\eta^\dagger_{-q},\quad\psi_{-q}=\cos\phi_q\eta_{-q}+\sin\psi_q\eta^\dagger_q
        \end{equation}
        Т.е. приводится к виду
        \begin{equation}
            H=\sum\epsilon_q\left(\psi^\dagger_q\psi_q-\frac{1}{2}\right)
        \end{equation}
        Глядя на это выражение, подумайте, как вычислить статистическую сумму одномерной квантовой цепочки Изинга. В частности, чему равно состояние с наименьшей энергией для различных знаков у констант связи.
    \end{itemize}
    \textbf{Решение.}
    \begin{itemize}
        \item[i)] Гамильтониан квантовой цепочки Изинга:
        \begin{equation}
            -H=K\sum\sigma^x_j\sigma^x_{j+1}+K^*\sum\sigma^z_j
        \end{equation} 
        Обратное преобразование Йордана-Вигнера:
        \begin{equation}
            \sigma^+_j=\left(e^{i\pi\sum\limits_{i<j}C_i^+С_i}\right)C^+_j,\quad\sigma_j^-=\left(e^{-i\pi\sum\limits_{i<j}C_i^+С_i}\right)C_j,\quad-\sigma_j^z=1-2C^\dagger_jC_j
        \end{equation}
        \begin{equation}
            \sigma^x_j=\sigma^+_j+\sigma^-_j
        \end{equation}
        \begin{equation}
            -H=K\sum\limits_j(C_{j+1}C_j+C^\dagger_{j+1}C_j+C^\dagger_jC_{j+1}+C^\dagger_jC^\dagger_{j+1})+K^*\sum\limits_j(2C^\dagger_jC_j-1)
        \end{equation}
        \item[ii)] Сделаем обратное преобразование Фурье:
        \begin{equation}
            C_j=\frac{1}{\sqrt{n}}e^{\frac{i\pi}{4}}\sum\limits_q \eta_qe^{-iqj}
        \end{equation}
        \begin{multline}
            -H=K\sum\limits_j\left(\frac{i}{n}\sum\limits_q \eta_qe^{-iq(j+1)}\sum\limits_{q'} \eta_{q'}e^{-iq'j}+\frac{1}{n}\sum\limits_q \eta^\dagger_qe^{iq(j+1)}\sum\limits_{q'} \eta_{q'}e^{-iq'j}+\right.\\\left.+\frac{1}{n}\sum\limits_q \eta^\dagger_qe^{iqj}\sum\limits_{q'} \eta_{q'}e^{-iq'(j+1)}-\frac{i}{n}\sum\limits_q \eta^\dagger_qe^{iqj}\sum\limits_{q'} \eta^\dagger_{q'}e^{iq'(j+1)}\right)+\\+K^*\sum\limits_j\left(\frac{2}{n}\sum\limits_q \eta^\dagger_qe^{iqj}\sum\limits_{q'} \eta_{q'}e^{-iq'j}-1\right)
        \end{multline}
        \begin{multline}
            -H=K\sum\limits_j\sum\limits_{q,q'}\left(\frac{i}{n}\eta_q\eta_{q'}e^{-iq(j+1)-iq'j}+ \eta^\dagger_q\eta_{q'}e^{iq(j+1)-iq'j}+\eta^\dagger_q\eta_{q'}e^{iqj}e^{-iq'(j+1)}-\right.\\\left.-\frac{i}{n}\eta^\dagger_q\eta^\dagger_{q'}e^{iqj} e^{iq'(j+1)}\right)+K^*\sum\limits_j\left(\frac{2}{n}\sum\limits_{q,q'}\eta^\dagger_q\eta_{q'}e^{iqj-iq'j}-1\right)
        \end{multline}
        \begin{equation}
            \sum\limits_j\eta_q\eta_{q'}e^{-iq(j+1)-iq'j}=e^{-iq}\sum\limits_j\eta_q\eta_{q'}e^{-i(q+q')j}=e^{-iq}n\delta_{q,-q'}
        \end{equation}
        Остальные суммы по $j$ берутся аналогично.
        \begin{multline}
            -H=K\sum\limits_q\left(i\eta_q\eta_{-q}e^{-iq}+\eta^\dagger_q\eta_{q}e^{iq}+\eta^\dagger_q\eta_{q}e^{-iq}-i\eta^\dagger_q\eta^\dagger_{-q}e^{iq}\right)+\\+K^*\left(2\sum\limits_{q}\eta^\dagger_q\eta_q-1\right)=K\sum\limits_q\left(i\eta_q\eta_{-q}e^{-iq}+2\cos q\eta^\dagger_q\eta_{q}-i\eta^\dagger_q\eta^\dagger_{-q}e^{iq}\right)+\\+K^*\left(2\sum\limits_{q}\eta^\dagger_q\eta_q-1\right)=\sum\limits_{q\in\{1,...,n\}}(2K^*\eta^\dagger_q\eta_q-K^*+2K\cos q\eta^\dagger_q\eta_q)+\\+\sum\limits_{q\in\{-\frac{n}{2},...,\frac{n}{2}\}}iK(e^{-iq}\eta_q\eta_{-q}-e^{iq}\eta^\dagger_q\eta^\dagger_{-q})
        \end{multline}
        \begin{equation}
            \boxed{H=\sum\limits_qH_q,\quad H_q=2(K^*-K\cos q)\eta^\dagger_q\eta_q+K\sin q(\eta^\dagger_q\eta^\dagger_{-q}-\eta_q\eta_{-q})-K^*}
        \end{equation}
        \item[iii)] Преобразование Боголюбова:
        \begin{equation}
            \psi_q=\cos\phi_q\eta_q-\sin\psi_q\eta^\dagger_{-q},\quad\psi_{-q}=\cos\phi_q\eta_{-q}+\sin\psi_q\eta^\dagger_q
        \end{equation}
        Обратное преобразование Боголюбова:
        \begin{equation}
            \eta_q=\psi_q\cos\phi_q+\psi^\dagger_{-q}\sin\phi_q,\quad\eta_{-q}=\psi_{-q}\cos\phi_q-\psi^\dagger_q\sin\phi_q
        \end{equation}
        \begin{equation}
            \eta^\dagger_q\eta_q=\psi_q\psi^\dagger_q\cos^2\phi_q+(\psi_q\psi_{-q}+\psi^\dagger_q\psi^\dagger_{-q})\cos\phi_q\sin\phi_q+\psi_{-q}\psi^\dagger_{-q}\sin^2\phi_q
        \end{equation}
        \begin{equation}
            \eta^\dagger_q\eta^\dagger_{-q}=\psi^\dagger_q\psi^\dagger_{-q}\cos^2\phi_q+(\psi_{-q}\psi^\dagger_{-q}-\psi_q\psi^\dagger_q)\cos\phi_q\sin\phi_q-\psi_q\psi_{-q}\sin^2\phi_q
        \end{equation}
        \begin{equation}
            \eta_q\eta_{-q}=\psi_q\psi_{-q}\cos^2\phi_q+(\psi_{-q}\psi^\dagger_{-q}-\psi_q\psi^\dagger_q)\cos\phi_q\sin\phi_q-\psi^\dagger_q\psi^\dagger_{-q}\sin^2\phi_q
        \end{equation}
        \begin{multline}
            H_q=2(K^*-K\cos q)(\psi_q\psi^\dagger_q\cos^2\phi_q+(\psi_q\psi_{-q}+\psi^\dagger_q\psi^\dagger_{-q})\cos\phi_q\sin\phi_q+\psi_{-q}\psi^\dagger_{-q}\sin^2\phi_q)+\\+K\sin q(\psi^\dagger_q\psi^\dagger_{-q}-\psi_q\psi_{-q})-K^*
        \end{multline}
        \begin{equation}
            \epsilon_q=2\sqrt{K^2+K^{*2}-2KK^*\cos ka}
        \end{equation}
        Пусть $\hat{N}_q=\psi^\dagger_q\psi_q$ -- оператор числа частиц. Статистическая сумма одномерной квантовой цепочки Изинга:
        \begin{equation}
            Z=\sum\exp\left(-\frac{\sum\limits_q\epsilon_q(N_q-\frac{1}{2})}{kT}\right)
        \end{equation}
        Состоянию с наименьшей энергией соответсвует $N_q=0$.
    \end{itemize}
    \item Рассмотрите квантовую цепочку более общего вида
    \begin{equation}
        H_{XYZ}=\sum\limits_{j=1}^n(J_x\sigma_j^x\sigma_{j+1}^x+J_y\sigma^y_j\sigma^y_{j+1}+J_z\sigma^z_j\sigma^z_{j+1})+h\sum\sigma_j^z
    \end{equation}
    Пусть, для простоты, наложены периодические граничные условия $\sigma^\alpha_{j+n}=\sigma^\alpha_j$ для $\alpha=\{x,y,z\}$.
    \begin{itemize}
        \item[i)] Подумайте, можно ли с помощью фермионизации точно решить такую цепочку при произвольных параметрах $J_x$, $J_y$, $J_z$, $h$? Как в принципе можно пытаться решать такие модели с помощью фермионов?
        \item[ii)] Для случая $J_z=0$ рассмотрим параметризацию $J_x=J(1+\gamma)/2$, $J_y=J(1-\gamma)/2$. В модели
        \begin{equation}
            H_{XY}=J\sum\limits_{j=1}^n\left(\frac{1+\gamma}{2}\sigma_j^x\sigma^x_{j+1}+\frac{1-\gamma}{2}\sigma_j^y\sigma_{j+1}^y\right)+h\sum\sigma^z_j
        \end{equation}
        есть точки $d=1$ квантовой цепочки Изинга. Какие? Мы получили квантовую цепочку Изинга из двумерной классической модели Изинга. Какая интерпретация была у параметра $h$?
        \item[iii)] Примените преобразование Йордана-Вигнера, игнорируя граничные члены. Расскажите словами, без формул, как можно диагонализовать $XY$ цепочку.
        \item[iv)] Рассмотрите преобразование Йордана-Вигнера для $H_{XY}$ с учётом граничных членов. Вы должны увидеть, что в фермионный гамильтониан входит оператор
        \begin{equation}
            \mu_n=\prod\limits_{j=1}^n\sigma_j^z=\prod(1-2\psi^\dagger_j\psi_j)
        \end{equation}
        Проверьте, что данный оператор коммутирует с гамильтонианом и пространство фермионов может быть расщеплено на два подпространства с чётным и нечётныи числом частиц. В каждом из секторов тогда диагонализация проводится стандартным образом.
    \end{itemize}
    \textbf{Решение.}
    \begin{itemize}
        \item[i)] При помощи фермионизации можно точно решить цепочку при некотором соотношении на $J_x$, $J_y$, $J_z$, чтобы существовал поворот, который уберёт слагаемые $\sigma^z_j\sigma^z_{j+1}$.\\
        После того, как слагаемых $\sigma^z_j\sigma^z_{j+1}$ не будет, можно сделать преобразования Фурье и Боголюбова.
        \item[ii)] $\gamma\pm 1$ -- точки 1D модели Изинга. $h$ интерпретировалось как магнитное поле.
        \item[iii)] Преобразование Йордана-Вигнера:
        \begin{multline}
            -H_{XY}=J\sum\limits_{j=1}^n\left(\frac{1+\gamma}{2}(\sigma^+_j+\sigma^-_j)(\sigma^+_{j+1}+\sigma^-_{j+1})+\frac{1-\gamma}{2i}(\sigma^+_j-\sigma^-_j)(\sigma^+_{j+1}-\sigma^-_{j+1})\right)+\\+h\sum\limits_{j=1}^n(-1+2C^\dagger_jC_j)=J\sum\limits_{j=1}^n\left(\left(\frac{1+\gamma}{2}+\frac{1-\gamma}{2i}\right)\sigma^+_j\sigma^+_{j+1}+\left(\frac{1+\gamma}{2}-\frac{1-\gamma}{2i}\right)\sigma^+_j\sigma^-_{j+1}+\right.\\\left.+\left(\frac{1+\gamma}{2}-\frac{1-\gamma}{2i}\right)\sigma^-_j\sigma^+_{j+1}+\left(\frac{1+\gamma}{2}+\frac{1-\gamma}{2i}\right)\sigma^-_j\sigma^-_{j+1}\right)+h\sum\limits_{j=1}^n(-1+2C^\dagger_jC_j)=\\=-J\sum\limits_{j=1}^n\left(\left(\frac{1+\gamma}{2}+\frac{1-\gamma}{2i}\right)C_j^+C_{j+1}^+e^{iC^+_jC_j}+\left(\frac{1+\gamma}{2}-\frac{1-\gamma}{2i}\right)C_j^+C_{j+1}e^{iC^+_jC_j}\right.+\\+\left.\left(\frac{1+\gamma}{2}-\frac{1-\gamma}{2i}\right)C_jC^+_{j+1}e^{iC^+_jC_j}+\left(\frac{1+\gamma}{2}+\frac{1-\gamma}{2i}\right)C_jC^+_{j+1}e^{iC^+_jC_j}\right)+\\+h\sum\limits_{j=1}^n(-1+2C^\dagger_jC_j)
        \end{multline}
        Пусть $a=\frac{1+\gamma}{2}+\frac{1-\gamma}{2i}$, $b=\frac{1+\gamma}{2}-\frac{1-\gamma}{2i}$. Учитывая, что $e^{iC^+_jC_j}=1-2C^+_jC_j$, получим
        \begin{multline}
            -H_{XY}=J\sum\limits_{j=1}^naC^+_jC^+_{j+1}+bC^+_jC_{j+1}+a(C_jC_{j+1}-2C_jC_{j+1}(1-C_jC^+_j))+\\+b(C_jC^+_{j+1}-2C_jC^+_{j+1}(1-C_jC^+_j))+h\sum\limits_{j=1}^n(-1+2C^\dagger_jC_j)=J\sum\limits_{j=1}^n(a(C^+_jC^+_{j+1}+C_jC_{j+1})+\\+b(C^+_jC_{j+1}+C_jC^+_{j+1}))+h\sum\limits_{j=1}^n(-1+2C^\dagger_jC_j)
        \end{multline}
        Гамильтониан свёлся к предыдущей задаче. Далее нужно сделать преобразование Фурье, Боголюбова и получить сумму по фермионным осцилляторам.
        \item[iv)] Учтём граничные члены.
        \begin{multline}
            -H_{XY}=J\sum\limits_{j=1}^n(a(C^+_jC^+_{j+1}+C_jC_{j+1})+b(C^+_jC_{j+1}+C_jC^+_{j+1}))+h\sum\limits_{j=1}^n(-1+2C^\dagger_jC_j)+\\+(-1)^NJ(aC_n^+C_1^++bC_n^+C_1+bC_1^+C_n+aC_1C_n)
        \end{multline}
        $\hat{N}=\sum\limits_jC_j^+C_j$ -- оператор числа фермионов.
        \begin{equation}
            e^{i\pi\hat{N}}=\prod\limits_je^{i\pi C^+_jC_j}=\prod\limits_j(-\sigma_j^z)=(-1)^{2n}\prod\limits_j\sigma_j^z=\hat{\mu}_n
        \end{equation}
        \begin{multline}
            -H_{XY}=J\sum\limits_{j=1}^n(a(C^+_jC^+_{j+1}+C_jC_{j+1})+b(C^+_jC_{j+1}+C_jC^+_{j+1}))+h\sum\limits_{j=1}^n(-1+2C^\dagger_jC_j)+\\+\hat{\mu}_nJ(aC_n^+C_1^++bC_n^+C_1+bC_1^+C_n+aC_1C_n)
        \end{multline}
        \begin{equation}
            \hat{\mu}_n\sigma_j^{x,y}\hat{\mu}_n=-\sigma_j^{x,y},\quad\hat{\mu}_n\sigma_j^z\hat{\mu}_n=-\sigma_j^z\rightarrow\hat{\mu}_n\hat{H}_{XY}\hat{\mu}_n=\hat{H}_{XY}
        \end{equation}
        \begin{equation}
            \boxed{[\hat{\mu}_n,\hat{H}_{XY}]=0}
        \end{equation}
        \begin{equation}
            \hat{\mu}_n=\frac{1}{2}(1+e^{i\pi\hat{N}})+\frac{1}{2}(1-e^{i\pi\hat{N}})=\hat{\mu}_n^\text{чёт}+\hat{\mu}_n^\text{нечет}
        \end{equation}
        $\hat{H}_{XY}$ расщепляется на 2 подпространства $\hat{H}^\text{чёт}_{XY}$ и $\hat{H}^\text{нечет}_{XY}$:
        \begin{equation}
            \hat{H}^\text{чёт}_{XY}=\hat{\mu}_n^\text{чёт}\hat{H}_{XY}\hat{\mu}_n^\text{чёт},\quad\hat{H}^\text{нечет}_{XY}=\hat{\mu}_n^\text{нечет}\hat{H}_{XY}\hat{\mu}_n^\text{нечет}
        \end{equation}
    \end{itemize}
\end{enumerate}
\section{Свободная энергия и корреляторы модели Изинга}
Мы продолжаем изучать двумерную модель Изинга без магнитного поля в свободно фермионном поле.
\begin{enumerate}
    \item Диагонализация трансфер матрицы и свободная энерния.\\
    На языке фермионов $\eta_k$ в импульсном представлении, удовлетворяющем каноническим антикоммутационным соотношениям, задача диагонализации эффективно факторизуется и сводится к диагонализации на каждом из узлов
    \begin{equation}
        T\sim\prod_{0\leq k\pi}(V_1)_k(V_2)_k
    \end{equation}
    Здесь
    \begin{equation}
        (V_1)_k=\exp(2K^*(1-\eta^\dagger_k\eta_k-\eta^\dagger_{-k}\eta_{-k}))
    \end{equation}
    \begin{equation}
        (V_2)_k=\exp(2K\cos\pi k(\eta^\dagger_k\eta_k+\eta^\dagger_{-k}\eta_{-k}-1)+2K\sin\pi k(\eta_k\eta_{-k}+\eta_k^\dagger\eta_{-k}^\dagger))
    \end{equation}
    Рассмотрим вектора в фермионном представлении с данным фиксированным параметром $k$. Операторы $(V_{1,2})_k$ действуют на вектора с импульсом $p\neq\pm k$ тривиально. Поэтому для решения задачи диагонализации нам необходимо вычислить действие на векторах следующего вида
    \begin{equation}
        \ket{1}_k=\ket{\text{vac}},\quad\ket{2}_k=\eta^\dagger_{-k}\eta^\dagger_k\ket{\text{vac}},\quad\ket{3}_k=\eta_k^\dagger\ket{\text{vac}},\quad\ket{4}_k=\eta^\dagger_{-k}\ket{\text{vac}}
    \end{equation}
    Вычислите матричные элементы трансфер матрицы в базисе таких векторов явно.
    \begin{itemize}
        \item[i)] Пусть
        \begin{equation}
            (V_1)_k=\exp(2K^*v_1)
        \end{equation}
        Проверьте, что
        \begin{equation}
            v_1\ket{1}_k=\ket{1}_k,\quad v_1\ket{2}_k=-\ket{2}_k,\quad v_1\ket{3}_k=0,\quad v_1\ket{4}_k=0
        \end{equation}
        и что матрица $(V_1)_k$ уже диагональна в этом базисе:
        \begin{equation}
            \exp2K^*v_1=\exp(\text{diag}(2K^*,-2K^*,0,0))=\text{diag}(\exp(2K^*),\exp(-2K^*),1,1)
        \end{equation}
        \item[ii)] Пусть
        \begin{equation}
            V_2=\exp(2Kv_2)
        \end{equation}
        где
        \begin{equation}
            v_2=\cos k(\eta^\dagger_k\eta_k+\eta^\dagger_{-k}\eta_{-k}-1)+\sin k(\eta_k\eta_{-k}+\eta^\dagger_k\eta_{-k}^\dagger)
        \end{equation}
        Найдите, что для вектора $\ket{1}$ оператор $v_2$ задаётся нетривиально:
        \begin{equation}
            v_2\ket{1}_k=-\cos k\ket{1}_k+\sin k\ket{2}_k
        \end{equation}
        \item[iii)] Сделайте аналогичное упражнение для всех остальных векторов. В частности, проверьте, что
        \begin{equation}
            v_2\ket{1}_k=-\cos k\ket{1}_k+\sin k\ket{2}_k,\quad v_2\ket{2}_k=\sin k\ket{1}_k+\cos k\ket{2}_k
        \end{equation}
        \begin{equation}
            v_2\ket{3}_k=0,\quad v_2\ket{4}_k=0
        \end{equation}
        \item[iv)] Сконцентрируйте внимание на первых двух векторах $\ket{1}$, $\ket{2}$. Проверьте, что экспонцениирование нетривиальной $2\times2$ матрицы приводит к следующему результату
        \begin{multline}
            \exp2K\begin{pmatrix}
                -\cos k &\sin k\\
                \sin k & \cos k
            \end{pmatrix}=\sum\limits_{j=0}^\infty\frac{(2K)^j}{j!}\begin{pmatrix}
                -\cos k &\sin k\\
                \sin k & \cos k
            \end{pmatrix}^j=\\=\begin{pmatrix}
                \cosh 2K-\sinh2K\cos k & \sinh2K\sin k\\
                \sinh2K\sin k & \cosh 2K+\sinh2K\cos k
            \end{pmatrix}
        \end{multline}
        Принимая во внимание также матрицу $V_1$, мы видим, что необходимо диагонализовать для каждого $k$ матрицу вида
        \begin{equation}
            1\otimes...\otimes1\otimes(V_1V_2)_k\otimes1\otimes...\otimes1
        \end{equation}
        которая сводится к диагонализации матрицы $2\times2$ в $k$-ой компоненте
        \begin{equation}
            \begin{pmatrix}
                e^{2K^*}(\cosh2K-\sinh2K\cos k) & e^{2K^*}\sinh2K\sin k\\
                e^{-2K^*}\sinh2K\sin k & e^{2K^*}(\cosh2K+\sinh2K\cos k)
            \end{pmatrix}
        \end{equation}
        Отсюда находятся собственные значения $e^{\pm\varepsilon_k}$ и собственные векторы (а также матрицу повоторов). Таким образом, собственные значения у произведения $V_1V_2$ имеют вид (их всего $2^n$)
        \begin{equation}
            \prod\limits_ke^{(\pm)\varepsilon_k},
        \end{equation}
        что решает задачу нахождения следа трансфер матрицы.
    \end{itemize}
    \textbf{Решение.}
    \begin{itemize}
        \item[i)] Понижающие операторы:
        \begin{equation}
            \eta_{\pm k}\ket{\text{vac}}=0
        \end{equation}
        \begin{equation}
            v_1=1-\eta^\dagger_k\eta_k-\eta^\dagger_{-k}\eta_{-k}
        \end{equation}
        \begin{equation}
            \boxed{v_1\ket{1}_k=(1-\eta^\dagger_k\eta_k-\eta^\dagger_{-k}\eta_{-k})\ket{1}_k=\ket{1}_k}
        \end{equation}
        \begin{equation}
            v_1\ket{2}_k=(1-\eta^\dagger_k\eta_k-\eta^\dagger_{-k}\eta_{-k})\eta^\dagger_{-k}\eta^\dagger_k\ket{1}_k
        \end{equation}
        \begin{equation}
            \eta^\dagger_k\eta_k\eta^\dagger_{-k}\eta^\dagger_k=\eta^\dagger_k(1-\eta^\dagger_{-k}\eta_k)\eta^\dagger_k=\eta_k^2-\eta^\dagger_k\eta^\dagger_{-k}(1-\eta^\dagger_k\eta_k)=-\eta^\dagger_k\eta^\dagger_{-k}+\eta^\dagger_k\eta^\dagger_{-k}\eta^\dagger_k\eta_k
        \end{equation}
        \begin{equation}
            \eta^\dagger_{-k}\eta_{-k}\eta^\dagger_{-k}\eta^\dagger_k=\eta^\dagger_{-k}(1-\eta^\dagger_{-k}\eta_{-k})\eta^\dagger_k=\eta^\dagger_{-k}\eta^\dagger_k-\eta^{\dagger2}_{-k}\eta_{-k}\eta^\dagger_k=\eta^\dagger_{-k}\eta^\dagger_k
        \end{equation}
        \begin{equation}
            \boxed{v_1\ket{2}_k=\ket{2}_k-2\ket{2}_k=-\ket{2}_k}
        \end{equation}
        \begin{equation}
            v_1\ket{3}_k=(1-\eta^\dagger_k\eta_k-\eta^\dagger_{-k}\eta_{-k})\eta^\dagger_k\ket{1}_k
        \end{equation}
        \begin{equation}
            \eta^\dagger_k\eta_k\eta^\dagger_k=(1-\eta_k\eta^\dagger_k)\eta^\dagger_k,\quad\eta^\dagger_{-k}\eta_{-k}\eta^\dagger_k=-\eta^\dagger_{-k}\eta^\dagger_k\eta_{-k}
        \end{equation}
        \begin{equation}
            \boxed{v_1\ket{3}_k=\eta^\dagger_k\ket{1}_k-\eta^\dagger_k\ket{1}_k=0}
        \end{equation}
        \begin{equation}
            v_1\ket{4}_k=(1-\eta^\dagger_k\eta_k-\eta^\dagger_{-k}\eta_{-k})\eta^\dagger_{-k}\ket{1}_k
        \end{equation}
        \begin{equation}
            \eta^\dagger_k\eta_k\eta^\dagger_{-k}=-\eta^\dagger_k\eta^\dagger_{-k}\eta_k,\quad\eta^\dagger_{-k}\eta_{-k}\eta^\dagger_{-k}=\eta^\dagger_{-k}(1-\eta^\dagger_{-k}\eta_{-k})
        \end{equation}
        \begin{equation}
            \boxed{v_1\ket{4}_k=\eta^\dagger_{-k}\ket{1}_k-\eta^\dagger_{-k}\ket{1}_k=0}
        \end{equation}
        Поэтому матрица $(V_1)_k$ диагональна в этом базисе:
        \begin{equation}
            \exp2K^*v_1=\exp(\text{diag}(2K^*,-2K^*,0,0))=\text{diag}(\exp(2K^*),\exp(-2K^*),1,1)
        \end{equation}
        \item[ii)] 
        \begin{equation}
            v_2\ket{1}_k=(\cos k(\eta^\dagger_k\eta_k+\eta^\dagger_{-k}\eta_{-k}-1)+\sin k(\eta_k\eta_{-k}+\eta^\dagger_k\eta_{-k}^\dagger))\ket{1}_k
        \end{equation}
        \begin{equation}
            \boxed{v_2\ket{1}_k=-\cos k\ket{1}_k+\sin k\ket{2}_k}
        \end{equation}
        \item[iii)]
        \begin{equation}
            v_2\ket{2}_k=(\cos k(\eta^\dagger_k\eta_k+\eta^\dagger_{-k}\eta_{-k}-1)+\sin k(\eta_k\eta_{-k}+\eta^\dagger_k\eta_{-k}^\dagger))\eta^\dagger_{-k}\eta^\dagger_k\ket{1}_k
        \end{equation}
        \begin{equation}
            \eta^\dagger_k\eta_k\eta^\dagger_{-k}\eta^\dagger_k=-\eta^\dagger_k\eta^\dagger_{-k}\eta_k\eta^\dagger_k=-\eta^\dagger_k\eta^\dagger_{-k}(1-\eta^\dagger_k\eta_k)=\eta^\dagger_{-k}\eta^\dagger_k+\eta^\dagger_k\eta^\dagger_{-k}\eta^\dagger_k\eta_k
        \end{equation}
        \begin{equation}
            \eta^\dagger_{-k}\eta_{-k}\eta^\dagger_{-k}\eta^\dagger_k=\eta^\dagger_{-k}(1-\eta^\dagger_{-k}\eta_{-k})\eta^\dagger_k=\eta^\dagger_{-k}\eta^\dagger_k
        \end{equation}
        \begin{equation}
            \eta_k\eta_{-k}\eta^\dagger_{-k}\eta^\dagger_k=\eta_k(1-\eta^\dagger_{-k}\eta_{-k})\eta^\dagger_k=1-\eta^\dagger_k\eta_k+\eta_k\eta^\dagger_{-k}\eta^\dagger_k\eta_{-k}
        \end{equation}
        \begin{equation}
            \boxed{v_2\ket{2}_k=\sin k\ket{1}_k+\cos k\ket{2}_k}
        \end{equation}
        \begin{equation}
            v_2\ket{3}_k=(\cos k(\eta^\dagger_k\eta_k+\eta^\dagger_{-k}\eta_{-k}-1)+\sin k(\eta_k\eta_{-k}+\eta^\dagger_k\eta_{-k}^\dagger))\eta^\dagger_k\ket{1}_k
        \end{equation}
        \begin{equation}
            \eta^\dagger_k\eta_k\eta^\dagger_k=\eta^\dagger_k(1-\eta^\dagger_k\eta_k),\quad\eta^\dagger_{-k}\eta_{-k}\eta^\dagger_k=-\eta^\dagger_{-k}\eta^\dagger_k\eta_{-k} 
        \end{equation}
        \begin{equation}
            \eta_k\eta_{-k}\eta^\dagger_k=-\eta_k\eta^\dagger_k\eta_{-k},\quad\eta^\dagger_k\eta_{-k}^\dagger\eta^\dagger_k=-\eta^\dagger_k\eta^\dagger_k\eta_{-k}^\dagger=0
        \end{equation}
        \begin{equation}
            \boxed{v_2\ket{3}_k=0}
        \end{equation}
        \begin{equation}
            v_2\ket{4}_k=(\cos k(\eta^\dagger_k\eta_k+\eta^\dagger_{-k}\eta_{-k}-1)+\sin k(\eta_k\eta_{-k}+\eta^\dagger_k\eta_{-k}^\dagger))\eta^\dagger_{-k}\ket{1}_k
        \end{equation}
        \begin{equation}
            \eta^\dagger_k\eta_k\eta^\dagger_{-k}=-\eta^\dagger_k\eta^\dagger_{-k}\eta_k,\quad\eta^\dagger_{-k}\eta_{-k}\eta^\dagger_{-k}=\eta^\dagger_{-k}(1-\eta^\dagger_{-k}\eta_{-k})
        \end{equation}
        \begin{equation}
            \eta_k\eta_{-k}\eta^\dagger_{-k}=\eta_k(1-\eta^\dagger_{-k}\eta_{-k}),\quad\eta^\dagger_k\eta_{-k}^\dagger\eta^\dagger_{-k}=0
        \end{equation}
        \begin{equation}
            \boxed{v_2\ket{4}_k=0}
        \end{equation}
        \item[iv)] Приведём матрицу к диагональной форме:
        \begin{equation}
            2K\begin{pmatrix}
                -\cos k &\sin k\\
                \sin k & \cos k
            \end{pmatrix}=\begin{pmatrix}
                \frac{\sin k}{\cos k+1} & \frac{\sin k}{\cos k-1}\\
                1 & 1
            \end{pmatrix}\begin{pmatrix}
                2K & 0\\
                0 & -2K
            \end{pmatrix}\begin{pmatrix}
                \frac{1-\cos2k}{4\sin k} & \frac{1+\cos k}{2}\\
                \frac{\cos2k-1}{4\sin k} & \frac{1-\cos k}{2}
            \end{pmatrix}
        \end{equation}
        \begin{multline}
            \exp\left(2K\begin{pmatrix}
                -\cos k &\sin k\\
                \sin k & \cos k
            \end{pmatrix}\right)=\begin{pmatrix}
                \frac{\sin k}{\cos k+1} & \frac{\sin k}{\cos k-1}\\
                1 & 1
            \end{pmatrix}\begin{pmatrix}
                e^{2K} & 0\\
                0 & e^{-2K}
            \end{pmatrix}\begin{pmatrix}
                \frac{1-\cos2k}{4\sin k} & \frac{1+\cos k}{2}\\
                \frac{\cos2k-1}{4\sin k} & \frac{1-\cos k}{2}
            \end{pmatrix}=\\=\begin{pmatrix}
                \cosh 2K-\sinh2K\cos k & \sinh2K\sin k\\
                \sinh2K\sin k & \cosh 2K+\sinh2K\cos k
            \end{pmatrix}
        \end{multline}
    \end{itemize}
    \item $^*$ Двухочечные спаривания фермионов.\\
    Мы хотим изучить спиновые корреляторы -- вакуумные средние операторов
    \begin{equation}
        \sigma^{(x)}_j\sigma^{(x)}_{j'}=C^{(x)}_j\exp\left(i\pi\sum\limits_{k=j}^{j'-1}C^\dagger_kC_k\right)С^{(x)}_{j'},
    \end{equation}
    где использовано преобразование Йордана-Вигнера и введены обозначения
    \begin{equation}
        C^{(x)}_j=C_j+C^\dagger_j,\quad iC^{(u)}=C^\dagger_j-C_j
    \end{equation}
    для фермионов, удовлетворяющих каноническим антикоммутационным соотношениям. Напомним, что
    \begin{equation}
        (-)^{C^\dagger_kC_k}=(C_k^\dagger+C_k)(C_k^\dagger-C_k)=C^{(x)}_k(iC^{(y)}_k)
    \end{equation}
    Мы имеем следующие операторы
    \begin{equation}
        \sigma^{(x)}_j\sigma^{(x)}_{j'}=C^{(x)}_j\left(\prod\limits_{s=j}^{j'-1}(C^{x}_s(iC^{y}_s))\right)C^{(x)}_{j'}=iC^{(y)}_jC^{(x)}_{j+1}iC^{(y)}_{j+1}...iC^{(y)}_{j'-1}C^{(x)}_{j'},
    \end{equation}
    у которых необходимо найти матричные элементы над вакуумным вектором. Для этого мы совершаем преобразование Фурье, переходя к фермионам в импульсном пространстве $\eta_{\pm k}$ и совершаем поворот, переходя к свободным фермионам $\chi_{\pm k}$ с диагональным гамильтонианом.
    \begin{equation}
        C_j=\frac{1}{\sqrt{n}}e^{-\frac{i\pi}{4}}\sum\limits_ke^{ikj}(\cos\phi_k\chi_k-\sin\phi_k\chi^\dagger_{-k})
    \end{equation}
    По конструкции для операторов $\Psi_k$ вектор $\Omega_0$ -- вакуумный
    \begin{equation}
        \chi_k\ket{\Omega_0}=0
    \end{equation}
    \begin{itemize}
        \item[i)] Используя фермионные формулы, заданные выше, найти, что необходимые нам двухточечные средни задаются
        \begin{equation}
            \braket{\Omega_0|(iC_j^{(y)})(iC_s^{(y)})|\Omega_0}=0
        \end{equation}
        \begin{equation}
            \braket{\Omega_0|C^{(x)}_jC^{(x)}_s|\Omega_0}=0
        \end{equation}
        \item[ii)] Аналогично предыдущему упражнению, найдите, что нетривиальные вклады приходят из спариваний
        \begin{equation}
            \braket{\Omega_0|(iC^{(y)}_j)C_s^{(x)}|\Omega_0}=A_{s-j-1}
        \end{equation}
        где $s>j$ и коэффициенты $A_{s-j}$ определяются через элементы матричного поворота $\phi_k$ как
        \begin{equation}
            A_{s-j}=-\frac{1}{n}\sum\limits_ke^{-ik(s-j-1)}e^{-2i\phi_k}
        \end{equation}
    \end{itemize}
    \textbf{Решение.}
    \begin{itemize}
        \item[i)] 
        \begin{equation}
            C_j^\dagger\ket{\Omega_0}=\frac{1}{\sqrt{n}}e^{\frac{i\pi}{4}}\sum\limits_ke^{-ikj}\cos\varphi_k\chi^\dagger_k
        \end{equation}
        \begin{equation}
            C_s\ket{\Omega_0}=-\frac{1}{\sqrt{n}}e^{-\frac{i\pi}{4}}\sum\limits_ke^{iks}\sin\varphi_k\chi^\dagger_{-k}
        \end{equation}
        \begin{equation}
            \braket{\Omega_0|C^\dagger_jC^\dagger_s|\Omega_0}=\left<\Omega_0\left|-\frac{i}{n}\sum\limits_{k,l}e^{-iks-ilj}\sin\varphi_l\chi_{-l}\chi^\dagger_k\cos\varphi_k\right|\Omega_0\right>
        \end{equation}
        \begin{equation}
            \chi_{-l}\chi^\dagger_k=\delta_{-l,k}\chi_{-l}\chi^\dagger_k,\quad\varphi_{-k}=-\varphi_k
        \end{equation}
        \begin{equation}
            \braket{\Omega_0|C^\dagger_jC^\dagger_s|\Omega_0}=\left<\Omega_0\left|\frac{i}{n}\sum\limits_{k,l}e^{-iks+ikj}\sin\varphi_k\chi_k\chi^\dagger_k\cos\varphi_k\right|\Omega_0\right>
        \end{equation}
        По аналогии,
        \begin{equation}
            \braket{\Omega_0|C_jC_s|\Omega_0}=\left<\Omega_0\left|-\frac{i}{n}\sum\limits_{k,l}e^{-iks+ikj}\sin\varphi_k\chi_k\chi^\dagger_k\cos\varphi_k\right|\Omega_0\right>
        \end{equation}
        \begin{equation}
            \braket{\Omega_0|C_jC_s|\Omega_0}+\braket{\Omega_0|C^\dagger_jC^\dagger_s|\Omega_0}=\frac{1}{n}\braket{\Omega_0|\Omega_0}\sum\limits_k\sin2\varphi_k\frac{e^{-ik(s-j)}-e^{-ik(s-j)}}{2i}=0
        \end{equation}
        \begin{multline}
            \braket{\Omega_0|-C_jC_s|\Omega_0}=-\braket{\Omega_0|\frac{1}{\sqrt{n}}e^\frac{i\pi}{4}\sum\limits_ke^{-ikj}(-\sin\varphi_k)\chi_{-k}\left(-\frac{1}{\sqrt{n}}\right)e^{-\frac{i\pi}{4}}\sum\limits_le^{ils}\sin\varphi_m\chi^\dagger_{-m}|\Omega_0}=\\=-\braket{\Omega_0|\frac{1}{n}\sum\limits_ke^{-ikj+iks}\sin^2\varphi_k\chi_{-k}\chi^\dagger_{-k}|\Omega_0}
        \end{multline}
        \begin{multline}
            \braket{\Omega_0|-C_jC^\dagger_s|\Omega_0}=-\braket{\Omega_0|\frac{1}{\sqrt{n}}e^{-\frac{i\pi}{4}}\sum\limits_ke^{ikj}\cos\varphi_k\chi_k\frac{1}{\sqrt{n}}e^{\frac{i\pi}{4}}\sum\limits_le^{-ils}\cos\varphi_m\chi^\dagger_{-m}|\Omega_0}=\\=-\braket{\Omega_0|\frac{1}{n}\sum\limits_ke^{ikj-iks}\cos^2\varphi_k\chi_k\chi^\dagger_k|\Omega_0}
        \end{multline}
        Поскольку $j<s$, то
        \begin{equation}
            -C^\dagger_jC_s-C_jC^\dagger_s=-\frac{1}{n}\braket{\Omega_0|\Omega_0}\sum\limits_ke^{ik(j-s)}=0
        \end{equation}
        Таким образом,
        \begin{equation}
            \boxed{\braket{\Omega_0|(iC^{(y)}_j)(iC^{(y)}_s)|\Omega_0}=0}
        \end{equation}
        \begin{equation}
            C^{(x)}_jC^{(x)}_s=(C^\dagger_j+C_j)(C^\dagger_s+C_s)=C^\dagger_jC^\dagger_s+C_jC_s+C^\dagger_jC_s+C_jC^\dagger_s
        \end{equation}
        Первое со вторым, третье с четвёртым слагаемые уходят, мы получаем
        \begin{equation}
            \boxed{\braket{\Omega_0|C^{(x)}_jC^{(x)}_s|\Omega_0}=0}
        \end{equation}
        \item[ii)]
        \begin{equation}
            (iC_j^{(y)})C^{(x)}_s=(C^\dagger_j-C_j)(C^\dagger_s+C_s)=C^\dagger_jC^\dagger_s-C_jC_s+C^\dagger_jC_s-C_jC^\dagger_s
        \end{equation}
        Воспользуемся пунктом $i$:
        \begin{equation}
            \braket{\Omega_0|C^\dagger_jC^\dagger_s-C_jC_s|\Omega_0}=\frac{1}{n}\braket{\Omega_0|\Omega_0}i\sum\limits_k e^{ik(s-j)}\sin2\varphi_k
        \end{equation}
        \begin{equation}
            \braket{\Omega_0|C^\dagger_jC_s-C_jC^\dagger_s|\Omega_0}=-\frac{1}{n}\braket{\Omega_0|\Omega_0}i\sum\limits_k e^{-ik(s-j)}\cos2\varphi_k
        \end{equation}
        \begin{multline}
            \braket{\Omega_0|(C^\dagger_j-C_j)(C^\dagger_s+C_s)|\Omega_0}=\braket{\Omega_0|(iC^{(y)}_s)(C^{(x)}_s)|\Omega_0}=\\=\frac{1}{n}\sum\limits_ke^{-ik(s-j)}(-\cos2\varphi_k+i\sin2\varphi_k)=-\frac{1}{n}\sum\limits_ke^{ik(s-j)}e^{-2i\varphi_k}
        \end{multline}
        \begin{equation}
            \boxed{\braket{\Omega_0|(iC^{(y)}_s)(C^{(x)}_s)|\Omega_0}=A_{s-j-1}}
        \end{equation}
        \begin{equation}
            \boxed{A_{s-j-1}=-\frac{1}{n}\sum\limits_ke^{ik(s-j)}e^{-2i\varphi_k}}
        \end{equation}
    \end{itemize}
    \item
    \item Спонтанная намагниченность. Рассмотрим, для простоты случай $K=L$. Здесь и далее мы введём следующие обозначения
    \begin{equation}
        t=\tanh K=e^{-2K^*},\quad t^*=\tanh K^*=e^{-2K}
    \end{equation}
    Наши двухточечные спаривания в терминах матрицы $A_{ij}=A_{j-i}$ задаются в термодинамическом пределе как
    \begin{equation}
        A_j=\int\limits_{-\pi}^\pi\frac{dk}{2\pi}e^{-ikj}f(e^{ik})
    \end{equation}
    где функция $f(z)$ имеет вид
    \begin{equation}
        f(z)=\left(\frac{(t-t^*z)(tt^*-z)}{(tz-t^*)(tt^*z-1)}\right)^\frac{1}{2}
    \end{equation}
    Мы хотим, используя лемму Сзего, вычислить предел двухточечной функции при $T<T_c$
    \begin{equation}
        \mathcal{M}_0^2=\lim\limits_{l\rightarrow\infty}\braket{\sigma_j\sigma_{j+l}}=\lim\limits_{l\rightarrow\infty}\det A
    \end{equation}
    \begin{itemize}
        \item[i)] Положим, что $T<T_c$. Проверьте аналитичность функции $f(z)$ в кольце малой окрестности окружности $|z|=1$.
        \item[ii)] Разлагая в ряд логарифм от $f$
        \begin{equation}
            \log f(z)=h(z)=\sum\limits_{n=-\infty}^\infty h_nz^n
        \end{equation}
        найдите необходимые для нас коэффициенты $h_n$.
        \item[iii)] Из предыдущих результатов и леммы Szego вычислите, что спонтанная намагниченность задаётся выражением
        \begin{equation}
            \mathcal{M}_0=\pm(1-\sinh2K^{-4})^\frac{1}{8}
        \end{equation}
    \end{itemize}
    \textbf{Решение.}
    \begin{itemize}
        \item[i)] Условия Коши-Римана
        \begin{equation}
            \frac{\partial f}{\partial x}+i\frac{\partial f}{\partial y}=0,\quad\frac{\partial f}{\partial y}-i\frac{\partial f}{\partial x}=0
        \end{equation}
        выполняются для $f(z)$. Нули функции $f(z)$
        \begin{equation}
            z=\frac{z^*}{t},\quad z=\frac{1}{t^*t},\quad
        \end{equation}
        Нули $f'(z)$:
        \begin{equation}
            z=\frac{t}{t^*},\quad z=tt^*
        \end{equation}
        Т.к. $T<T_c$, то $\frac{t^*}{t}<1$ и $tt^*<1$. Пусть $\varepsilon_1=\frac{t-t^*}{2t}$, $\varepsilon_2=\frac{t-t^*}{2t^*}$, тогда
        \begin{equation}
            1-\varepsilon_1=\frac{t+t^*}{2t}<1,\quad 1+\varepsilon_2=\frac{t+t^*}{2t^*}>1
        \end{equation}
        В кольце $1-\varepsilon_1<z<1+\varepsilon_2$ нет особых точек.
        \item[ii)] Проведём разрез по отрицательным действительным числам, выберем ветвь логарифма: $\log z=\log x$.
        \begin{multline}
            h(z)=\frac{1}{2}\left(\log\left(1-\frac{t^*z}{t}\right)+\log\left(1+\frac{t^*t}{z}\right)-\log\left(1-\frac{t^*}{zt}\right)-\log(1-ztt^*)\right)=\\=\frac{1}{2}\left(-\sum\limits_{n=1}^\infty\frac{z^n}{n}\left(\frac{t^*}{t}\right)^n+\sum\limits_{n=1}^\infty\frac{z^{*n}}{n}(tt^*)^n+\sum\limits_{n=1}^\infty\frac{z^{-n}}{n}((-tt^*)^n+(tt^*)^n)\right)
        \end{multline}
        \begin{equation}
            \boxed{h_n=-\frac{1}{2n}\begin{cases}
                \left(\frac{t^*}{t}\right)^n(t^{2n}-1),\quad n>0\\
                0,\quad\quad\quad\quad\quad\quad n=0\\
                \frac{1-t^{2n}}{(tt^*)^n},\quad n<0
            \end{cases}}
        \end{equation}
        \item[iii)]
        \begin{equation}
            \mathcal{M}_0^2=\lim\limits_{l\rightarrow\infty}\braket{\sigma_j\sigma_{j+l}}=\lim\limits_{l\rightarrow\infty}\det A
        \end{equation}
        \begin{equation*}
            \mathcal{M}_0^2=e^{\sum\limits_{n=1}^\infty nh_nh_{-n}}=\exp\left(\frac{1}{4}\log(1-(tt^*)^2)-\frac{1}{2}\log(1-(t^*)^2)+\frac{1}{4}\log\left(1-\frac{(t^*)^2}{t^2}\right)\right)
        \end{equation*}
        \begin{equation}
            \mathcal{M}_0^2=\left(\frac{(1-\tanh^2K)(1-\frac{e^{-4K}}{\tanh^2K})}{(1-e^{-4K})^2}\right)^\frac{1}{4}
        \end{equation}
        \begin{equation}
            \boxed{\mathcal{M}_0=\pm(1-\sinh(2K)^{-4})^\frac{1}{8}}
        \end{equation}
    \end{itemize}
    \item Показатели двумерной модели Изинга в нулевом магнитном поле ($K=L$).
    \begin{itemize}
        \item[i)] Показатель $\alpha$. Рассмотрите точное выражение для свободной энергии
        \begin{equation*}
            \frac{f}{kT}=-\log\sqrt{2\sinh2K}-\int\limits_0^{2\pi}\frac{dk}{4\pi}
            \text{arccosh}(\cosh2K^*\cosh2K-\sinh2K^*\sinh2K\cos k)
        \end{equation*}
        Изучите сингулярность (в критической точке) для теплоёмкости, а также проверьте, что критический показатель теплоёмкости
        \begin{equation}
            \alpha=0
        \end{equation}
        \item[ii)] Используя точный ответ, найдите показатель намагниченности $\beta$ в двумерной модели Изинга в нулевом магнитном поле
        \begin{equation}
            \beta=\frac{1}{8}
        \end{equation}
    \end{itemize}
    \textbf{Решение.}
    \begin{itemize}
        \item[i)]
        \begin{equation*}
            \frac{f}{kT}=-\log\sqrt{2\sinh2K}-\int\limits_0^{2\pi}\frac{dk}{4\pi}
            \text{arccosh}(\cosh2K^*\cosh2K-\sinh2K^*\sinh2K\cos k)
        \end{equation*}
        \begin{equation}
            K=\frac{J}{kT}
        \end{equation}
        \begin{equation}
            C=-T^2\frac{\partial^2f}{\partial T^2}=\frac{k^2}{\pi}\left(\frac{2J}{kT_c}\right)^2\left(-\log\left(1-\frac{T}{T_c}\right)+\log\left(\frac{kT_c}{2J}\right)-\left(1+\frac{\pi}{4}\right)\right)
        \end{equation}
        \begin{equation}
            C\sim|t|^{-\alpha},\quad t=1-\frac{T}{T_c}
        \end{equation}
        \begin{equation}
            \boxed{\alpha=0}
        \end{equation}
        \item[ii)]
        \begin{equation}
            \mathcal{M}_0=\pm(1-\sinh(2K)^{-4})^\frac{1}{8},\quad T<T_c
        \end{equation}
        \begin{equation}
            M(0;T)\sim(-t)^\beta
        \end{equation}
        \begin{equation}
            \boxed{\beta=\frac{1}{8}}
        \end{equation}
    \end{itemize}
\end{enumerate}
\section{Уравнение звезда-треугольник и YBE}
\begin{enumerate}
    \item Критическая температура на гексагональной (и треугольной) решётке.
    \begin{itemize}
        \item[i)] Попробуйте вывести уравнение звезда-треугольник: найти константы связи $K_i$ в терминах $L_i$
        \begin{equation}
            w(\sigma_i,\sigma_j,\sigma_k)=\sum\limits_{\sigma_l=\pm}W(\sigma_l|\sigma_i,\sigma_j,\sigma_k)=2\cosh(L_1\sigma_i+L_2\sigma_j+L_3\sigma_k)
        \end{equation}
        При фиксированных внешних спинах $\sigma_i$, $\sigma_j$, $\sigma_k$ зная $\{L_j\}$, найти величины $R$, $\{K_j\}$ такие, что
        \begin{equation}
            w(\sigma_i,\sigma_j,\sigma_k)=R\exp(K_1\sigma_j\sigma_k+K_2\sigma_k\sigma_i+K_3\sigma_i\sigma_j)
        \end{equation}
        \item[ii)] Используя дуальность Краммерса-Ваннье и уравнение звезда-треугольник, вычислите критическую температуру в модели Изинга на треугольной решётке без магнитного поля и найдите $T_c$, предполагая существование единственной критической токи в однородной модели.
    \end{itemize}
    \textbf{Решение.}
    \begin{itemize}
        \item[i)] Решение уравнения звезда-треугольник:
        \begin{equation}
            \sinh2K_j\sinh2L_j=\frac{1}{k}=\frac{\sinh2L_1\sinh2L_2\sinh2L_3}{2(cc_1c_2c_3)^\frac{1}{2}},\quad j=1,2,3
        \end{equation}
        где
        \begin{equation}
            c=\cosh(L_1+L_2+L_3),\quad c_i=\cosh(-L_i+L_j+L_k)
        \end{equation}
        \begin{equation}
            R^2=2k\sinh2L_1\sinh2L_2\sinh2L_3
        \end{equation}
        \item[ii)] В критической точке $K_i=K$, $L_i=L$.
        \begin{equation}
            c=\cosh3L,\quad c_i=\cosh L
        \end{equation}
        \begin{equation}
            \frac{1}{k}=\frac{\sinh^32L}{2(\cosh3L\cosh^3L)^\frac{1}{2}}=1,\quad \sinh2K=\frac{1}{k\sinh2L}=\frac{1}{\sinh2L}
        \end{equation}
        \begin{equation}
            R^2=2k\sinh^32L=\frac{2}{\sinh^32K}
        \end{equation}
        \begin{equation}
            \sinh2L=\sqrt{\cosh^22L-1}
        \end{equation}
        \begin{equation}
            \frac{1}{\sinh2K}=e^{-K}\sqrt{\frac{2}{\sinh^32K}\left(\frac{e^{-2K}}{2\sinh^32K}-1\right)}
        \end{equation}
        \begin{equation}
            \frac{e^{-4K}}{\sinh^42K}-\frac{2e^{-2K}}{\sinh(2K)}=1
        \end{equation}
        Пусть $z=e^{-4K}$, тогда
        \begin{equation}
            z^4-6z^2+8z-3=0\rightarrow z_1=-3,\quad z_{2,3,4}=1
        \end{equation}
        \begin{equation}
            K_c=\frac{\log3}{4}\rightarrow \boxed{T_c=\frac{4J}{k\log3}}
        \end{equation}
    \end{itemize}
    \textbf{Коммутирующие трансфер-матрицы}\\
    Мы хотим обсудить важные следствия уравнения звезда-треугольник (аналог уравнения Янга-Бакстера) для коммутации трансфер матриц в модели Изинга на квадратной решётке. Это позволит нам написать функциональные уравнения для собственных значений трансфер матриц и вычислить их явно.\\
    Рассмотрите диагональ-диагональ трансфер матрицы в двумерной модели Изинга в нулевом магнитном поле, предполагая в них периодические граничные условия.
    \item Запишите диагональ-диагональ матрицы $V(K,L)$ и $W(K,L)$, определив явно их матричные элементы $V_{\sigma'\sigma}(K,L)$ и $W_{\sigma''\sigma'}(K,L)$ между состояниями
    \begin{equation}
        \sigma=\{\sigma_1,...,\sigma_n\},\quad\sigma'=\{\sigma'_1,...,\sigma'_n\},\quad\sigma''=\{\sigma''_1,...,\sigma''_n\}
    \end{equation}
    в терминах $K$, $L$. Выше использованы периодические граничные условия $\sigma_{n+1}=\sigma_1$, $\sigma''_{n+1}=\sigma''_1$. Мы считаем, что матрицы $V$, $W$ действуют в пространстве спинов $\sigma$ размерностью $2^n$ и принадлежат $Mat(2^n\times2^n)$.\\
    По смыслу умножение на эти матрицы <<присоединяет>> к решётке целый ряд спинов. Они могут рассматриваться как некий оператор трансляции по решётке в вертикальном направлении.\\
    Рассмотрите оператор сдвига $P\in Mat(2^n\times2^n)$ в горизонтальном направлении, определив его как
    \begin{equation}
        P_{\sigma',\sigma}=\delta_{\sigma'_2,\sigma_1}\delta_{\sigma'_3,\sigma_2}...\delta_{\sigma'_{n+1},\sigma_n}
    \end{equation}
    Сопряжение оператором $P$ переводит спины с номерами $1,2,...,n$ в $2,3,...,n+1$.
    \begin{itemize}
        \item[i)] Проверьте, что матрица $P$ коммутирует с диагональ-диагональ трансфер матрицами.
        \item[ii)] Проверьте прямым вычислением, что
        \begin{equation}
            W=VP
        \end{equation}
        \item[iii)] Введите оператор $R\in Mat(2^n\times2^n)$, который меняет знак всех спинов $\sigma_j$ в ряду на $-\sigma_j$. Запишите вид этого оператора при действии на $\sigma$. Покажите, что оператор $R$ коммутирует с $V$ и $W$. В частности, покажите, что
        \begin{equation}
            V(K,L)R=V(-K,-L),\quad W(K,L)R=W(-K,-L)
        \end{equation}
    \end{itemize}
    \textbf{Решение.}
    \begin{itemize}
        \item[i)] 
        \begin{equation}
            V_{\sigma',\sigma}=\exp(L\sigma_1\sigma'_1+K\sigma_2\sigma'_1+...K\sigma_1\sigma'_n)
        \end{equation}
        \begin{equation}
            W_{\sigma'',\sigma'}=\exp(K\sigma_1\sigma'_1+L\sigma_2\sigma'_2+...+L\sigma'_n\sigma''_n)
        \end{equation}
        Проверим, что $P^{-1}VP=V$:
        \begin{multline}
            P_{\sigma'',\sigma}^{-1}V_{\sigma',\sigma}P_{\sigma',\sigma'''}=\exp\left(\sum\limits_{j=1}^n(L\sigma''_{j+1}\sigma'''_{j+1}+K\sigma''_{j+2}\sigma'''_{j+1})\right)=\\=\exp\left(\sum\limits_{j=1}^n(L\sigma''_j\sigma'''_j+K\sigma''_{j+1}\sigma'''_j)\right)=V_{\sigma'',\sigma'''}
        \end{multline}
        Следовательно $[V,P]=0$. Аналогично, $[W,P]=0$.
        \item[ii)] Вычислим $VP$:
        \begin{multline}
            V_{\sigma'',\sigma}P_{\sigma',\sigma}=\exp\left(\sum\limits_{j=1}^n(L\sigma'_{j-1}\sigma''_j+K\sigma'_j\sigma''_j)\right)=\\=\exp\left(\sum\limits_{j=1}^n(L\sigma'_j\sigma''_{j+1}+K\sigma'_j\sigma''_j)\right)=W_{\sigma'',\sigma'}
        \end{multline}
        \item[iii)]
        \begin{equation}
            R_{\sigma',\sigma}=\delta_{-\sigma'_1,\sigma_1}\delta_{-\sigma'_2,\sigma_2}...\delta_{-\sigma'_n,\sigma_n}
        \end{equation}
        \begin{equation}
            V_{\sigma',\sigma}(K,L)R_{\sigma',\sigma''}=\exp(-\sum\limits_{j=1}^n(L\sigma_j\sigma''_j+K\sigma_{j+1}\sigma'_j))=V_{\sigma'',\sigma}(-K,-L)
        \end{equation}
        Аналогично, $W(K,L)R=W(-K,-L)$.
        \begin{equation}
            R^{-1}_{\sigma',\sigma}=\delta_{\sigma'_1,-\sigma_1}\delta_{\sigma'_2,-\sigma_2}...\delta_{\sigma'_n,-\sigma_n}
        \end{equation}
        \begin{equation}
            R^{-1}_{\sigma'',\sigma}V_{\sigma,\sigma'}R_{\sigma',\sigma'''}=V_{\sigma'',\sigma'''}
        \end{equation}
        поскольку знаки $K$ и $L$ меняются дважды.
        \begin{equation}
            [R,V]=0,\quad[R,W]=0
        \end{equation}
    \end{itemize}
    \item
    \item
    \item
    \item В случае, когда температура не равна критической, попробуйте ввести параметризацию
    \begin{equation}
        \sinh2K=-i\text{sn}(iu),\quad\sinh2L=\frac{1}{k(-i\text{sn}(iu))}
    \end{equation}
    через эллиптические функции Якоби $\text{sn(u)}$, $\text{cn}(u)$, $\text{dn}(u)$. Используя свойства
    \begin{equation}
        \text{sn}^2u+\text{cn}^2u=1,\quad\text{dn}^2u+k^2\text{sn}^2u=1
    \end{equation}
    проверьте, что
    \begin{equation}
        e^{\pm2K}=\text{cn}(iu)\mp i\text{sn}(iu),\quad e^{\pm2L}=ik^{-1}\frac{\text{dn}(iu)\pm1}{\text{sn}(iu)}
    \end{equation}
    Это удобная параметризация больцмановских весов в модели Изинга для некритических температур. Функции Якоби выражаются через отношения функций $\theta_1,...,\theta_4$. Дальнейший анализ собственных значений в некритическом случае идёт аналогично случаю $k=1$ с той разницей, что вместо тригономестрических функций $\sin u$ используются тета-функции $\theta_1(u)$, зависящие от $k$ и стремящиеся в критическом пределе к $\sin u$. Соответственно, требуется двоякопериодичность, а факторизация производится в произведение тета функций.\\
    \textbf{Решение.}\\
    \begin{equation}
        e^{\pm2K}=\cosh2K\pm\sinh2K=\sqrt{1+\sinh^22K}\pm\sinh2K
    \end{equation}
    \begin{equation}
        \boxed{e^{\pm2K}=\sqrt{1-\text{sn}^2(iu)}\mp i\text{sn}(iu)=\text{cn}(iu)\mp i\text{sn}(iu)}
    \end{equation}
    \begin{equation}
        e^{\pm2L}=\cosh2L\pm\sinh2L=\sqrt{1+\sinh^22L}\pm\sinh2L
    \end{equation}
    \begin{equation}
        \boxed{e^{\pm2L}=\sqrt{1-\frac{1}{k^2\text{sn}^2(iu)}}\pm\frac{i}{k\text{sn}(iu)}=ik^{-1}\frac{\text{dn}(iu)\pm1}{\text{sn}(iu)}}
    \end{equation}
    \item Модель Изинга как частный случай модели IRF
    \begin{itemize}
        \item[i)] Рассмотрите уравнение Янга-Бакстера в IRF форме в данном случае и проверьте, что оно в самом деле сводится к уравнению звезда-треугольник с условием
        \begin{equation}
            \sinh(2K_j)\sinh(2L_j)=\frac{1}{k}
        \end{equation}
        \item[ii)] Рассмотрите ряд-в-ряд трансфер матрицу модели и покажите, что в случае модели Изинга она сводится к диагональ-диагональ трансфер матрице модели Изинга. Покажите, что в этом случае коммутационные соотношения для матриц $T_{ba}$ сводятся к соотношениям для диагональ-диагональ трансфер матриц Изинга.
        \item[iii)] Рассмотрите удвоенную модель Изинга. Используя предыдущие результаты, докажите соответствующее уравнение Янга-Бакстера в IRF-форме.
    \end{itemize}
    \textbf{Решение.}
    \begin{itemize}
        \item[i)] Уравнение Янга-Бакстера в IRF форме:
        \begin{equation*}
            \sum\limits_{s_7}W''\begin{pmatrix}
                s_1 & s_6\\
                s_2 & s_7
            \end{pmatrix}W'\begin{pmatrix}
                s_6 & s_5\\
                s_7 & s_4
            \end{pmatrix}W\begin{pmatrix}
                s_7 & s_4\\
                s_2 & s_3
            \end{pmatrix}=\sum\limits_{s_7}W\begin{pmatrix}
                s_6 & s_5\\
                s_1 & s_7
            \end{pmatrix}W'\begin{pmatrix}
                s_1 & s_7\\
                s_2 & s_3
            \end{pmatrix}W''\begin{pmatrix}
                s_7 & s_5\\
                s_3 & s_4
            \end{pmatrix}
        \end{equation*}
        Больцмановские веса IRF-модели:
        \begin{equation}
            W\begin{pmatrix}
                * & s_1\\
                s_2 & *
            \end{pmatrix}=e^{Ls_1s_2},\quad W\begin{pmatrix}
                s_1 & *\\
                * & s_2
            \end{pmatrix}=e^{Ks_1s_2}
        \end{equation}
        Зафиксировав $s_2$, $s_4$, $s_6$, получим уравнение:
        \begin{equation}
            \boxed{\sum\limits_{s_7}e^{K''s_1s_7}e^{L's_5s_7}e^{Ks_3s_7}=e^{Ls_1s_5}e^{K's_1s_3}e^{L''s_3s_5}}
        \end{equation}
        \item[ii)] Ряд-в-ряд трансфер матрица:
        \begin{equation}
            T_{\vec{b}\vec{a}}=W\begin{pmatrix}
                b_1 & b_2\\
                a_1 & a_2
            \end{pmatrix}W\begin{pmatrix}
                b_2 & b_3\\
                a_2 & a_3
            \end{pmatrix}...W\begin{pmatrix}
                b_n & b_1\\
                a_n & a_1
            \end{pmatrix}
        \end{equation}
        Фиксируем половину узлов ($b_1$, $a_2$, $b_3$, $a_4$,...) и получим диагональ-диагональ трансфер матрицу (переобозачим $a_{2k+1}=\sigma_k$, $b_{2k}=\sigma'_k$):
        \begin{equation}
            \boxed{T_{\vec{b}\vec{a}}=e^{L\sigma_1\sigma'_1}e^{K\sigma'_1\sigma_2}...e^{K\sigma'_{n/2}\sigma_1}=V_{\sigma',\sigma}}
        \end{equation}
        \item[iii)] В удвоенной модели Изинга больцмановские веса IRF-модели:
        \begin{equation}
            W\begin{pmatrix}
                s_4 & s_3\\
                s_1 & s_2
            \end{pmatrix}=e^{Ls_1s_3+Ks_2s_4}
        \end{equation}
        Уравнение Янга-Бакстера:
        \begin{equation}
            \sum\limits_{s_7}e^{K''s_1s_7+L''s_2s_6}e^{L's_5s_7+K's_4s_6}e^{Ks_3s_7+Ls_2s_4}=\sum\limits_{s_7}e^{Ls_1s_5+Ks_6s_7}e^{K's_1s_3+L's_2s_7}e^{L''s_3s_5+K''s_4s_7}
        \end{equation}
        Зафиксируем $s_2$, $s_4$, $s_6$:
        \begin{equation}
            \boxed{\sum\limits_{s_7}e^{K''s_1s_7}e^{L's_5s_7}e^{Ks_3s_7}=e^{Ls_1s_5}e^{K's_1s_3}e^{L''s_3s_5}}
        \end{equation}
        Зафиксируем $s_1$, $s_3$, $s_5$:
        \begin{equation}
            \boxed{e^{L''s_2s_6}e^{K's_4s_6}e^{Ls_2s_4}=\sum\limits_{s_7}e^{Ks_6s_7}e^{L's_2s_7}e^{K''s_4s_7}}
        \end{equation}
    \end{itemize}
    \item Двумерная модель Изинга связано со свободно фермионной точкой восьми вершинной модели. В вершинных моделях спиновые переменные $\sigma=\pm$ ассоциированы с рёбрами решётки, а не с узлами. В моделях на квадратной решётке будем предполагать, что локальное взаимодействие описывается больцмановскими весами, ассоциированными с узлами или вершинами решётки. А сами больцмановские веса зависят от 4 спинов на рёбрах решётки вокруг вершины.
    \begin{equation}
        R^{\sigma_3\sigma_4}_{\sigma_1\sigma_2}=e^{-\varepsilon(\sigma_1,\sigma_2,\sigma_3,\sigma_4)/kT}
    \end{equation}
    В общем случае в ситуации, когда спины принимают 2 значения ($\sigma=\pm1$), существует 16 возможных значений больцмановских весов $R_{\sigma_1\sigma_2}^{\sigma_3\sigma_4}$. Из-за симметрий или каких-либо других условий число нетривиальных значений может стать меньше. Например, в модели без внешнего поля часто налагается условие симметрии при обращении всех знаков
    \begin{equation}
        R^{-\sigma_3-\sigma_4}_{-\sigma_1-\sigma_2}=R^{\sigma_3\sigma_4}_{\sigma_1\sigma_2}
    \end{equation}
    Для заданной конфигурации спинов на решётке весом конфигурации будет произведение по всем вершинам соответствующих локальных больцмановских весов. Статистическая сумма тогда -- сумма по всем спинам в таких конфигурациях
    \begin{equation}
        Z=\sum\limits_{spins}e^{-\sum\limits_{sites}\varepsilon(\sigma_1,\sigma_2,\sigma_3,\sigma_4)/kT}=\sum\limits_{spins}\prod_{sites}e^{-\varepsilon(\sigma_1,\sigma_2,\sigma_3,\sigma_4)/kT}
    \end{equation}
    В терминах больцмановских весов $R$ статистическая сумма имеет вид
    \begin{equation}
        Z=\sum\limits_{spins}\prod_{sites}R^{\sigma_3\sigma_4}_{\sigma_1\sigma_2}
    \end{equation}
    Соответствующим образом определяются корреляционные функции и другие объекты статистической механики.\\
    Рассмотрим удвоенную модель Изинга: 2 невзаимодействующие между собой модели Изинга, живущие на разных подрешётках, предполагая, что больцмановский вес имеет вид
    \begin{equation}
        W\begin{pmatrix}
            s_4 & s_3\\
            s_1 & s_2
        \end{pmatrix}=e^{Ls_1s_3+Ks_2s_4},\quad s_j\in\{\pm1\}
    \end{equation}
    Пусть 2 невзаимодействующие друг с другом модели Изинга определяются, как и ранее, константами взаимодействия $K$ и $L$. Рассмотрим конфигурацию 4 спинов. Вес такой конфигурации
    \begin{equation}
        e^{Ls_1s_3+Ks_2s_4}
    \end{equation}
    Запишем этот вес как $R^{\sigma_3\sigma_4}_{\sigma_1\sigma_2}$, сделав переход от спинов $s_1,...,s_4$ к спинам на рёбрах $\sigma_1,...,\sigma_4$ по следующему правилу. Если 2 изинговских спина $s_i$, $s_j$ разделены ребром $\sigma$, то значение спина на ребре равно произведению $\sigma=s_is_j$:
    \begin{equation}
        \sigma_1=s_1s_2,\quad\sigma_2=s_1s_4,\quad\sigma_3=s_2s_3,\quad\sigma_4=s_3s_4
    \end{equation}
    Обратное преобразование от переменных $\{\sigma_j\}$ к переменным $\{s_j\}$ требует, чтобы ограничение
    \begin{equation}
        \sigma_1\sigma_2\sigma_3\sigma_4=1
    \end{equation}
    \begin{itemize}
        \item[i)] Проверьте, что в силу симметрий исходной удвоенной модели Изинга есть 8 ненулевых спиновых конфигураций вокруг вершины $a$, $b$, $c$, $d$:
        \begin{equation}
            a=R^{++}_{++}=R^{--}_{--}=e^{K+L},\quad b=R^{+-}_{+-}=R^{-+}_{-+}=e^{-K-L}
        \end{equation}
        \begin{equation}
            c=R^{-+}_{+-}=R^{+-}_{-+}=e^{-K+L},\quad d=R^{--}_{++}=R^{++}_{--}=e^{K-L}
        \end{equation}
        Такие ненулевые конфигурации удобно представить в виде матрицы $4\times4$. Именно, запишем оператор $R$ в базисе векторов $(++,+-,-+,--)$ нумеруемых двойным индексом. В таком базисе матрица больцмановских весов имеет вид
        \begin{equation}
            R=\begin{pmatrix}
                a & 0 & 0 & d\\
                0 & b & c & 0\\
                0 & c & b & 0\\
                d & 0 & 0 & a\\
            \end{pmatrix}
        \end{equation}
        \item[ii)] В свободно-фермионной точке, которая эквивалентна 2 невзаимодействующим моделям Изинга, рассмотрите уравнение Янга-Бакстера и покажите, что оно следует из уравнения звезда-треугольник для модели Изинга.
    \end{itemize}
    \textbf{Решение.}
    \begin{itemize}
        \item[i)] Поскольку $\sigma_1\sigma_2\sigma_3\sigma_4=1$, то среди $\sigma_i$ чётное число имеют 1 знак. Поэтому из $2^4$ вариантов остаётся $2^3=8$. Из них неодинаковых остаётся $2^2=4$, поскольку смена всех знаков не меняет $R^{\sigma_3\sigma_4}_{\sigma_1\sigma_2}$.
        \item[ii)] Переход от уравнении звезда-треугольник в модели Изинга к уравнению Янга-Бакстера происходит при сравнении соответствующих картинок.
    \end{itemize}
    \item Матричные обозначения.\\
    В случае шестивершинной модели есть сильное упрощение: $d=0$. Рассмотрим матрицу $R\in Mat(4\times4)$. Для шестивершинной модели попробуйте ввести матричные обозначения.
    \begin{itemize}
        \item[i)] Вообще говоря, мы изучаем матрицы (операторы), которые действуют на пространствах высокой размерности -- собрании спинов вдоль прямой, например. Такие пространства имеют естественную структруру тензорного произведения
        \begin{equation}
            \mathcal{F}=F_1\otimes F_2\otimes ...\otimes F_n
        \end{equation}
        Пусть $v_\pm$ -- 2 базисных вектора в $F_j\sim C^2$ (это представления $sl_2$ со спином вверх или вниз):
        \begin{equation}
            v_+=\begin{pmatrix}
                1\\
                0
            \end{pmatrix},\quad v_-=\begin{pmatrix}
                0\\
                1
            \end{pmatrix}
        \end{equation}
        Наша $R$-матрица $R:C^2\rightarrow C^2$ действует как оператор в тензорном произведении двух пространств по правилу
        \begin{equation}
            R(v_{\sigma_1}\otimes v_{\sigma_2})=R^{\sigma'_1\sigma'_2}_{\sigma_1\sigma_2}v_{\sigma'_1\sigma'_2},\quad R:F\otimes F\rightarrow F\otimes F
        \end{equation}
        В базисе векторов тензорного квадрата $(++,+-,-+,--)$ или $v_+\otimes v_+$, $v_+\otimes v_-$, $v_-\otimes v_+$, $v_-\otimes v_-$ матрица имеет следующий вид $4\times4$
        \begin{equation}
            R=\begin{pmatrix}
                a & 0 & 0 & 0\\
                0 & b & c & 0\\
                0 & c & b & 0\\
                0 & 0 & 0 & a\\
            \end{pmatrix},\quad a=R^{++}_{++}=R^{--}_{--}, b=R^{+-}_{+-}=R^{-+}_{-+}, c=R^{-+}_{+-}=R^{+-}_{-+}
        \end{equation}
        Альтернативно, мы можем представить матрицу $4\times4$ в виде суммы матричных произведений
        \begin{equation}
            R=\sum\limits_js_j\otimes t_j,\quad s_j,t_j\in Mat(2\times2)
        \end{equation}
        Проверьте, что $R$-матрицу шестивершинной модели можно представить в виде
        \begin{equation}
            R=\frac{a+b}{2}I\otimes I+\frac{a-b}{2}\sigma^z\otimes\sigma^z+c(\sigma^+\otimes\sigma^-+\sigma^-\otimes\sigma^+)
        \end{equation}
        \begin{equation}
            R=\begin{pmatrix}
                \frac{a+b}{2}I+\frac{a-b}{2}\sigma^z & c\sigma^-\\
                c\sigma^+ & \frac{a+b}{2}I-\frac{a-b}{2}\sigma^z
            \end{pmatrix}
        \end{equation}
        \item[ii)] Матричная форма уравнений Янга-Бакстера.
    \end{itemize}
    \textbf{Решение.}
    \begin{itemize}
        \item[i)]
        \begin{equation}
            R=\begin{pmatrix}
                a & 0 & 0 & 0\\
                0 & b & c & 0\\
                0 & c & b & 0\\
                0 & 0 & 0 & a\\
            \end{pmatrix}=\frac{a+b}{2}I\otimes I+\frac{a-b}{2}\sigma^z\otimes\sigma^z+c(\sigma^+\otimes\sigma^-+\sigma^-\otimes\sigma^+)
        \end{equation}
        \begin{equation}
             R=\begin{pmatrix}
                \frac{a+b}{2}I+\frac{a-b}{2}\sigma^z & c\sigma^-\\
                c\sigma^+ & \frac{a+b}{2}I-\frac{a-b}{2}\sigma^z
            \end{pmatrix}
        \end{equation}
        \item[ii)]
        \begin{multline}
            R''_{23}R_{13}R_{12}(v_\sigma\otimes v_\mu\otimes v_\nu)=R''_{23}R_{13}s'\otimes t'\otimes1(v_\sigma\otimes v_\mu\otimes v_\nu)=\\=R''_{23}s\otimes1\otimes t(v_{s'(\sigma)}\otimes v_{t'(\mu)}\otimes v_\nu)=v_{ss'(\sigma)}\otimes v_{s''t'(\mu)}\otimes v_{t''t(\nu)}
        \end{multline}
        Аналогично,
        \begin{equation}
            R'_{12}R_{13}R''_{23}(v_\sigma\otimes v_\mu\otimes v_\nu)=v_{s's(\sigma)}\otimes v_{t's''(\mu)}\otimes v_{tt''(\nu)}
        \end{equation}
        Таким образом, матричное уравнения Янга Бакстера:
        \begin{equation}
            \boxed{R''_{23}R_{13}R_{12}=R'_{12}R_{13}R''_{23}}
        \end{equation}
    \end{itemize}
\end{enumerate}
\section{Шестивершинная модель. YBE}
Мы рассматриваем шестивершинную модель с параметризацией
\begin{equation}
    R=\begin{pmatrix}
        a & 0 & 0 & 0\\
        0 & b & c & 0\\
        0 & c & b & 0\\
        0 & 0 & 0 & a
    \end{pmatrix}
\end{equation}
\begin{equation}
    R_{\sigma_1\sigma_2}^{\sigma_3\sigma_4}=e^{-\varepsilon(\sigma_1,\sigma_2,\sigma_3,\sigma_4)/kT}
\end{equation}
\begin{enumerate}
    \item
    \begin{itemize}
        \item[i)] Уравнение Янга-Бакстера имеет вид, представленный на рис. 2. Запишите это уравнение в компонентах, глядя на графическое уравнение.\\
        Заметим, что уравнение должно выполняться при выборе всех возможных комбинаций для внешних спинов $\sigma_1,...,\sigma_6$. Всего есть $2^6$ нелинейных уравнений.\\
        Для шестивершинной модели число уравнений будет меньшим, как и их структура будет проще. Мы рассматриваем шестивершинную модель с параметризацией
        \begin{equation}
            R=\begin{pmatrix}
                a & 0 & 0 & 0\\
                0 & b & c & 0\\
                0 & c & b & 0\\
                0 & 0 & 0 & a\\
            \end{pmatrix}
        \end{equation}
        и схожей параметризацией для $R$, $R'$: матрица $R'$ зависит от параметров $a'$, $b'$, $c'$, а матрица $R''$ зависит от параметров $a''$, $b''$, $c''$.
        \item[ii)] Рассмотрите в шестивершинной модели случай, когда все внешние спины равны $+1$. Смотрите рис. 3. Проверьте справедливость YBE в этом случае.
        \item[iii)] Рассмотрите уравнение Янга-Бакстера для случая, когда внешние спины выбраны как на верхней диаграмме рисунка 4. Выпишите соответствующее уравнение для весов $(a,b,c,a',b',...)$.
        \item[iv)] Повторите упражнение выше для случаев расстановки внешних спинов из рис. 5. С учётом предыдущего задания вы должны получить систему
        \begin{equation}
            -ca'a''+cb'b''+ac'c''=0
        \end{equation}
        \begin{equation}
            cc'b''+(ab'-a'b)c''=0
        \end{equation}
        \begin{equation}
            -c'ba''+ac'b''+cb'c''=0
        \end{equation}
        YBE в шестивершинной модели сводится к этим 3 уравнениям. Попробуйте проверить это утверждение для произвольной комбинации внешних спинов.
        \item[v)] Проверьте, что условие существования решения $a''$, $b''$, $c''$ можно записать как
        \begin{equation}
            \frac{a^2+b^2-c^2}{2ab}=\frac{a'^2+b'^2+c'^2}{2a'b'}=\Delta
        \end{equation}
    \end{itemize}
    \textbf{Решение.}
    \begin{itemize}
        \item[i)] Уравнение Янга-Бакстера в компонентах:
        \begin{equation}
            \boxed{\sum\limits_{\mu',\nu'\sigma'}(R'')^{\mu'\nu'}_{\mu\nu}(R)^{\sigma'\nu''}_{\sigma\nu'}(R')^{\sigma''\mu''}_{\sigma'\mu'}=\sum\limits_{\mu',\nu'\sigma'}(R')^{\sigma'\mu'}_{\sigma\mu}(R)^{\sigma''\nu'}_{\sigma'\nu}(R'')^{\mu''\nu''}_{\mu'\nu'}}
        \end{equation}
        \item[ii)] Случай, когда все внешние спины $\sigma,\mu,\nu,\sigma'',\mu'',\nu''=+$.
        \begin{equation}
            \sum\limits_{\mu',\nu',\sigma'}(R'')^{\mu'\nu'}_{++}(R)^{\sigma'+}_{+\nu'}(R')^{++}_{\sigma'\mu'}=\sum\limits_{\mu',\nu',\sigma'}(R')^{\sigma'\mu'}_{++}(R)^{+\nu'}_{\sigma'+}(R'')^{++}_{\mu'\nu'}
        \end{equation}
        \begin{equation}
            (R'')^{++}_{++}(R)^{++}_{++}(R')^{++}_{++}=(R')^{++}_{++}(R)^{++}_{++}(R'')^{++}_{++}\rightarrow \boxed{a''a'a=aa'a''}
        \end{equation}
        \item[iii)]
        \begin{equation}
            \sum\limits_{\mu',\nu',\sigma'}(R')^{\sigma'\mu'}_{++}(R)^{-\nu'}_{\sigma'-}(R'')^{++}_{\mu'\nu'}=(R')^{++}_{++}(R)^{-+}_{+-}(R'')^{++}_{++}
        \end{equation}
        \begin{equation}
            (R')^{++}_{++}(R)^{-+}_{+-}(R'')^{++}_{++}=(R')^{++}_{++}(R)^{-+}_{+-}(R'')^{++}_{++}\rightarrow\boxed{a'ca''=a'ca''}
        \end{equation}
        \item[iv)]
        \begin{equation}
            \sum\limits_{\sigma',\mu',\nu'}(R'')^{\mu'\nu'}_{+-}(R)^{\sigma'+}_{+\nu'}(R')^{-+}_{\sigma'\mu'}=(R')^{++}_{++}(R)^{-+}_{+-}(R'')^{++}_{++}
        \end{equation}
        \begin{equation}
            (R'')^{+-}_{+-}(R)^{-+}_{+-}(R')^{-+}_{-+}+(R'')^{-+}_{+-}(R)^{++}_{++}(R')^{-+}_{+-}=(R')^{++}_{++}(R)^{-+}_{+-}(R'')^{++}_{++}
        \end{equation}
        \begin{equation}
            \boxed{b''cb'+c''ac'=a'ca''}
        \end{equation}
        Аналогично получаются остальные уравнения:
        \begin{equation}
            \boxed{cc'b''+(ab'-a'b)c''=0,\quad-c'ba''+ac'b''+cb'c''=0}
        \end{equation}
        Эти уравнения подтверждаются для произвольной комбинации внешних спинов.
        \item[v)] Полученные уравнения можно записать в матричном виде:
        \begin{equation}
            \begin{pmatrix}
                -ca' & cb' & ac'\\
                0 & cc' & ab'-a'b\\
                c'b & ac' & cb'
            \end{pmatrix}\begin{pmatrix}
                a''\\
                b''\\
                c''\\
            \end{pmatrix}=0
        \end{equation}
        Условие существования решений:
        \begin{equation}
            \text{det}\begin{pmatrix}
                -ca' & cb' & ac'\\
                0 & cc' & ab'-a'b\\
                c'b & ac' & cb'
            \end{pmatrix}=0
        \end{equation}
        \begin{equation}
            \boxed{\frac{a^2+b^2-c^2}{2ab}=\frac{a'^2+b'^2+c'^2}{2a'b'}=\frac{a''^2+b''^2+c''^2}{2a''b''}=\Delta}
        \end{equation}
    \end{itemize}
    \item Рассмотрите параметризацию $R$ матрицы 6-вершинной модели в отсутствии внешнего поля в терминах целых функций.
    \begin{equation}
        a(u|\lambda)=\rho\sinh(\lambda+u),\quad b(u|\lambda)=\rho\sinh(u),\quad c(u|\lambda)=\rho\sinh(\lambda)
    \end{equation}
    и соответствующие выражения через $u'$, $u''$ (но то же самое значение $\lambda$) для $R$, $R'$.
    \begin{itemize}
        \item[i)] Выразите $\Delta$ через $\lambda$.
        \item[ii)] Рассмотрите уравнение Янга-Бакстера для матриц $R$, $R'$, $R''$ с одинаковым значением $\lambda$ и найдите связь между спектральными параметрами $u$, $u'$, $u''$.
        \item[iii)] $R$ матрицы 6-вершинной модели встречаются в разных ситуациях. Чем будет отличаться параметризация в терминах $\sin u$ функций от той, что дана выше? Выпишите выражение для $R$ матрицы 6-вершинной модели в терминах рациональных функций от $w=e^u$, $q=e^\lambda$. Подумайте, как можно определить $u\rightarrow\infty$ предел от $R$ матрицы. Как выглядит уравнение Янга-Бакстера в новых переменных?
    \end{itemize}
    \textbf{Решение.}
    \begin{itemize}
        \item[i)] Выразим $\Delta$ через $\lambda$:
        \begin{equation}
            \boxed{\Delta=\frac{a^2+b^2-c^2}{2ab}=\cosh\lambda}
        \end{equation}
        \item[ii)] Уравнение Янга-Бакстера:
        \begin{equation}
            b''cb'+c''ac'=a'ca''
        \end{equation}
        \begin{equation}
            \sinh\lambda\sinh u'\sinh u''+\sinh(\lambda+u)\sinh^2\lambda=\sinh\lambda\sinh(\lambda+u')\sinh(\lambda+u'')
        \end{equation}
        \begin{equation}
            \boxed{u=u'+u''}
        \end{equation}
        \item[iii)] Параметр $\Delta$ будет не больше 1.
        \begin{equation}
            a(u|\lambda)=\rho\sinh(\lambda+u)=\rho\frac{e^{\lambda+u}-e^{-\lambda-u}}{2}=\rho\frac{wq-w^{-1}q^{-1}}{2}
        \end{equation}
        \begin{equation}
            b(u|\lambda)=\rho\sinh(u)=\rho\frac{e^u-e^{-u}}{2}=\rho\frac{w-w^{-1}}{2}
        \end{equation}
        \begin{equation}
            c(u|\lambda)=\rho\sinh(\lambda)=\rho\frac{e^{\lambda}-e^{-\lambda)}}{2}=\rho\frac{q-q^{-1}}{2}
        \end{equation}
        При $u\rightarrow\infty$:
        \begin{equation}
            a(u|\lambda)=\frac{\rho wq}{2},\quad b(u|\lambda)=\frac{\rho w}{2},\quad c(u|\lambda)=\rho\frac{q-q^{-1}}{2}\ll a(u|\lambda),b(u|\lambda)
        \end{equation}
        \begin{equation}
            \boxed{R=\frac{\rho}{2}\begin{pmatrix}
                wq & 0 & 0 & 0\\
                0 & w & 0 & 0\\
                0 & 0 & w & 0\\
                0 & 0 & 0 & wq\\
            \end{pmatrix}}
        \end{equation}
        Уравнение Янга-Бакстера в новых переменных выглядит как тождественные равенства.
    \end{itemize}
    \item Мы изучаем матрицы (операторы), которые действуют на пространствах высокой размерности -- собрании спинов вдоль прямой, например. Такие пространства имеют естественную структуру тензорного произведения
    \begin{equation}
        \mathcal{F}=F_1\otimes F_2\otimes...\otimes F_n
    \end{equation}
    Пусть $v_\pm$ -- два базисных вектора в $H_j\sim C^2$ (это представления $sl_2$ со спином вверх или вниз):
    \begin{equation}
        v_+=\begin{pmatrix}
            1\\
            0
        \end{pmatrix},\quad v_-=\begin{pmatrix}
            0\\
            1
        \end{pmatrix}
    \end{equation}
    Наша $R$-матрица $R:C^2\rightarrow R^2$ действует как оператор в тензорном произведении двух пространств по правилу
    \begin{equation}
        R(v_{\sigma_1}\otimes v_{\sigma_2})R^{\sigma'_1\sigma'_2}_{\sigma_1\sigma_2}v_{\sigma'_1}\otimes v_{\sigma'_2}
    \end{equation}
    \begin{equation}
        R:F\otimes F\rightarrow F\otimes F
    \end{equation}
    \begin{itemize}
        \item[i)] Матрицу $4\times4$ в виде суммы матричных произведений
        \begin{equation}
            R=\sum\limits_js_j\otimes t_j,\quad s_j,t_j\in\text{Mat}(2\times2)
        \end{equation}
        где $s_i$ и $t_i$ -- матрицы $2\times2$. Проверьте, что $R$-матрицу шестивершинной модели можно представить в виде
        \begin{equation}
            R=\frac{a+b}{2}I\otimes I+\frac{a-b}{2}\sigma^z\otimes\sigma^z+c(\sigma^+\otimes\sigma^-+\sigma^-\otimes\sigma^+)
        \end{equation}
        Рассмотрим тензорное произведение 3 спиновых пространств
        \begin{equation}
            F_1\otimes F_2\otimes F_3
        \end{equation}
        Будем обозначать через $R_{12}$ действие оператора, который действует на первую и вторую компоненту как $R$ матрица, а на третью -- 1
        \begin{equation}
            R_{12}=R\otimes I=\sum\limits_js_j\otimes t_j\otimes1:F_1\otimes F_2\otimes F_3\rightarrow F_1\otimes F_2\otimes F_3
        \end{equation}
        Аналогично определим оператор $R_{23}$. Это оператор в тензорном кубе действует 1 на первую компоненту, а на вторую и третью -- как $R$ матрица
        \begin{equation}
            R_{23}=I\otimes R=\sum\limits_jI\otimes s_j\otimes t_j:F_1\otimes F_2\otimes F_3\rightarrow F_1\otimes F_2\otimes F_3
        \end{equation}
        Определим действие оператора $R_{13}$ как действующего $R$ матрицей на 1 и 3 компоненты тензорного произведения
        \begin{equation}
            R_{13}=\sum\limits_js_j\otimes I\otimes t_j:F_1\otimes F_2\otimes F_3\rightarrow F_1\otimes F_2\otimes F_3
        \end{equation}
        Рассмотрев явную запись уравнения Янга-Бакстера, можно подействовать левой и правой компонентами на вектор $v_{\sigma''}\otimes v_{\mu''}\otimes v_{\nu''}$. Используя разложение
        \begin{equation}
            R=\sum s_j\otimes t_j,\quad R'=\sum s'_j\otimes t'_j,\quad R''=\sum s''_j\otimes t''_j
        \end{equation}
        можно проверить в этих обозначений, что уравнение Янга-Бакстера имеет очень простой для запоминания вид
        \begin{equation}
            R''_{23}R_{13}R'_{12}=R'_{12}R_{13}R''_{23}
        \end{equation}
        Каждая из матриц имеет размер $2^3$, но строится из стандартной $R$ матрицы, как указано выше. Но данное соотношение теперь -- явное матричное тождество.
        \item[ii)] В лекциях сделано простое вычисление для одной из сторон YBE. Проверьте (для другой стороны YBE), что уравнение Янга-Бакстера для 6-вершинной модели в параметризации со спектральным параметром записываются как
        \begin{equation}
            R_{12}(u_1-u_2)R_{13}(u_1-u_3)R_{23}(u_2-u_3)=R_{23}(u_2-u_3)R_{13}(u_1-u_3)R_{12}(u_1-u_2)
        \end{equation}
        Если матричные обозначения понятны, то единственный вопрос -- связь между спектральными параметрами.
        \item[iii)] Используя $R$ матрицу $R(u)$, можно построить трансфер матрицу $T(u)$. Возьмём в случае 6-вершинной модели другую $R$ матрицу $R(u')$ с тем же самым параметром $\lambda$ (а значит и $\Delta$). Построим аналогично трансфер матрицу $T(u')$ с тем же самым $\lambda$.\\
        В произведении $T(u)T(u')$ аккуратно рассмотрите граничные спины, введите умножение на $R$ матрицу и аргументируйте, что 
        \begin{equation}
            T(u)T(u')=T(u')T(u)
        \end{equation}
        в случае периодической решётки.
    \end{itemize}
    \textbf{Решение.}
    \begin{itemize}
        \item[i)] Было проверено в предыдущем домашнем задании. \item[ii)]
        \begin{equation}
            R''_{23}R_{13}R'_{12}=R'_{12}R_{13}R''_{23}
        \end{equation}
        В параметризации со спектральным параметром
        \begin{equation}
            R_{23}(u'')R_{13}(u)R_{12}(u')=R_{12}(u')R_{13}(u)R_{23}(u'')
        \end{equation}
        Из уравнения Янга-Бакстера $u=u'+u''$, поэтому введём параметризацию $u=u_1-u_3$, $u'=u_1-u_2$, $u''=u_2-u_3$ (чтобы выполнялось $u=u'+u''$).
        \begin{equation*}
            \boxed{R_{12}(u_1-u_2)R_{13}(u_1-u_3)R_{23}(u_2-u_3)=R_{23}(u_2-u_3)R_{13}(u_1-u_3)R_{12}(u_1-u_2)}
        \end{equation*}
        \item[iii)]
        \begin{equation}
            T_{\vec{\sigma}}^{\vec{\sigma}'}=\sum\limits_{\vec{\mu}}R^{\sigma'_1\mu'_2}_{\sigma_1\mu'_1}R^{\sigma'_2\mu'_3}_{\sigma_2\mu'_2}...R^{\sigma'_n\mu'_1}_{\sigma_n\mu'_n}
        \end{equation}
        \begin{equation}
            (TT')^{\vec{\sigma}''}_{\vec{\sigma}}=\sum\limits_{\vec{\sigma}'}T^{\vec{\sigma}''}_{\vec{\sigma}'}T'^{\vec{\sigma}'}_{\vec{\sigma}}
        \end{equation}
        Из периодичности $\mu'_{n+1}=\mu'_1$, $\mu''_{n+1}=\mu''_1$, $\sigma_{n+1}=\sigma_1$, $\sigma'_{n+1}=\sigma'_1$. Из уравнения Янга-Бакстера
        \begin{equation}
            \sum\limits_{\vec{\sigma},\vec{\mu}',\vec{\mu}''}(R'')^{\mu''_1\mu'_1}_{\mu''_{n+1}\mu'_{n+1}}(R)^{\sigma'_1\mu'_2}_{\sigma'_1\mu'_1}(R')^{\sigma''_2\mu''_2}_{\sigma'_1\mu''_1}...=\sum\limits_{\vec{\sigma},\vec{\mu}',\vec{\mu}''}...(R')^{\sigma'_n\mu''_{n+1}}_{\sigma''_n\mu''_n}(R)^{\sigma_n\mu'_{n+1}}_{\sigma'_n\mu'_n}(R'')^{\mu''_1\mu'_1}_{\mu''_{n+1}\mu'_{n+1}}
        \end{equation}
        \begin{equation}
            T'^{\vec{\sigma}'}_{\vec{\sigma}}T^{\vec{\sigma}''}_{\vec{\sigma}''}=T^{\vec{\sigma}'}_{\vec{\sigma}}T'^{\vec{\sigma}''}_{\vec{\sigma}''}\rightarrow \boxed{T(u)T(u')=T(u')T(u)}
        \end{equation}
    \end{itemize}
    \item Оператор перестановки. Рассмотрите произведение двух двумерных векторов
    \begin{equation}
        x=\begin{pmatrix}
            x_+\\
            x_-
        \end{pmatrix},\quad y=\begin{pmatrix}
            y_+\\
            y_-
        \end{pmatrix},\quad z=x\otimes y,\quad z_{aj}=x_ay_j
    \end{equation}
    \begin{itemize}
        \item[i)] Напишите явно, в виде матрицы $4\times4$ оператор перестановки
        \begin{equation}
            P(x\otimes y)=y\otimes x
        \end{equation}
        через дельта-символы Кронекера. Проверьте, что
        \begin{equation}
            P^2=I
        \end{equation}
        \item[ii)] Напишите явно матричные элементы оператора перестановки $P^{bk}_{aj}$. Выпишите в компонентах действие оператора перестановки на матрицу $A^{bk}_{aj}\in\text{Mat}_{4\times4}$, действующую в тензорном квадрате $F_1\otimes F_2$:
        \begin{equation}
            (A_{21})_{aj}^{bk}=(PA_{12}P)^{bk}_{aj}
        \end{equation}
        \item[iii)] Рассмотрим теперь тензорное произведение $n$ копий двумерных пространств $F\otimes F\otimes...\otimes F$. Пусть на этом пространстве действуют матрицы
        \begin{equation}
            T^{\sigma'_1,\sigma'_2,...}_{\sigma_1,\sigma_2,...}\in\text{Mat}_{2^n\times2^n}
        \end{equation}
        ($n$-мерные обобщения матриц на $F\otimes F$)
        \begin{equation}
            v'_{\sigma_1,\sigma_2,...,\sigma_n}=T^{\sigma'_1,\sigma'_2,...,\sigma'_n}_{\sigma_1,\sigma_2,...,\sigma_n}v_{\sigma'_1,\sigma'_2,...,\sigma'_n}
        \end{equation}
        Как пример операторов такого вида, рассмотрим оператор перестановки, который меняет пространства $l$ и $k$:
        \begin{equation}
            P_{lk}=\prod\limits_{j=1,j\neq l,k}^n\delta^{\sigma'_j}_{\sigma_j}P^{\sigma'_l\sigma'_k}_{\sigma_l\sigma_k}
        \end{equation}
        Аналогичным образом определим действие матриц $A_{lk}$, построенные из матриц, действующих на тензорном квадрате $A:F\otimes F\rightarrow F\otimes F$
        \begin{equation}
            A_{lk}=\prod\limits_{j=1,j\neq l,k}^n\delta_{\sigma_j}^{\sigma'_j}A_{\sigma_l\sigma_k}^{\sigma'_l\sigma'_k}
        \end{equation}
        В дальнейшем нам понадобится понятие частичного следа. Частичным следом по $k$-му пространству у матрицы $T$ (дейстаующей в тензорном произведении $F^{\otimes n}$) определим как след по $F_k$ пространству
        \begin{equation}
            \text{Tr}_kT_{\sigma_1,\sigma_2,...}^{\sigma'_1,\sigma'_2,...}=\sum\limits_{\sigma_k}T^{\sigma'_1,\sigma'_2,...,\sigma_k,...}_{\sigma_1,\sigma_2,...,\sigma_k,...}
        \end{equation}
        Чтобы выучить обозначения, вычислите явно матрицы
        \begin{equation}
            P_{12}A_{13}P_{12},\quad P_{12}A_{23}A_{13}P_{12}
        \end{equation}
        в терминах матриц $A_{kl}$. Вычислите примеры частичного следа
        \begin{equation}
            \text{Tr}_k(P_{1k}),\quad\text{Tr}_1(P_{12}A_{13}A_{14})
        \end{equation}
    \end{itemize}
    \textbf{Решение.}
    \begin{itemize}
        \item[i)]
        \item[ii)] Матричные элементы оператора перестановки:
        \begin{equation}
            \boxed{P^{bk}_{aj}=\delta_a^k\delta_j^b}
        \end{equation}
        \begin{equation}
            (A_{21})_{aj}^{bk}=(PA_{12}P)^{bk}_{aj}=P^{bk}_{cd}(A_{12})^{cd}_{ef}P^{ef}_{aj}=\delta^k_c\delta^b_d(A_{12})^{cd}_{ef}\delta_a^f\delta_j^e=(A_{12})^{kb}_{ja}
        \end{equation}
        \begin{equation}
            \boxed{(A_{21})_{aj}^{bk}=(A_{12})^{kb}_{ja}}
        \end{equation}
        \item[iii)]
        \begin{multline}
            (P_{12}A_{13}P_{12})^{\sigma'_1...\sigma'_n}_{\mu_1...\mu_n}=\prod\limits_{j=3}^n\delta^{\sigma'_j}_{\sigma_j}P^{\sigma'_1\sigma'_2}_{\sigma_1\sigma_2}\prod\limits_{j=2,j\neq3}^n\delta^{\sigma_j}_{\lambda_j}A^{\sigma_1\sigma_3}_{\lambda_1\lambda_4}\prod\limits_{j=3}^n\delta^{\lambda_j}_{\mu_j}P^{\lambda_1\lambda_2}_{\mu_1\mu_2}=\\=P^{\sigma'_1\sigma'_2}_{\sigma_1\sigma_2}\delta^{\sigma'_3}_{\sigma_3}A^{\sigma_1\sigma_3}_{\lambda_1\lambda_3}\delta^{\sigma_2}_{\lambda_2}P^{\lambda_1\lambda_2}_{\mu_1\mu_2}\delta^{\lambda_3}_{\mu_3}\prod\limits_{j=4}^n\delta^{\sigma'_j}_{\mu_j}=\delta^{\sigma'_1}_{\sigma_2}\delta^{\sigma'_2}_{\sigma_1}A^{\sigma_1\sigma'_3}_{\lambda_1\mu_3}\delta^{\lambda_1}_{\mu_2}\delta^{\sigma_2}_{\mu_1}\prod\limits_{j=4}^n\delta^{\sigma'_j}_{\mu_j}
        \end{multline}
        \begin{equation}
            \boxed{(P_{12}A_{13}P_{12})^{\sigma'_1...\sigma'_n}_{\mu_1...\mu_n}=\delta^{\sigma'_1}_{\mu_1}A^{\sigma'_2\sigma'_3}_{\mu_2\mu_3}\prod\limits_{j=4}^n\delta^{\sigma'_j}_{\mu_j}}
        \end{equation}
        \begin{multline}
            (P_{12}A_{23}A_{13}P_{12})^{\sigma'_1...\sigma'_n}_{\nu_1...\nu_n}=(P_{12}A_{13}P_{12}P_{12}A_{23}P_{12})^{\sigma'_1...\sigma'_n}_{\nu_1...\nu_n}=(P_{12}A_{23}P_{12}A_{23})^{\sigma'_1...\sigma'_n}_{\nu_1...\nu_n}=\\=\prod\limits_{j=3}^n\delta^{\sigma'_j}_{\sigma_j}P^{\sigma'_1\sigma'_2}_{\sigma_1\sigma_2}\prod\limits_{j=1,j\neq2,3}\delta^{\sigma_j}_{\lambda_j}A^{\sigma_2\sigma_3}_{\lambda_2\lambda_3}\prod\limits_{j=3}^n\delta^{\lambda_j}_{\mu_j}P^{\lambda_1\lambda_2}_{\mu_1\mu_2}(A_{23})^{\mu_1...\mu_n}_{\nu_1...\nu_n}=(A_{13}A_{23})^{\sigma'_1...\sigma'_n}_{\mu_1...\mu_n}
        \end{multline}
        \begin{equation}
            \boxed{(P_{12}A_{23}A_{13}P_{12})^{\sigma'_1...\sigma'_n}_{\nu_1...\nu_n}=(A_{13}A_{23})^{\sigma'_1...\sigma'_n}_{\mu_1...\mu_n}}
        \end{equation}
        \begin{equation}
            (\text{Tr}_k(P_{1k}))^{\sigma'_1...\sigma'_{k-1}\sigma'_{k+1}...\sigma'_n}_{\sigma_1...\sigma_{k-1}\sigma_{k+1}...\sigma_n}=\delta^{\sigma'_1}_{\sigma_k}\delta^{\sigma_k}_{\sigma_1}\prod\limits_{j=2,j\neq k}^n\delta^{\sigma'_j}_{\sigma_j}=\prod\limits_{j=1,j\neq k}^n\delta^{\sigma'_j}_{\sigma_j}
        \end{equation}
        \begin{equation}
            \boxed{(\text{Tr}_k(P_{1k}))^{\sigma'_1...\sigma'_{k-1}\sigma'_{k+1}...\sigma'_n}_{\sigma_1...\sigma_{k-1}\sigma_{k+1}...\sigma_n}=\mathbb{I}^{\sigma'_1...\sigma'_{k-1}\sigma'_{k+1}...\sigma'_n}_{\sigma_1...\sigma_{k-1}\sigma_{k+1}...\sigma_n}}
        \end{equation}
        \begin{multline}
            \text{Tr}_1(P_{12}A_{13}A_{14})^{\sigma'_2...\sigma'_n}_{\sigma_2...\sigma_n}=\prod\limits_{j=3}^n\delta^{\sigma'_j}_{\mu_j}P^{\sigma_1\sigma'_2}_{\mu_1\mu_2}\prod\limits_{j=2,j\neq3}^n\delta^{\mu_j}_{\lambda_j}A^{\mu_1\mu_3}_{\lambda_1\lambda_3}\prod\limits_{j=2,j\neq4}^n\delta^{\lambda_j}_{\sigma_j}A^{\lambda_1\lambda_4}_{\sigma_1\sigma_4}=\\=\delta^{\sigma_1}_{\mu_2}\delta^{\sigma'_2}_{\mu_1}\delta^{\sigma'_3}_{\mu_3}\delta^{\sigma'_4}_{\mu_4}\delta^{\mu_2}_{\lambda_2}\delta^{\mu_4}_{\lambda_4}\delta^{\lambda_2}_{\sigma_2}\delta^{\lambda_3}_{\sigma_3}A^{\mu_1\mu_3}_{\lambda_1\lambda_3}A^{\lambda_1\lambda_4}_{\sigma_1\sigma_4}A^{\mu_1\mu_3}_{\lambda_1\lambda_3}A^{\lambda_1\lambda_4}_{\sigma_1\sigma_4}\prod\limits_{j=5}^n\delta^{\sigma'_j}_{\sigma_j}=(A_{23}A_{24})^{\sigma'_2...\sigma'_n}_{\sigma_2...\sigma_n}
        \end{multline}
        \begin{equation}
            \boxed{\text{Tr}_1(P_{12}A_{13}A_{14})^{\sigma'_2...\sigma'_n}_{\sigma_2...\sigma_n}=(A_{23}A_{24})^{\sigma'_2...\sigma'_n}_{\sigma_2...\sigma_n}}
        \end{equation}
    \end{itemize}
    \item Рассмотрите анзац для $R:F\otimes F\rightarrow F\otimes F$ матрицы в виде
    \begin{equation}
        R(u)=f(u)I+cP
    \end{equation}
    где $I$ -- тождественный оператор в пространстве $4\times4$, $c$ -- константа, а $P$ -- оператор перестановки в $F\otimes F$.
    \begin{itemize}
        \item[i)] Покажите, что при $c=0$, а также при $f(u)=0$ мы получаем простые решения уравнения Янга-Бакстера.
        \item[ii)] Подставьте анзац $R(u)=f(u)I+cP$ с неисчезающими обоими членами в YBE
        \begin{equation}
            R_{12}(u_1-u_2)R_{13}(u_1-u_3)R_{23}(u_2-u_3)=R_{23}(u_2-u_3)R_{13}(u_1-u_3)R_{12}(u_1-u_2)
        \end{equation}
        разлагая уравнение по переменной $c$, найдите функциональное уравнение на $f(u)$. Решите его (найдите рациональное решение уравнения Янга-Бакстера -- случай $XXX$-цепочки -- запишите его явно в виде матрицы $4\times4$).
        \item[iii)] Будет ли решение верным, если в качестве векторного пространства взять не спин 1/2, а произвольное пространство размерности $k$? Выпишите явно пример (рациональной) $R$ матрицы для пространств $3\times3$. 
    \end{itemize}
    \textbf{Решение.}
    \begin{itemize}
        \item[i)] Уравнение Янга-Бакстера:
        \begin{equation}
            R_{12}(u_1-u_2)R_{13}(u_1-u_3)R_{23}(u_2-u_3)=R_{23}(u_2-u_3)R_{13}(u_1-u_3)R_{12}(u_1-u_2)
        \end{equation}
        При $c=0$:
        \begin{equation}
            f(u_1-u_2)f(u_1-u_3)f(u_2-u_3)I=f(u_2-u_3)f(u_1-u_3)f(u_1-u_2)I
        \end{equation}
        При $f(u)=0$:
        \begin{equation}
            c^3P_{12}P_{13}P_{23}=c^3P_{23}P_{13}P_{12}
        \end{equation}
        \begin{equation}
            P_{12}P_{13}P_{23}P_{12}=P_{13}P_{23}
        \end{equation}
        Поменяем местами индексы 1 и 2 в правой части:
        \begin{equation}
            P_{12}P_{13}P_{23}P_{12}=P_{23}P_{13}
        \end{equation}
        \item[ii)] Подставим анзац:
        \begin{multline}
            f(u_1-u_2)f(u_1-u_3)f(u_2-u_3)+сf(u_1-u_2)f(u_2-u_3)P_{13}+\\+cf(u_1-u_3)f(u_2-u_3)P_{12}+cf(u_1-u_2)f(u_1-u_3)P_{23}+c^2f(u_2-u_3)P_{12}P_{13}+\\+c^2f(u_1-u_2)P_{13}P_{23}+c^2f(u_1-u_3)P_{12}P_{23}+c^3P_{12}P_{13}P_{23}=\\=f(u_1-u_2)f(u_1-u_3)f(u_2-u_3)+f(u_2-u_3)f(u_1-u_2)P_{13}+\\+сf(u_1-u_3)f(u_1-u_2)P_{23}+cf(u_2-u_3)f(u_1-u_3)P_{12}+c^2f(u_1-u_3)P_{23}P_{13}+\\+c^2f(u_2-u_3)P_{13}P_{12}+c^2f(u_1-u_3)P_{23}P_{12}+c^3P_{12}P_{13}P_{23}
        \end{multline}
        \begin{equation}
            P_{12}P_{13}=P_{13}P_{23}=P_{23}P_{12}=P_{231}
        \end{equation}
        \begin{equation}
            P_{12}P_{23}=P_{23}P_{13}=P_{13}P_{12}=P_{312}
        \end{equation}
        Приравняем коэффициент при $cP_{12}$:
        \begin{equation}
            f(u_1-u_3)f(u_2-u_3)=f(u_2-u_3)f(u_1-u_3)
        \end{equation}
        Приравняем коэффициент при $cP_{23}$:
        \begin{equation}
            f(u_1-u_2)f(u_1-u_3)=f(u_1-u_3)f(u_1-u_2)
        \end{equation}
        Приравняем коэффициент при $cP_{13}$:
        \begin{equation}
            f(u_1-u_2)f(u_2-u_3)=f(u_2-u_3)f(u_1-u_2)
        \end{equation}
        Приравняем коэффициент при $c^2P_{123}$:
        \begin{equation}
            f(u_1-u_3)=f(u_1-u_2)+f(u_2-u_3)
        \end{equation}
        Приравняем коэффициент при $c^2P_{312}$:
        \begin{equation}
            f(u_2-u_3)+f(u_1-u_2)=f(u_1-u_3)
        \end{equation}
        При $u_2=0$:
        \begin{equation}
            f(u_1-u_3)=f(u_1-u_2)+f(u_2-u_3)\rightarrow f(u)=au+b
        \end{equation}
        Подставив решение в функциональное уравнение, получим
        \begin{equation}
            b=0
        \end{equation}
        \begin{equation}
            \boxed{R(u)=f(u)I+cP=\begin{pmatrix}
                au+c & 0 & 0 & 0\\
                0 & au & c & 0\\
                0 & c & au & 0\\
                0 & 0 & 0 & au+c
            \end{pmatrix}}
        \end{equation}
        \item[iii)] Пример для пространств $3\times3$:
        \begin{equation}
            \boxed{R=\begin{pmatrix}
                f(u)+c & 0 & 0 & 0 & 0 & 0 & 0 & 0 & 0\\
                0 & f(u) & 0 & c & 0 & 0 & 0 & 0 & 0\\
                0 & 0 & f(u) & 0 & 0 & 0 & c & 0 & 0\\
                0 & c & 0 & f(u) & 0 & 0 & 0 & 0 & 0\\
                0 & 0 & 0 & 0 & f(u)+c & 0 & 0 & 0 & 0\\
                0 & 0 & 0 & 0 & 0 & f(u) & 0 & c & 0\\
                0 & 0 & c & 0 & 0 & 0 & f(u) & 0 & 0\\
                0 & 0 & 0 & 0 & 0 & c & 0 & f(u) & 0\\
                0 & 0 & 0 & 0 & 0 & 0 & 0 & 0 & f(u)+c\\
            \end{pmatrix}}
        \end{equation}
    \end{itemize}
\end{enumerate}
\section{Алгебра $RLL=LLR$}
\begin{enumerate}
    \item 
    \item $^*$ XXZ Гамильтониан.
    \begin{itemize}
        \item[i)] Покажите, что оператор $R_{ij}(0)$, действующий на компонентах $i$ и $j$ тензорного произведения
        \begin{equation}
            F_1\otimes F_2\otimes F_3
        \end{equation}
        является с точностью до константы оператором перестановки $P_{ij}$.
        \item[ii)] Найдите явно матрицу
        \begin{equation}
            \frac{dR(u)}{du}|_{u=0}
        \end{equation}
        \item[iii)] Найдите явно вид операторов $T(0)$ и $T^{-1}(0)$ при действии на спиновые переменные из $\mathcal{F}$.
        \item[iv)] Рассмотрите разложение трансфер матрицы модели по степеням спектрального параметра
        \begin{equation}
            \log(T^{-1}(0)T(u))=1-\sum\limits_{j=1}^\infty I_ju^j
        \end{equation}
        Почему $[I_i,I_j]=0$? В каком пространстве действуют операторы $I_j$? Если $I_1$ осуществляет трансляции, то какой может быть смысл у старших интегралов? В разложении в ряд будет бесконечное число операторов $I_j$. До термодинамического предела в динамической системе конечное число степеней свободы (флуктуирующих спинов). Как мы нашли бесконечномерную симметрию в конечной системе? Что вы думаете по этому поводу?
        \item[v)] Найдите явно первый нетривиальный оператор $I_1$ (дифференцированием и наложением условия $u=0$) и сравните результат с гамильтонианом $XXZ$ цепочки.
        \item[vi)] Подумайте над видом следующего нетривиального интеграла движения $I_2$ в данном разложении. Видна ли локальность интегралов движения? Попробуйте вычислить какой-то старший интеграл движения для простого случая $XX$ модели, в которой $\Delta=0$.
    \end{itemize}
    \textbf{Решение.}
    \begin{itemize}
        \item[i)]
        \begin{equation}
            R(u)=\rho\begin{pmatrix}
                \sinh(\lambda+u) & 0 & 0 & 0\\
                0 & \sinh u & \sinh\lambda & 0\\
                0 & \sinh\lambda & \sinh u & 0\\
                0 & 0 & 0 & \sinh(\lambda+u)
            \end{pmatrix}
        \end{equation}
        \begin{equation}
            R(0)=\rho\begin{pmatrix}
                \sinh\lambda & 0 & 0 & 0\\
                0 & 0 & \sinh\lambda & 0\\
                0 & \sinh\lambda & 0 & 0\\
                0 & 0 & 0 & \sinh\lambda
            \end{pmatrix}=\rho\sinh\lambda P
        \end{equation}
        \begin{equation}
            R_{12}=R\otimes I,\quad R_{23}=I\otimes R
        \end{equation}
        \item[ii)]
        \begin{equation}
            \frac{dR}{du}=\rho\begin{pmatrix}
                \cosh(\lambda+u) & 0 & 0 & 0\\
                0 & \cosh u & 0 & 0\\
                0 & 0 & \cosh u & 0\\
                0 & 0 & 0 & \cosh(\lambda+u)
            \end{pmatrix}
        \end{equation}
        \begin{equation}
            \boxed{\frac{dR}{du}|_{u=0}=\rho\begin{pmatrix}
                \cosh\lambda & 0 & 0 & 0\\
                0 & 1 & 0 & 0\\
                0 & 0 & 1 & 0\\
                0 & 0 & 0 & \cosh\lambda
            \end{pmatrix}}
        \end{equation}
        \item[iii)]
        \begin{equation}
            \boxed{(T(0))^{\sigma'_1...\sigma'_n}_{\sigma_1...\sigma_n}=\sinh^n\lambda P^{\mu_2\sigma'_1}_{\mu_1\sigma_1}P^{\mu_3\sigma'_2}_{\mu_2\sigma_2}...P^{\mu_1\sigma'_n}_{\mu_n\sigma_n}=\sinh^n\lambda\prod\limits_{j=1}^n\delta_{\sigma_j}^{\sigma'_{j+1}}}
        \end{equation}
        \begin{equation}
            \boxed{(T^{-1}(0))^{\sigma'_1...\sigma'_n}_{\sigma_1...\sigma_n}=\sinh^{-n}\lambda P^{\mu_n\sigma'_1}_{\mu_1\sigma_1}P^{\mu_1\sigma'_2}_{\mu_2\sigma_2}...P^{\mu_{n-1}\sigma'_n}_{\mu_n\sigma_n}=\sinh^n\lambda\prod\limits_{j=1}^n\delta_{\sigma_j}^{\sigma'_{j-1}}}
        \end{equation}
        \item[iv)] Покажем, что из $[T(u),T(v)]=0$ следует, что $[T^{-1}(u),T(v)]=0$:
        \begin{equation}
            [T(u),T(v)]=T(u)T(v)-T(v)T(u)=0
        \end{equation}
        Домножим на $T^{-1}(u)$ слева:
        \begin{equation}
            T(v)-T^{-1}(u)T(v)T(u)=0
        \end{equation}
        Домножим на $T^{-1}(u)$ справа:
        \begin{equation}
            T(v)T^{-1}(u)-T^{-1}(u)T(v)=[T^{-1}(u),T(v)]=0
        \end{equation}
        \begin{equation}
            T^{-1}(0)T(u)=1-\sum\limits_{j=1}^\infty\frac{1}{j!}H_ju^j,\quad [H_i,H_j]=0
        \end{equation}
        \begin{multline}
            \log(T^{-1}(0)T(u))=\log\left(1-\sum\limits_{j=1}^\infty\frac{1}{j!}H_ju^j\right)=\sum\limits_{n=1}^\infty\left(\frac{(-1)^{n+1}}{n!}(-1)^n\left(\sum\limits_{j=1}^\infty\frac{1}{j!}H_ju^j\right)^n\right)=\\=-\sum\limits_{n=1}^\infty\frac{1}{n!}\left(\sum\limits_{j=1}^\infty\frac{1}{j!}H_ju^j\right)^n
        \end{multline}
        С другой стороны,
        \begin{equation}
            \log(T^{-1}(0)T(u))=1-\sum\limits_{j=1}^\infty I_ju^j
        \end{equation}
        \begin{equation}
            -\sum\limits_{n=1}^\infty\frac{1}{n!}\left(\sum\limits_{j=1}^\infty\frac{1}{j!}H_ju^j\right)^n=1-\sum\limits_{j=1}^\infty I_ju^j\rightarrow I_j=I_j(H_j)
        \end{equation}
        \begin{equation}
            [H_i,H_j]=0\rightarrow\boxed{[I_i,I_j]=0}
        \end{equation}
        Т.к. $T:\mathcal{F}\rightarrow\mathcal{F}$, то $I_j$ действуют в $\mathcal{F}$. Интегралы движения $I_j$ ($j\neq1$) не являются функционально независимыми, поскольку
        \begin{equation}
            \frac{d\mathcal{O}}{dt}=i[I_1,\mathcal{O}]
        \end{equation}
        Т.е. симметрия не бесконечномерная.
        \item[v)] 
        \begin{equation}
            I_1=\frac{d}{du}(\log(T^{-1}(0)T(u)))|_{u=0}=(T^{-1}(0)T(0))^{-1}((T^{-1}(0))'T(0)+T^{-1}(0)T'(0))
        \end{equation}
        С точностью до множителя это совпадает с
        \begin{equation}
            H_{XXZ}=-J\sum\limits_{j=1}^n(\sigma_j^x\sigma^x_{j+1}+\sigma_j^y\sigma^y_{j+1}+\Delta \sigma_j^z\sigma^z_{j+1})
        \end{equation}
        \item[vi)]
        \begin{equation}
            \frac{dI_2}{dt}=i[I_1,I_2]
        \end{equation}
        Локальность следует из вида гамильтониана (зависит от конечного числа рядом стоящих спинов).
        \begin{equation}
            I_{n+1}=[B,I_n],\quad B=\frac{1}{2i}\sum\limits_{j=1}^NJ(\sigma^x_j\sigma^x_{j+1}+\sigma^y_j\sigma^y_{j+1}+\sigma^z_j\sigma^z_{j+1})
        \end{equation}
        \begin{equation}
            \sigma^\alpha_j=I\otimes...\otimes I\otimes\sigma^\alpha\otimes I\otimes...\otimes I
        \end{equation}
        \begin{equation}
            I_3=\sum\limits_{j=1}^n([s_j,s_{j+1}],s_{j+2}),\quad s_j=\begin{pmatrix}
                \sigma^x_j\\
                \sigma^y_j\\
                \sigma^z_j
            \end{pmatrix}
        \end{equation}
        \begin{equation}
            \boxed{I_3=\sum\limits_{j=1}^N\det\begin{pmatrix}
                \sigma^x_j & \sigma^y_j & \sigma^z_j\\
                \sigma^x_{j+1} & \sigma^y_{j+1} & \sigma^z_{j+1}\\
                \sigma^x_{j+2} & \sigma^y_{j+2} & \sigma^z_{j+2}\\
            \end{pmatrix}}
        \end{equation}
    \end{itemize}
    \item Матричные произведения.
    \begin{itemize}
        \item[i)] Рассмотрите матрицу монодромии и уравнения коммутации матриц с разными спектральными параметрами
        \begin{equation}
            \sum\limits_{\mu\mu'}R^{\mu\mu'}_{\alpha\beta}(v-u)\mathcal{T}^\nu_\mu(v)\mathcal{T}^{\nu'}_{\mu'}(u)=\sum\limits_{\mu\mu'}\mathcal{T}^{\mu'}_\beta(u)\mathcal{T}^\mu_\alpha(v)R^{\nu\nu'}_{\mu\mu'}(v-u)
        \end{equation}
        Пусть по определению
        \begin{equation}
            \mathcal{T}_1=\mathcal{T}\otimes1,\quad \mathcal{T}_2=1\otimes\mathcal{T}
        \end{equation}
        Запишите матрицы $\mathcal{T}_1$, $\mathcal{T}_2$ как операторно-значные матрицы $4\times4$. Проверьте явным вычислением, что $RTT$ алгебра может быть записана в виде
        \begin{equation}
            R_{12}(v-u)\mathcal{T}_1(v)\mathcal{T}_2(u)=\mathcal{T}_2(u)\mathcal{T}_1(v)R_{12}(v-u)
        \end{equation}
        \item[ii)] Запишите матрицу монодромии через произведение $R_{ij}$ в матричном виде.
    \end{itemize}
    \textbf{Решение.}
    \begin{itemize}
        \item[i)]
        \begin{equation}
            \boxed{\mathcal{T}_1=\mathcal{T}\otimes1=\begin{pmatrix}
                A & 0 & B & 0\\
                0 & A & 0 & B\\
                C & 0 & D & 0\\
                0 & C & 0 & D
            \end{pmatrix},\quad \mathcal{T}_2=1\otimes\mathcal{T}=\begin{pmatrix}
                A & B & 0 & 0\\
                C & D & 0 & 0\\
                0 & 0 & A & B\\
                0 & 0 & C & D
            \end{pmatrix}}
        \end{equation}
        \begin{multline}
            R^{\mu\mu'}_{\alpha\beta}(v-u)\mathcal{T}^\nu_\mu(v)\mathcal{T}^{\nu'}_{\mu'}(u)=R^{\mu\mu'}_{\alpha\beta}(v-u)\mathcal{T}^{\nu'}_{\kappa'}(u)\delta^{\kappa'}_{\mu'}\mathcal{T}^\kappa_\mu(v)\delta^\nu_\kappa=\\=((1\otimes \mathcal{T}(u))(\mathcal{T}(v)\otimes1)R(v-u))^{\nu\nu'}_{\mu\mu'}=(\mathcal{T}_2(u)\mathcal{T}_1(v)R_{12}(v-u))^{\nu\nu'}_{\mu\mu'}
        \end{multline}
        \begin{multline}
            \mathcal{T}^{\mu'}_\beta(u)\mathcal{T}^\mu_\alpha(v)R^{\nu\nu'}_{\mu\mu'}(v-u)=R^{\nu\nu'}_{\mu\mu'}(v-u)\mathcal{T}^{\kappa'}_{\beta}(u)\delta^{\mu'}_{\kappa'}\mathcal{T}^\mu_\kappa(v)\delta^\kappa_\alpha=\\=(R(v-u)(\mathcal{T}(v)\otimes1)(1\otimes \mathcal{T}(u)))^{\nu\nu'}_{\alpha\beta}=(R_{12}(v-u)\mathcal{T}_1(v)\mathcal{T}_2(u))^{\nu\nu'}_{\alpha\beta}
        \end{multline}
        \begin{equation}
            \boxed{R_{12}(v-u)\mathcal{T}_1(v)\mathcal{T}_2(u)=\mathcal{T}_2(u)\mathcal{T}_1(v)R_{12}(v-u)}
        \end{equation}
        \item[ii)] Матрица монодромии:
        \begin{equation}
            (\mathcal{T}^{\sigma'_1...\sigma'_n}_{\sigma_1...\sigma_n})^\nu_\mu=\sum\limits_{\mu_2=\pm1}...\sum\limits_{\mu_n=\pm1}R_{\mu\sigma_1}^{\mu_2\sigma'_1}R_{\mu_2\sigma_2}^{\mu_3\sigma'_2}...R_{\mu_n\sigma_n}^{\nu\sigma'_n}
        \end{equation}
        \begin{equation}
            \boxed{\mathcal{T}=R_{12}...R_{N-1N}}
        \end{equation}
    \end{itemize}
    \item Ассоциативность
    \begin{itemize}
        \item[i)] Рассмотрите формальную квадратичную алгебру $L$-операторов для шестивершинной модели
        \begin{equation}
            R_{12}(u_1-u_2)L_1(u_1)L_2(u_2)=L_2(u_2)L_1(u_1)R_{12}(u_1-u_2)
        \end{equation}
        Рассмотрите произведение 3 операторов
        \begin{equation}
            L_1(u_1)L_2(u_2)L_3(u_3)
        \end{equation}
        Используя соотношения в квадратичной алгебре, попробуйте переписать этот элемент через
        \begin{equation}
            L_3(u_3)L_2(u_2)L_1(u_1)
        \end{equation}
        Сделать это можно двумя способами. Коммутируя операторы 1 с 2, а потом с 3. Или коммутируя 2 и 3, а потом 1. Будет ли результат зависеть от порядка коммутации (является ли алгебра ассоциативной?)
        \item[ii)] Рассмотрите квадратичную алгебру вершинных операторов $\Phi_\pm$ квантовой аффинной алгебры вида
        \begin{equation}
            \Phi_{\sigma}(u_1)\Phi_\mu(u_2)=R^{\sigma'\mu'}_{\sigma\mu}(u_1-u_2)\Phi_{\mu'}(u_2)\Phi_{\sigma'}(u_1)
        \end{equation}
        где операторы $\Phi_\mu$ действуют в 'квантовом пространстве' и некоммутативны. Покажите ассоциативность этой алгебры.
    \end{itemize}
    \textbf{Решение.}
    \begin{itemize}
        \item[i)]
        \item[ii)]
        \begin{equation}
            \Phi_{\sigma}(u_1)\Phi_\mu(u_2)=R^{\sigma'\mu'}_{\sigma\mu}(u_1-u_2)\Phi_{\mu'}(u_2)\Phi_{\sigma'}(u_1)
        \end{equation}
        \begin{equation*}
            (\Phi_1\Phi_2)\Phi_3=R_{12}\Phi_2\Phi_1\Phi_3=R_{12}\Phi_2R_{13}\Phi_3\Phi_1=R_{12}R_{13}\Phi_2\Phi_3\Phi_1=R_{12}R_{13}R_{23}\Phi_3\Phi_2\Phi_1
        \end{equation*}
        \begin{equation*}
            \Phi_1(\Phi_2\Phi_3)=\Phi_1R_{23}\Phi_3\Phi_2=R_{13}R_{23}\Phi_3\Phi_1\Phi_2=R_{12}R_{13}R_{23}\Phi_3\Phi_2\Phi_1
        \end{equation*}
        Таким образом,
        \begin{equation}
            \boxed{(\Phi_1\Phi_2)\Phi_3=\Phi_1(\Phi_2\Phi_3)}
        \end{equation}
    \end{itemize}
    \item YBE and Braid group
    \begin{itemize}
        \item[i)] Введём модификацию больцмановских весов
        \begin{equation}
            \check{R}=PR,\quad \check{R}^{\sigma_3,\sigma_4}_{\sigma_1,\sigma_2}=R^{\sigma_4,\sigma_3}_{\sigma_1,\sigma_2}
        \end{equation}
        Проверьте, что тогда уравнения Янга-Бакстера записывается как
        \begin{equation}
            (1\otimes\check{R}(u))(\check{R}(u+v)\otimes1)(1\otimes\check{R}(v))=(\check{R}(v)\otimes1)(1\otimes\check{R}(u+v))(\check{R}(u)\otimes1)
        \end{equation}
        \item[ii)] Определим
        \begin{equation}
            \check{R}_j(u)=1\otimes...\otimes\check{R}(u)\otimes...\otimes1,
        \end{equation}
        где $\check{R}$ действует на $j$, $j+1$ компоненты. Проверьте, что тогда
        \begin{equation}
            \check{R}_j(u)\check{R}_{j+1}(u+v)\check{R}_j(v)=\check{R}_{j+1}(v)\check{R}_j(u+v)\check{R}_{j+1}(u)
        \end{equation}
        и
        \begin{equation}
            \check{R}_j(u)\check{R}_i(v)=\check{R}_i(v)\check{R}_j(u),\quad|i-j|\geq2
        \end{equation}
        Это очень похоже на определяющее соотношение группы кос (если игнорировать спектральные параметры).
        \item[iii)] Определим предел группы кос как
        \begin{equation}
            (R^B)^{\sigma_3\sigma_4}_{\sigma_1\sigma_2}=P\lim\limits_{u\rightarrow\infty}e^{-|u|}e^{-\frac{1}{2}u(\sigma_1-\sigma_4)}R^{\sigma_3\sigma_4}_{\sigma_1\sigma_2}
        \end{equation}
        Проверьте явным вычислением, что тогда
        \begin{equation}
            R^B=\begin{pmatrix}
                q^\frac{1}{2}&&&\\
                &0&q^{-\frac{1}{2}}&&\\
                &q^{-\frac{1}{2}} & (q-q^{-1})q^{-\frac{1}{2}}&\\
                &&&q^{\frac{1}{2}}
            \end{pmatrix},\quad q=e^\lambda
        \end{equation}
        Проверьте справедливость уравнения Янга-Бакстера. Тем самым проверьте, что представление группы кос можно задать в терминах постоянных $R$ матриц
        \begin{equation}
            \tau_j=1\otimes...\otimes1\otimes R^B_j\otimes1\otimes...\otimes1
        \end{equation}
    \end{itemize}
    \textbf{Решение.}
    \begin{itemize}
        \item[i)]
        \item[ii)] Уравнение Янга-Бакстера в этих обозначениях:
        \begin{equation}
            \check{R}_{1+1}(u)\check{R}_1(u+v)\check{R}_{1+1}(v)=\check{R}_1(u)\check{R}_{1+1}(u+v)\check{R}_1(v)
        \end{equation}
        Умножая слева нужное число раз на 1, получим
        \begin{equation}
            \boxed{\check{R}_j(u)\check{R}_{j+1}(u+v)\check{R}_j(v)=\check{R}_{j+1}(v)\check{R}_j(u+v)\check{R}_{j+1}(u)}
        \end{equation}
        Поскольку при $|i-j|\geq2$ тензорные произведения пространств не пересекаются, то
        \begin{equation}
            \boxed{\check{R}_j(u)\check{R}_i(v)=\check{R}_i(v)\check{R}_j(u)}
        \end{equation}
        \item[iii)]
        \begin{equation}
            R^{\sigma_3\sigma_4}_{\sigma_1\sigma_2}=
            \begin{pmatrix}
                \sinh(u+\lambda) & 0 & 0 & 0\\
                0 & \sinh u & \sinh\lambda & 0\\
                0 & \sinh\lambda & \sinh u & 0\\
                0 & 0 & 0 & \sinh(u+\lambda)
            \end{pmatrix}_{\sigma_1\sigma_2}^{\sigma_3\sigma_4}
        \end{equation}
        \begin{equation}
            \boxed{R^B=\begin{pmatrix}
                \frac{q}{2} & 0 & 0 & 0\\
                0 & 0 & \frac{1}{2} & 0\\
                0 & \frac{1}{2} & \frac{1}{2}(q-q^{-1}) & 0\\
                0 & 0 & 0 & \frac{q}{2}
            \end{pmatrix}=\frac{1}{2}q^{\frac{1}{2}}\begin{pmatrix}
                q^\frac{1}{2}&&&\\
                &0&q^{-\frac{1}{2}}&&\\
                &q^{-\frac{1}{2}} & (q-q^{-1})q^{-\frac{1}{2}}&\\
                &&&q^{\frac{1}{2}}
            \end{pmatrix}}
        \end{equation}
    \end{itemize}
    \item Для диагонализации трансфер матрицы нам будет полезно понимать, как элементы матрицы монодромии действуют на векторах простого типа. Вычислите явно действие элементов $A$, $B$, $C$ матрицы монодромии на векторе
    \begin{equation}
        \Psi_0=v^\dagger\otimes...\otimes v^\dagger,\quad v^+=\begin{pmatrix}
            1\\
            0
        \end{pmatrix}
    \end{equation}
    \textbf{Решение.}\\
    \begin{equation}
        (\Psi_0)^{\sigma_1,...,\sigma_n}=\delta^{\sigma_1}_+...\delta^{\sigma_n}_+
    \end{equation}
    \begin{equation}
        A=\mathcal{T}^+_+,\quad A^{\sigma'_1...\sigma'_n}_{\sigma_1...\sigma_n}=\sum\limits_{\mu_2=\pm1}...\sum\limits_{\mu_n=\pm1}R^{\mu_2\sigma'_1}_{+\sigma_1}R^{\mu_3\sigma'_2}_{\mu_2\sigma_2}...R^{+\sigma'_n}_{\mu_n\sigma_n}
    \end{equation}
    \begin{equation}
        (A\Psi_0)^{\sigma'_1,...,\sigma'_n}=\sum\limits_{\mu_2=\pm1}...\sum\limits_{\mu_n=\pm1}R^{\mu_2\sigma'_1}_{++}R^{\mu_3\sigma'_2}_{\mu_2+}...R^{+\sigma'_n}_{\mu_n+}
    \end{equation}
    $R^{\mu_2\sigma'_1}_{++}\neq0$ при $\mu_2=\sigma'_1=+$, $R^{\mu_3\sigma'_2}_{++}\neq0$ при $\mu_3=\sigma'_2=+$ и т.д.
    \begin{equation}
        (A\Psi_0)^{\sigma'_1,...,\sigma'_n}=R^{++}_{++}...R^{++}_{++}\delta^{\sigma'_1}_+...\delta^{\sigma'_n}_+
    \end{equation}
    \begin{equation}
        \boxed{A(u)\Psi_0=a(u)^n\Psi_0}
    \end{equation}
    \begin{equation}
        B=\mathcal{T}^-_+,\quad B^{\sigma'_1...\sigma'_n}_{\sigma_1...\sigma_n}=\sum\limits_{\mu_2=\pm1}...\sum\limits_{\mu_n=\pm1}R^{\mu_2\sigma'_1}_{+\sigma_1}R^{\mu_3\sigma'_2}_{\mu_2\sigma_2}...R^{-\sigma'_n}_{\mu_n\sigma_n}
    \end{equation}
    \begin{equation}
        (B\Psi_0)^{\sigma'_1,...,\sigma'_n}=\sum\limits_{\mu_2=\pm1}...\sum\limits_{\mu_n=\pm1}R^{\mu_2\sigma'_1}_{++}R^{\mu_3\sigma'_2}_{\mu_2+}...R^{-\sigma'_n}_{\mu_n+}
    \end{equation}
    $R^{\mu_2\sigma'_1}_{++}\neq0$ при $\mu_2=\sigma'_1=+$, $R^{\mu_3\sigma'_2}_{++}\neq0$ при $\mu_3=\sigma'_2=+$ и т.д.
    \begin{equation}
        (B\Psi_0)^{\sigma'_1,...,\sigma'_n}=R^{++}_{++}...R^{-\sigma'_n}_{++}\delta^{\sigma'_1}_+...\delta^{\sigma'_{n-1}}_+
    \end{equation}
    $\forall\sigma'_n$ выполняется $R^{-\sigma'_n}_{++}=0$, поэтому
    \begin{equation}
        \boxed{B(u)\Psi_0=0}
    \end{equation}
    \begin{equation}
        C=\mathcal{T}^+_-,\quad C^{\sigma'_1...\sigma'_n}_{\sigma_1...\sigma_n}=\sum\limits_{\mu_2=\pm1}...\sum\limits_{\mu_n=\pm1}R^{\mu_2\sigma'_1}_{-\sigma_1}R^{\mu_3\sigma'_2}_{\mu_2\sigma_2}...R^{+\sigma'_n}_{\mu_n\sigma_n}
    \end{equation}
    \begin{multline}
        (C\Psi_0)^{\sigma'_1,...,\sigma'_n}=\sum\limits_{\mu_2=\pm1}...\sum\limits_{\mu_n=\pm1}R^{\mu_2\sigma'_1}_{-+}R^{\mu_3\sigma'_2}_{\mu_2+}...R^{+\sigma'_n}_{\mu_n+}=\\=\sum\limits_{\mu_2=\pm1}...\sum\limits_{\mu_n=\pm1}(b\delta^{\mu_2}_-\delta^{\sigma'_1}_++c\delta^{\mu_2}_+\delta^{\sigma'_1}_-)R^{\mu_3\sigma'_2}_{\mu_2+}...R^{+\sigma'_n}_{\mu_n+}=ca^{n-1}\begin{pmatrix}
            0\\
            1
        \end{pmatrix}\otimes\begin{pmatrix}
            1\\
            0
        \end{pmatrix}\otimes...\otimes\begin{pmatrix}
            1\\
            0
        \end{pmatrix}+\\+b\delta^{\sigma'_1}_+\sum\limits_{\mu_3=\pm1}...\sum\limits_{\mu_n=\pm1}(b\delta^{\mu_3}_-\delta^{\sigma'_2}_++c\delta^{\mu_3}_+\delta^{\sigma'_2}_-)R^{\mu_3\sigma'_2}_{\mu_2+}...R^{+\sigma'_n}_{\mu_n+}
    \end{multline}
    \begin{equation}
        \boxed{C(u)\Psi_0=c\sum\limits_{k=1}^n\left(b^{k-1}a^{n-k}\prod\limits_{j=1}^n\left((1-\delta_{kj})\begin{pmatrix}
            1\\0
        \end{pmatrix}+\delta_{kj}\begin{pmatrix}
            0\\1
        \end{pmatrix}\right)\right)}
    \end{equation}
    \end{enumerate}
\section{Квантовые группы}
\begin{enumerate}
    \item Проверьте формулу для коумножения. Пусть операторы $L^x$ -- $L$ в представлении $\mathcal{F}_x$:
    \begin{equation}
        L^x:F\otimes\mathcal{F}_x\rightarrow F\otimes\mathcal{F}_x
    \end{equation}
    а операторы $L^y$ -- $L$ в представлении $\mathcal{F}_y$:
    \begin{equation}
        L^y:F\otimes\mathcal{F}_y\rightarrow F\otimes\mathcal{F}_y
    \end{equation}
    Будучи представлениями 1 алгебры, они удовлетворяют соотношениям
    \begin{equation}
        R_{12}(v-u)L^x_1(v)L^x_2(u)=L^x_2(u)L^x_1(v)R_{12}(v-u)
    \end{equation}
    \begin{equation}
        R_{12}(v-u)L^y_1(v)L^y_2(u)=L^y_2(u)L^y_1(v)R_{12}(v-u)
    \end{equation}
    Введите оператор
    \begin{equation}
        L^{xy}=L^xL^y:F\otimes\mathcal{F}_x\otimes\mathcal{F}_y\rightarrow F\otimes\mathcal{F}_x\otimes\mathcal{F}_y
    \end{equation}
    и проверьте, что это представление $RLL$ алгебры на тензорном произведении $\mathcal{F}_x\otimes\mathcal{F}_y$.\\
    \textbf{Решение.}\\
    Т.к. $L^x$ и $L^y$ действуют на разных множествах, то $[L^x,L^y]=0$. На $\mathcal{F}$ они действуют одинаково.
    \begin{equation}
        R_{12}L_1^{xy}L_2^{xy}=R_{12}L_1^xL_1^yL_2^xL_2^y
    \end{equation}
    \begin{equation}
        L^y_1L^x_2=L^x_2L^y_1\rightarrow R_{12}L_1^{xy}L_2^{xy}=R_{12}L_1^xL_2^xL_1^yL_2^y
    \end{equation}
    Из уравнения Янга-Бакстера:
    \begin{equation}
        R_{12}L^x_1L^x_2=L^x_2L^x_1R_{12}\rightarrow R_{12}L_1^{xy}L_2^{xy}=L^x_2L^x_1R_{12}L_1^yL_2^y
    \end{equation}
    Из уравнения Янга-Бакстера:
    \begin{equation}
        R_{12}L^y_1L^y_2=L^y_2L^y_1R_{12}\rightarrow R_{12}L_1^{xy}L_2^{xy}=L^x_2L^x_1L^y_2L^y_1R_{12}
    \end{equation}
    \begin{equation}
        L^x_1L^y_2=L^y_2L^x_1\rightarrow R_{12}L_1^{xy}L_2^{xy}=L^x_2L^y_2L^x_1L^y_1R_{12}=L^{xy}_2L^{xy}_1R_{12}
    \end{equation}
    \begin{equation}
        \boxed{R_{12}L_1^{xy}L_2^{xy}=L^{xy}_2L^{xy}_1R_{12}}
    \end{equation}
    \item Проверьте, что $q$-детерминант алгебры $RTT=TTR$ для $GL_q(2,C)$ принадлежит центру. Выпишите явно действие антипода на генераторы $\alpha,...,\delta$ (необходимо найти обратную матрицу). Проверьте аксиомы для антипода для $SL_q(2,C)$.\\
    \textbf{Решение.}
    \begin{equation}
        \text{Det}_qL=\alpha\delta-q^{-1}\beta\gamma
    \end{equation}
    $Det_qL$ лежит в центре, поскольку $RTT=TTR$ -- алгебра для $GL_q(2,C)$, если для любого элемента алгебры выполняется
    \begin{equation}
        L\text{Det}_qL=\text{Det}_qLL
    \end{equation}
    \begin{equation}
        \begin{pmatrix}
            \alpha & \beta\\
            \gamma & \delta
        \end{pmatrix}\text{Det}_qL=\text{Det}_qL\begin{pmatrix}
            \alpha & \beta\\
            \gamma & \delta
        \end{pmatrix}
    \end{equation}
    Проверим равенство покомпонентно:
    \begin{multline}
        \text{Det}_qL\alpha=\alpha\delta\alpha-q^{-1}\beta\gamma\alpha=\alpha(\alpha\delta-(q^{-1}-q)\beta\gamma)-q^{-1}\beta\gamma\alpha=\\=\alpha(\alpha\delta-q^{-1}\beta\gamma)+q\alpha\beta\gamma-q^{-1}q^2\alpha\beta\gamma=\alpha\text{Det}_qL
    \end{multline}
    \begin{equation}
        \text{Det}_qL\beta=\alpha\delta\beta-q^{-1}\beta\gamma\beta=qq^{-1}\beta\alpha\delta-q^{-1}\beta\beta\gamma=\beta\text{Det}_qL
    \end{equation}
    \begin{equation}
        \text{Det}_qL\gamma=\alpha\delta\gamma-q^{-1}\beta\gamma\gamma=qq^{-1}\alpha\delta-q^{-1}\gamma\beta\gamma=\gamma\text{Det}_qL
    \end{equation}
    \begin{multline}
        \text{Det}_qL\delta=\alpha\delta\delta-q^{-1}\beta\gamma\delta=\delta\alpha\delta+(q^{-1}-q)\beta\gamma\delta-q^{-1}\beta\gamma\delta=\\=\delta\alpha\delta-qq^{-1}q^{-1}\delta\beta\gamma=\delta\text{Det}_qL
    \end{multline}
    \begin{equation}
        S(L_\mu^\nu)=(L^{-1})_\mu^\nu\rightarrow S\begin{pmatrix}
            \alpha & \beta\\
            \gamma & \delta
        \end{pmatrix}=\frac{1}{\alpha\delta-\beta\gamma}\begin{pmatrix}
            \delta & -\beta\\
            -\gamma & \alpha
        \end{pmatrix}
    \end{equation}
    Переопределим $\text{Det}_q(L)$, чтобы
    \begin{equation}
        S\begin{pmatrix}
            \alpha & \beta\\
            \gamma & \delta
        \end{pmatrix}=\begin{pmatrix}
            \delta & -\beta\\
            -\gamma & \alpha
        \end{pmatrix}
    \end{equation}
    \begin{equation}
        S(L_1L_2)=(L_1L_2)^{-1}=L^{-1}_2L^{-1}_1=S(L_2)S(L_1)
    \end{equation}
    \begin{equation}
        m(S\otimes I)\Delta L=m(S\otimes I)(L_\mu^{\nu'}L_{\nu'}^\nu)=(S(L_\mu^{\nu'}))I=\delta_\mu^\nu I
    \end{equation}
    \item
    \item Проверьте явно, что аксиомы алгебры Хопфа выполняются для универсальной обёртывающей алгебры $U_q(sl_2)$.\\
    \textbf{Решение.}\\
    Проверим аксиомы алгебры Хопфа:
    \begin{enumerate}
        \item \begin{equation}
            m(A\otimes I)\Delta(x)=\varepsilon(x)I
        \end{equation}
        \begin{multline*}
            m(A\otimes I)\Delta(s^\pm)=m(A\otimes I)(s^\pm\otimes q^{s^z}+q^{-s^z}\otimes s^\pm)=m(-q^{\pm1}s^\pm\otimes q^{s^z}+q^{s^z}\otimes s^{\pm})
        \end{multline*}
        \begin{equation*}
            m(A\otimes I)\Delta(s^+)=-qs^+q^{s^z}+q^{s^z}s^+=-q\begin{pmatrix}
                    0 & 1\\
                    0 & 0
                \end{pmatrix}\begin{pmatrix}
                    q^\frac{1}{2} & 0\\
                    0 & q^{-\frac{1}{2}}
                \end{pmatrix}+\begin{pmatrix}
                    q^\frac{1}{2} & 0\\
                    0 & q^{-\frac{1}{2}}
                \end{pmatrix}\begin{pmatrix}
                    0 & 1\\
                    0 & 0
                \end{pmatrix}=0
        \end{equation*}
        \begin{equation*}
            m(A\otimes I)\Delta(s^-)=-q^{-1}s^-q^{s^z}+q^{s^z}s^-=-q^{-1}\begin{pmatrix}
                    0 & 0\\
                    1 & 0
                \end{pmatrix}\begin{pmatrix}
                    q^\frac{1}{2} & 0\\
                    0 & q^{-\frac{1}{2}}
                \end{pmatrix}+\begin{pmatrix}
                    q^{\frac{1}{2}} & 0\\
                    0 & q^{-\frac{1}{2}}
                \end{pmatrix}\begin{pmatrix}
                    0 & 0\\
                    1 & 0
                \end{pmatrix}=0
        \end{equation*}
        \begin{equation}
            \boxed{m(A\otimes I)\Delta(s^\pm)=0}
        \end{equation}
        \begin{equation}
            \boxed{m(A\otimes I)\Delta(s^z)=m(A\otimes I)(q^{s^z}\otimes q^{s^z})=1}
        \end{equation}
        \item
        \begin{equation}
            m(I\otimes\varepsilon)\Delta(x)=m(\varepsilon\otimes I)\Delta(x)=x
        \end{equation}
        \begin{equation}
            \boxed{m(I\otimes\varepsilon)\Delta(s^\pm)=m(s^\pm\otimes1)=s^\pm}
        \end{equation}
        \begin{equation}
            \boxed{m(I\otimes\varepsilon)\Delta(q^{s^z})=m(q^{s^z}\otimes1)=q^{s^z}}
        \end{equation}
        \item Коассоциативность
        \begin{equation}
            \Delta(q^{s^z})=q^{s^z}\otimes q^{s^z}
        \end{equation}
        \begin{equation}
            \boxed{(I\otimes\Delta)(q^{s^z}\otimes q^{s^z})=(\Delta\otimes I)(q^{s^z}\otimes q^{s^z})=q^{s^z}\otimes q^{s^z}\otimes q^{s^z}}
        \end{equation}
        Коассоциативность для $s^\pm$ была проверена на занятии.
        \item Антипод
        \begin{equation}
            A(s^\pm q^{s^z})=A(s^\pm)A(q^{s^z})
        \end{equation}
        \begin{multline}
            m(I\otimes A)\Delta(s^\pm q^{s^z})=m(I\otimes A)(s^\pm q^{s^z}\otimes q^{s^z}q^{s^z}+1\otimes s^\pm q^{s^z})=\\=m(s^\pm q^{s^z}\otimes q^{-s^z}q^{-s^z}+1\otimes s^\pm q^{s^z})=s^\pm q^{-s^z}+s^\pm q^{s^z}=0
        \end{multline}
        \begin{equation}
            \boxed{m(I\otimes A)\Delta(s^\pm q^{s^z})=0}
        \end{equation}
    \end{enumerate}
    \item Явно выпишите противоположное коумножение $\tilde\Delta(g)=P\Delta(g)$ для $U_q(sl_2)$. (Универсальная) $R$ матрица определяется как сплетающий оператор
    \begin{equation}
        R\Delta(g)=\tilde\Delta(g)R
    \end{equation}
    Проверьте в случае спина $1/2$ сплетающее свойство.\\
    \textbf{Решение.}\\
    \begin{equation}
        q^{s^z}=\begin{pmatrix}
            q & 0\\
            0 & q^{-1}
        \end{pmatrix}
    \end{equation}
    \begin{equation}
        \Delta(q^{s^z})=q^{s^z}\otimes q^{s^z}=\begin{pmatrix}
            q^2 & 0 & 0 & 0\\
            0 & 1 & 0 & 0\\
            0 & 0 & 1 & 0\\
            0 & 0 & 0 & q^{-2}
        \end{pmatrix}
    \end{equation}
    \begin{equation}
        \Delta(s^\pm)=s^\pm\otimes q^{s^z}+q^{-s^z}\otimes s^\pm
    \end{equation}
    \begin{equation}
        \Delta(s^+)=\begin{pmatrix}
            0 & 0 & q & 0\\
            0 & 0 & 0 & q^{-1}\\
            0 & 0 & 0 & 0\\
            0 & 0 & 0 & 0
        \end{pmatrix}+\begin{pmatrix}
            0 & q^{-1} & 0 & 0\\
            0 & 0 & 0 & 0\\
            0 & 0 & 0 & q\\
            0 & 0 & 0 & 0
        \end{pmatrix}=\begin{pmatrix}
            0 & q^{-1} & q & 0\\
            0 & 0 & 0 & q^{-1}\\
            0 & 0 & 0 & q\\
            0 & 0 & 0 & 0
        \end{pmatrix}
    \end{equation}
    Аналогично,
    \begin{equation}
        \Delta(s^-)=\begin{pmatrix}
            0 & 0 & q & 0\\
            q^{-1} & 0 & 0 & 0\\
            q & 0 & 0 & 0\\
            0 & q^{-1} & q & 0
        \end{pmatrix}
    \end{equation}
    \begin{equation}
        \tilde\Delta(q^{s^z})=P\Delta(q^{s^z})=P\begin{pmatrix}
            q^2 & 0 & 0 & 0\\
            0 & 1 & 0 & 0\\
            0 & 0 & 1 & 0\\
            0 & 0 & 0 & q^{-2}
        \end{pmatrix}=\begin{pmatrix}
            q^2 & 0 & 0 & 0\\
            0 & 0 & 1 & 0\\
            0 & 1 & 0 & 0\\
            0 & 0 & 0 & q^{-2}
        \end{pmatrix}
    \end{equation}
    \begin{equation}
        \tilde\Delta(s^+)=P\Delta(s^+)=P\begin{pmatrix}
            0 & q^{-1} & q & 0\\
            0 & 0 & 0 & q^{-1}\\
            0 & 0 & 0 & q\\
            0 & 0 & 0 & 0
        \end{pmatrix}=\begin{pmatrix}
            0 & q^{-1} & q & 0\\
            0 & 0 & 0 & q\\
            0 & 0 & 0 & q^{-1}\\
            0 & 0 & 0 & 0
        \end{pmatrix}
    \end{equation}
    \begin{equation}
        \tilde\Delta(s^-)=P\Delta(s^-)=P\begin{pmatrix}
            0 & 0 & 0 & 0\\
            q^{-1} & 0 & 0 & 0\\
            q & 0 & 0 & 0\\
            0 & q^{-1} & q & 0
        \end{pmatrix}=\begin{pmatrix}
            0 & 0 & 0 & 0\\
            q & 0 & 0 & 0\\
            q^{-1} & 0 & 0 & 0\\
            0 & q^{-1} & q & 0
        \end{pmatrix}
    \end{equation}
    Проверим в случае спина $1/2$ сплетающее свойство.
    \begin{equation}
        R=q^{-\frac{1}{2}}\begin{pmatrix}
            q & 0 & 0 & 0\\
            0 & 1 & q-q^{-1} & 0\\
            0 & 0 & 1 & 0\\
            0 & 0 & 0 & q
        \end{pmatrix}
    \end{equation}
    \begin{equation}
        R\Delta(q^{s^z})=R\begin{pmatrix}
            q^2 & 0 & 0 & 0\\
            0 & 1 & 0 & 0\\
            0 & 0 & 1 & 0\\
            0 & 0 & 0 & q^{-2}
        \end{pmatrix}=q^{-\frac{1}{2}}\begin{pmatrix}
            q^3 & 0 & 0 & 0\\
            0 & 1 & q-q^{-1} & 0\\
            0 & 0 & 1 & 0\\
            0 & 0 & 0 & q^{-1}
        \end{pmatrix}
    \end{equation}
    \begin{equation}
        \tilde\Delta(q^{s^z})R=\begin{pmatrix}
            q^2 & 0 & 0 & 0\\
            0 & 0 & 1 & 0\\
            0 & 1 & 0 & 0\\
            0 & 0 & 0 & q^{-2}
        \end{pmatrix}R=q^{-\frac{1}{2}}\begin{pmatrix}
            q^3 & 0 & 0 & 0\\
            0 & 0 & 1 & 0\\
            0 & 1 & q-q^{-1} & 0\\
            0 & 0 & 0 & q^{-1}
        \end{pmatrix}
    \end{equation}
    \begin{equation}
        R\Delta(s^+)=R\begin{pmatrix}
            0 & q^{-1} & q & 0\\
            0 & 0 & 0 & q^{-1}\\
            0 & 0 & 0 & q\\
            0 & 0 & 0 & 0
        \end{pmatrix}=q^{-\frac{1}{2}}\begin{pmatrix}
            0 & 1 & q^2 & 0\\
            0 & 0 & 0 & q^{-1}+q^2-1\\
            0 & 0 & 0 & q\\
            0 & 0 & 0 & 0
        \end{pmatrix}
    \end{equation}
    \begin{equation}
        \tilde\Delta(s^+)R=\begin{pmatrix}
            0 & q^{-1} & q & 0\\
            0 & 0 & 0 & q\\
            0 & 0 & 0 & q^{-1}\\
            0 & 0 & 0 & 0
        \end{pmatrix}R=q^{-\frac{1}{2}}\begin{pmatrix}
            0 & q^{-1} & q+1-q^{-2} & 0\\
            0 & 0 & 0 & q^2\\
            0 & 0 & 0 & 1\\
            0 & 0 & 0 & 0
        \end{pmatrix}
    \end{equation}
    \begin{equation}
        R\Delta(s^-)=R\begin{pmatrix}
            0 & 0 & 0 & 0\\
            q^{-1} & 0 & 0 & 0\\
            q & 0 & 0 & 0\\
            0 & q^{-1} & q & 0
        \end{pmatrix}=q^{-\frac{1}{2}}\begin{pmatrix}
            0 & 0 & 0 & 0\\
            q^{-1}+q^2-1 & 0 & 0 & 0\\
            q & 0 & 0 & 0\\
            0 & 1 & q^2 & 0
        \end{pmatrix}
    \end{equation}
    \begin{equation}
        \tilde\Delta(s^-)R=\begin{pmatrix}
            0 & 0 & 0 & 0\\
            q & 0 & 0 & 0\\
            q^{-1} & 0 & 0 & 0\\
            0 & q^{-1} & q & 0
        \end{pmatrix}R=q^{-\frac{1}{2}}\begin{pmatrix}
            0 & 0 & 0 & 0\\
            q^2 & 0 & 0 & 0\\
            1 & 0 & 0 & 0\\
            0 & q^{-1} & q+1-q^{-1} & 0
        \end{pmatrix}
    \end{equation}
    \item
    \begin{itemize}
        \item[i)] Для построения представлений в теории $sl_2$ важную роль играет квадратичный оператор Казимира. Этот оператор коммутирует со всеми элементами алгебры. Найдите квантовую деформацию оператора Казимира для $U_q(sl_2)$. (Центр алгебры).
        \item[ii)] Опишите одномерное, двумерное и трёхмерное представления $U_q(sl_2)$.
    \end{itemize}
    \textbf{Решение.}
    \begin{itemize}
        \item[i)] Рассмотрим $sl_2$.
        \begin{equation}
            [H,E]=E,\quad [H,F]=-F,\quad [E,F]=2H
        \end{equation}
        Оператор Казимира:
        \begin{equation}
            C=\frac{1}{4}(EF+FE)+\frac{H^2}{2}
        \end{equation}
        Рассмотрим $U_q(sl_2)$.
        \begin{equation}
            [H,E]=E,\quad [H,F]=-F,\quad [E,F]=\frac{q^{2H}-q^{-2H}}{q-q^{-1}}
        \end{equation}
        Т.к. коммутаторы $[H,E]$ и $[H,F]$ не изменились, а коммутатор $[C_q,H]$ зависит только от $E$ и $F$, то $C_q=EF+FE+\varphi(H)$. Будем искать $\varphi(H)$ в виде
        \begin{equation}
            \varphi(H)=\sum\limits_{n=0}^\infty\phi_nH^n
        \end{equation}
        \begin{equation}
            [E,C_q]=0
        \end{equation}
        \begin{equation}
            EEF+EFE+E\varphi(H)=EFE+FEE+\varphi(H)E
        \end{equation}
        \begin{equation}
            EEF-FEE=[\varphi(H),E]\rightarrow EEF-(EF-[E,F])E=[\varphi(H),E]
        \end{equation}
        \begin{equation}
            E[E,F]+[E,F]E=[\varphi(H),E]
        \end{equation}
        \begin{equation}
            EH^n=(HE-E)H^{n-1}=(H-1)EH^{n-1}=(H-1)^nE\rightarrow E\varphi(H)=\varphi(H-1)E
        \end{equation}
        \begin{equation}
            [\varphi(H),E]=(\varphi(H)-\varphi(H-1))E    
        \end{equation}
        \begin{equation}
            [E,F]=\frac{q^{2H}-q^{-2H}}{q-q^{-1}}
        \end{equation}
        \begin{equation}
            Eq^{2H}=Ee^{2H\log q}=E\sum\limits_{n=0}^\infty\log^nq2^n\frac{H^n}{n!}=\sum\limits_{n=0}^\infty\log^nq2^n\frac{(H-1)^n}{n!}E=q^{2H-1}E
        \end{equation}
        \begin{equation}
            E[E,F]+[E,F]E=\left(\frac{q^{2(H-1)}-q^{-2(H-1)}}{q-q^{-1}}+\frac{q^{2H}-q^{-2H}}{q-q^{-1}}\right)E
        \end{equation}
        \begin{equation}
            \varphi(H)-\varphi(H-1)=\frac{q^{2(H-1)}-q^{-2(H-1)}}{q-q^{-1}}+\frac{q^{2H}-q^{-2H}}{q-q^{-1}}
        \end{equation}
        Ищем решение в виде $\varphi(H)=Ae^{2H}+Be^{-2H}$.
        \begin{equation}
            A=B=\frac{q^2+1}{q^2-1}\left(q-\frac{1}{q}\right)^{-1}
        \end{equation}
        \begin{equation}
            \boxed{C_q=EF+FE+\frac{q^2+1}{q^2-1}\left(q-\frac{1}{q}\right)^{-1}(e^{2H}+e^{-2H}),\quad [C_q,F]=0}
        \end{equation}
    \end{itemize}
\end{enumerate}
\section{Алгебраический анзатц Бете}
\begin{enumerate}
    \item $RTT=TTR$ алгебра.
    \begin{itemize}
        \item[i)] Выведите уравнения
        \begin{equation}
            [B(u),B(v)]=0,\quad [C(u),C(v)]=0,\quad [D(u),D(v)]=0
        \end{equation}
        \item[ii)] Выведите явно соотношения
        \begin{equation}
            [T(u),T(v)]=0
        \end{equation}
        \begin{equation}
            a(v-u)B(v)A(u)=A(u)B(v)b(v-u)+B(u)A(v)c(v-u)
        \end{equation}
        \begin{equation}
            a(u-v)B(v)D(u)=D(u)B(v)b(u-v)+B(u)D(v)c(u-v)
        \end{equation}
    \end{itemize}
    \textbf{Решение.}
    \begin{itemize}
        \item[i)] $RTT=TTR$.
        \begin{equation}
            T_\mu^\nu=\begin{pmatrix}
                A & B\\
                C & D
            \end{pmatrix}=\begin{pmatrix}
                T^+_+ & T^-_+\\
                T^+_- & T^-_-
            \end{pmatrix},\quad R^{\sigma_3\sigma_4}_{\sigma_1\sigma_2}(u)=\begin{pmatrix}
                a(u) & 0 & 0 & 0\\
                0 & b(u) & c(u) & 0\\
                0 & c(u) & b(u) & 0\\
                0 & 0 & 0 & a(u)
            \end{pmatrix}
        \end{equation}
        \begin{equation}
            R^{\mu\mu'}_{\alpha\beta}(v-u)T^{\nu'}_{\mu'}(v)T^\nu_\mu(u)=R^{\nu\nu'}_{\mu\mu'}(v-u)T^\mu_\alpha(v)T^{\mu'}_\beta(v)        
        \end{equation}
        Пусть $\alpha=\beta=+$, $\nu=\nu'=-$, тогда
        \begin{equation}
            R^{--}_{\mu\mu'}(v-u)T^\mu_+(u)T^{\mu'}_+(v)=a(v-u)B(u)B(v)
        \end{equation}
        \begin{equation}
            R^{\mu\mu'}_{++}(v-u)T^-_{\mu'}(v)T^-_\mu(u)=a(v-u)B(v)B(u)
        \end{equation}
        \begin{equation}
            \boxed{[B(u),B(v)]=0}
        \end{equation}
        Пусть $\alpha=\beta=-$, $\nu=\nu'=+$, тогда
        \begin{equation}
            R^{++}_{\mu\mu'}(v-u)T^\mu_-(u)T^{\mu'}_-(v)=a(v-u)C(u)C(v)
        \end{equation}
        \begin{equation}
            R^{\mu\mu'}_{--}(v-u)T^+_{\mu'}(v)T^+_\mu(u)=a(v-u)C(v)C(u)
        \end{equation}
        \begin{equation}
            \boxed{[C(u),C(v)]=0}
        \end{equation}
         Пусть $\alpha=\beta=-$, $\nu=\nu'=-$, тогда
        \begin{equation}
            R^{--}_{\mu\mu'}(v-u)T^\mu_-(u)T^{\mu'}_-(v)=a(v-u)D(u)D(v)
        \end{equation}
        \begin{equation}
            R^{\mu\mu'}_{++}(v-u)T^-_{\mu'}(v)T^-_\mu(u)=a(v-u)D(v)D(u)
        \end{equation}
        \begin{equation}
            \boxed{[D(u),D(v)]=0}
        \end{equation}
        \item[ii)]
        \begin{equation}
            T(u)=A(u)+D(u),\quad T(v)=A(v)+D(v)
        \end{equation}
        \begin{equation}
            [T(u),T(v)]=[A(u)+D(u),A(v)+D(v)]=[A(u),D(v)]+[D(u),A(v)]
        \end{equation}
        Пусть $\alpha=\nu=+$, $\beta=\nu'=-$, тогда
        \begin{equation}
            R^{\mu\mu'}_{+-}(v-u)T^-_{\mu'}(v)T^+_\mu(u)=c(v-u)D(v)A(u)+c(v-u)B(v)C(u)
        \end{equation}
        \begin{equation}
            R^{+-}_{\mu\mu'}(v-u)T^\mu_+(u)T^{\mu'}_-(v)=b(v-u)A(u)D(v)+c(v-u)B(u)C(v)
        \end{equation}
        \begin{equation}
            b(v-u)[D(v),A(u)]=c(v-u)(B(u)C(v)-B(v)C(u))
        \end{equation}
        Пусть $\alpha=\nu=-$, $\beta=\nu'=+$, тогда
        \begin{equation}
            R^{\mu\mu'}_{-+}(v-u)T^+_{\mu'}(v)T^-_\mu(u)=b(v-u)A(v)D(u)+c(v-u)C(v)B(u)
        \end{equation}
        \begin{equation}
            R^{-+}_{\mu\mu'}(v-u)T^\mu_-(u)T^{\mu'}_+(v)=b(v-u)D(u)A(v)+c(v-u)C(u)B(v)
        \end{equation}
        \begin{equation}
            b(v-u)[A(v),D(u)]=c(v-u)(C(u)B(v)-C(v)B(u))
        \end{equation}
        \begin{equation*}
            [T(u),T(v)]=-\frac{c(v-u)}{b(v-u)}(B(u)C(v)-B(v)C(u))-\frac{c(v-u)}{b(v-u)}(C(u)B(v)-C(v)B(u))
        \end{equation*}
        \begin{equation}
            \boxed{[T(u),T(v)]=0}
        \end{equation}
        Пусть $\alpha=\nu=\beta=+$, $\nu'=-$, тогда
        \begin{equation}
            R^{\mu\mu'}_{++}(v-u)T^-_{\mu'}(v)T^+_\mu(u)=a(v-u)B(v)A(u)
        \end{equation}
        \begin{equation}
            R^{+-}_{\mu\mu'}(v-u)T^\mu_+(u)T^{\mu'}_+(v)=b(v-u)A(u)B(v)+C(v-u)B(u)A(v)
        \end{equation}
        \begin{equation}
            \boxed{a(v-u)B(v)A(u)=b(v-u)A(u)B(v)+C(v-u)B(u)A(v)}
        \end{equation}
        Пусть $\beta=\nu=\nu'=-$, $\alpha=+$, тогда
        \begin{equation}
            R^{--}_{\mu\mu'}(v-u)T^\mu_+(u)T^{\mu'}_-(u)=a(v-u)B(u)D(v)
        \end{equation}
        \begin{equation}
            R^{\mu\mu'}_{+-}(v-u)T^-_{\mu'}(v)T^-_\mu(v)=b(v-u)D(v)B(u)+C(v-u)B(v)D(u)
        \end{equation}
        Меняя $u$ и $v$ местами, получим
        \begin{equation}
            \boxed{a(u-v)B(v)D(u)=b(u-v)D(u)B(v)+C(u-v)B(u)D(v)}
        \end{equation}
    \end{itemize}
    \item
    \item Одно 'частичное' возбуждение в алгебраическом анзаце Бете очень простое, оно было разобрано на лекции. Повторите вывод ещё раз. Используя $RLL=LLR$ соотношения
    \begin{equation}
        [A(u),A(v)]=0,\quad [D(u),D(v)]=0,\quad [B(u),B(v)]=0
    \end{equation}
    \begin{equation}\label{eq2}
        a(v-u)B(v)A(u)=A(u)B(v)b(v-u)+B(u)A(v)c(v-u)
    \end{equation}
    \begin{equation}\label{eq3}
        a(u-v)B(v)D(u)=D(u)B(v)b(u-v)+B(u)D(v)c(u-v)
    \end{equation}
    а также
    \begin{equation}
        T(u)\Psi_0=(A(u)+D(u))\Psi_0=(a^n(u)+b^n(u))\Psi_0
    \end{equation}
    Рассмотрите вектор $\Phi=B(v_1)\Psi_0$. Подействуйте на него трансфер матрицей и найдите условие диагонализации. Сколько имеется решений уравнения для данного случая? Попробуйте решить уравнение (если трудно, то хотя бы в пределе $\Delta=1$ из упр. 6).\\
    \textbf{Решение.}
    \begin{equation}
        \Psi_0=\begin{pmatrix}
            1\\0
        \end{pmatrix}\otimes...\otimes\begin{pmatrix}
            1\\0
        \end{pmatrix}
    \end{equation}
    Подействуем на вектор $\Phi=B(v_1)\Psi_0$ трансфер-матрицей:
    \begin{equation}
        T(u)\Phi=T(u)B(v_1)\Psi_0=(A(u)+D(u))B(v_1)\Psi_0
    \end{equation}
    Из (\ref{eq2}) и (\ref{eq3}):
    \begin{equation}
        A(u)B(v_1)=\frac{a(v_1-u)}{b(v_1-u)}B(v_1)A(u)-\frac{c(v_1-u)}{b(v_1-u)}B(u)A(v_1)
    \end{equation}
    \begin{equation}
        D(u)B(v_1)=\frac{a(u-v_1)}{b(u-v_1)}B(v_1)D(u)-\frac{c(u-v_1)}{b(u-v_1)}B(u)D(v_1)
    \end{equation}
    \begin{multline}
        T(u)\Phi=\left(\frac{a(v_1-u)}{b(v_1-u)}B(v_1)A(u)+\frac{a(u-v_1)}{b(u-v_1)}B(v_1)D(u)\right)\Psi_0-\\-\left(\frac{c(v_1-u)}{b(v_1-u)}B(u)A(v_1)+\frac{c(u-v_1)}{b(u-v_1)}B(u)D(v_1)\right)\Psi_0=\\=\left(\frac{a(v_1-u)}{b(v_1-u)}a^n(u)+\frac{a(u-v_1)}{b(u-v_1)}b^n(u)\right)\Phi-\\-\left(\frac{c(v_1-u)}{b(v_1-u)}a^n(v_1)+\frac{c(u-v_1)}{b(u-v_1)}b^n(v_1)\right)B(u)\Psi_0
    \end{multline}
    Условие диагонализации:
    \begin{equation}
        \frac{c(v_1-u)}{b(v_1-u)}a^n(v_1)+\frac{c(u-v_1)}{b(u-v_1)}b^n(v_1)=0
    \end{equation}
    \begin{equation}
        \frac{a(v_1)}{b(v_1)}\in\{e^{i\frac{2\pi k}{n}}|k\in\{0,...,n-1\}\}
    \end{equation}
    \begin{equation}
        L_j=\begin{pmatrix}
            \frac{a+b}{2}I_j+\frac{a-b}{2}\sigma^z_j & C\sigma^-_j\\
            C\sigma^+_j & \frac{a+b}{2}I_j-\frac{a-b}{2}\sigma^z_j
        \end{pmatrix}
    \end{equation}
    \begin{equation}
        A_j(v_1)\begin{pmatrix}
            1\\0
        \end{pmatrix}=a(v_1)\begin{pmatrix}
            1\\0
        \end{pmatrix},\quad D_j(v_1)\begin{pmatrix}
            1\\0
        \end{pmatrix}=b(v_1)\begin{pmatrix}
            1\\0
        \end{pmatrix}
    \end{equation}
    \begin{equation}
        \begin{cases}
            a(v_1)=\sinh(v_1+\lambda)\\
            b(v_1)=\sinh(v_1)
        \end{cases}
    \end{equation}
    \begin{equation}
        a(v_1)=e^{i\frac{2\pi k}{n}}b(v_1)\rightarrow\sinh(v_1+\lambda)=e^{i\frac{2\pi k}{n}}\sinh(v_1)
    \end{equation}
    \begin{equation}
        \sinh v_1\cosh\lambda+\cosh v_1\sinh\lambda=e^{i\frac{2\pi k}{n}}\sinh(v_1)\rightarrow\tanh v_1=-\frac{\sinh\lambda}{\cosh\lambda-e^{i\frac{2\pi k}{n}}}
    \end{equation}
    \item В условиях предыдущего упражнения рассмотрите вектор $\Phi=B(v_1)B(v_2)\Psi_0$. Попробуйте найти действие оператора $A(u)+D(u)$ на этот вектор. Выделите и выпишите диагональную часть, т.е. вектора вида $\varepsilon(u,v_1,v_2)\Phi$.\\
    Выпишите явно нежелательные (нарушающие условия диагональности) члены вида
    \begin{multline}
        K_1(u,v_1,v_2)B(u)B(v_1)A(v_2)+K_2(u,v_1,v_2)B(u)B(v_2)A(v_1)+\\+K'_1(u,v_1,v_2)B(u)B(v_1)D(v_2)+K'_2(u,v_1,v_2)B(u)B(v_2)D(v_1)
    \end{multline}
    (где $K_i$ -- функции) и проверьте, что условие зануления нежелательных членов приводит к уравнению анзаца Бете с $l=2$. Подумайте, как обобщить результат на $l=3$ и выше.\\
    \textbf{Решение.}\\
    Подействуем на вектор $\Phi=B(v_1)B(v_2)\Psi_0$ трансфер-матрицей:
    \begin{equation}
        T(u)\Phi=T(u)B(v_1)B(v_2)\Psi_0=(A(u)+D(u))B(v_1)B(v_2)\Psi_0
    \end{equation}
    Из (\ref{eq2}) и (\ref{eq3}):
    \begin{equation}
        A(u)B(v_1)=\frac{a(v_1-u)}{b(v_1-u)}B(v_1)A(u)-\frac{c(v_1-u)}{b(v_1-u)}B(u)A(v_1)
    \end{equation}
    \begin{equation}
        D(u)B(v_1)=\frac{a(u-v_1)}{b(u-v_1)}B(v_1)D(u)-\frac{c(u-v_1)}{b(u-v_1)}B(u)D(v_1)
    \end{equation}
    \begin{multline}
        T(u)\Phi=\left(\frac{a(v_1-u)}{b(v_1-u)}B(v_1)A(u)+\frac{a(u-v_1)}{b(u-v_1)}B(v_1)D(u)\right)B(v_2)\Psi_0-\\-\left(\frac{c(v_1-u)}{b(v_1-u)}B(u)A(v_1)+\frac{c(u-v_1)}{b(u-v_1)}B(u)D(v_1)\right)B(v_2)\Psi_0=\\=\left(\frac{a(v_1-u)}{b(v_1-u)}\frac{a(v_2-u)}{b(v_2-u)}a^n(u)+\frac{a(u-v_1)}{b(u-v_1)}\frac{a(u-v_2)}{b(u-v_2)}b^n(u)\right)\Phi+\\+\left(-\frac{a(v_1-u)}{b(v_1-u)}\frac{c(v_2-u)}{b(v_2-u)}a^n(v_2)+\frac{c(v_1-u)}{b(v_1-u)}\frac{c(v_2-v_1)}{b(v_2-v_1)}a^n(v_2)-\right.\\\left.-\frac{a(u-v_1)}{b(u-v_1)}\frac{c(u-v_2)}{b(u-v_2)}b^n(v_2)+\frac{c(u-v_1)}{b(u-v_1)}\frac{c(v_1-v_2)}{b(v_1-v_2)}b^n(v_2)\right)B(u)B(v_1)\Psi_0+\\+\left(-\frac{c(v_1-u)}{b(v_1-u)}\frac{a(v_2-v_1)}{b(v_2-v_1)}a^n(v_1)-\frac{c(u-v_1)}{b(u-v_1)}\frac{a(v_1-v_2)}{b(v_1-v_2)}b^n(v_1)\right)B(u)B(v_2)\Psi_0
    \end{multline}
    Условия зануления недиагональных членов:
    \begin{equation}
        \begin{cases}
            \frac{a^n(v_2)}{b^n(v_2)}=-\left(\frac{a(u-v_1)}{b(u-v_1)}\frac{c(u-v_2)}{b(u-v_2)}-\frac{c(u-v_1)}{b(u-v_1)}\frac{c(v_1-v_2)}{b(v_1-v_2)}\right)/\left(\frac{a(v_1-u)}{b(v_1-u)}\frac{c(v_2-u)}{b(v_2-u)}-\frac{c(v_1-u)}{b(v_1-u)}\frac{c(v_2-v_1)}{b(v_2-v_1)}\right)\\
            \frac{a^n(v_1)}{b^n(v_1)}=-\frac{c(u-v_1)}{b(u-v_1)}\frac{a(v_1-v_2)}{b(v_1-v_2)}/\frac{c(v_1-u)}{b(v_1-u)}\frac{a(v_2-v_1)}{b(v_2-v_1)}
        \end{cases}
    \end{equation}
    В тригонометрической параметризации $\Delta>1$:
    \begin{equation}
        a(u)=\sinh(\lambda+u),\quad b(u)=\sinh(u),\quad c(u)=\sinh(\lambda)
    \end{equation}
    Система преобразуется к виду:
    \begin{equation}
        \begin{cases}
            \frac{a^n(v_2)}{b^n(v_2)}=-\frac{a(v_1-v_2)}{a(v_2-v_1)}\\
            \frac{a^n(v_1)}{b^n(v_1)}=-\frac{a(v_2-v_1)}{a(v_1-v_2)}
        \end{cases}
    \end{equation}
    Получилось уравнение анзаца Бете с $l=2$.\\
    Обобщение для $l\geq3$:
    \begin{equation}
        \frac{a^n(v_i)}{b^n(v_i)}=\prod\limits_{j\neq i}\frac{a(v_i-v_j)}{a(v_j-v_i)}\frac{b(v_j-v_i)}{b(v_i-v_j)}=(-1)^{l-1}\prod\limits_{j\neq i}\frac{a(v_i-v_j)}{a(v_j-v_i)},\quad i,j\in\{1,2\}
    \end{equation}
    \item
    \item Предел к $\Delta=1$.
    \begin{itemize}
        \item[i)] Рассмотрите случаи $\Delta=1$, взяв следующую параметризацию (см. упр. 7), сделав рескейлинг и совершив предел
        \begin{equation}
            \lambda\rightarrow\pi-\delta, \quad u\rightarrow u\delta,\quad \delta\rightarrow0
        \end{equation}
        Выпишите уравнения анзатца Бете в этом пределе. После переобозначений вы должны получить систему уравнений (анзатц Бете для $XXX$ цепочки)
        \begin{equation}
            \left(\frac{v_i+i}{v_i-i}\right)^n=(-1)^{l-1}\prod\limits_{j=1,j\neq i}^l\frac{v_i-v_j+2i}{v_i-v_j-2i}
        \end{equation}
        \item[ii)] Струнные рещения. Рассмотрите двухчастичный анзатц Бете для $XXX$-цепочки
        \begin{equation}
            \left(\frac{v_1+i}{v_1-i}\right)^n=\frac{v_1-v_2+2i}{v_1-v_2-2i},\quad\left(\frac{v_2+i}{v_2-i}\right)^n=\frac{v_2-v_1+2i}{v_2-v_1-2i}
        \end{equation}
        Поищите в термодинамическом пределе корни, для которых $\text{Im}(v_1)\neq0$. Вы должны получить решения, которые имеют одинаковую вещественную часть, но разные мнимые части. Причём мнимые части будут расположены на комплексной плоскости $v$ симметрично относительно оси $\text{Re}(v)$ (<<струнные решения>>).
        \item[iii)] Попробуйте повторить упражнение $ii)$ для $l$-частичного анзатца.
    \end{itemize}
    \textbf{Решение.}
    \begin{itemize}
        \item[i)] Параметризация:
        \begin{equation}
            a(u)=\sin(\lambda-u),\quad b(u)=\sin(u),\quad c(u)=\sin(\lambda),\quad\Delta=-\cos\lambda
        \end{equation}
        Рескейлинг и предел:
        \begin{equation}
            \lambda\rightarrow\pi-\delta, \quad u\rightarrow u\delta,\quad \delta\rightarrow0
        \end{equation}
        Анзатц Бете:
        \begin{equation}
            \frac{a^n(v_i)}{b^n(v_i)}=(-1)^{l-1}\prod\limits_{j=1,j\neq i}^l\frac{a(v_i-v_j)}{a(v_j-v_i)}
        \end{equation}
        \begin{equation}
            \frac{\sin^n(\lambda-v_i)}{\sin^n(v_i)}=(-1)^{l-1}\prod\limits_{j=1,j\neq i}^l\frac{\sin(\lambda-(v_i-v_j))}{\sin(\lambda-(v_j-v_i))}
        \end{equation}
        \begin{equation}
            \frac{\sin^n(\delta(1+v_i))}{\sin^n(v_i\delta)}=(-1)^{l-1}\prod\limits_{j=1,j\neq i}^l\frac{\sin(\delta(1+v_i-v_j))}{\sin(\delta(1+v_j-v_i))}
        \end{equation}
        \begin{equation}
            \lim\limits_{\delta\rightarrow0}\frac{\sin^n(\delta(1+v_i))}{\sin^n(v_i\delta)}=\left(\frac{1+v_i}{v_i}\right)^n
        \end{equation}
        \begin{equation}
            \lim\limits_{\delta\rightarrow0}\frac{\sin(\delta(1+v_i-v_j))}{\sin(\delta(1+v_j-v_i))}=\frac{1+v_i-v_j}{1+v_j-v_i}
        \end{equation}
        \begin{equation}
            \left(\frac{1+v_i}{v_i}\right)^n=(-1)^{l-1}\prod\limits_{j=1,j\neq i}^l\frac{1+v_i-v_j}{1+v_j-v_i}
        \end{equation}
        Пусть $u_i=2iv_i$, тогда
        \begin{equation}
            \left(\frac{u_i+i}{u_i-i}\right)^n=(-1)^{l-1}\prod\limits_{j=1,j\neq i}^l\frac{1+\frac{u_i}{2i}-\frac{u_j}{2i}}{1+\frac{u_j}{2i}-\frac{u_i}{2i}}=(-1)^{2l-2}\prod\limits_{j=1,j\neq i}^l\frac{u_i-u_j+2i}{u_i-u_j-2i}
        \end{equation}
        Анзатц Бете для $XXX$:
        \begin{equation}
            \boxed{\left(\frac{u_i+i}{u_i-i}\right)^n=\prod\limits_{j=1,j\neq i}^l\frac{u_i-u_j+2i}{u_i-u_j-2i}}
        \end{equation}
        \item[ii)] Пусть $u_1=a_1+ib_1$, $u_2=a_2+ib_2$, тогда
        \begin{equation}
            \left|\frac{u_1+i}{u_1-i}\right|^2=\left|\frac{a_1+i(b_1+1)}{a_1+i(b_1-1)}\right|^2=\frac{a^2_1+(b_1+1)^2}{a^2_1+(b_1-1)^2}\begin{cases}
                >1,\quad b_1>0\\
                <1,\quad b_1<0
            \end{cases}
        \end{equation}
        При $b_1<0$ в термодинамическом пределе $\lim\limits_{n\rightarrow\infty}\left|\frac{u_1+i}{u_1-i}\right|^{2n}=0$.
        \begin{equation}
            \lim\limits_{n\rightarrow\infty}\left|\frac{u_1-u_2+2i}{u_1-u_2-2i}\right|^2=\frac{(a_1-a_2)^2+(b_1-b_2+2)^2}{(a_1-a_2)^2+(b_1-b_2-2)^2}=0\rightarrow\begin{cases}
                a_1=a_2\\
                b_1=b_2-2
            \end{cases}
        \end{equation}
        \begin{equation}
            \lim\limits_{n\rightarrow\infty}\frac{v_1-v_2+2i}{v_1-v_2-2i}\frac{v_2-v_1+2i}{v_2-v_1-2i}=1
        \end{equation}
        \begin{multline}
            \lim\limits_{n\rightarrow\infty}\left(\frac{v_1+i}{v_1-i}\right)^n\left(\frac{v_2+i}{v_2-i}\right)^n=\lim\limits_{n\rightarrow\infty}\left(\frac{a_1+i(b_1+1)}{a_1+i(b_1-1)}\frac{a_1+i(b_1+3)}{a_1+i(b_1+1)}\right)^n=\\=\lim\limits_{n\rightarrow\infty}\left(\frac{a_1+i(b_1+3)}{a_1+i(b_1-1)}\right)^n=1\rightarrow b_1=-1
        \end{multline}
        \begin{equation}
            \boxed{u_1=a-i,\quad u_2=a+i}
        \end{equation}
        При $b_1>0$ $u_1$ и $u_2$ поменяются местами.
        \item[iii)] Повторяя аналогично действия для проищвольного $l$, получим, что при чётном $l$ будет $\frac{l}{2}$ струнных решений, а при нечётном -- $\frac{l-1}{2}$ струнных и 1 вещественное.
    \end{itemize}
    \item Режимы. Параметризация.
    \begin{enumerate}
        \item[i)] Проверьте, что удобная параметризация 6-вершинной модели
        \begin{equation}
            \Delta=\frac{a^2+b^2-c^2}{2ab}
        \end{equation}
        для случая вещественных положительных параметров $u,\lambda$ задаётся
        \begin{itemize}
            \item Ферромагнитный режим $\Delta>1$.\\
            $a(u)=\sinh(\lambda+u)$, $b(u)=\sinh(u)$, $c(u)=\sinh(\lambda)$, $\Delta=\cosh\lambda$.
            \item Антиферромагнитный режим $\Delta<-1$.\\
            $a(u)=\sinh(\lambda-u)$, $b(u)=\sinh(u)$, $c(u)=\sinh(\lambda)$, $\Delta=-\cosh\lambda$.
            \item Gapless режим $-1<\Delta<1$.\\
            $a(u)=\sin(\lambda-u)$, $b(u)=\sin(u)$, $c(u)=\sin(\lambda)$, $\Delta=-\cos\lambda$.
        \end{itemize}
        \item[ii)] Обсудите, какие веса доминируют в каком режиме (проверьте, когда $a>b+c$, когда $c>a+b$ и т.д.), а также обсудите, как устроено основное состояние в теории.
    \end{enumerate}
    \textbf{Решение.}
    \begin{itemize}
        \item[i)] $\Delta>1$:
        \begin{equation}
            \Delta=\frac{a^2+b^2-c^2}{2ab}=\frac{\sinh^2(\lambda+u)+\sinh^2(u)-\sinh^2(\lambda)}{2\sinh(\lambda+u)\sinh(u)}=\cosh\lambda
        \end{equation}
        $\Delta<1$:
        \begin{equation}
            \Delta=\frac{a^2+b^2-c^2}{2ab}=\frac{\sinh^2(\lambda-u)+\sinh^2(u)-\sinh^2(\lambda)}{2\sinh(\lambda-u)\sinh(u)}=-\cosh{\lambda}
        \end{equation}
        $-1<\Delta<1$:
        \begin{equation}
            \Delta=\frac{a^2+b^2-c^2}{2ab}=\frac{\sin^2(\lambda-u)+\sin^2(u)-\sin^2(\lambda)}{2\sin(\lambda-u)\sin(u)}=-\cos\lambda
        \end{equation}
        Равенства проверены в WM.
        \item[ii)] При $\Delta>1$ выполняется $a>b+c$ и $b>a+c$. Состояние $\Psi_0$ имеет наименьшую энергию. Основной вклад в статистическую сумму вносят конфигурации либо со всеми $+$, либо со всеми $-$.\\
        При $\Delta<1$ выполняется $a<b+c$ или $a+c<b$. $\Psi_0$ не является вакуумом, это вектор с максимальной энергией. Вакуум бесспиновый, строится из векторов $v^+\otimes v^-\otimes v^+\otimes...$\\
        При $-1<\Delta<1$ выполняется $a+b<c$ или $a<b+c$. $\Psi_0$ не является вакуумом. Существует бесщелевой режим с нетривиальным вакуумом.
    \end{itemize}
\end{enumerate}
\section{XXZ и координатный анзац Бете}
Рассмотрим гамильтониан $XXZ$ цепочки с периодическими граничными условиями
\begin{equation}
    H_{XXZ}=-\sum\limits_{j=1}^n\left(\sigma^+_j\sigma^-_{j+1}+\sigma^-_j\sigma^+_{j+1}+\frac{\Delta}{2}(\sigma^z_j\sigma^z_{j+1}-1)\right)        
\end{equation}
\begin{enumerate}
    \item Гамильтониан, псевдовакуум, симметрии.
    \begin{itemize}
        \item[i)] Покажите, что оператор спина $S^z=\frac{1}{2}\sum\sigma^z$ коммутирует с гамильтонианом и что собственные вектора $H_{XXZ}$ разбиваются на сектора с данным значением спина.
        \item[ii)] Пусть псевдовакуум задаётся как
        \begin{equation}
            \Psi_0=v^+\otimes...\otimes v^+,\quad v^+=\begin{pmatrix}
                1\\0
            \end{pmatrix}
        \end{equation}
        Найдите спин и энергию такого состояния.
        \item[iii)] Изучите действие гамильтониана $XXZ$ на одночастичных возбуждениях вида
        \begin{equation}
            \Psi_j=\sigma^-_j\Psi_0
        \end{equation}
        Определим собственные вектора в виде плоской волны
        \begin{equation}
            \Psi(k)=\sum\limits_{j=1}^na_j(k)\Psi_j
        \end{equation}
        где $a_j(k)=e^{ikj}$. Покажите, что данные вектора удовлетворяют соотношению
        \begin{equation}
            H_{XXZ}\Psi(k)=\varepsilon(k)\Psi(k)
        \end{equation}
        с собственными энергиями
        \begin{equation}
            \varepsilon(k)=2(\Delta-\cos k),\quad k=\frac{2\pi}{n}J,\quad J=0,1,...,n-1
        \end{equation}
    \end{itemize}
    \textbf{Решение.}
    \begin{itemize}
        \item[i)] 
        \begin{multline}
            [\sigma^z_j,\sigma^+_{j-1}\sigma^-_j+\sigma^-_{j-1}\sigma^+_j+\frac{\Delta}{2}(\sigma^z_{j-1}\sigma^z_j-1)+\sigma^+_j\sigma^-_{j+1}+\sigma^-_j\sigma^+_{j+1}+\frac{\Delta}{2}(\sigma^z_j\sigma^z_{j+1}-1)]=\\=\sigma^+_{j-1}[\sigma^z_j,\sigma^-_j]+\sigma^-_{j-1}[\sigma^z_j,\sigma^+_j]+[\sigma^z_j,\sigma^+_j]\sigma^-_{j+1}+[\sigma^z_j,\sigma^-_j]\sigma^+_{j+1}=\\=-2\sigma^+_{j-1}\sigma^-_j+2\sigma^-_{j-1}\sigma^+_j+2\sigma^+_j\sigma^-_{j+1}-2\sigma^-_j\sigma^+_{j+1}
        \end{multline}
        \begin{equation}
            \boxed{[S,H_{XXZ}]=0}
        \end{equation}
        Таким образом, собственные вектора $H_{XXZ}$ разбиваются на сектора с данным значением спина.
        \item[ii)] Подействуем на псевдовакуум оператором спина:
        \begin{equation}
            S^z\Psi_0=\frac{1}{2}\sum\limits_{j=1}^n\sigma^z(v^+\otimes...\otimes v^+)=\frac{n}{2}(v^+\otimes...\otimes v^+)=\frac{n}{2}\Psi_0
        \end{equation}
        Спин равен $\frac{n}{2}$. Подействуем на псевдовакуум гамильтонианом:
        \begin{equation}
            H_{XXZ}\Psi_0=0
        \end{equation}
        Энергия равна 0.
        \item[iii)]
        \begin{equation}
            H_{XXZ}\Psi(k)=(-a_{j-1}-a_{j+1}+2\Delta a_j)\Psi(k)
        \end{equation}
        Выбирая анзатц плоской волны $a_j=e^{ikj}$, получим
        \begin{equation}
            \boxed{\varepsilon(k)=2(\Delta-\cos k)}
        \end{equation}
    \end{itemize}
    \item Две частицы.\\
    Начнём изучение более сложных возбуждений, спектр которых опредееляется уравнениями анзатца Бете.\\
    Рассмотрим возбуждения (двухмагнонного типа) над (псевдо)вакуумом со спином $\frac{n}{2}-2$ вида
    \begin{equation}
        \Psi_{j_1j_2}=\sigma^-_{j_1}\sigma^-_{j_2}\Psi_0
    \end{equation}
    Это вектора вида
    \begin{equation}
        \Psi_{j_1j_2}=v^+\otimes...\otimes v^+\otimes v^-\otimes v^+\otimes...\otimes v^+\otimes v^-\otimes v^+\otimes...\otimes v^+
    \end{equation}
    где $v^-$ стоят на $j_1$-ом и $j_2$-ом местах. Для поиска собственных векторов в данном секторе рассмотрим сумму
    \begin{equation}
        \Psi(k_1,k_2)=\sum\limits_{j_1<j_2}a_{j_1j_2}(k_1,k_2)\Psi_{j_1j_2}
    \end{equation}
    Мы хотим, чтобы Гамильтониан цепочки действовал бы на эти вектора по правилу
    \begin{equation}
        H_{XXZ}\Psi(k_1,k_2)=\varepsilon(k_1,k_2)\Psi(k_1,k_2)
    \end{equation}
    \begin{itemize}
        \item[i)] Найдите действие оператора $H_{XXZ}$ на состояния
        \begin{equation}
            \Psi_{j_1j_2}=\sigma^-_{j_1}\sigma^-_{j_2}\Psi_0
        \end{equation}
        с $j_1+1<j_2$. Вы должны получить
        \begin{equation}
            H_{XXZ}\Psi_{j_1j_2}=-(\Psi_{j_1+1,j_2}+\Psi_{j_1-1,j_2}+\Psi_{j_1,j_2+1}+\Psi_{j_1,j_2-1}-4\Delta\Psi_{j_1,j_2})
        \end{equation}
        \item[ii)] Найдите действие оператора $H_{XXZ}$ на состояния
        \begin{equation}
            \Psi_{j,j+1}=\sigma^-_j\sigma^-_{j+1}\Psi_0
        \end{equation}
        Вы должны получить
        \begin{equation}
            H_{XXZ}\Psi_{j,j+1}=-(\Psi_{j,j+2}+\Psi_{j-1,j+1}-2\Delta\Psi_{j,j+1})
        \end{equation}
        \item[iii)] Произведите пересуммирование и проверьте, что
        \begin{multline}
            H_{XXZ}\Psi(k)=-\sum\limits_{j_1+1<j_2}(a_{j_1+1,j_2}+a_{j_1-1,j_2}+a_{j_1,j_2-1}+a_{j_1,j_2+1}-4\Delta a_{j_1,j_2})\Psi_{j_1,j_2}-\\-\sum\limits_j(a_{j-1,j+1}+a_{j,j+2}-2\Delta a_{j,j+1})\Psi_{j,j+1}
        \end{multline}
        \item[iv)] Рассмотрите анзатц Бете
        \begin{equation}
            a_{j_1j_2}=A_{12}(k_1,k_2)e^{i(j_1k_1+j_2k_2)}+A_{21}(k_1,k_2)e^{i(j_1k_2+j_2k_1)}
        \end{equation}
        где коэффициенты $A_{12}$, $A_{21}$ зависят лишь от импульсов $k_1$, $k_2$, но не зависят от координат $j_1$, $j_2$. Проверьте, что
        \begin{multline}
            -\sum\limits_{j_1+1<j_2}(a_{j_1+1,j_2}+a_{j_1,j_2+1}+a_{j_1-1,j_2}+a_{j_1,j_2-1}-4\Delta a_{j_1,j_2})\Psi_{j_1,j_2}=\\=(\varepsilon(k_1)+\varepsilon(k_2))\sum\limits_{j_1+1<j_2}a_{j_1j_2}\Psi_{j_1j_2}
        \end{multline}
        где $\epsilon(k)$ -- энергии одночастичных возбуждений. Рассмотрите условие
        \begin{equation}
            -\sum\limits_j(a_{j-1,j+1}+a_{j,j+2}-2\Delta a_{j,j+1})\Psi_{j,j+1}=(\varepsilon(k_1)+\varepsilon(k_2))\sum\limits_ja_{j,j+1}\Psi_{j,j+1}
        \end{equation}
        Используя явный вид для $\varepsilon(k)$, найдите, что коэффициенты $A_{12}$, $A_{21}$ должны удовлетворять соотношению
        \begin{equation}
            S(k_1,k_2)=\frac{A_{12}(k_1,k_2)}{A_{21(k_1,k_2)}}=-\frac{1+e^{i(k_1+k_2)}-2\Delta e^{ik_1}}{1+e^{i(k_1+k_2)}-2\Delta e^{ik_2}}
        \end{equation}
    \end{itemize}
    \textbf{Решение.}
    \begin{itemize}
        \item[i)] 
    \end{itemize}
    \item
    \item
    \begin{itemize}
        \item[i)] Проверьте в условиях предыдущего упражнения, что матрица рассеяния -- это чистая фаза.
        \item[ii)] Введите тригонометрическую параметризацию для импульсов, переходя к быстротным переменным
        \begin{equation}
            k_1=k(u_1),\quad k_2=k(u_2),\quad \Delta=-\cos\lambda
        \end{equation}
        \begin{equation}
            e^{ik(u)}=\frac{\sin(\frac{\lambda}{2}+iu)}{\sin(\frac{\lambda}{2}-iu)}
        \end{equation}
        Проверьте, что матрица рассеяния (квази)частиц имеет вид
        \begin{equation}
            e^{iS(k_1,k_2)}=-\frac{\sin(\lambda+i(u_1-u_2))}{\sin(\lambda-i(u_1-u_2))}
        \end{equation}
        Выпишите уравнение анзаца Бете в этой параметризации. Решите их для случая $\Delta=0$.
        \item[iii)] Введите быстротную параметризацию $XXX$ модели (сделав перемасштабирование $u\rightarrow u\lambda$ и взяв предел $\lambda\rightarrow0$). Нарисуйте зависимость имульса от быстроты. Выпишите уравнение анзаца Бете.
    \end{itemize}
    \textbf{Решение.}
    \begin{itemize}
        \item[i)] 
        \begin{equation}
            e^{iS(k_1,k_2)}e^{ik_1n}=1,\quad e^{iS(k_1,k_2)}e^{ik_2n}=1
        \end{equation}
        \begin{equation}
            \begin{cases}
                k_1n+S(k_1,k_2)=2\pi n,\\
                k_2m+S(k_1,k_2)=2\pi m.
            \end{cases}\quad n,m\in\mathbb{Z}
        \end{equation}
        Поэтому $\text{Im}(S)=0$ и матрица рассеяния -- фаза.
        \item[ii)]
        \begin{equation}
            e^{iS(k_1,k_2)}=\frac{A_{12}(k_1,k_2)}{A_{21}(k_1,k_2)}=-\frac{1+e^{i(k_1+k_2)}-2\Delta e^{ik_1}}{1+e^{i(k_1+k_2)}-2\Delta e^{ik_2}}
        \end{equation}
        \begin{equation}
            e^{iS(k_1,k_2)}=-\frac{1+\frac{\sin(\frac{\lambda}{2}+iu_1)}{\sin(\frac{\lambda}{2}-iu_1)}\frac{\sin(\frac{\lambda}{2}+iu_2)}{\sin(\frac{\lambda}{2}-iu_2)}+2\cos\lambda\frac{\sin(\frac{\lambda}{2}+iu_1)}{\sin(\frac{\lambda}{2}-iu_1)}}{1+\frac{\sin(\frac{\lambda}{2}+iu_1)}{\sin(\frac{\lambda}{2}-iu_1)}\frac{\sin(\frac{\lambda}{2}+iu_2)}{\sin(\frac{\lambda}{2}-iu_2)}+2\cos\lambda\frac{\sin(\frac{\lambda}{2}+iu_2)}{\sin(\frac{\lambda}{2}-iu_2)}}
        \end{equation}
        После упрощений в WM получим
        \begin{equation}
            \boxed{e^{iS(k_1,k_2)}=-\frac{\sin(\lambda+i(u_1-u_2))}{\sin(\lambda-i(u_1-u_2))}}
        \end{equation}
        \begin{equation}
            e^{ik_1n}=e^{-iS(k_1,k_2)},\quad e^{ik_2n}=e^{-iS(k_1,k_2)}
        \end{equation}
        При $\Delta=0$, $\lambda=\frac{\pi}{2}$:
        \begin{equation}
             e^{iS(k_1,k_2)}=1\rightarrow\quad \boxed{\begin{cases}
                 \left(\frac{\sin(\frac{\pi}{4}+iu_1)}{\sin(\frac{\pi}{4}-iu_1)}\right)^n=1,\\
                 \left(\frac{\sin(\frac{\pi}{4}+iu_2)}{\sin(\frac{\pi}{4}-iu_2)}\right)^n=1
             \end{cases}}
        \end{equation}
        \item[iii)] Быстротная параметризация $XXX$-модели $u\rightarrow u\lambda$, $\lambda\rightarrow0$.
        \begin{equation}
            e^{iS(k_1,k_2)}=-\lim\limits_{\lambda\rightarrow0}\frac{\sin(\lambda+i(u_1-u_2)\lambda)}{\sin(\lambda-i(u_1-u_2)\lambda)}=-1
        \end{equation}
    \end{itemize}
    \item Анзац Бете для $m>2$.
        \begin{itemize}
            \item[i)] Рассмотрите сектор со спином $\frac{n}{2}-3$ и возбуждения вида
            \begin{equation}
                \Psi_{j_1j_2j_3}=\sigma^-_{j_1}\sigma^-_{j_2}\sigma^-_{j_3}\Psi_0
            \end{equation}
            Пусть волновая функция Бете
            \begin{equation}
                \Phi(k_1,k_2,k_3)=\sum\limits_{j_1<j_2<j_3}a_{j_1j_2j_3}\Psi_{j_1,j_2,j_3}
            \end{equation}
            определена фиксированием коэффициентов $a_{j_1j_2j_3}$ как
            \begin{multline}
                a_{j_1j_2j_3}=e^{i(j_1k_1+j_2k_2+j_3k_3)}+S_{12}e^{i(j_1k_2+j_2k_1+j_3k_3)}+S_{23}e^{i(j_1k_1+j_2k_3+j_3k_2)}+\\+S_{12}S_{23}e^{i(j_1k_2+j_2k_3+j_3k_1)}+S_{13}S_{23}e^{i(j_1k_3+j_2k_1+j_3k_2)}+S_{12}S_{13}S_{23}e^{i(j_1k_3+j_2k_2+j_3k_1)},\quad S_{ij}:=S(k_i,k_j)
            \end{multline}
            Попробуйте наложить периодические граничные условия и выписать уравнения анзатца Бете для квазиимпульсов в этом случае.
            \item[ii)] Попробуйте обосновать уравнения анзатца Бете в общем случае. Выпишите их в логарифмическом виде. Найдите полный импульс системы.
        \end{itemize}
        \textbf{Решение.}
        \begin{itemize}
            \item[i)]
            В случае спина $\frac{n}{2}-3$:
            \begin{equation}
            \boxed{\begin{cases}
                e^{ik_1n}=e^{iS(k_1,k_2)}e^{iS(k_1,k_3)},\\
                e^{ik_2n}=e^{iS(k_2,k_1)}e^{iS(k_2,k_3)},\\
                e^{ik_3n}=e^{iS(k_3,k_1)}e^{iS(k_3,k_2)}
            \end{cases}}
            \end{equation}
            \item[ii)] Модель должна учитывать взаимодействие только между соседними сипинами, оба из которых могут находиться в возбуждённом или невозбуждённом состояниях.
            Уравнения анзатца Бете в общем случае:
            \begin{equation}
                \boxed{e^{ik_in}=\prod\limits_{j=1,j\neq i}e^{iS(k_i,k_j)},\quad i=1,...,m}
            \end{equation}
            В логарифмическом виде:
            \begin{equation}
                \boxed{k_in=2\pi J_i+\sum\limits_{j\neq i}S(k_i,k_j)}
            \end{equation}
            Полный импульс системы:
            \begin{equation}
                k=\sum\limits_{i}k_i=\frac{2\pi}{n}\sum\limits_{i=1}J_i+\frac{1}{n}\sum\limits_{i=1}^n\left(\sum\limits_{j\neq i} S(k_i,k_j)\right)
            \end{equation}
            \begin{equation}
                \boxed{k=\frac{2\pi}{n}\sum\limits_{i=1}J_i}
            \end{equation}
        \end{itemize}
\end{enumerate}
\end{document}
