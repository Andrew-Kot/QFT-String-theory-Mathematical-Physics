\documentclass[12pt]{article}

% report, book
%  Русский язык

\usepackage{hyperref,bookmark}
\usepackage[warn]{mathtext} %русский язык в формулах
\usepackage[T2A]{fontenc}			% кодировка
\usepackage[utf8]{inputenc}			% кодировка исходного текста
\usepackage[english,russian]{babel}	% локализация и переносы
\usepackage[title,toc,page,header]{appendix}
\usepackage{amsfonts}


% Математика
\usepackage{amsmath,amsfonts,amssymb,amsthm,mathtools} 
%%% Дополнительная работа с математикой
%\usepackage{amsmath,amsfonts,amssymb,amsthm,mathtools} % AMS
%\usepackage{icomma} % "Умная" запятая: $0,2$ --- число, $0, 2$ --- перечисление

\usepackage{cancel}%зачёркивание
\usepackage{braket}
%% Шрифты
\usepackage{euscript}	 % Шрифт Евклид
\usepackage{mathrsfs} % Красивый матшрифт


\usepackage[left=2cm,right=2cm,top=1cm,bottom=2cm,bindingoffset=0cm]{geometry}
\usepackage{wasysym}

%размеры
\renewcommand{\appendixtocname}{Приложения}
\renewcommand{\appendixpagename}{Приложения}
\renewcommand{\appendixname}{Приложение}
\makeatletter
\let\oriAlph\Alph
\let\orialph\alph
\renewcommand{\@resets@pp}{\par
  \@ppsavesec
  \stepcounter{@pps}
  \setcounter{subsection}{0}%
  \if@chapter@pp
    \setcounter{chapter}{0}%
    \renewcommand\@chapapp{\appendixname}%
    \renewcommand\thechapter{\@Alph\c@chapter}%
  \else
    \setcounter{subsubsection}{0}%
    \renewcommand\thesubsection{\@Alph\c@subsection}%
  \fi
  \if@pphyper
    \if@chapter@pp
      \renewcommand{\theHchapter}{\theH@pps.\oriAlph{chapter}}%
    \else
      \renewcommand{\theHsubsection}{\theH@pps.\oriAlph{subsection}}%
    \fi
    \def\Hy@chapapp{appendix}%
  \fi
  \restoreapp
}
\makeatother
\newtheorem{resh}{Решение}
\newtheorem{theorem}{Теорема}
\newtheorem{predl}[theorem]{Предложение}
\newtheorem{sled}[theorem]{Следствие}

\theoremstyle{definition}
\newtheorem{zad}{Задача}[section]
\newtheorem{upr}[zad]{Упражнение}
\newtheorem{defin}[theorem]{Определение}

\title{Решение заданий\\ ОП "Квантовая теория поля, теория струн и математическая физика"\\[2cm]
Алгебры и группы Ли,\\ теория представлений алгебр Ли I\\ (Б.Л. Фейгин)}
\author{Коцевич Андрей Витальевич, Б02-920с}
\date{6 семестр, 2022}

\begin{document}
\setlength{\parindent}{0pt}
\maketitle
\newpage
\tableofcontents{}
\newpage
\section{Теоретический минимум}
\subsection{Основные сведения из теории групп}
\begin{defin}
\textit{Группой} называется множество $G$ с бинарной операцией $\cdot$: $G\times G\rightarrow G$, если выполнены следующие аксиомы:
\begin{enumerate}
    \item (Ассоциативность) $\forall a,b,c\in G\hookrightarrow a\cdot(b\cdot c)=(a\cdot b)\cdot c$.
    \item ($\exists$ единицы) $\exists e\in G:\forall a\in G\hookrightarrow a\cdot e=e\cdot a=a$.
    \item ($\exists$ обратного элемента) $\forall a\in G\hookrightarrow\exists b$: $a\cdot b=e$.
\end{enumerate}
\end{defin}
Из аксиом группы следует, что а) единичный элемент единственный, б) обратный элемент удовлетворяет также свойству $a^{-1}\cdot a=e$.\\
\textit{Примеры}:
\begin{enumerate}
    \item Группа целых чисел $\mathbb{Z}$ операцией сложения. Аналогично, группа вещественных $\mathbb{R}$ или комплексных $\mathbb{C}$ чисел с операцией сложения.\\
    Все ненулевые вещественные (или комплексные) числа с операцией умножения.
    \item Циклическая группа из $n$ элементов $C_n=\{e, r, ... , r^{n-1}\}$, где $r^n = e$. Элементы группы можно представлять как повороты на угол $\frac{2\pi k}{n}$ вокруг начала координат.
    \item Группа перестановок $S_n$.
\end{enumerate}
Элементами группы могут быть матрицы. Перечислим основные матричные группы:
\begin{enumerate}
    \item Полная линейная группа $GL_n(\mathbb{R})$ -- группа всех невырожденных матриц с ненулевым определителем.
    ДОПИСАТЬ!!!
\end{enumerate}
\begin{defin}
Группа называется \textit{коммутативной} (или \textit{абелевой}), если дополнительно выполнена аксиома: $\forall a,b\in G\hookrightarrow a\cdot b=b\cdot a$.
\end{defin}
\begin{defin}
\textit{Порядком элемента} $g\in G$ называют наименьшее натуральное $n$: $g^n=e$. Если такого $n$ не существует, то говорят что порядок равен бесконечности.\\
Количество элементов в группе называется \textit{порядком группы}. Обозначение: $|G|$.
\end{defin}
\begin{defin}
Множество элементов $s_1,...,s_k\in G$ навается \textit{образующими группы} $G$, если любой элемент $g\in G$ может быть представлен в виде $g=s^{\pm 1}_{i_1}...s^{\pm1}_{i_l}$. Здесь среди индексов $i_1,...i_l$ могут быть одинаковые.
\end{defin}
\begin{defin}
    Подмножество $H\subset G$ называется \textit{подгруппой}, если $\forall a,b\in H\hookrightarrow a\cdot b\in H$ и $a^{-1}\in H$.
\end{defin} 
Так как аксиомы группы выполняются в $G$, то они выполняются и в $H$, т.е. любая подгруппа является группой.
\begin{defin}
    \textit{Гомоморфизм групп} из $(G,\cdot)$ в $(H,*)$ -- функция $\varphi:G\rightarrow H:\forall u,v\in G\hookrightarrow \varphi(u\cdot v)=\varphi(u)*\varphi(v)$. 
\end{defin}
Отсюда можно вывести, что $\varphi$ отображает нейтральный элемент $e_G$ группы $G$ в нейтральный элемент $e_H$ группы $H$, а также отображает обратные элементы в обратные:
\begin{equation}
    \varphi(g^{-1})=\varphi(g)^{-1}
\end{equation}
Можно сказать, что $\varphi$ сохраняет групповую структуру.
\begin{defin}
    Две группы $(G,\cdot)$ и $(H,*)$ называются \textit{изоморфными}, если $\exists$ биекция $\varphi:G\rightarrow H:\forall a,b\in G\hookrightarrow\varphi(a\cdot b)=\varphi(a)*\varphi(b)$.
\end{defin}
\begin{defin}
    \textit{Прямым произведением} групп $G$ и $H$ называется множество пар $G\times H=\{(g,h)|g\in G, h\in H\}$ с операцией $(g_1,h_1)\cdot(g_2,h_2)=(g_1g_2,h_1h_2)$.
\end{defin}
\begin{theorem}
    Любая абелева группа с конечным числом образующих изоморфна произведению циклических групп $G\simeq\mathbb{Z}^n\times C_{n_1}\times...\times C_{n_l}$.
\end{theorem}
Теорема в частности утверждает, что любая конечная абелева группа изоморфна произведению конечных циклических групп $G\simeq C_{n_1}\times...\times C_{n_l}$. При этом порядок группы $G$ равен $|G| = n_1\cdot...\cdot n_l$.\\
Пользуясь этой теоремой, можно перечислить абелевы группы данного порядка. Ясно, что существует ровно одна абелева группа порядка 1, так же для групп порядка 2 и порядка 3. Cуществует две неизоморфныее
абелевы группы порядка 4: $C_2\times C_2$ и $C_4$ (в группе $C_2\times C_2$ любой элемент в квадрате равен 1, а в группе $C_4$ нет). Для порядка 5 абелева группа одна: $C_5$, для
порядка 6 есть уже два кандидата $C_2\times C_3$ и $C_6$, но они изоморфны.
\begin{predl}
    Группы $C_m \times C_n$ и $C_{mn}$ изоморфны $\Leftrightarrow$ $m$ и $n$ взаимно просты.
\end{predl}
\begin{defin}
\textit{Действие группы $G$ на множестве $X$} -- отображение $\rho:G\times X\rightarrow X$:
\begin{enumerate}
    \item $\forall g,h\in G, x\in X\hookrightarrow\rho(g,\rho(h,x))=\rho(gh,x)$.
    \item $\forall x\in X\hookrightarrow\rho(e,x)=x$.
\end{enumerate}
\end{defin}
Для упрощения обозначений обычно действие группы записывается как умножение слева $\rho(g,x)=gx$. Тогда первое свойство -- это просто ассоциативность, а второе это свойство единичного элемента.\\
\textit{Пример}: группа $S_n$ действует на множестве $\{1,2,...,n\}$, переставляя его элементы.
\begin{defin}
Пусть группа $G$ действует на множестве $X$.\\
\textit{Орбита} элемента $x\in X$ -- множество $Gx=\{y\in X|\exists g\in G, y=\rho(g,x)\}$. Множество всех орбит обозначается $X/G$.\\
\textit{Стабилизатор} элемента $x\in X$ -- множество $G_x=\{g\in G|\rho(g,x)=x\}$.
\end{defin}
\textit{Примеры:}
\begin{enumerate}
    \item Группа $S_n$ действует на множестве $\{1,2,...,n\}$. Тогда орбита любой точки $i$ совпадает со всем множеством $\{1,2,...,n\}$, а стабилизатор изоморфен $S_{n-1}$.
    \item Группа $C_n$ действует на $\mathbb{R}^2$ поворотами на угол $2\pi/n$. Тогда орбитой точки будут вершины правильного $n$-угольника. Множество орбит можно отождествить с точками угла $2\pi/n$, в котором стороны угла между собой склеены. Таким образом, множество орбит -- это конус.
\end{enumerate}
\begin{predl}
    $\forall$ точки $x\in X$ стабилизатор $G_x$ являетя подгруппой в $G$.
\end{predl}
\begin{predl}
    Любые две орбиты $Gx$ и $Gy$ или не пересекаются, или совпадают.
\end{predl}
Можно ввести отношение $x\sim y$, если $x\in Gy$ и проверить, что оно является рефлексивным, симметричным и транзитивным. Т.е. является отношением эквивалентности.
\begin{theorem}
    Пусть $G$ -- конечная группа. $\forall x\in X\hookrightarrow|G| = |Gx|\cdot|G_x|$.
\end{theorem}
\begin{defin}
Группа $G$ действует на себе \textit{умножением слева} по формуле $\rho(g,x)=gx$.
\end{defin}
Полезно ограничить это действие на подгруппу $H\subset G$, т.е. рассмотреть действие $G$ на $H$ по формуле $\rho(h,x)=hx$.
\begin{defin}
Орбиты для этого действия -- множества $Hg=\{hg|h\in H\}$ -- называются \textit{правыми классами смежности}.
\end{defin}
Помимо действия умножения слева можно определить действие справа.
\begin{defin}
Группа $G$ действует на себе \textit{умножением справа} по формуле $\rho(g,x)=xg^{-1}$.
\end{defin}
\begin{defin}
Группа $G$ действует на себе \textit{сопряжениями} по формуле $\rho(g,x)=gxg^{-1}$. Орбиты для действия группы на себе сопряжениями называются \textit{классами сопряженности}.
\end{defin}
Надо проверить корректность определения, т.е. что получается действительно действие:
\begin{equation}
    \rho(g_1,\rho(g_2,x))=g_1g_2xg^{-1}_2g^{-1}_1=(g_1g_2)x(g_1g_2)^{-1}=\rho(g_1g_2,x)
\end{equation}
По аналогии с векторными пространствами естественно хотеть ввести структуру умножения на классах смежности. Для этого надо чтобы $g_1H\cdot g_2H=g_1g_2H$. Т.е. $\forall h_1,h_2\in H\hookrightarrow\exists h\in H$: $g_1h_1g_2h_2=g_1g_2h$. Значит, $g_2^{-1}h_1g_2=hh_2^{-1}$. Т.е. нужно, чтобы $gHg^{-1}\subset H$.
\begin{defin}
Подгруппа $N\subset G$ называется \textit{нормальной} (иногда говорят \textit{инвариантной}), если $\forall g\in G\hookrightarrow gNg^{-1}=N$. Это иногда записывают $N\lhd G$.
\end{defin}
Свойства нормальных подгрупп:
\begin{enumerate}
    \item Если $N\lhd G$, то $N$ является объединением каких-то классов сопряжённости.
    \item Если $N\lhd G$. то левые и правые смежные классы совпадают $gN=Ng$.
\end{enumerate}
\begin{proof}
Возьмем произвольный элемент $gn\in gN$ левого класса смежности. Тогда $gn=gng^{-1}g\in Ng$, т.к. $gng^{-1}\in N$ в силу нормальности $N$.
\end{proof}
Оба этих свойства можно было взять в качестве определения нормальной подгруппы.
\begin{predl}
Если $N\lhd G$, то смежные классы по $N$ образуют группу. Эта группа называется \textit{факторгруппой} и обозначается $G/N$.
\end{predl}
Операция умножения вводится по формуле
\begin{equation}
    g_1N\cdot g_2N=g_1g_2N
\end{equation}
Корректность определения следует из нормальности $N$.
\begin{defin}
Группа называется \textit{простой}, если в ней нет нетривиальных нормальных подгрупп (отличных от всей группы и единичной подгруппы).
\end{defin}
\begin{defin}
    Пусть даны две группы $G$ и $H$ и гомоморфизм $\phi$ из группы $G$ в группу автоморфизмов $H$, т.е. $\forall g\in G\hookrightarrow\exists$ биекция $\phi_g:H\rightarrow H:$
    \begin{equation}
        \phi_g(h_1h_2)=\phi_g(h_1)\phi_g(h_2),\quad \phi_{g_1}(\phi_{g_2}(h))=\phi_{g_1g_2}(h),\quad\forall g,g_1,g_2\in G, h,h_1,h_2\in H
    \end{equation}
    \textit{Полупрямым произведением} $G\ltimes H$ называется группа, элементами которой являются всевозможные пары $\{(g,h)\}\in G\times H$ с умножением
    \begin{equation}
        (g_1,h_1)\cdot(g_2,h_2)=(g_1g_2,\phi_{g_2^{-1}}(h_1)h_2)
    \end{equation}
\end{defin}
В полупрямом произведении $G\ltimes H$ есть подгруппы изоморфные $G$ и $H$, но только подгруппа изоморфная $H$ является нормальной. Смысл отображения $\phi_g$ -- это сопряжение $H$ посредством элемента $g \in G$.
\begin{defin}
    \textit{Внутренний автоморфизм}, определённый элементом $a\in G$, -- отображение сопряжением:
    \begin{equation}
        f_a:G\rightarrow G,\quad f_a(x)=a^{-1}xa
    \end{equation}
\end{defin}
Композиция двух внутренних автоморфизмов снова является внутренним автоморфизмом и набор всех внутренних автоморфизмов группы $G$ сам по себе тоже является группой и обозначается $\text{Inn}(G)$.\\
$\text{Inn}(G)$ является нормальной подгруппой полной группы автоморфизмов $\text{Aut}(G)$ группы $G$. Группа внешних автоморфизмов $\text{Out}(G)$ -- это факторгруппа $\text{Out}(G) = \text{Aut}(G)/\text{Inn}(G)$.
Группа внешних автоморфизмов отражает, насколько много автоморфизмов $G$ являются внутренними.
\begin{defin}
    Центр группы $G$ -- множество элементов группы, коммутирующих со всеми её элементами:
    \begin{equation}
        Z(G)=\{z\in G|\forall g\in G\hookrightarrow zg=gz\}
    \end{equation}
\end{defin}
Центр группы всегда является её подгруппой: всегда содержит нейтральный элемент (т.к. он коммутирует с любым элементом группы по определению), замкнут относительно групповой операции и вместе с входящими элементами содержит их обращения. Центр группы всегда является её нормальной подгруппой, поскольку он замкнут относительно сопряжения.\\
\begin{equation}
    G/Z(G)\simeq \text{Inn}(G)
\end{equation}
\textit{Примеры:}
\begin{enumerate}
    \item Центр абелевой группы $G$: $Z(G)=G$.
    \item Центр неабелевой простой группы $G$ тривиален: $Z(G)=\{e\}$.
    \item Центр полной линейной группы $GL_n(\mathbb{F})$ -- множество скалярных матриц: $Z(GL_n(\mathbb{F}))=\{sI_n|s\in\mathbb{F}\setminus\{0\}\}$.
    \item Центр ортогональной группы $O(n)$ является $\{-I_n,I_n\}$.
\end{enumerate}
Если у группы $G$ есть нормальная подгруппа $N$, то по ней можно отфакторизовать $G/N$ и сама группа может оказаться равной 
\begin{itemize}
    \item произведению групп $G=G/N\times N$. Тогда $N$ и $G/N$ -- нормальные подгруппы $G$ и изучение $G$ сводится к изучению $N$ и $G/N$.
    \item полупрямому произведению групп $G=G/N\ltimes N$.
    \item ни тому, ни другому. Тогда будем говорить, что группа $G$ получается <<склейкой>> $N$ и $G/N$.
\end{itemize}
Линейная группа $GL(n,\mathbb{F})$ не является простой, у неё есть нормальная подгруппа (и даже центр) $Z(GL_n(\mathbb{F}))=\{sI_n|s\in\mathbb{F}\setminus\{0\}\}$. Факторгруппа по ней является простой группой -- проективной линейной группой:
\begin{equation}
    PGL(n,\mathbb{F})=GL(n,\mathbb{F})/\{sI_n|s\in\mathbb{F}\setminus\{0\}\}
\end{equation}
Специальная линейная группа $SL(2,\mathbb{R})$ не является простой, у неё есть нормальная подгруппа (и даже центр) $Z(SL(2,\mathbb{R}))=\left\{\begin{pmatrix}
    1 & 0\\
    0 & 1
\end{pmatrix},\begin{pmatrix}
    -1 & 0\\
    0 & -1
\end{pmatrix}\right\}\simeq\mathbb{Z}_2$. Факторгруппа по ней является простой группой -- проективной специальной линейной группой:
\begin{equation}
    PSL(2,\mathbb{R})=SL(2,\mathbb{R})/\mathbb{Z}_2
\end{equation}
\subsection{Однородное пространство}
Часто в физике группы возникают как группы симметрий какой-то физической задачи в пространстве. Среди движений пространства можно выделить трансляции и движения сохраняющие некоторую точку, которую удобно считать началом координат. Также движения разделяются на движения сохраняющие ориентацию (они также называются собственными) и несохраняющие ориентацию. Все движения пространства описаны:
\begin{itemize}
    \item Движения плоскости, сохраняющие начало координат: повороты вокруг $R_\alpha$ начала координат на угол $\alpha$ (сохраняют ориентацию) и симметрии $S_l$ относительно прямых $l$, проходящих через начало координат (меняют ориентацию).
    \item Движения трехмерного пространства, сохраняющие начало координат: повороты $R_{l,\alpha}$ вокруг осей, проходящих через начало координат (сохраняют ориентацию) и \textit{зеркальные повороты} $S_{\pi,\alpha}$ -- композиции симметрии относительно плоскости $\pi$ и поворота вокруг перпендикулярной ей оси на угол $\alpha$, плоскость и ось пересекаются в начале координат, зеркальные повороты меняются ориентацию.
\end{itemize}
Конечные группы возникают как подгруппы группы симметрий. Примеры:
\begin{itemize}
    \item Диэдральная группа $D_n$ -- группа симметрий правильного $n$-угольника.
    \item Группа симметрий правильного многогранника, т.е. группа движений трехмерного пространства переводящих правильный многогранник в себя.
\end{itemize}
Однородное пространство неформально можно описать как пространство, в котором все точки одинаковы, то есть существует симметрия пространства, переводящая любую точку в другую.
\begin{defin}
Действие $G$ на $X$ \textit{транзитивно}, если $\forall x,y\in X\hookrightarrow\exists g\in G: \rho(g,x)=y$. Т.е. все элементы $X$ лежат в одной орбите.
\end{defin}
\begin{defin}
\textit{Однородное} пространство -- непустое многообразие или топологическое пространство $X$, на котором $G$ действует транзитивно.
\begin{itemize}
    \item Элементы $X$ -- \textit{точки} однородного пространства.
    \item Элементы $G$ -- \textit{симметрии} пространства, сама группа $G$ -- \textit{группа движений} или \textit{основная группа} однородного пространства.
    \item Если множество $X$ наделено дополнительной структурой, например, метрикой, топологией или гладкой структурой, то обычно предполагается, что действие $G$ сохраняет эту структуру. Например, в случае метрики действие предполагается изометрическим. Аналогично, если $X$ является гладким многообразием, то элементы группы являются диффеоморфизмами.
\end{itemize}
\end{defin}
Однородное пространство с основной группой $G$ можно отождествить со множеством правых классов смежности стабилизатора $H$: $\mathcal{M}\cong G/H$.\\
Примеры метрических однородных пространств:
\begin{itemize}
    \item Евклидово пространство $\mathbb{E}^n$ с действием группы изометрий; стабилизатор этого действия -- группа $O(n)$ ортогональных преобразований.
    \item Стандартная сфера $\mathbb{S}^n$ со следующими действиями:
    \begin{itemize}
        \item Группы $O(n)$ ортогональных преобразований; стабилизатор этого действия изоморфен группе $O(n-1)$.
        \item Группы $SO(n)$ -- специальной ортогональной группы; стабилизатор этого действия изоморфен группе $SO(n-1)$.
    \end{itemize}
\end{itemize}
\subsection{Проективное пространство и преобразование}
Понятие проективного пространства возникло из визуального эффекта перспективы, когда кажется, что параллельные линии пересекаются на бесконечности. Его можно рассматривать как расширение евклидова пространства или, в общем случае, аффинного пространства с бесконечно удаленными точками.
\begin{defin}
Для векторного пространства $V$ над полем $K$ \textit{проективное пространство $P(V)$} -- фактормножество $(V\setminus\{0\})/\sim$ с отношением эквивалентности $x\sim y$, если $\exists\lambda\in K$: $x = \lambda y$. Если $\dim V=n+1$, то $\dim P(V)=n$, а само проективное пространство обозначается $KP^n$.\\
Эквивалентно, проективное пространство $P(V)$ -- пространство, состоящее из одномерных подпространств (прямых, проходящих через 0) векторного пространства $V$. Прямые пространства $V$ над $K$ -- точки проективного пространства $P(V)$.
\end{defin}
Классификация при малых $n$:
\begin{itemize}
    \item $n=0$: пространство состоит из 1 точки.
    \item $n=1$: проективная прямая $\mathbb{R}P^1=\mathbb{R}^1\cup\{\infty\; удал.\; точка\}$.\\ Вещественная $\mathbb{R}P^1$ диффеоморфна окружности $S^1$. Комплексная $\mathbb{C}P^1$ (сфера Римана) диффеоморфна двумерной сфере $S^2$. Кватернионная $\mathbb{H}P^1$ диффеоморфна $S^4$.
    \item $n=2$: проективная плоскость $\mathbb{R}P^2=\mathbb{R}^2\cup\mathbb{R}P^1$.\\
    Проективная плоскость состоит из прямых, проходящих через начало координат. Прямая в проективной плоскости -- плоскость, проходящая через начало координат. Из определения следует, что все прямые в $\mathbb{R}P^2$ персекаются.
\end{itemize}
\begin{defin}
\textit{Проективное преобразование плоскости} -- взаимно-однозначное отображение $\phi$: $\pi \to \pi$ проективной плоскости $\pi$  на себя, при котором для любой прямой $l\in \pi$  образ $\phi(l)$ также является прямой.
\end{defin}
\subsection{Геометрия Лобачевского}
Гиперболическая геометрия является неевклидовой геометрией. Постулат параллельности евклидовой геометрии заменяется следующим: <<$\forall$ заданной прямой $l$ и точки $P$, не лежащей на $l$, в плоскости, содержащей $P$ и $l$, $\exists$ по крайней мере 2 различные прямые, проходящие через $P$, которые не пересекают $P$>>.\\
Отсюда следует, что через $P$ проходит бесконечное число компланарных прямых, не пересекающих $l$. Эти непересекающиеся линии делятся на два класса:
\begin{itemize}
    \item Две из прямых являются предельными параллелями: по одной в направлении каждой из идеальных точек на <<концах>> $l$, асимптотически приближающихся к $l$, но не пересекающие её.
    \item Все остальные непересекающиеся линии имеют точку минимального расстояния и расходятся с обеих сторон этой точки и называются \textit{ультрапараллельными}.
\end{itemize}
Одиночные линии в гиперболической геометрии обладают точно такими же свойствами, как отдельные прямые в евклидовой геометрии. Например, две точки однозначно определяют линию, а сегменты линии можно удлинять до бесконечности. Две пересекающиеся прямые обладают теми же свойствами, что и две пересекающиеся прямые в евклидовой геометрии. Например, две различные прямые могут пересекаться не более чем в одной точке, пересекающиеся прямые образуют равные противоположные углы.\\
Когда вводится третья линия, то могут быть свойства пересекающихся линий, отличные от пересекающихся линий в евклидовой геометрии. Например, для данных двух пересекающихся прямых существует бесконечно много прямых, не пересекающих ни одну из данных прямых.\\
Существуют несколько моделей, используемых для гиперболической геометрии.
\subsubsection{Модель диска Пуанкаре}
Модель диска Пуанкаре представляет собой модель двумерной гиперболической геометрии, в которой все точки находятся внутри единичного диска, а прямые линии представляют собой либо дуги окружности, содержащиеся внутри диска, которые ортогональны окружности, либо диаметры единичного круга.\\
Группа сохраняющих ориентацию изометрий модели диска задается проективной специальной унитарной группой $PSU(1,1)=SU(1,1)/\{\pm I\}$.



\subsubsection{Модель полуплоскости Пуанкаре}
Модель полуплоскости Пуанкаре -- половина евклидовой плоскости $H=\{(x,y)|y>0;x,y\in\mathbb{R}\}$. Модель полуплоскости является пределом модели диска Пуанкаре, граница которого касается оси абсцисс в той же точке, а радиус модели диска стремится к бесконечности.\\
Гиперболические линии представляют собой либо полуокружности, ортогональные оси абсцисс, либо лучи, перпендикулярные оси абсцисс. Длина интервала на луче задается логарифмической мерой, поэтому она инвариантна относительно гомотетий $(x,y)\rightarrow(\lambda x,\lambda y),\lambda>0$.\\
Эта модель является конформной, т.е. углы, измеренные в точке в модели такие же, как и в реальной гиперболической плоскости. Все изометрии в рамках этой модели -- преобразования Мёбиуса плоскости ($z=x+iy$)
\begin{equation}
    f(z)=\frac{az+b}{cz+d},\quad ad-bc>0
\end{equation}
$\forall$ преобразованию $f(z)$ соответствует матрица
\begin{equation}
    M=\begin{pmatrix}
        a & b\\
        c & d
    \end{pmatrix}\in SL(2,\mathbb{R})
\end{equation}
Композиции отображений $f_2(f_1(z))$ соответствует матричное умножение $M_2M_1$:
\begin{equation}
    f_2(f_1(z))=\frac{a_2(\frac{a_1z+b_1}{c_1z+d_1})+b_2}{c_2(\frac{a_1z+b_1}{c_1z+d_1})+d_2}=\frac{(a_1a_2+b_2c_1)z+a_2b_1+b_2d_1}{(a_1c_2+c_1d_2)z+c_2b_1+d_2d_1}
\end{equation}
\begin{equation}
    M_2M_1=\begin{pmatrix}
        a_2 & b_2\\
        c_2 & d_2
        \end{pmatrix}\begin{pmatrix}
        a_1 & b_1\\
        c_1 & d_1
    \end{pmatrix}=\begin{pmatrix}
        a_1a_2+b_2c_1 & a_2b_1+b_2d_1\\
        a_1c_2+c_1d_2 & c_2b_1+d_2d_1
    \end{pmatrix}
\end{equation}
Также сущестует симметрия $(a,b,c,d)\rightarrow(-a,-b,-c,-d)$, не меняющая $f(z)$, т.е. группа изометрий -- $PSL(2,\mathbb{R})=SL(2,\mathbb{R})/\{\pm I\}$ -- проективная специальная линейная группа.
\subsubsection{Модель Бельтрами-Клейна}
Модель Бельтрами-Клейна -- модель гиперболической геометрии, в которой точки представлены точками внутри единичного диска, а прямые изображаются хордами.\\
Эта модель не является конформной, что означает, что углы искажены, а окружности на гиперболической плоскости в модели вообще не являются круглыми. Только круги, центр которых находится в центре граничного круга, не искажаются.
\subsubsection{Модель полушария}
Модель полушария использует верхнюю половину единичной сферы $\{(x,y,z)|x^2+y^2+z^2=1,z>0\}$. Гиперболические линии представляют собой полуокружности, ортогональные границе полусферы.\\
Модель полусферы является частью сферы Римана и разные проекции дают разные модели гиперболической плоскости:
\begin{itemize}
    \item Стереографическая проекция из $(0,0,-1)$ на плоскость $z=0$ проецирует соответствующие точки на модель диска Пуанкаре.
    \item Стереографическая проекция из $(-1,0,0)$ на плоскость $x=1$ проецирует соответствующие точки на модель полуплоскости Пуанкаре.
    \item Ортогональная проекция на плоскость $z=C$ проецирует соответствующие точки на модель Бельтрами–Клейна.
\end{itemize}

\subsection{Группы Ли}
\begin{defin}
\textit{Группой Ли} называется группа $G$ со структурой гладкого многообразия,
совместной с групповыми операциями. Это означает, что
\begin{enumerate}
    \item отображение умножения $m: G \times G \rightarrow G, \;m(g, h) := gh$ гладко;
    \item отображение обращения $s : G \rightarrow G,\; s(g) := g^{-1}$ гладко.
\end{enumerate}
\end{defin}
Определение означает, что элементы группы должны быть представлены как функции $g(\alpha_1,...,\alpha_d)$ от какого-то набора вещественных параметров $\alpha_1,...,\alpha_d$. Требуется, чтобы матричные элементы как функции от $\alpha_1,...,\alpha_d$ были гладкими. Также требуется, чтобы функции $\gamma_i(\alpha_1,...,\alpha_d,\beta_1,...,\beta_d)$, определенные при помощи умножения в группе
\begin{equation}
    g(\alpha_1,...,\alpha_d)g(\beta_1,...,\beta_d)=g(\gamma_1,...,\gamma_d)
\end{equation}
были гладкими. Аналогично, требуется, чтобы функции $\delta_i(\alpha_1,...,\alpha_d)$, определённые при помощи операции взятия обратного элемента в группе
\begin{equation}
    g(\alpha_1,...,\alpha_d)^{-1}=g(\delta_1,...,\delta_d)
\end{equation}
были гладкими.\\
Правильно думать, что параметры $\alpha_1,...,\alpha_d$ принадлежат некоторому открытому подмножеству в $\mathbb{R}^d$ и $g$ осуществляет гладкую биекцию между этим открытым множеством и окрестностью единицы в группе $G$.
\begin{defin}
\textit{Размерностью} группы Ли называется число параметров $d$.
\end{defin}
%В частности, всякий элемент $g \in G$ задает следующие диффеоморфизмы: $L_g$: $G \rightarrow G$, $L_g(h)=gh$ и $R_g$: $G \rightarrow G$, $R_g(h)=hg$.\\
Примеры:
\begin{enumerate}
    \item любая дискретная (в частности, конечная) группа;
    \item группа $\mathbb{R}$ по сложению;
    \item группа $\mathbb{R}_+$ положительных вещественных чисел по умножению;
    \item группа $S^1$ комплексных чисел, по модулю равных 1, по умножению.
\end{enumerate}
\begin{defin}
\textit{Подгруппой Ли} в группе Ли называется замкнутое гладкое подмногообразие, замкнутое относительно операции умножения.
\end{defin}
Примеры:
\begin{enumerate}
    \item Подгруппа $SL_n(\mathbb{R}) \subset GL_n(\mathbb{R})$ или $SL_n(\mathbb{C}) \subset GL_n(\mathbb{C})$;
    \item Подгруппа $N_+(n)\subset GL_n$, состоящая из верхнетреугольных матриц с единицами на диагонали;
    \item Подгруппа $B_+(n) \subset GL_n$, состоящая из обратимых верхнетреугольных матриц;
    \item Подгруппа $O_n(\mathbb{R}) \subset GL_n(\mathbb{R})$, состоящая из операторов, сохраняющих евклидово скалярное произведение (т.е. $A$: $A^{-1} = A^T$), а также $SO_n(\mathbb{R}) := O_n(\mathbb{R}) \cap SL_n(\mathbb{R})$;
    \item Более общо, $O_{n,k}(\mathbb{R}) \subset GL_{n+k}(\mathbb{R})$ -- подгруппа, сохраняющая квадратичную форму сигнатуры $(n, k)$.
\end{enumerate}
Рассмотрим подробнее некоторые матричные группы Ли:
\begin{enumerate}
    \item $SO(2)$ --  группа ортогональных преобразований плоскости с определителем 1. Геометрически элементы группы $SO(2)$ -- повороты на углы $0\leq\alpha < 2\pi$, матрица имеет вид
    \begin{equation}
        g(\alpha)=\begin{pmatrix}
        \cos\alpha & -\sin\alpha\\
        \sin\alpha & \cos\alpha
        \end{pmatrix}
    \end{equation}
    Матрицы $g(\alpha)$ удовлетворяют соотношению $g(\alpha+\beta)=g(\alpha)g(\beta)$. Если продифференцировать последнее равенство по $\beta$ и положить $\beta=0$, то получаем
    \begin{equation}
        g'(\alpha)=g(\alpha)g'(0)=g(\alpha)\begin{pmatrix}
        0 & -1\\
        1 & 0
        \end{pmatrix}
    \end{equation}
    Единственным решение этого уравнения, удовлетворяющим начальному условию $g(0)=E$, является матричная экспонента
    \begin{equation}
        g(\alpha)=\exp\left(\alpha\begin{pmatrix}
        0 & -1\\
        1 & 0
        \end{pmatrix}\right)=\begin{pmatrix}
        \cos\alpha & -\sin\alpha\\
        \sin\alpha & \cos\alpha
        \end{pmatrix}
    \end{equation}
    В данном случае матрица $\begin{pmatrix}
    0 & -1\\
    1 & 0\end{pmatrix}$
    является инфинитезимальным генератором группы $SO(2)$, т.к. имеем соотношение $g(\alpha)=\exp(\alpha g'(0))$. При этом $g'(0)$ должна удовлетворять соотношению $\exp(2\pi g'(0))=E$, поэтому её собственные значения равны $ik_1,...,ik_N$, где $k_1,...,k_N\in\mathbb{Z}$.
    \item Группа всех невырожденных матриц $GL_n(\mathbb{R})$. В качестве параметров $\alpha_1,...,\alpha_d$ можно взять все матричные элементы. Размерность группы равна $n^2$. Аналогично, группа всех невырожденных комплексных матриц $GL_n(\mathbb{C})$ имеет вдвое большую размерность $2n^2$.
    \item Группа матриц с единичным определителем $SL_n(\mathbb{R})$. Она задается одним уравнением $\det(g)-1=0$. По теореме о неявной функции можно взять $n^2-1$ матричных элементов и тогда оставшийся выражается через них при помощи гладкой функции. Эти $n^2-1$ элементов и можно взять в качестве локальных параметров, размерность группы равна $n^2-1$. Единственное, что надо проверить, что дифференциал не равен нулю. На более конкретном языке это означает, что есть ненулевая частная производная.\\
    Проверим это сначала для случая $n=2$. Тогда
    \begin{equation}
        \delta(\det g-1)=g_{11}\delta g_{22}+g_{22}\delta g_{11}-g_{12}\delta g_{21}-g_{21}\delta g_{12}
    \end{equation}
    Мы видим, что $\delta(\det g-1) = 0$ только если все матричные элементы $g$ равны нулю, но такая матрица не лежит в $SL_2(\mathbb{R})$.\\
    Для произвольной матрицы $g$ легко видеть, что $\delta \det g=\sum\limits_{i,j}G^{ij}\delta g_{ij}$, где $G^{ij}$ -- алгебраическое дополнение к матричному элементу $g_{ij}$. Т.к. $\det g = 1$, то одно из этих дополнений не равно 0, значит дифференциал невырожден.\\
    В частности в точке $g=E$, мы имеем
    \begin{equation}
        \delta \det g=\delta g_{11}+...+\delta g_{nn}=\text{Tr} \delta g
    \end{equation}
    Т.е. мы получили, что частные производные по координатам $g_{ii}$ не равны нулю, в качестве локальных координат можно взять все координаты, кроме любой из них.
    \item Через $O(n)$ обозначается группа всех ортогональных матриц. Ортогональные матрицы задаются уравнением $gg^T=E$. Т.к. матрица $gg^T$ -- симметрична, то уравнение $gg^T=E$ являет собой $\frac{n(n+1)}{2}$ уравнений на матричные элементы матрицы $g$, которых всего $n^2$. По теореме о неявной функции матрицы из $O(n)$ могут локально быть выражены через $n^2-\frac{n(n+1)}{2}=\frac{n(n-1)}{2}$ параметров.\\
    Но для того, чтобы применить теорему о неявной функции надо проверить, что дифференциалы этих уравнений линейно независимы. Рассмотрим малое приращение $g=E+t\delta g+o(t)$. Подставляя это в уравнение на $g$, мы получаем, что в первом порядке $\delta g+\delta g^T=0$. Это система линейных уравнений на $\delta g$, её решения это кососимметричные матрицы, которые образуют пространство размерности $\frac{n(n-1)}{2}$. Значит ранг системы равен $\frac{n(n+1)}{2}$ и совпадает с количеством уравнений, что и требовалось показать.
    \item Через $SO(n)$ обозначается подгруппа $O(n)$, состоящая из ортогональных матриц с определителем 1. Соотношение $gg^T=E$ означает, что $\det g=\pm 1$. Поэтому условие единичного детерминанта исключает матрицы с детерминантом $-1$, но не уменьшает число параметров $\frac{n(n-1)}{2}$, которые требуются для характеристики матрицы $g$. Другими словами, $\dim SO(n)=\dim O(n)=\frac{n(n-1)}{2}$.
    \item Через $U(n)$ обозначается группа всех унитарных матриц. Унитарные матрицы задаются уравнением $gg^*=E$. Т.к. матрица $gg^*$ -- эрмитова, то уравнение $gg^*=E$ являет собой $n^2$ уравнений на матричные элементы матрицы $g$, которых всего $n^2$. По теореме о неявной функции матрицы из $U(n)$ могут локально быть выражены через $2n^2-n^2=n^2$ параметров.\\
    Но для того, чтобы применить теорему о неявной функции надо проверить, что дифференциалы этих уравнений линейно независимы. Рассмотрим малое приращение $g=E+t\delta g+o(t)$. Подставляя это в уравнение на $g$, мы получаем, $gg^*=(E+t\delta g+o(t))(E+t\delta g^*+o(t))=E+t(\delta g+\delta g^*)+o(t)=E$ что в первом порядке даёт $\delta g+\delta g^*=0$. Это система линейных уравнений на $\delta g$, её решения это антиэрмитовы матрицы, которые образуют пространство размерности $n^2$. Значит ранг системы равен $n^2$ и совпадает с количеством уравнений, что и требовалось показать.
    \item Через $SU(n)$ обозначается подгруппа $U(n)$, состоящая из унитарных матриц с определителем 1. Соотношение $gg^*=E$ означает, что $|\det g|=1$. Поэтому условие единичного детерминанта накладывает одно дополнительное ограничение на матрицу $g$. Вычитая 1 из числа $n^2$ параметров унитарной матрицы, находим, что унитарная матрица с 1 детерминантом характеризуется $n^2-1$ параметрами. Другими словами, $\dim SU(n)=\dim U(n)-1=n^2-1$.
\end{enumerate}
Выпишем отдельно размерности групп Ли:
\begin{table}[h!]
    \centering
    \begin{tabular}{|c|c|c|c|}\hline
     Группа Ли над $\mathbb{R}$   & Размерность & Группа Ли над $\mathbb{C}$    & Размерность\\ \hline
     $GL(n,\mathbb{R})$ & $n^2$ & $GL(n,\mathbb{C})$ & $2n^2$\\ \hline
     $SL(n,\mathbb{R})$ & $n^2-1$ & $SL(n,\mathbb{C})$ & $2n^2-2$\\ \hline
     $O(n)$ & $\frac{n(n-1)}{2}$ & & \\ \hline
     $SO(n)$ & $\frac{n(n-1)}{2}$ & & \\ \hline
     & & $U(n)$ & $n^2$\\ \hline
     & & $SU(n)$ & $n^2-1$\\ \hline
    \end{tabular}
    \caption{Размерности групп Ли}
    \label{tab:my_label}
\end{table}
\begin{defin}
\textit{Гомоморфизмом} групп Ли называется гладкий гомоморфизм групп. Биективный гомоморфизм называется \textit{изоморфизмом}.
\end{defin}
Примеры:
\begin{enumerate}
    \item Отображение $\mathbb{R}\rightarrow\mathbb{R}_+$, $t\rightarrow e^t$ -- изоморфизм групп Ли.
    \item Отображение $\mathbb{R}\rightarrow S^1$, $t \rightarrow e^{2\pi it}$ -- гомоморфизм групп Ли с ядром $\mathbb{Z} \subset \mathbb{R}$.
    \item Отображение det: $GL(n,\mathbb{R})\rightarrow \mathbb{R}\setminus\{0\}$ -- гомоморфизм групп Ли с ядром $SL(n,\mathbb{R})$.
\end{enumerate}
Рассмотрим все возможные гладкие кривые $g(t)$, где $g(0)=E$. При малых $t$ эта кривая имеет вид $g(t)=E+At+o(t)$, где $A=g'(0)$.
\begin{defin}
Множество таких $A$ называется \textit{касательным пространством} к $G$ в точке $E$ и обозначается $T_EG$. 
\end{defin}
Заметим, что $T_EG$ является векторным пространством. Действительно, если есть две кривые $g_1(t)=E+A_1t+o(t)$ и $g_2(t)=E+A_2t+o(t)$, то их прозведение имеет вид $g_1(t)g_2(t)=E+(A_1+A_2)t+o(t)$. Значит, если $A_1,A_2\in T_E G$, то $A_1+A_2\in T_EG$. Кроме того, если рескалироть параметр $t$, т.е. взять кривую $g_3(t)=g_1(\lambda t)=E+\lambda A_1t+o(t)$, то мы получаем, что если $A_1\in T_EG$, то $\lambda A_1\in T_EG$.\\
Укажем, что это за векторные пространства для примеров выше. В случае группы $G=SO(2)$ оно порождено матрицей $\begin{pmatrix}
0 & -1\\
1 & 0
\end{pmatrix}$. В случае группы $G=SL(n,\mathbb{R})$ мы нашли, что матрица $A$ должна удовлетворять условию $\text{Tr}A = 0$. В случае $G = SO(n)$ матрица $A$ антисимметрична $A=-A^T$. В случае $G = SU(n)$ матрица $A$ антиэрмитова $A=-A^*$.\\
Аналогично можно определить касательное пространство к любой точке $g\in G$ (подобно тому как есть касательное пространство к сфере в любой её точке). Это касательное пространство обозначается $T_gG$, оно всегда будет векторным пространством, для этого структура группы на самом деле не нужна. Но для следующих свойств $T_EG$ структура группы уже является необходимой.\\
Рассмотрим гладкую кривую $g(t)=E+At+o(t)$. Тогда $\forall h\in G$ кривая $\tilde{g}(t)=hg(t)h^{-1}=E+hAh^{-1}t+o(t)$ тоже является гладкой и $\tilde{g}(0)=E$. То есть, мы доказали, что если $A\in T_EG$ и $h\in G$, то элемент $hAh^{-1}\in T_EG$. Значит, пространство $T_EG$ имеет структуру представления группы $G$. Такое представление есть для любой группы Ли G, оно называется \textit{присоединенным представлением}.\\
Пусть теперь элемент $h$ также зависит от параметра, другими словами, рассмотрим кривую $h(s) = E + Bs + o(s)$. Тогда $\forall s\hookrightarrow h(s)Ah(s)^{-1}\in T_EG$.
\begin{equation}
    h(s)Ah(s)^{-1} = (E + Bs + o(s))A(E-Bs + o(s)) = A + (BA-AB)s + o(s)
\end{equation}
Дифференцируя по $s$, мы получаем, что $BA-AB\in T_EG$. Это выражение называется коммутатором матриц $B$, $A$ и обозначается $[B, A]$.\\
Резюмируя, мы получили, что векторное пространство $T_EG$ является замкнутым относительно действия группы $G$ сопряжениями и взятия коммутатора.
\begin{defin}
\textit{Представлением} группы Ли $G$ в векторном пространстве $V$ называется гомоморфизм групп Ли $\rho: G \rightarrow GL(V)$.
\end{defin}
Примеры:
\begin{enumerate}
    \item \textit{Тавтологическое} представление $GL_n(\mathbb{R})$ в пространстве $\mathbb{R}^n$.
    \item \textit{Присоединенное} представление $GL_n(\mathbb{R})$ в пространстве $n\times n$-матриц сопряжениями.
    \item Ограничение этих представлений на какую-либо подгруппу Ли в $GL_n(\mathbb{R})$.
\end{enumerate}
\begin{defin}
Пусть $G$ -- группа Ли. $\forall a\in G$ рассмотрим
\begin{equation}
    \text{Ad}_a(\xi)=\frac{d}{dt}(ae^{\xi t}a^{-1})|_{t=0}
\end{equation}
Полученное действие
\begin{equation}
    \text{Ad}:G\rightarrow Gl(\mathfrak{g}),\quad \text{Ad}(a)=\text{Ad}_a
\end{equation}
называется \textit{присоединённым представлением}. Образ группы Ли $G$ при присоединённом представлении называется \textit{присоединённой группой группы $G$} и обозначается $\text{Ad}\;G$.
\end{defin}
Если $G\subset GL(V)$ -- линейная группа в пространстве $V$, то
\begin{equation}
    \text{Ad}_aX=aXa^{-1}
\end{equation}
\begin{predl}
    Всякая одномерная группа Ли изоморфна $\mathbb{R}$ по сложению или $S^1$ по умножению.
\end{predl}
\subsection{Алгебры Ли}
\begin{defin}
Пусть $A$ -- векторное пространство над полем $K$, снабжённое билинейной операцией $\cdot:A\times A\rightarrow A$ (умножением), т.е. $\forall x,y,z\in A,a,b\in K$
\begin{enumerate}
    \item $(x+y)\cdot z=x\cdot z+y\cdot z$
    \item $x\cdot(y+z)=x\cdot y+x\cdot z$
    \item $(ax)\cdot(by)=ab(x\cdot y)$
\end{enumerate}
Тогда $A$ называется \textit{алгеброй над $K$}.
\end{defin}
\begin{defin}
Алгебра называется \textit{ассоциативной}, если операция умножения в ней ассоциативна.
\end{defin}
\begin{defin}
\textit{Алгеброй Ли} называется векторное пространство с билинейной операцией (коммутатором) $[\cdot,\cdot]$, удовлетворяющей следующим аксиомам:
\begin{enumerate}
    \item (кососимметричность или антикоммутативность) $[x, y] = -[y, x]$.
    \item (тождество Якоби) $[x, [y, z]] + [y, [z, x]] + [z, [x, y]] = 0$.
\end{enumerate}
\end{defin}
Например, коммутатор матриц $[A, B] = AB - BA$ удовлетворяет антикоммутативности и тождеству Якоби.
\begin{defin}
    \textit{Дифференцированием} алгебры $A$ называется линейный оператор $D: A \rightarrow A$, для которого выполнено тождество Лейбница: $D(a\cdot b) = D(a)\cdot b + a\cdot D(b)$.
\end{defin} 
Все дифференцирования алгебры A образуют векторное пространство.
\begin{predl}
Пусть $D_1$, $D_2$ -- дифференцирования алгебры A. Тогда $D_1D_2-D_2D_1$ -- тоже дифференцирование алгебры A.
\end{predl}
\begin{defin}
\textit{Центром алгебры} Ли $\mathfrak{g}$ называется подпространство $\mathfrak{Z}(\mathfrak{g})\subset \mathfrak{g}$, состоящее из элементов $x \in \mathfrak{g}:[x,y]=0\;\forall y\in\mathfrak{g}$.
\end{defin}
\begin{defin}
Пусть $\mathfrak{g}_1$, $\mathfrak{g}_2$ -- алгебры Ли. Линейное отображение $\varphi:\mathfrak{g}_1 \rightarrow \mathfrak{g}_2$ называется \textit{гомоморфизмом алгебр Ли}, если $\varphi([x, y]) = [\varphi(x), \varphi(y)]\;\forall x,y\in\mathfrak{g}_1$. \textit{Ядром} гомоморфизма $\varphi:\mathfrak{g}_1 \rightarrow \mathfrak{g}_2$ называется подпространство $\text{Ker}\;\varphi = \{x \in\mathfrak{g}_1| \varphi(x) = 0\}$. \textit{Образом} гомоморфизма $\varphi: \mathfrak{g}_1 \rightarrow \mathfrak{g}_2$ называется подпространство $\text{Im}\;\varphi = \{\varphi(x)| x\in \mathfrak{g}_1\}\subset\mathfrak{g}_2$. Взаимно однозначный гомоморфизм алгебр Ли называется \textit{изоморфизмом алгебр Ли}.
\end{defin}
\begin{defin}
С каждым элементом $x$ алгебры Ли $\mathfrak{g}$ можно связать присоединенный оператор $ad_x:\mathfrak{g}\rightarrow\mathfrak{g}$, переводящий $y\in\mathfrak{g}$ в $[x, y]\in \mathfrak{g}$ для всевозможных $y \in \mathfrak{g}$.
\end{defin}
\subsection{Представления алгебр Ли}
\begin{defin}
\textit{Представлением} алгебры Ли $\mathfrak{g}$ в векторном пространстве $V$ называется пара $(V,\rho)$, где гомоморфизм алгебр Ли $\rho:\mathfrak{g}\rightarrow\mathfrak{gl}(V)$ (т.е. $\forall x,y\in\mathfrak{g}\hookrightarrow\rho([x,y])=[\rho(x),\rho(y)]$).
\end{defin}
\begin{defin}
\textit{Подпредставлением} представления $(V,\rho)$ называется подпространство $V'\subset V$, инвариантное относительно всех операторов $\rho(x),\;x \in\mathfrak{g}$.
\end{defin}
\begin{defin}
Представление называется \textit{неприводимым}, если любое его подпредставление есть либо $V$, либо 0.
\end{defin}
\begin{defin}
Пусть $(V_1, \rho_1)$ и $(V_2, \rho_2)$ -- представления алгебры Ли $\mathfrak{g}$. \textit{Прямая сумма} этих представлений -- это представление в пространстве $V_1 \oplus V_2:\rho(x)(v_1 \oplus v_2) = \rho_1(x)v_1\oplus \rho_2(x)v_2$.
\end{defin}
\begin{defin}
Представление называется \textit{неразложимым}, если оно не представляется в виде прямой суммы собственных подпредставлений.
\end{defin}
\begin{defin}
Представление называется \textit{вполне приводимым}, если каждое его подпредставление выделяется прямым слагаемым (т.е. для любого подпредставления $V'\subset V$ существует подпредставление $V''\subset V$ такое, что $V = V' \oplus V''$).
\end{defin}
Примеры представлений:
%ДОПИСАТЬ!!!
\begin{defin}
Пусть $(V_1, \rho_1)$ и $(V_2, \rho_2)$ -- представления алгебры Ли $\mathfrak{g}$. Тензорное произведение этих представлений -- это представление алгебры Ли $\mathfrak{g}$ в пространстве $V_1\otimes V_2$, в котором элементы алгебры Ли действуют по правилу Лейбница
\begin{equation}
    \rho(x)=\rho_1(x)\otimes1+1\otimes\rho_2(x)\quad\forall x\in\mathfrak{g}
\end{equation}
\end{defin}
\subsection{Универсальная обёртывающая алгебра}
Построение универсальной обёртывающей алгебры пытается процесс построения $A_L$ из $A$: для данной алгебры Ли $\mathfrak{g}$ над $K$ находят <<наиболее общую>> ассоциативную $K$-алгебру $U(\mathfrak{g}):$ алгебра Ли $U_L(\mathfrak{g})$ содержит $\mathfrak{g}$. Важное ограничение -- сохранение теории представлений: представления $\mathfrak{g}$ соотносятся точь-в-точь так же как и модули над $U(\mathfrak{g})$. В типичном контексте, где $\mathfrak{g}$ задаётся инфинитезимальными преобразованиями, элементы $U(\mathfrak{g})$ действуют как дифференциальные операторы всех порядков.
\begin{defin}
\textit{Универсальной обертывающей алгеброй} алгебры Ли $\mathfrak{g}$ называется пара $(U(\mathfrak{g}),\varepsilon)$, где $U(\mathfrak{g})$ -- ассоциативная алгебра с единицей, $\varepsilon:\mathfrak{g}\rightarrow U_L(\mathfrak{g})$ -- гомоморфизм алгебр Ли (т.е. $\varepsilon([x,y])=[\varepsilon(x),\varepsilon(y)]$), обладающий следующим \textit{универсальным свойством}: $\forall$ ассоциативной алгебры $A$ и гомоморфизма алгебр Ли $\varphi:\mathfrak{g}\rightarrow A_L$ (т.е. $\varphi([x,y])=[\varphi(x),\varphi(y)]$) $\exists !$ гомоморфизм ассоциативных алгебр $\Phi:U(\mathfrak{g}) \rightarrow A:\varphi=\Phi\circ\varepsilon$ (рис. 1).
\end{defin}
\begin{figure}
    \centering
    \includegraphics[scale=0.3]{1.jpg}
    \caption{Универсальное свойство}
    \label{fig:my_label}
\end{figure}
Из универсального свойства, в частности, следует, что любое представление алгебры Ли $\mathfrak{g}$ $\varphi:\mathfrak{g}\rightarrow \text{End}(V)$ несет структуру представления ассоциативной алгебры $U(\mathfrak{g})$, причем любой гомоморфизм представлений алгебры Ли $\mathfrak{g}$ есть также и гомоморфизм представлений $U(\mathfrak{g})$. Для этого нужно в качестве ассоциативной алгебры взять $A=\text{End}(V)$.
\begin{predl}
Универсальная обёртывающая алгебра единственна с точностью до изоморфизма.
\begin{proof}
    От противного, пусть $\exists$ две универсальные обёртывающие алгебры $(U_1(\mathfrak{g}),\varepsilon_1)$ и $(U_2(\mathfrak{g}),\varepsilon_2)$. Тогда из универсального свойства (рис. 2) следует единственность.
\end{proof}
\end{predl}
\begin{figure}
    \centering
    \includegraphics[scale=0.3]{2.jpg}
    \caption{Единственность универсальной обёртывающей}
    \label{fig:my_label}
\end{figure}
\begin{predl}
    Универсальная обёртывающая алгебра $\exists$ $\forall$ алгебры Ли $\mathfrak{g}$.
    \begin{proof}
        Зададим алгебру $U(\mathfrak{g})$ явно образующими и соотношениями. Пусть $T(\mathfrak{g})=\mathbb{C}\oplus \mathfrak{g}\oplus\mathfrak{g}\otimes \mathfrak{g}\oplus...$ -- тензорная алгебра пространства $\mathfrak{g}$ (т.е. свободная ассоциативная алгебра, порождённая пространством $\mathfrak{g}$) и пусть $J\subset T(\mathfrak{g})$ -- двусторонний идеал, порождённый элементами $x\otimes y-y\otimes x-[x,y]\;\forall x,y\in \mathfrak{g}$. Тогда ассоциативная алгебра $U(\mathfrak{g}):=T(\mathfrak{g})/J$ с тождественным отображением $\epsilon:\mathfrak{g}\rightarrow\mathfrak{g}\subset T(\mathfrak{g})$ обладает требуемым универсальным свойством. Иначе говоря, пусть $x_1,...,x_n$ -- базис в алгебре Ли $\mathfrak{g}$ и пусть $[x_i,x_j]=\sum\limits_{k=1}^nc^k_{ij}x_k$. Тогда $U(\mathfrak{g})$ -- ассоциативная алгебра с образующими $x_1,...,x_n$ и определяющими соотношениями $x_ix_j-x_jx_i=\sum\limits_{k=1}^nc^k_{ij}x_k$, причём $\epsilon(x_i)=x_i$: $U(\mathfrak{g})=\mathbb{C}\braket{x_1,...,x_n}/(x_ix_j-x_jx_i-\sum\limits_kc^k_{ij}x_k)$.
    \end{proof}
\end{predl}
\textit{Пример.} Для абелевой группы Ли $L$ с базисом $x_1,...,x_n$ универсальная обёртывающая алгебра $U(L)$ -- симметрическая алгебра $S(L):=T(L)/(x_ix_j-x_jx_i)$, т.е. алгебра многочленов $\mathbb{C}[x_1,...,x_n]$.
\begin{predl}
    Пусть $x_1,...,x_n$ -- базис алгебры Ли $\mathfrak{g}$. Элементы вида $x_1^{k_1}x_2^{k_2}...x_n^{k_n}$ образуют полную систему в универсальной обёртывающей алгебре $U(\mathfrak{g})$ (любой элемент $U(\mathfrak{g})$ может быть линейно выражен через такие упорядоченные мономы).
    \begin{proof}
        Мономы вида $x_{i_1}x_{i_2}...x_{i_N}$ образуют полную систему в универсальной обёртывающей алгебре $U(\mathfrak{g})$. Пусть $\sum\limits_{i=1}^nk_i=N$ -- степень упорядоченного монома. Докажем утверждение по индукции.
        \begin{itemize}
            \item База индукции: $N=1$, верно.
            \item Пусть верно для всех $\sum\limits_{i=1}^nk_i<N$.
            \item Докажем, что верно для $\sum\limits_{i=1}^nk_i=N$. Если в какой-то части монома $x_{i_1}x_{i_2}...x_{i_N}$ индексы расположены не по возрастанию, то их можно переставить, используя коммутационное соотношение $x_{i_1}x_{i_2}=x_{i_2}x_{i_1}+[x_{i_1},x_{i_2}]$. При этом получится упорядоченный моном и мономы меньшего размера, для которых утверждение доказано.
        \end{itemize}
    \end{proof}
\end{predl}
\begin{theorem}[Пуанкаре-Биркгофа-Витта]
Если $x_1,...,x_n$ -- базис в алгебре Ли $\mathfrak{g}$, то мономы $x_1^{k_1}x_2^{k_2}...x_n^{k_n}$ образуют базис в пространстве $U(\mathfrak{g})$.\\
Эквивалентная формулировка. Пусть $i:S(\mathfrak{g})\rightarrow T(\mathfrak{g})$ -- вложение (симметризация):
\begin{equation}
    i(v_1\cdot v_2\cdot...\cdot v_n)=\frac{1}{n!}\sum\limits_{\sigma\in S_n}v_{\sigma(1)}\otimes v_{\sigma(2)}\otimes...\otimes v_{\sigma(n)}
\end{equation}
Пусть $\tau:T(\mathfrak{g})\rightarrow U(\mathfrak{g})$ -- отображение факторизации. Тогда $\sigma=\tau\circ i$ -- отображение симметризации -- является изоморфизмом векторных пространств и представлений алгебры Ли $\mathfrak{g}$ (см. рис. 3). На самом деле это изоморфизм $\mathfrak{g}$-модулей относительно присоединённого действия.
\end{theorem}
%\subsection{Модуль Верма}
\section{Задачи к зачёту}
\begin{enumerate}
    \item Постройте сюрьективный гомоморфизм $SL(2,\mathbb{R})$ на $SO(2,1)$. Найдите ядро.
    \item То же для групп $SL(2,\mathbb{C})$ и $SO(3,1)$.
    \item Найдите максимальную компактную подгруппу в $SL(2, \mathbb{C})$ и покажите,что факторпространство изоморфно пространству Лобачевского.
    \item Покажите, что на факторпространстве из предыдущей задачи имееся единственная инвариантная метрика. Найдите все инвариантные дифференциальные операторы (метрика единственна с точностью до константы). Найдите геодезические.
    \item Найдите все подмодули в тензорном произведении двумерного представления $\mathfrak{sl}(2)$ на модуль Верма (ответ зависит от старшего веса модуля Верма).
    \item Докажите что квадратичный казимир порождает центр универсальной обертывающей алгебры $\mathfrak{sl}(2)$.
    \item Найдите кубический центральный элемент в универсальной обертывающей для $\mathfrak{sl}(3)$.
    \item Найдите 2 модуля Верма для $\mathfrak{sl}(2)$, тензорное произведение которых не вполне приводимо.
    \item Покажите, что на группе $SU(2)$ имеется биинвариантная метрика. Найдите геодезические.
    \item Функции на группе $SU(2)$ -- представление той же группы, которая действует сопряжением. Разложить его на неприводимые.
    \item Универсальная обертывающая $\mathfrak{sl}(2)$ -- представление $\mathfrak{sl}(2)$, которое действует посредством коммутатора. Разложить его на неприводимые.
\end{enumerate}
\end{document}

%https://math.hse.ru/bac3-12-lie, https://math.hse.ru/lie_2015
