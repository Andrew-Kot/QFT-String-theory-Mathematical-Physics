\documentclass[12pt]{article}

% report, book
%  Русский язык

%\usepackage{bookmark}

\usepackage[T2A]{fontenc}			% кодировка
\usepackage[utf8]{inputenc}			% кодировка исходного текста
\usepackage[english,russian]{babel}	% локализация и переносы
\usepackage[title,toc,page,header]{appendix}
\usepackage{amsfonts}
\usepackage{hyperref,bookmark}


% Математика
\usepackage{amsmath,amsfonts,amssymb,amsthm,mathtools} 
%%% Дополнительная работа с математикой
%\usepackage{amsmath,amsfonts,amssymb,amsthm,mathtools} % AMS
%\usepackage{icomma} % "Умная" запятая: $0,2$ --- число, $0, 2$ --- перечисление

\usepackage{cancel}%зачёркивание
\usepackage{braket}
%% Шрифты
\usepackage{euscript}	 % Шрифт Евклид
\usepackage{mathrsfs} % Красивый матшрифт


\usepackage[left=2cm,right=2cm,top=1cm,bottom=2cm,bindingoffset=0cm]{geometry}
\usepackage{wasysym}

%размеры
\renewcommand{\appendixtocname}{Приложения}
\renewcommand{\appendixpagename}{Приложения}
\renewcommand{\appendixname}{Приложение}
\makeatletter
\let\oriAlph\Alph
\let\orialph\alph
\renewcommand{\@resets@pp}{\par
  \@ppsavesec
  \stepcounter{@pps}
  \setcounter{subsection}{0}%
  \if@chapter@pp
    \setcounter{chapter}{0}%
    \renewcommand\@chapapp{\appendixname}%
    \renewcommand\thechapter{\@Alph\c@chapter}%
  \else
    \setcounter{subsubsection}{0}%
    \renewcommand\thesubsection{\@Alph\c@subsection}%
  \fi
  \if@pphyper
    \if@chapter@pp
      \renewcommand{\theHchapter}{\theH@pps.\oriAlph{chapter}}%
    \else
      \renewcommand{\theHsubsection}{\theH@pps.\oriAlph{subsection}}%
    \fi
    \def\Hy@chapapp{appendix}%
  \fi
  \restoreapp
}
\makeatother
\newtheorem{theorem}{Теорема}
\newtheorem{predl}[theorem]{Предложение}
\newtheorem{sled}[theorem]{Следствие}

\theoremstyle{definition}
\newtheorem{zad}{Задача}[section]
\newtheorem{upr}[zad]{Упражнение}
\newtheorem{defin}[theorem]{Определение}

\title{Решение заданий\\ ОП "Квантовая теория поля, теория струн и математическая физика"\\[2cm]
Введение в теорию групп (4 семестр, М.А. Берштейн)}
\author{Коцевич Андрей Витальевич, группа Б02-920}
\date{\today\; Версия 12.0}

\begin{document}

\maketitle
\newpage
\newpage
\tableofcontents{}
\newpage
\maketitle
\section*{Введение}
Все задания, которые я присылаю, выполнены и написаны мной самостоятельно!
\newpage
\section{Определение группы. Группа перестановок.}
\begin{upr}
\begin{itemize}
\item[a)] $\alpha=(1,3,5)(2,4,7)$, $\beta=(1,4,7)(2,3,5,6)$.
\begin{equation}
\alpha\beta=\left(
\begin{array}{ccccccc}
1 & 2 & 3 & 4 & 5 & 6 & 7\\
4 & 3 & 5 & 7 & 6 & 2 & 1\\
7 & 5 & 1 & 2 & 6 & 4 & 3\\
\end{array}
\right)=\left(
\begin{array}{ccccccc}
1 & 2 & 3 & 4 & 5 & 6 & 7\\
7 & 5 & 1 & 2 & 6 & 4 & 3\\
\end{array}
\right)
\end{equation}
\begin{equation}
    \boxed{\alpha\beta=(1,7,3)(2,5,6,4)}
\end{equation}
\item[б)]
\begin{equation}
    \left(
    \begin{array}{ccccccc}
    1 & 2 & 3 & 4 & 5 & 6 & 7\\
    5 & 6 & 4 & 7 & 2 & 1 & 3\\
    \end{array}
    \right)=(1,5,2,6)(3,4,7)
\end{equation}
\begin{equation}
    \boxed{\text{ord}\left(
    \begin{array}{ccccccc}
    1 & 2 & 3 & 4 & 5 & 6 & 7\\
    5 & 6 & 4 & 7 & 2 & 1 & 3\\
    \end{array}
    \right)=\text{НОК}(4,3)=12}
\end{equation}
где $\text{ord}\;\sigma$ -- порядок элемента $\sigma$.
\end{itemize}
\end{upr}
\begin{zad}
\begin{itemize}
\item[a)]
\begin{predl}
Любая транспозиция является нечётной перестановкой.
\end{predl}
\begin{proof}
Пусть $\sigma$ -- транспозиция. По определению транспозиции:
\begin{equation}
    \sigma=
    \left(
\begin{array}{ccccccc}
1 & \hdots & a & \hdots & b & \hdots & n\\
1 & \hdots & b & \hdots & a & \hdots & n\\
\end{array}
\right)
\end{equation}
Число инверсий $|\sigma|=1+2(b-a-1)$, т.к. инверсиями будут следующие пары чисел: первое число из множества чисел между $a$ и $b$ (их $b-a-1$ штук), второе число из множества $\{a,b\}$ (таких инверсий $2(b-a-1)$ штук), а также $a$ и $b$ (1 инверсия). Тогда по определению $\sigma$ является нечётной перестановкой.
\end{proof}
\item[б)] Для начала докажем следующее предложение:
\begin{predl}\label{pr1}
Умножение на транспозицию меняет чётность перестановки.
\end{predl}
\begin{proof}
Пусть $\sigma$ -- транспозиция. По определению транспозиции:
\begin{equation}
    \sigma=
    \left(
\begin{array}{ccccccc}
1 & \hdots & a & \hdots & b & \hdots & n\\
1 & \hdots & b & \hdots & a & \hdots & n\\
\end{array}
\right)
\end{equation}
Транспозицию $\sigma$ можно разложить в элементарные транспозиции. Их количество: $1+2(b-a-1)$ ($b-a-1$ элементарную транспозицию нужно сделать, чтобы $a$ и $b$ стали соседними, потом 1 транспозицию, чтобы поменять их местами, и ещё $b-a-1$, чтобы вернуть $a$ и $b$ на новое место). Каждая элементарная транспозиция меняет чётность, значит нечётное их число также меняет чётность. 
\end{proof}
\begin{theorem}\label{pr2}
Пусть перестановка разложена в произведение транспозиций. Тогда её чётность равна чётности количества этих транспозиций.
\end{theorem}
\begin{proof}
Пусть количество транспозиций $k$. Докажем утверждение по индукции. База: $k=0$ (0 транспозиций дают чётную перестановку). Прежположим, что для $k-1$ утверждение верно. Тогда для $k$ утверждение верно, т.к. домножение на транспозицию меняет чётность перестановки (по предложению \ref{pr1}), а числа $k-1$ и $k$ также имеют противоположные чётности.
\end{proof}
\item[в)] Пусть $\sigma=(i_1,...,i_d)$. Разложим $\sigma$ в произведение транспозиций: $\sigma=(i_1,i_d)(i_1,i_{d-1})...(i_1,i_2)$. Всего $d-1$ транспозиций. Если число транспозиций $d-1$ чётно, то и $\sigma$ чётная (по теореме \ref{pr2}); если $d-1$ нечётно, то и $\sigma$ нечётная ($|\sigma|=d-1$). Таким образом, если $d$ и $\sigma$ имеют противоположные чётности.
\end{itemize}
\end{zad}
\begin{zad}
Ответ на вопрос про собственные числа даёт характеристическое уравнение:
\begin{equation}
    \text{det}(R(\sigma)-\lambda E)=0
\end{equation}
Пусть перестановка $\sigma$ расладывается в произведение циклов $\sigma=\prod\limits_{i=1}^l\sigma_i$, где все $\sigma_i$ -- циклы, $l$ -- их число.
\begin{defin}
\textit{Циклический тип перестановки} -- данные о том, сколько циклов каждой длины присутствует в разложении перестановки через циклы.
\end{defin}
Пусть длина цикла $\sigma_i$ равна $k_i$, количество циклов длины $k$ в произведении равно $m_k$. Разумеется, $\sum\limits_{i=1}^l k_i=\sum\limits_{k=1}^n m_kk=n$. Характеристическое уравнение инвариантно относительно выбора базиса. Значит выберем его так, чтобы матрица стала блочно-диагональной матрицей с блоками $R(\sigma_i)$, соответствующими $i$ циклу:
\begin{equation}
    R(\sigma)=\left(
\begin{array}{cccc}
R(\sigma_1) & 0 & \ldots & 0\\
0 & R(\sigma_2) & \ldots & 0\\
\vdots & \vdots & \ddots & \vdots\\
0 & 0 & \ldots & R(\sigma_l)
\end{array}
\right)
\end{equation}
Матрица $R(\sigma_i)_{k_i\times k_i}$ будет выглядеть так:
\begin{equation}
    R(\sigma_i)=\left(
\begin{array}{cccccc}
0 & 1 & 0 & \ldots & 0 & 0\\
0 & 0 & 1 & \ldots & 0 & 0\\
0 & 0 & 0 & \ldots & 0 & 0\\
\vdots & \vdots &\vdots & \ddots & \vdots & \vdots\\
0 & 0 & 0 & \ldots & 0 & 1\\
1 & 0 & 0 & \ldots & 0 & 0
\end{array}
\right)
\end{equation}
\begin{equation}
    \text{det}(R(\sigma)-\lambda E)=\prod_{i=1}^l\text{det}(R(\sigma_i)-\lambda E)
\end{equation}
Определитель посчитаем, раскрыв по первому столбцу:
\begin{equation}
    \left|
\begin{array}{cccccc}
-\lambda & 1 & 0 & \ldots & 0 & 0\\
0 & -\lambda & 1 & \ldots & 0 & 0\\
0 & 0 & -\lambda & \ldots & 0 & 0\\
\vdots & \vdots &\vdots & \ddots & \vdots & \vdots\\
0 & 0 & 0 & \ldots & -\lambda & 1\\
1 & 0 & 0 & \ldots & 0 & -\lambda
\end{array}
\right|=(-\lambda)\left|
\begin{array}{cccc}
-\lambda & 1 & \ldots  & 0\\
0 & -\lambda & \ldots  & 0\\
\vdots &\vdots & \ddots & \vdots\\
0 & 0 & \ldots  & -\lambda
\end{array}
\right|+(-1)^{k_i-1}\left|
\begin{array}{cccc}
1 & 0 & \ldots  & 0\\
-\lambda & 1 & \ldots  & 0\\
\vdots &\vdots & \ddots & \vdots\\
0 & 0 & \ldots & 1
\end{array}
\right|
\end{equation}
\end{zad}
\begin{equation}
\text{det}(R(\sigma_i)-\lambda E)=(-\lambda)(-\lambda)^{k_i-1}+(-1)^{k_i-1}=(-1)^{k_i}(\lambda^{k_i}-1)
\end{equation}
\begin{equation}
    \text{det}(R(\sigma)-\lambda E)=\prod_{i=1}^l(-1)^{k_i}(\lambda^{k_i}-1)=(-1)^{\sum\limits_{i=1}^l k_i}\prod_{i=1}^l(\lambda^{k_i}-1)=(-1)^n\prod_{i=1}^l(\lambda^{k_i}-1)
\end{equation}
Сгруппируем множители:
\begin{equation}
    \text{det}(R(\sigma)-\lambda E)=(-1)^n\prod_{k=1}^n(\lambda^k-1)^{m_k}=0
\end{equation}
Теперь мы можем выписать все собственные значения. Как видно, все $\lambda_k$ -- корни из 1 степени $k$:
\begin{equation}
    \boxed{\lambda_{k,t}=\exp\left(\frac{2\pi t}{k}i\right), \quad t\in\{0,...,k-1\}}
\end{equation}
$m_k$ -- количество $\lambda_{k,0},...,\lambda_{k,k-1}$. Всего собственных значений $\sum\limits_{k=1}^nm_kk=n$, как и должно быть (матрица $n\times n$ содержит $n$ комплексных собственных чисел).
\begin{zad}[$^*$]
Для начала заметим, что при $n=1$ перестановка всего одна -- тождественная. Для неё нет понятия инверсии, т.к. для нет $i$ и $j$ ($i<j$). Элемент всего 1 и 2 разных выбрать нельзя.\\
Далее будем рассматривать случай $n\geq 2$ (обычно такой случай сразу и рассматривается). Пусть число чётных перестановок $n_1$, число нечётных $n_2$. Домножим все чётные перестановки на любую транспозицию (например, (1, 2)). Все они станут нечётными, а значит $n_1\leq n_2$. Делая то же самое с нечётными перестановками, получим неравенство $n_2\leq n_1$. Таким образом, в группе $S_n$ число чётных перестановок равно числу нечётных: $n_1=n_2=\frac{n!}{2}$ ($n!$ чётен при $n\geq2$).
%Для начала докажем следующее предложение:
%Расположим все $n!$ перестановок без повторений в таком порядке, чтобы соседние перестановки отличались на одну транспозицию (а значит имели противоположные чётности). Сначала запишем любую перестановку (уже есть $1!$), затем перестановку, в нижней строке которой поменяем местами предпоследний и последний элементы (уже есть $2!$), затем остальные перестановки, в которых переставлены только последние 3 элемента: переносим левый из 3 на 1 вправо и делаем перестановку остальных двух, потом опять его вправо и перестановку остальных (уже $3!$) и т.д. аналогичным образом, соблюдая правило про чётность соседних перестановок. Таким образом, все перестановки можно выписать в ряд, в котором за каждой чётной перестановкой стоит нечётная и за каждой нечётной чётная. Т.к. $n!$ чётно при $n\geq 2$, то число чётных и нечётных перестановок совпадает (и равно $\frac{n!}{2}$).
\end{zad}
\section{Абелевы группы. Действие группы на множестве.}
\begin{upr}
\begin{itemize}
    \item[a)] В группе $Z^*_{10}=\{1,3,7,9\}$ 2 элемента имеют порядок 4: 3 ($3^4=81=1 \text{ mod } 10$) и 7 ($7^4=2401=1\text{ mod }10$), а значит $Z^*_{10}\simeq C_4$ (по предложению 3 лекции 2).
    \item[б)] $Z^*_{15}=\{1,2,4,7,8,11,13,14\}$. Группа не является циклической, поскольку в ней не существует элемента, степени которого порождают всю группу (порядка 8): $\text{ord } 1=1$, $\text{ord } 2=4$, $\text{ord } 4=2$, $\text{ord } 7=4$, $\text{ord } 8=4$, $\text{ord } 11=2$, $\text{ord } 13=4$, $\text{ord } 14=2$.
\end{itemize}
\end{upr}
\begin{upr}
    $|G|=48$ -- порядок группы симметрий куба. По теореме 12 лекции 2: $|Gx||G_x|=|G|=48$.
\begin{itemize}
    \item[а)] Пусть $x\in X$ -- произвольное ребро. Симметриями оно может быть переведено в любое другое, а значит $|Gx|=12$ (в кубе 12 рёбер). $|G_x|=\frac{|G|}{|Gx|}=4$. Найдём эти 4 симметрии: тождественное, поворот на $180^\circ$ вокруг прямой $l$, проходящей через центр ребра $x$ и центр его противоположного на и эти повороты со симметрией относительно плоскости, проходящей через $l$ и параллельной 2 граням куба. Повороты образуют группу, изоморфную $C_2$ (два поворота на $180^\circ$ эквивалентны тождественному), симметрии тоже. Таким образом, 
    \begin{equation}
        \boxed{G_x\simeq C_2\times C_2\simeq D_2}
    \end{equation}
    \item[б)] Пусть $x\in X$ -- произвольная грань. Симметриями она может быть переведена в любую другую, а значит $|Gx|=6$ (в кубе 6 граней). $|G_x|=\frac{|G|}{|Gx|}=8$. Найдём эти 8 симметрий. Это будет группа симметрий квадрата (диэдра $D_4$): тождественное, повороты на $90^\circ$, $180^\circ$, $270^\circ$ вокруг прямой $l$, проходящей через центр грани $x$ и центр её противоположной и эти повороты со симметрией относительно плоскости, проходящей через $l$ и параллельной 2 граням куба. Таким образом,
    \begin{equation}
        \boxed{G_x\simeq D_4 \simeq C_2\ltimes C_4}
    \end{equation}
\end{itemize}
\end{upr}
\begin{upr}
Порядок группы $S_3$: $|S_3|=3!=6$. По теореме 12 лекции 2: $|S_3x||{S_3}_x|=|S_3|=6$. Следовательно, стабилизатор группы ${S_3}_x$ (являющийся подгруппой $S_3$ по предложению 10 лекции 2), имеет порядок, являющийся делителем 6. Всего 4 варианта: $1, 2, 3, 6$.
\begin{equation}
    {S_3}_x=\{g\in S_3|gxg^{-1}=x\}
\end{equation}
Найдём элементы $g$, удовлетворяющие соотношению $gxg^{-1}=x$. Домножим на $g$ справа:
\begin{equation}
    gx=xg
\end{equation}
Среди таких $g$ могут быть: $e$ ($eg=ge=g$), $x$ ($xx=x^2$), $x^{-1}$ ($xx^{-1}=x^{-1}x=e$). Их уже 3 штуки. Рассмотрим несколько случаев:
\begin{enumerate}
    \item $e=x=x^{-1}$. С $e$ коммутирует любой элемент группы, следовательно $|{S_3}_e|=6$ и \begin{equation}
        \boxed{{S_3}_e=S_3}
    \end{equation} 
    Со всей группой в $S_3$ коммутирует только $e$ (это будет показано в п. 2), а значит при $x\neq e$ $|{S_3}_x<6|$. Из этого и того, что варианта всего 4 (1, 2, 3, 6) следует, что при $x\neq e$ $|{S_3}_x\leq 3|$.
    \item $x=x^{-1}\rightarrow x^2=e$. Такими элементами являются транспозиции: $(1,2)$, $(1,3)$, $(2,3)$. Может оказаться, что порядок их стабилизаторов не 2, а 3. Проверим, что транспозиции с другими элементами не коммутируют на примере $(1,2)$ (для $(1,3)$ и $(2,3)$ всё аналогично):
    \begin{equation}
        (1,2)(1,3)=\left(
    \begin{array}{ccc}
    1 & 2 & 3\\
    3 & 2 & 1\\
    3 & 1 & 2\\
    \end{array}
    \right)=(1,3,2)\neq (1,2,3)=\left(
    \begin{array}{ccc}
    1 & 2 & 3\\
    2 & 1 & 3\\
    2 & 3 & 1\\
    \end{array}
    \right)=(1,3)(1,2)
    \end{equation}
    \begin{equation}
        (1,2)(2,3)=\left(
    \begin{array}{ccc}
    1 & 2 & 3\\
    1 & 3 & 2\\
    2 & 3 & 1\\
    \end{array}
    \right)=(1,2,3)\neq (1,3,2)=\left(
    \begin{array}{ccc}
    1 & 2 & 3\\
    2 & 1 & 3\\
    3 & 1 & 2\\
    \end{array}
    \right)=(2,3)(1,2)
    \end{equation}
    \begin{equation}
        (1,2)(1,2,3)=\left(
    \begin{array}{ccc}
    1 & 2 & 3\\
    2 & 3 & 1\\
    1 & 3 & 2\\
    \end{array}
    \right)=(2,3)\neq (1,3)=\left(
    \begin{array}{ccc}
    1 & 2 & 3\\
    2 & 1 & 3\\
    3 & 2 & 1\\
    \end{array}
    \right)=(1,2,3)(1,2)
    \end{equation}
    \begin{equation}
        (1,2)(1,3,2)=\left(
    \begin{array}{ccc}
    1 & 2 & 3\\
    3 & 1 & 2\\
    3 & 2 & 1\\
    \end{array}
    \right)=(1,3)\neq (2,3)=\left(
    \begin{array}{ccc}
    1 & 2 & 3\\
    2 & 1 & 3\\
    1 & 3 & 2\\
    \end{array}
    \right)=(1,3,2)(1,2)
    \end{equation}
    Следовательно $|S_{3_{(1,2)}}|=|S_{3_{(1,3)}}|=|S_{3_{(2,3)}}|=2$ и
    \begin{equation}
        \boxed{S_{3_{(1,2)}}=\{e,(1,2)\},\quad S_{3_{(1,3)}}=\{e,(1,3)\},\quad S_{3_{(2,3)}}=\{e,(2,3)\}}
    \end{equation}
    \item $e\neq x\neq x^{-1}$. Такими элементами являются оставшиеся циклы длины 3: $(1,2,3)$, $(1,3,2)$ ($(1,2,3)^{-1}=(1,3,2)$). Значит, $|S_{3_{(1,2,3)}}|=|S_{3_{(1,3,2)}}|=3$ (мы уже предоставили 3 разных $g$, а больше быть не может) и
    \begin{equation}
        \boxed{S_{3_{(1,2,3)}}=S_{3_{(1,3,2)}}=\{e,(1,2,3),(1,3,2)\}}
    \end{equation}
\end{enumerate}
Таким образом, мы рассмотрели все 6 перестановок и нашли их стабилизаторы.\\
Определение орбиты:
\begin{equation}
    S_3x=\{y\in S_3|\exists g\in S_3: y=gxg^{-1}\}
\end{equation}
Рассмотрим также 3 случая:
\begin{enumerate}
    \item $x=e$, $|{S_3}_e|=6$: $|S_3e|=\frac{|S_3|}{|{S_3}_e|}=1$. Значит нужно найти единственный элемент $y\in S_3e$: $y=geg^{-1}=gg^{-1}=e$.
    \begin{equation}
        \boxed{S_3e=\{e\}}
    \end{equation}
    \item Транспозиции, $|{S_3}_x|=2$: $|S_3x|=\frac{|S_3|}{|{S_3}_x|}=3$. Значит нужно найти 3 элемента $y\in S_3x$: $y=gxg^{-1}$. Для примера возьмём $x=(1,2)$. $g=(1,2)$: $y=(1,2)(1,2)(1,2)^{-1}=(1,2)$; $g=(1,3)$: $y=(1,3)(1,2)(1,3)^{-1}=(2,3)$; $g=(2,3): y=(2,3)(1,2)(2,3)^{-1}=(1,3)$. Все 3 различных элемента $y$ найдены. Орбиты $(1,3)$ и $(2,3)$ точно такие же по предложению 11 лекции 2 (они пересекаются, а значит должны совпадать).
    \begin{equation}
        \boxed{S_3(1,2)=S_3(1,3)=S_3(2,3)=\{(1,2),(1,3),(2,3)\}}
    \end{equation}
    \item Циклы длины 3, $|{S_3}_x|=3$: $|S_3x|=\frac{|S_3|}{|{S_3}_x|}=2$. Значит нужно найти 2 элемента $y\in S_3x$: $y=gxg^{-1}$. Для примера возьмём $x=(1,2,3)$. $g=(1,2,3)$: $y=(1,2,3)(1,2,3)(1,2,3)^{-1}=(1,2,3)$; $g=(1,2)$: $y=(1,2)(1,2,3)(1,2)^{-1}=(1,3,2)$. Оба различных элемента $y$ найдены. Орбита $(1,3,2)$ точно такая же по предложению 11 лекции 2 (они пересекаются, а значит должны совпадать).
    \begin{equation}
        \boxed{S_3(1,2,3)=S_3(1,3,2)=\{(1,2,3),(1,3,2)\}}
    \end{equation}
    Множество орбит:
    \begin{equation}
        \boxed{S_3/S_3=\{\{e\},\{(1,2),(1,3),(2,3)\},\{(1,2,3),(1,3,2)\}\}}
    \end{equation}
\end{enumerate}
\end{upr}
\begin{zad}
Пусть $|G|=54$ -- абелева группа. Найдём все возможные такие группы. Воспользуемся теоремой 7 лекции 2:
\begin{equation}
    G=C_{n_1}\times...\times C_{n_l}
\end{equation}
При этом $\prod\limits_{i=1}^ln_i=54$. Разложим 54 на простые множители: $54=2\cdot 3^3$. Воспользуемся теоремой 7 лекции 2 для того, чтобы найти все неизоморфные группы:
\begin{enumerate}
    \item Группа $C_{2}\times C_{27}\simeq C_{54}$ (т.к. $\text{НОД}(2,27)=1$).
    \item Группа $C_{2}\times C_{3}\times C_9\simeq C_{6}\times C_9$ (т.к. $\text{НОД}(2, 3)=1$).
    \item Группа $C_{2}\times C_{3}\times C_3\times C_3\simeq C_{6}\times C_3\times C_3$ (т.к. $\text{НОД}(2, 3)=1$).
\end{enumerate}
Всего 3 неизоморфные абелевы группы (они неизоморфны, поскольку $\text{НОД}(3, 3)=3\neq 1$).
\end{zad}
\begin{zad}
\begin{predl}
Для любой точки $x\in X$ стабилизатор $G_x$ является подгруппой в $G$.
\begin{proof}
Проверим свойства подгруппы:
\begin{enumerate}
    \item Пусть $g_1, g_2\in G_x$, тогда $g_1x=x$, $g_2x=x$. $(g_1g_2)x=g_1(g_2x)=g_1x=x$, поэтому $g_1g_2\in G_x$.
    \item Пусть $g\in G_x$, тогда $gx=x$. $x=g^{-1}gx=g^{-1}x$, поэтому $g^{-1}\in G_x$.
\end{enumerate}
\end{proof}
\end{predl}
\end{zad}
\section{Теорема Лагранжа, классы сопряженности, нормальные подгруппы, полупрямое произведение.}
\begin{upr}
\begin{predl}
\begin{equation}\label{eq1}
    (a_1,b_1)\cdot(a_2,b_2)=(a_1a_2,\phi_{a_2^{-1}}(b_1)b_2)
\end{equation}
Полупрямое произведение, заданное формулой (\ref{eq1}) является группой.
\end{predl}
\begin{proof}
Проверим свойства группы:
\begin{enumerate}
    \item Ассоциативность:\\
    $((a_1,b_1)\cdot(a_2,b_2))\cdot(a_3,b_3)=(a_1a_2,\phi_{a_2^{-1}}(b_1)b_2)\cdot(a_3,b_3)=(a_1a_2a_3,\phi_{a_3^{-1}}(\phi_{a_2^{-1}}(b_1)b_2)b_3)=(a_1a_2a_3,\phi_{a_3^{-1}}(\phi_{a_2^{-1}}(b_1))\phi_{a_3^{-1}}(b_2)b_3)=(a_1a_2a_3,\phi_{a_3^{-1}a_2^{-1}}(b_1)\phi_{a_3^{-1}}(b_2)b_3)$.\\
    $(a_1,b_1)\cdot((a_2,b_2)\cdot(a_3,b_3))=(a_1,b_1)\cdot(a_2a_3,\phi_{a_3^{-1}}(b_2)b_3)=(a_1a_2a_3,\phi_{(a_2a_3)^{-1}}(b_1)\phi_{a_3^{-1}}(b_2)b_3)=(a_1a_2a_3,\phi_{a_3^{-1}a_2^{-1}}(b_1)\phi_{a_3^{-1}}(b_2)b_3)$.
    \item Существование единицы $(e,e)$:\\
    $(a,b)\cdot(e,e)=(ae,\phi_{e^{-1}}(b)e)=(a,b)$.\\
    $(e,e)\cdot(a,b)=(ea,\phi_{a^{-1}}(e)b)=(a,b)$.
    \item Существование обратного $(a^{-1},\phi_a(b^{-1}))$:\\
    $(a,b)\cdot(a^{-1},\phi_a(b^{-1}))=(aa^{-1},\phi_a(b)\phi_a(b^{-1}))=(e,e)$.\\
    $(a^{-1},\phi_a(b^{-1}))\cdot(a,b)=(a^{-1}a,\phi_{a^{-1}}(\phi_a(b^{-1}))b)=(e,e)$.
\end{enumerate}
\end{proof}
\end{upr}
\begin{zad}
\begin{itemize}
    \item[а)]
    \begin{predl}
    Пусть $G$ -- группа движений, сохраняющих правильный тетраэдр. Действие $G$ на множестве вершин тетраэдра задаёт изоморфизм $G$ и $S_4$.
    \end{predl}
    \begin{proof}
    Для доказательства изоморфизма выпишем соответствие между перестановками различных циклических типов в $S_4$ и различными движениями, сохраняющими правильный тетраэдр:
    \begin{enumerate}
        \item $e$ (единичный, 1 шт.) -- тождественное движение (ничего не делает с тетраэдром).
        \item $(a,b)$ (транспозиции, 6 шт.) -- отражение относительно плоскости $\pi$, проходящей через ребро (вершины которого остаются на месте) и центр противопложного ребра (вершины которого меняются местами).
        \item $(a,b)(c,d)$ (произведение транспозиций, 3 шт.) -- повороты на $\pi$ вокруг прямой $l_1$, проходящей через центры противоположных рёбер (вершины которых и будут меняться местами).
        \item $(a,b,c)$ (цикл длины 3, 8 шт.) -- поворот на $2\pi/3$ или $4\pi/3$ вокруг прямой $l_2$, проходящей через вершину и центр противоположной грани (вершины этой грани и будут циклически меняться).
        \item $(a,b,c,d)$ (цикл длины 4, 6 шт.) -- зеркальный повороте на $\pi/2$ с прямой $l_3$, проходящей через середины двух противоположных ребер и плоскостью симметрии, проходящей через середины остальных ребер.
    \end{enumerate}
    Таким образом, построено однозначное соответствие между перестановками и движениями, сохраняющее групповые операции, а значит изоморфизм групп построен.
    \end{proof}
    По предложению 7 лекции 3 перестановки сопряжены тогда и только тогда, когда имеют одинаковую циклическую структуру. А значит классами сопряжённости являются 5 видов движения, описанных выше.
    \item[б)] В $G_0$ будут входить движения вида 1, 3 и 4 (см. список выше), поскольку среди них нет отражений и зеркальных поворотов. Всего 12 штук. Поскольку $G_0$ состоит из классов сопряжённости, то она нормальна. Также можно заметить, что она изоморфна знакопеременной группе $A_4$, которая является нормальной подгруппой $S_4$.\\
    Проверим, какие классы сопряжённости из $G$ перейдут в $G_0$. Конечно, класс $\{e\}$ сохранится. Класс, состоящий из произведений транспозиций сохранится также, поскольку из одного произведения можно получить остальные 2 при сопряжениях с циклами длины 3. $(1,3,4)(1,2)(3,4)(1,3,4)^{-1}=(1,4)(2,3)$, $(1,4,3)(1,2)(3,4)(1,4,3)^{-1}=(1,3)(2,4)$ (это все случаи, если учесть, что $(1,3,4)^{-1}=(1,4,3)$) (для сокращения записи циклы написаны в $A_4$, а не в $G_0$).\\
    Рассмотрим циклы длины 3. $|G_0|=12$, а значит класс сопряжённости (орбита) не может состоять из 8 элементов (12 на 8 не делится). В $G_0$ невозможно получить каждый цикл длины 3 из каждого, т.е. циклы длины 3 создадут 2 класса сопряжённости в $G_0$, которые соответствуют поворотам на $\frac{2\pi}{3}$ и $\frac{4\pi}{3}$ вокруг прямой, проходящей через вершину и центр противоположной грани тетраэдра (по 4 в каждом классе): $\{(1,2,3),(1,4,2),(1,3,4),(2,4,3)\}$, $\{(1,3,2),(1,2,4),(2,3,4),(1,4,3)\}$ (для сокращения записи циклы написаны в $A_4$, а не в $G_0$).\\
    Таким образом, всего будет 4 класса сопряженности в $G_0$ (и соответственно в $A_4$).
\end{itemize}
\end{zad}
\begin{zad}
\begin{itemize}
    \item[а)] Группой симметрии правильного $n$-угольника является группа диэдра $D_n$ ($n$ поворотов и $n$ осевых отражений). Основания могут поменяться местами при помощи отражения относительно плоскости $\pi$, проходящей через середины боковых рёбер. Тождественная симметрия (которая ничего не делает с призмой) и отражение образуют группу $C_2$. Поскольку все симметрии из $D_{nh}$ можно представить в виде произведения элементов групп $D_n$ и $C_2$, группы $D_n$ и $C_2$ коммутативны и пересечение групп тривиально (по $e$), то
    \begin{equation}
        D_{nh}=D_n\times C_2
    \end{equation}
    \begin{equation}
        \boxed{|D_{nh}|=|D_n|\cdot|C_2|=4n}
    \end{equation}
    Порядок можно было получить проще: только при 2 симметриях вершина $x$ переходит сама в себя (тождественная и отражение относительно плоскости, проходящей через $x$). Значит $G_x=2$. Любая вершина при симметриях может перейти в любую, значит $Gx=2n$.
    \begin{equation}
        \boxed{|D_{nh}|=|G_x|\cdot|Gx|=4n}
    \end{equation}
    \item[б)]
    \begin{equation}
        \boxed{D_{3h}\simeq D_3\times C_2\simeq C_3\ltimes C_2\times C_2}
    \end{equation}
    Сначала докажем следующее предложение:
    \begin{predl}
        $D_6$ изоморфно $D_3\times C_2$.
    \end{predl}
    \begin{proof}
        Заметим, что $D_6$ -- группа симметрий правильного шестиугольника, а его вершины являются вершинами двух равносторонних треугольников. Группа симметрии правильного треугольника -- $D_3$. Один треугольник можно перевести в другой при помощи поворота на $\pi$ вокруг прямой $l$, проходящей через центр шестиугольника и перпендикулярной его плоскости. Тождественная симметрия и такой поворот образуют $C_2$. Поскольку все симметрии из $D_6$ можно представить в виде произведения элементов групп $D_3$ и $C_2$, группы $D_3$ и $C_2$ коммутативны и пересечение групп тривиально (по $e$), то утверждение доказано.
    \end{proof}
    Учтём, что $D_{3h}\simeq D_3\times C_2$ и получим
    \begin{equation}
        \boxed{D_{3h}\simeq D_3\times C_2\simeq D_6}
    \end{equation}
\end{itemize}
\end{zad}
\begin{zad}
\begin{itemize}
    \item[а] Движения получаются четырёх видов: трансляции относительно векторов решетки, симметрии относительно горизонтальных и вертикальных осей и центральные симметрии относительно точек половинной решётки (повороты на $\pi$ вокруг прямой, проходящей через центр решётки перпендикулярно ей, можно получить из симметрий).
    \begin{enumerate}
        \item Группа трансляций $T$ порождена сдвигами на порождающие вектора решётки $e_1$ и $e_2$. Её элемент:
        \begin{equation}
            t_{n_1,n_2,n_3,n_4}:\left\{
            \begin{array}{l}
            x_1 \rightarrow x_1+n_1e_1+n_2e_2\\
            x_2 \rightarrow x_2+n_3e_1+n_4e_2
            \end{array}
            \right.,\quad n_i\in\mathbb{Z}
        \end{equation}
        \item Общая симметрия относительно вертикальных осей:
        \begin{equation}
            s^v_{l_1,l_2,l_3,l_4}:\left\{
            \begin{array}{l}
            x_1 \rightarrow -x_1+l_1e_1+l_2e_2\\
            x_2 \rightarrow x_2+l_3e_1+l_4e_2
            \end{array}
            \right.,\quad l_i\in\mathbb{Z}
        \end{equation}
        где $x_1$, $x_2$ -- координаты вектора $x$.
       \item Общая симметрия относительно горизонтальных осей:
        \begin{equation}
            s^h_{k_1,k_2,k_3,k_4}:\left\{
            \begin{array}{l}
            x_1 \rightarrow x_1+k_1e_1+k_2e_2\\
            x_2 \rightarrow -x_2+k_3e_1+k_4e_2
            \end{array}
            \right.,\quad k_i\in\mathbb{Z}
        \end{equation}
        \item Общая центральная симметрия:
        \begin{equation}
            s_{m_1,m_2,m_3,m_4}: \left\{
            \begin{array}{l}
            x_1 \rightarrow -x_1+m_1e_1+m_2e_2\\
            x_2 \rightarrow -x_2+m_3e_1+m_4e_2
            \end{array}
            \right.,\quad m_i\in\mathbb{Z}
        \end{equation}
    \end{enumerate}
    \begin{predl}
         Подгруппа $T$ является нормальной
    \end{predl}
    \begin{proof}
        При сопряжениях с движениями любых видов трансляция перейдёт в трансляцию (коэффициенты перед $x_1$ и $x_2$ будут либо 1, либо $(-1)^2=1$, а значит $x\rightarrow x+a$, где $a$ -- некоторый вектор с целыми координатами).
    \end{proof}
    Движения различных типов ($T$, 3 вида симметрий) -- 4 класса смежности по $T$. Факторгруппа $G/T$ состоит из 4 элементов:
    \begin{enumerate}
        \item \begin{equation}
            e=t_{0,0,0,0}:x\rightarrow x
        \end{equation}
        \item 
        \begin{equation}
            s^v_{0,0,0,0}:\left\{
            \begin{array}{l}
            x_1 \rightarrow -x_1\\
            x_2 \rightarrow x_2
            \end{array}
            \right.
        \end{equation}
       \item 
        \begin{equation}
            s^h_{0,0,0,0}:\left\{
            \begin{array}{l}
            x_1 \rightarrow x_1\\
            x_2 \rightarrow -x_2
            \end{array}
            \right.
        \end{equation}
        \item
        \begin{equation}
            s_{0,0,0,0}: x \rightarrow -x
        \end{equation}
    \end{enumerate}
    Такая факторгруппа $G/T\simeq C_2\times C_2$ (таблицы умножения совпадают). Рассмотрим подгруппу $G_0$ движений, сохраняющих начало координат.\\
    Пусть $G_0$ -- подгруппа движений, сохраняющих начало координат. Группа $G\simeq G_0\ltimes T$, $G_0$ действует на $T$ следующим образом: $\phi_e(t_{n_1,n_2,n_3,n_4})=t_{n_1,n_2,n_3,n_4}$, $\phi_{s_0^v}(t_{n_1,n_2,n_3,n_4})=t_{-n_1,-n_2,n_3,n_4}$, $\phi_{s_0^h}(t_{n_1,n_2,n_3,n_4})=t_{n_1,n_2,-n_3,-n_4}$ и $\phi_{s_0}(t_{n_1,n_2,n_3,n_4})=t_{-n_1,-n_2,-n_3,-n_4}$:
    \begin{equation*}
        \phi_e(t_{n_1,n_2,n_3,n_4})(x)=et_{n_1,n_2,n_3,n_4}e(x)=t_{n_1,n_2,n_3,n_4}(x)
    \end{equation*}
    \begin{multline*}
        \phi_{s_0^v}(t_{n_1,n_2,n_3,n_4})(x_1,x_2)^T=s^v_{0,0,0,0}t_{n_1,n_2,n_3,n_4}s^v_{0,0,0,0}(x_1,x_2)^T=s^v_{0,0,0,0}t_{n_1,n_2,n_3,n_4}(-x_1,x_2)^T=\\=s^v_{0,0,0,0}(-x_1+n_1e_1+n_2e_2, x_2+n_3e_1+n_4e_2)^T=(x_1-n_1e_1-n_2e_2,x_2+n_3e_1+n_4e_2)^T=t_{-n_1,-n_2,n_3,n_4}(x)
    \end{multline*}
    \begin{multline*}
        \phi_{s_0^h}(t_{n_1,n_2,n_3,n_4})(x_1,x_2)^T=s^h_{0,0,0,0}t_{n_1,n_2,n_3,n_4}s^h_{0,0,0,0}(x_1,x_2)^T=s^h_{0,0,0,0}t_{n_1,n_2,n_3,n_4}(x_1,-x_2)^T=\\=s^h_{0,0,0,0}(x_1+n_1e_1+n_2e_2,-x_2+n_3e_1+n_4e_2)^T=(x_1+n_1e_1+n_2e_2,x_2-n_3e_1-n_4e_2)^T=t_{n_1,n_2,-n_3,-n_4}(x)
    \end{multline*}
    \begin{multline*}
        \phi_{s_0}(t_{n_1,n_2,n_3,n_4})(x_1,x_2)^T=s_{0,0,0,0}t_{n_1,n_2,n_3,n_4}s_{0,0,0,0}(x_1,x_2)^T=s_{0,0,0,0}t_{n_1,n_2,n_3,n_4}(-x_1,-x_2)^T=\\=s_{0,0,0,0}(-x_1+n_1e_1+n_2e_2,-x_2+n_3e_1+n_4e_2)^T=(x_1-n_1e_1-n_2e_2,x_2-n_3e_1-n_4e_2)^T=\\=t_{-n_1,-n_2,-n_3,-n_4}(x_1,x_2)^T
    \end{multline*}
    \item[б)*] Найдём классы сопряжённости в $G$. Они также будут совпадать с 4 типами движений: трансляции и 3 симметрии (см. список выше), поскольку невозможно получить из движения одного типа сопряжением движение другого типа (т.к. при сопряжениях знак при $x_1$ и $x_2$ меняется чётное число раз: 0 (если то, чем мы сопрягаем не меняет там знак) или 2 (если то, чем мы сопрягаем меняет там знак)). Т.е. если например трансляция не меняла знаки при $x_1$ и $x_2$. то и при сопряжении чем угодно менять знак не будет; если симметрия меняла знак при $x_1$ и не меняла при $x_2$, то и при сопряжении чем угодно будет менять при $x_1$ и не будет при $x_2$. Т.е. всего 4 класса сопряжённости (трансляции и 3 вида симметрий).
\end{itemize}
\end{zad}
\section{Разные конструкции.}
\begin{zad}
Пусть $\varphi$ -- гомоморфизм из $S_4$ в $S_3$. По предложению 2 лекции 4 $\text{Ker}\;\varphi$ -- нормальная подгруппа в $S_4$. Значит, $\text{Ker}\;\varphi$ является объединением каких-то классов сопряжённости. $|S_4|=4!=24$, $|S_3|=3!=6$. Тогда по предложению 5 лекции 4 $|S_4|=|\text{Ker}\;\varphi||S_3|$, откуда $|\text{Ker}\;\varphi|=\frac{|S_4|}{|S_3|}=4$. Классы сопряжённости $S_4$ рассмотрены в задаче 3.2. Всего 2 класса имеют мощность, не превышающую 4: $\{e\}$ и $\{(1,2)(3,4),(1,3)(2,4),(1,4)(2,3)\}$, они и будут составлять ядро (их суммарная мощность равна 4): $\text{Ker}\;\varphi=\{e,(1,2)(3,4),(1,3)(2,4),(1,4)(2,3)\}$.\\
Воспользуемся действием $S_4$ на пространстве многочленов от четырёх переменных вида $P=x_1x_2+x_3x_4$, переставляя переменные. Ещё есть 2 многочлена такого вида, не равные $P$: $Q=x_1x_3+x_2x_4$ и $R=x_1x_4+x_2x_3$. Все элементы ядра соответствуют перестановке $P\rightarrow P$, $Q\rightarrow Q$, $R\rightarrow R$ ($\varphi(\text{Ker}\;\varphi)=e$). Транспозиции $(1,2)$, $(3,4)$ соответствуют перестановке $(Q,R)$ ($\varphi((1,2))=\varphi((3,4))=(Q,R)$). Транспозиции $(1,3)$, $(2,4)$ соответствуют перестановке $(P,R)$ ($\varphi((1,3))=\varphi((2,4))=(P,R)$). Транспозиции $(1,4)$, $(2,3)$ соответствуют перестановке $(P,Q)$ ($\varphi((1,4))=\varphi((2,3))=(P,Q)$). Циклы длины 3 $(1,3,2), (1,4,3), (1,2,4)$, $(2.3.4)$ соответствуют перестановке $(P,Q,R)$ ($\varphi((1,3,2))=\varphi((1,4,3))=\varphi((1,2,4))=\varphi((2,3,4))=(P,Q,R)$). Циклы длины 3 $(1,2,3), (1,3,4), (1,4,2)$, $(2,4,3)$ соответствуют перестановке $(P,R,Q)$ ($\varphi((1,3,2))=\varphi((1,4,3))=\varphi((1,2,4))=\varphi((2,3,4))=(P,R,Q)$). Циклы длины 4 $(1,2,4,3)$, $(1,3,4,2)$ соответствуют перестановке $(P,Q)$ ($\varphi((1,2,4,3))=\varphi((1,3,4,2))=(P,Q)$). Циклы длины 4 $(1,2,3,4)$, $(1,4,3,2)$ соответствуют перестановке $(P,R)$ ($\varphi((1,2,3,4))=\varphi((1,4,3,2))=(P,R)$). Циклы длины 4 $(1,3,2,4)$, $(1,4,2,3)$ соответствуют перестановке $(Q,R)$ ($\varphi((1,3,2,4))=\varphi((1,4,2,3))=(Q,R)$). Таким образом, построено отображение из $S_4$ в $S_3$. Оно сюръективно, поскольку $\text{Im}\;\varphi=S_3$. Также $\forall a,b\in S_4\;\varphi(ab)=\varphi(a)\varphi(b)$, поскольку $a$ переставит индексы в многочленах (этому соответствует перестановка многочленов $\varphi(a)$), $b$ переставит индексы в многочленах (этому соответствует перестановка многочленов $\varphi(b)$), а перестановке индексов $ab$ соответствует перестановка многочленов $\varphi(ab)$. Значит $\varphi$ -- действительно сюръективный гомоморфизм из $S_4$ в $S_3$.\\
Ядро гомоморфизма:
\begin{equation}
    \boxed{\text{Ker}\;\varphi=\{e,(1,2)(3,4),(1,3)(2,4),(1,4)(2,3)\}}
\end{equation}
\end{zad}
\begin{zad}
Рассмотрим две подгруппы $\mathbb{C}^*$: $\mathbb{R}_+$ и $U(1)$ (\textit{унитарная группа порядка 1} -- подгруппа комплексных чисел, по модулю равных 1 (на комплексной плоскости это окружность радиуса 1): $U(1)=\{z\in\mathbb{C}:|z|=1, z=\exp(i\theta)\}$). $\mathbb{R}_+$ и $U(1)$ коммутируют (комплексные числа коммутируют при умножении). $\mathbb{R}_+\cap U(1)=\{1\}$ (луч из центра окружности пересекается с окружностью в 1 точке). $\mathbb{C}^*=\mathbb{R}_+\cdot U(1)$ (любое комплексное число представимо в экспотенциальной форме: $z=r\exp(i\theta)$). Тогда по предложению 7 лекции 4: $\mathbb{C}^*\simeq \mathbb{R}_+ \times U(1)$. По предложению 10 лекции 4:
\begin{equation}
    \boxed{\mathbb{C}^*/\mathbb{R}_+\simeq U(1)}
\end{equation}
\end{zad}
\begin{zad}
\begin{itemize}
    \item[а)] Любой элемент $D_n$ может быть представлен в виде $r^b$ или $sr^b$.\\ Найдём классы сопряжённости $r^b$.
    \begin{equation}
        r^ar^br^{-a}=r^b, \quad sr^ar^b(sr^a)^{-1}=sr^{a+b}r^{-a}s^{-1}=sr^bs^{-1}=r^{-b}
    \end{equation}
    Таким образом, если $n$ -- чётное, то классы сопряжённости: $\{e\}, \{r,r^{-1}\}, \{r^2,r^{-2}\},...,\{r^{\frac{n}{2}}\}$ ($\frac{n}{2}+1$ штук); если $n$ -- нечётное, то классы сопряжённости: $\{e\}, \{r,r^{-1}\}, \{r^2,r^{-2}\},...,\{r^{\frac{n-1}{2}},r^{-\frac{n-1}{2}}\}$ ($\frac{n+1}{2}$ штук).\\
    Найдём классы сопряжённости $sr^b$.
    \begin{equation}
        r^asr^br^{-a}=r^asr^{b-a}=sr^{b-2a},\quad sr^asr^b(sr^a)^{-1}=sr^asr^{b-a}s=sr^{2a-b}
    \end{equation}
    Таким образом, если $n$ -- чётное, то классы сопряжённости: $\{sr, sr^3, ..., sr^{n-1}\}$ ($b$ нечётное) и $\{s, sr^2, ..., sr^{n-2}\}$ ($b$ чётное); если $n$ -- нечётное, то класс сопряжённости 1 (не зависит от чётности $b$): $\{s,sr,...,sr^{n-1}\}$.
    \item[б)] Рассмотрим 4 возможных случая:
    \begin{enumerate}
        \item $x=r^a$, $y=r^b$.
        \begin{equation}
            xyx^{-1}y^{-1}=r^ar^br^{-a}r^{-b}=r^0=e
        \end{equation}
        \item $x=sr^a$, $y=r^b$.
        \begin{equation}
            xyx^{-1}y^{-1}=sr^ar^b(sr^a)^{-1}r^{-b}=sr^{a+b}r^{-a}s^{-1}r^{-b}=sr^bsr^{-b}=r^{-2b}
        \end{equation}
        \item $x=r^a$, $y=sr^b$.
        \begin{equation}
            xyx^{-1}y^{-1}=r^asr^br^{-a}(sr^b)^{-1}=r^asr^{-a}s=r^{2a}
        \end{equation}
        \item $x=sr^a$, $y=sr^a$.
        \begin{equation}
            xyx^{-1}y^{-1}=sr^asr^b(sr^a)^{-1}(sr^b)^{-1}=r^{b-a}r^{-a}sr^{-b}s=r^{2(b-a)}
        \end{equation}
    \end{enumerate}
    Во всех возможных случаях коммутаторы элементов -- повороты в чётной степени. При определённых $a$ и $b$ среди них есть $r^2$, который порождает коммутант.\\
    Если $n$ -- чётное, то коммутант состоит из всех поворотов в чётной степени:
    \begin{equation}
        \boxed{[D_n,D_n]=\{e,r^2,r^4,...,r^{n-2}\},\quad n=2k,k\in\mathbb{Z}}
    \end{equation}
    Если $n$ -- нечётное, то $r^{n-1}r^2=r^1$ и коммутант состоит из всех поворотов:
    \begin{equation}
        \boxed{[D_n,D_n]=\{e,r,r^2,...,r^{n-1}\},\quad n=2k+1,k\in\mathbb{Z}}
    \end{equation}
    \item[в)*] 
    \begin{predl}
        $D_n$ изоморфно полупрямому произведению $C_n$ и $C_2$.
    \end{predl}
    \begin{proof}
        В группе $D_n$ содержатся повороты (подгруппа $C_n$) и симметрия (подгруппа $C_2$). Они пересекаются только по $e$. Любой элемент $D_n$ -- $r^b$ или $sr^b$. Группа $C_n$ является нормальной подгруппой в $D_n$, поскольку является объединением классов сопряжённости из $D_n$ (см. п. а). По предложению 9 лекции 4, предложение доказано.
    \end{proof}
    \begin{equation}
        \boxed{D_n\simeq C_2\ltimes C_n}
    \end{equation}
\end{itemize}
\end{zad}
\begin{zad}
\begin{itemize}
    \item[а)]
    \begin{predl}
    Пусть $G_0$ -- группа собственных движений, сохраняющих куб. Действие $G_0$ на множестве диагоналей куба задаёт изоморфизм $G_0$ и $S_4$.
    \end{predl}
    \begin{proof}
    Для доказательства изоморфизма выпишем соответствие между перестановками различных циклических типов в $S_4$ и различными собственными движениями, сохраняющими куб:
    \begin{enumerate}
        \item $e$ (единичный, 1 шт.) -- тождественное движение (ничего не делает с кубом).
        \item $(a,b)$ (транспозиции, 6 шт.) -- повороты на $\pi$ вокруг прямой $l_1$, проходящей через центры противоположных рёбер.
        \item $(a,b)(c,d)$ (произведение транспозиций, 3 шт.) -- повороты на $\pi$ вокруг прямой $l_2$, проходящей через центры противоположных граней.
        \item $(a,b,c)$ (цикл длины 3, 8 шт.) -- поворот на $\frac{2\pi}{3}$, $\frac{4\pi}{3}$ вокруг прямой $l_3$, проходящей через диагональ куба.
        \item $(a,b,c,d)$ (цикл длины 4, 6 шт.) -- повороты на $\frac{\pi}{2}$, $\frac{3\pi}{2}$ вокруг прямой $l_2$, проходящей через центры противоположных граней.
    \end{enumerate}
    Таким образом, построено однозначное соответствие между перестановками и движениями, сохраняющее групповые операции, а значит изоморфизм групп построен.
    \end{proof}
    \item[б)]
    По предложению 7 лекции 3 перестановки сопряжены тогда и только тогда, когда имеют одинаковую циклическую структуру. А значит классами сопряжённости являются 5 видов движения, описанных выше.
    \item[в)]
    \begin{predl}
        G изоморфно $G_0\times C_2$.
    \end{predl}
    \begin{proof}
        В $G$ тождественное движение и центральная симметрия образуют $C_2$.
        Все движения из $G$ можно представить в виде произведения собственных движений из $G_0$ и элементов $C_2$. Подгруппы $G_0$ и $C_2$ коммутируюет и пересекаются только по $e$. По предложению 7 лекции 4 предложение доказано.
    \end{proof}
    \begin{equation}
        \boxed{G\simeq G_0\times C_2}
    \end{equation}
    В $G_0$ 5 классов сопряжённости, в $C_2$ -- 2 класса ($\{e\}$ и $\{r\}$). Пусть $x=(g_x,c_x)\in G_0\times C_2$, $y=(g_y,c_y)\in G_0\times C_2$. Найдём классы сопряжённости в $G_0\times C_2$.
    \begin{equation}
        xyx^{-1}=(g_x,c_x)(g_y,c_y)(g_x,c_x)^{-1}=(g_xg_yg_x^{-1},c_xc_yc_x^{-1})
    \end{equation}
    Как видно, если $g_x$ и $g_y$ сопряжены и $c_x$ и $c_y$ сопряжены, то и $(g_x,c_x)$ и $(g_y,c_y)$ сопряжены и наоборот. Т.е. классы сопряжения произведения групп -- произведения классов сопряжённости этих групп. Поэтому число классов сопряжённости $2\cdot 5=10$.
\end{itemize}
\end{zad}
\section{Представления групп.}
\begin{upr}
\begin{itemize}
    \item[а)] Регулярное представление $C_3$:
    \begin{equation}
        e\rightarrow
        \left(
    \begin{array}{ccc}
    1 & 0 & 0\\
    0 & 1 & 0\\
    0 & 0 & 1\\
    \end{array}
    \right)\quad r\rightarrow
        \left(
    \begin{array}{ccc}
    0 & 0 & 1\\
    1 & 0 & 0\\
    0 & 1 & 0\\
    \end{array}
    \right)\quad r^2\rightarrow
        \left(
    \begin{array}{ccc}
    0 & 1 & 0\\
    0 & 0 & 1\\
    1 & 0 & 0\\
    \end{array}
    \right)
    \end{equation}
    Найдём инвариантные подпространства в этом представлении (кроме тривиальных $\{0\}$ и $\mathbb{C}^3$). Для этого найдём собственные векторы матриц $\rho(r)$ и $\rho(r^2)$ (для единичной матрицы любой вектор собственный с собственным значением 1).
    \begin{equation}
        \text{det}(\rho(r)-\lambda E)=\left|
    \begin{array}{ccc}
    -\lambda & 0 & 1\\
    1 & -\lambda & 0\\
    0 & 1 & -\lambda\\
    \end{array}
    \right|=-\lambda^3+1=0
    \end{equation}
    \begin{equation}
        \lambda_0=1, \lambda_1=e^{\frac{2\pi i}{3}}, \lambda_2=e^{\frac{4\pi i}{3}}
    \end{equation}
    \begin{equation}
        \lambda_0=1: \left(
    \begin{array}{ccc}
    -1 & 0 & 1\\
    1 & -1 & 0\\
    0 & 1 & -1\\
    \end{array}
    \right)\left(
    \begin{array}{c}
    h_{01}\\
    h_{02}\\
    h_{03}\\
    \end{array}
    \right)=\left(
    \begin{array}{c}
    0\\
    0\\
    0\\
    \end{array}
    \right)\rightarrow h_0=\left(
    \begin{array}{c}
    1\\
    1\\
    1\\
    \end{array}
    \right)
    \end{equation}
    \begin{equation}
        \lambda_1=e^{\frac{2\pi i}{3}}: \left(
    \begin{array}{ccc}
    -e^{\frac{2\pi i}{3}} & 0 & 1\\
    1 & -e^{\frac{2\pi i}{3}} & 0\\
    0 & 1 & -e^{\frac{2\pi i}{3}}\\
    \end{array}
    \right)\left(
    \begin{array}{c}
    h_{11}\\
    h_{12}\\
    h_{13}\\
    \end{array}
    \right)=\left(
    \begin{array}{c}
    0\\
    0\\
    0\\
    \end{array}
    \right)\rightarrow h_1=\left(
    \begin{array}{c}
    1\\
    e^{\frac{4\pi i}{3}}\\
    e^{\frac{2\pi i}{3}}\\
    \end{array}
    \right)
    \end{equation}
    \begin{equation}
        \lambda_2=e^{\frac{4\pi i}{3}}: \left(
    \begin{array}{ccc}
    -e^{\frac{4\pi i}{3}} & 0 & 1\\
    1 & -e^{\frac{4\pi i}{3}} & 0\\
    0 & 1 & -e^{\frac{4\pi i}{3}}\\
    \end{array}
    \right)\left(
    \begin{array}{c}
    h_{21}\\
    h_{22}\\
    h_{23}\\
    \end{array}
    \right)=\left(
    \begin{array}{c}
    0\\
    0\\
    0\\
    \end{array}
    \right)\rightarrow h_2=\left(
    \begin{array}{c}
    1\\
    e^{\frac{2\pi i}{3}}\\
    e^{\frac{4\pi i}{3}}\\
    \end{array}
    \right)
    \end{equation}
    Заметим, что эти же вектора и значения являются собственными и для $\rho(r^2)$:
    \begin{equation}
        \rho(r^2)h_0=\lambda_0h_0,\quad \rho(r^2)h_1=\lambda_2h_1,\quad \rho(r^2)h_2=\lambda_1h_2
    \end{equation}
    Таким образом,
    \begin{equation}
        \boxed{V_0=\braket{\left(
    \begin{array}{c}
    1\\
    1\\
    1\\
    \end{array}
    \right)},\quad V_1=\braket{\left(
    \begin{array}{c}
    1\\
    e^{\frac{4\pi i}{3}}\\
    e^{\frac{2\pi i}{3}}\\
    \end{array}
    \right)},\quad V_2=\braket{\left(
    \begin{array}{c}
    1\\
    e^{\frac{2\pi i}{3}}\\
    e^{\frac{4\pi i}{3}}\\
    \end{array}
    \right)}}
    \end{equation}
    являются одномерными инвариантными представлениями.
    \begin{equation}
        \mathbb{C}^3=V_0\oplus V_1\oplus V_2 
    \end{equation}
    Разложим представление $C_n$ в сумму $R_j$:
    \begin{enumerate}
        \item В $V_0$: $R_0(e)=1$, $R_0(r)=1$, $R_0(r^2)=1$.
        \item В $V_1$: $R_1(e)=1$, $R_1(r)=e^{\frac{2\pi i}{3}}$, $R_1(r^2)=e^{\frac{4\pi i}{3}}$.
        \item В $V_2$: $R_2(e)=1$, $R_2(r)=e^{\frac{4\pi i}{3}}$, $R_2(r^2)=e^{\frac{2\pi i}{3}}$.
    \end{enumerate}
    Заметим, что для $R_j$ можно записать общую формулу:
    \begin{equation}
        R_j(r^m)=e^{\frac{2\pi i}{n}mj}
    \end{equation}
    \begin{equation}
        \boxed{R=R_0\oplus R_1\oplus R_2}
    \end{equation}
    \item[б)] В базисе $h_0$, $h_1$, $h_2$:
    \begin{equation}
        \rho(e)=\left(
    \begin{array}{ccc}
    1 & 0 & 0\\
    0 & 1 & 0\\
    0 & 0 & 1\\
    \end{array}
    \right),\quad
    \rho(r)=\left(
    \begin{array}{ccc}
    1 & 0 & 0\\
    0 & e^{\frac{2\pi i}{3}} & 0\\
    0 & 0 & e^{\frac{4\pi i}{3}}\\
    \end{array}
    \right),\quad
    \rho(r^2)=\left(
    \begin{array}{ccc}
    1 & 0 & 0\\
    0 & e^{\frac{4\pi i}{3}} & 0\\
    0 & 0 & e^{\frac{2\pi i}{3}}\\
    \end{array}
    \right)
    \end{equation}
    Матрица перехода к этому базису:
    \begin{equation}
        \boxed{\varphi=\left(
    \begin{array}{ccc}
    1 & 1 & 1\\
    1 & e^{\frac{4\pi i}{3}} & e^{\frac{2\pi i}{3}}\\
    1 & e^{\frac{2\pi i}{3}} & e^{\frac{4\pi i}{3}}\\
    \end{array}
    \right)}
    \end{equation}
\end{itemize}
\end{upr}
\begin{zad}
\begin{itemize}
    \item[а)] 
    \begin{predl}
        Характеры сопряжённых элементов равны.
    \end{predl}
    \begin{proof}
        \begin{equation*}
            \text{tr}(\rho(gxg^{-1}))=\text{tr}(\rho(g)\rho(x)\rho(g^{-1}))=\text{tr}(\rho(g)\rho(x)\rho(g)^{-1})=\text{tr}(\rho(g)\rho(g)^{-1}\rho(x))=\text{tr}(\rho(x))
        \end{equation*}
    \end{proof}
    \item[б)]
    \begin{equation}
        U_2=\left\{\sum\limits_{i=1}^3x_ie_i|\sum_{i=1}^3x_i=0\right\}
    \end{equation}
    Воспользуемся базисом:
    \begin{equation}
        e'=e_1-e_2=\left(
    \begin{array}{c}
    1\\
    -1\\
    0\\
    \end{array}
    \right),\quad e''=e_1-e_3=\left(
    \begin{array}{c}
    1\\
    0\\
    -1\\
    \end{array}
    \right)
    \end{equation}
    Любой вектор может быть выражен через базисные вектора:
    \begin{equation}
        \left(
    \begin{array}{c}
    x_1\\
    x_2\\
    x_3\\
    \end{array}
    \right)=-x_2\left(
    \begin{array}{c}
    1\\
    -1\\
    0\\
    \end{array}
    \right)-x_3\left(
    \begin{array}{c}
    1\\
    0\\
    -1\\
    \end{array}
    \right)=-x_2e'-x_3e''
    \end{equation}
    Выпишем, во что будут переходить $e'$ и $e''$ при домножении слева на $\rho(g)$ (из перестановочного представления): $\rho(e)e'=(1,-1,0)^T=e'$, $\rho(e)e''=(1,0,-1)^T=e''$, $\rho((1,2))e'=(-1,1,0)^T=-e'$, $\rho((1,2))e''=(0,1,-1)^T=e''-e'$, $\rho((1,3))e'=(0,-1,1)^T=e'-e''$, $\rho((1,3))e''=(-1,0,1)^T=-e''$, $\rho((2,3))e'=(1,0,-1)^T=e''$, $\rho((2,3))e''=(1,-1,0)^T=e'$, $\rho((1,2,3))e'=(0,1,-1)^T=e''-e'$, $\rho((1,2,3))e''=(-1,1,0)^T=-e'$, $\rho((1,3,2))e'=(-1,0,1)^T=-e''$, $\rho((1,3,2))e''=(0,-1,1)^T=e'-e''$.\\
    Представление $U_2$:
    \begin{equation}
        \boxed{e\rightarrow\left(
    \begin{array}{cc}
    1 & 0\\
    0 & 1\\
    \end{array}
    \right),\quad (1,2)\rightarrow\left(
    \begin{array}{cc}
    -1 & -1\\
    0 & 1\\
    \end{array}
    \right),\quad (1,3)\rightarrow\left(
    \begin{array}{cc}
    1 & 0\\
    -1 & -1\\
    \end{array}
    \right)}
    \end{equation}
    \begin{equation}
    \boxed{(2,3)\rightarrow\left(
    \begin{array}{cc}
    0 & 1\\
    1 & 0\\
    \end{array}
    \right),\quad (1,2,3)\rightarrow\left(
    \begin{array}{cc}
    -1 & -1\\
    1 & 0\\
    \end{array}
    \right),\quad (1,3,2)\rightarrow\left(
    \begin{array}{cc}
    0 & 1\\
    -1 & -1\\
    \end{array}
    \right)}
    \end{equation}
    \item[в)] \textit{Тривиальное представление:}
    \begin{equation}
        \rho(g)=1\rightarrow \boxed{\chi_V(g)=\text{Tr}\;1=1}
    \end{equation}
    \textit{Перестановочное представление} $S_3$ задаётся матрицами ($\rho(g)e_x=e_{gx}$):
    \begin{equation}
        e\rightarrow
        \left(
    \begin{array}{ccc}
    1 & 0 & 0\\
    0 & 1 & 0\\
    0 & 0 & 1\\
    \end{array}
    \right)\quad (1,2)\rightarrow
        \left(
    \begin{array}{ccc}
    0 & 1 & 0\\
    1 & 0 & 0\\
    0 & 0 & 1\\
    \end{array}
    \right)\quad (1,3)\rightarrow
        \left(
    \begin{array}{ccc}
    0 & 0 & 1\\
    0 & 1 & 0\\
    1 & 0 & 0\\
    \end{array}
    \right)
    \end{equation}
    \begin{equation}
        (2,3)\rightarrow
        \left(
    \begin{array}{ccc}
    1 & 0 & 0\\
    0 & 0 & 1\\
    0 & 1 & 0\\
    \end{array}
    \right)\quad (1,2,3)\rightarrow
        \left(
    \begin{array}{ccc}
    0 & 0 & 1\\
    1 & 0 & 0\\
    0 & 1 & 0\\
    \end{array}
    \right)\quad (1,3,2)\rightarrow
        \left(
    \begin{array}{ccc}
    0 & 1 & 0\\
    0 & 0 & 1\\
    1 & 0 & 0\\
    \end{array}
    \right)
    \end{equation}
    \begin{equation}
        \boxed{\chi_V(e)=3,\quad\chi_V((a,b))=1,\quad\chi_V((a,b,c))=0}
    \end{equation}
    \textit{Регулярное представление} ($ge_x=e_{gx}$).\\
    $|S_3|=6$. Рассмотрим пространство $V=\mathbb{C}^6$. Элементы $S_3$ представим в виде: $e=(1,0,0,0,0,0)$, $(1,2)=(0,1,0,0,0,0)$, $(2,3)=(0,0,1,0,0,0)$, $(1,3)=(0,0,0,1,0,0)$, $(1,2,3)=(0,0,0,0,1,0)$, $(1,3,2)=(0,0,0,0,0,1)$.\\
    Все матрицы регулярного представления состоят из 0 и 1, в каждой строке и столбце по одной 1. Если характер элемента не равен 0, то на диагонали по крайней мере есть одна 1. Пусть она будет стоять на $m$ месте. Это значит, что $ge_m=e_m$, что может быть только при $g=e$. Следовательно, при $g\neq e$ характер $\rho(g)$ равен 0.
    \begin{equation}
        \boxed{\chi_V(e)=6,\quad \chi_V(g)=0,\;g\neq e}
    \end{equation}
    \textit{Представление $U_2$}:
    \begin{equation}
        \boxed{\chi_V(e)=2,\quad\chi_V((a,b))=0,\quad\chi_V((a,b,c))=-1}
    \end{equation}
\end{itemize}
\end{zad}
\begin{zad}
\begin{itemize}
    \item[а)] $R_\alpha$ -- вращение трёхмерного пространства относительно некоторой оси на угол $\alpha$. В базисе, в котором один из базисных векторов ($\vec{k}$) проходит через эту ось, $R_\alpha$ записывается в виде
    \begin{equation}
        R_\alpha=\left(
    \begin{array}{ccc}
    \cos\alpha & -\sin\alpha & 0\\
    \sin\alpha & \cos\alpha & 0\\
    0 & 0 & 1\\
    \end{array}
    \right)
    \end{equation}
    \begin{equation}
        \boxed{\text{tr}R_\alpha=2\cos\alpha+1}
    \end{equation}
    Такой след будет иметь матрица $R_\alpha$ в любом базисе, поскольку следы сопряжённых матриц равны.\\
    $S_\alpha$ -- зеркальный поворот -- композиция вращения трёхмерного пространства относительно некоторой оси на угол $\alpha$ и отражения относительно плоскости перпендикулярной этой оси. В базисе, в котором один из базисных векторов ($\vec{k}$) проходит через эту ось, $S_\alpha$ записывается в виде
    \begin{equation}
        S_\alpha=\left(
    \begin{array}{ccc}
    \cos\alpha & -\sin\alpha & 0\\
    \sin\alpha & \cos\alpha & 0\\
    0 & 0 & -1\\
    \end{array}
    \right)
    \end{equation}
    \begin{equation}
        \boxed{\text{tr}S_\alpha=2\cos\alpha-1}
    \end{equation}
    Такой след будет иметь матрица $S_\alpha$ в любом базисе, поскольку следы сопряжённых матриц равны.
    \item[б)] Выпишем следы матриц $\rho_3$ для всех симметрий, сохраняющих правильный тетраэдр:
    \begin{enumerate}
        \item $e$ (единичный, 1 шт.) -- тождественное движение (ничего не делает с тетраэдром).
        %\begin{equation}
         %   \rho_3(e)=R_0=\left(
        %\begin{array}{ccc}
        %1 & 0 & 0\\
        %0 & 1 & 0\\
        %0 & 0 & 1\\
        %\end{array}
        %\right)
        %\end{equation}
        \begin{equation}
            \text{tr}R_0=2\cos 0+1=3
        \end{equation}
        \begin{equation}
            \boxed{\chi_{\mathbb{R}^3}(e)=3}
        \end{equation}
        \item $(a,b)$ (транспозиции, 6 шт.) -- отражение относительно плоскости $\pi$, проходящей через ребро (вершины которого остаются на месте) и центр противопложного ребра (вершины которого меняются местами).
        %\begin{equation}
         %   \rho_3(a,b)=S_0=\left(
        %\begin{array}{ccc}
        %1 & 0 & 0\\
        %0 & 1 & 0\\
        %0 & 0 & -1\\
        %\end{array}
        %\right)
        %\end{equation}
        \begin{equation}
            \text{tr}S_0=2\cos 0-1=1
        \end{equation}
        \begin{equation}
            \boxed{\chi_{\mathbb{R}^3}((a,b))=1}
        \end{equation}
        \item $(a,b)(c,d)$ (произведение транспозиций, 3 шт.) -- повороты на $\pi$ вокруг прямой $l_1$, проходящей через центры противоположных рёбер (вершины которых и будут меняться местами).
        %\begin{equation}
         %   \rho_3((a,b)(c,d))=R_\pi=\left(
        %\begin{array}{ccc}
        %-1 & 0 & 0\\
        %0 & -1 & 0\\
        %0 & 0 & 1\\
        %\end{array}
        %\right)
        %\end{equation}
        \begin{equation}
            \text{tr}R_\pi=2\cos\pi+1=-1
        \end{equation}
        \begin{equation}
            \boxed{\chi_{\mathbb{R}^3}((a,b)(c,d))=-1}
        \end{equation}
        \item $(a,b,c)$ (цикл длины 3, 8 шт.) -- поворот на $2\pi/3$ или $4\pi/3$ вокруг прямой $l_2$, проходящей через вершину и центр противоположной грани (вершины этой грани и будут циклически меняться).
         %\begin{equation}
          %  \rho_3((a,b)(c,d))=R_{\frac{2\pi}{3}}=\left(
        %\begin{array}{ccc}
        %-\frac{1}{2} & -\frac{\sqrt{3}}{2} & 0\\
        %\frac{\sqrt{3}}{2} & -\frac{1}{2} & 0\\
        %0 & 0 & 1\\
        %\end{array}
        %\right)
        %\end{equation}
        \begin{equation}
            \text{tr}R_{\frac{2\pi}{3}}=\text{tr}R_{\frac{4\pi}{3}}=2\cos{\frac{2\pi}{3}}+1=0
        \end{equation}
        \begin{equation}
            \boxed{\chi_{\mathbb{R}^3}((a,b,c))=0}
        \end{equation}
        \item $(a,b,c,d)$ (цикл длины 4, 6 шт.) -- зеркальный повороте на $\pi/2$ с прямой $l_3$, проходящей через середины двух противоположных ребер и плоскостью симметрии, проходящей через середины остальных ребер.
        \begin{equation}
            \text{tr}S_{\frac{\pi}{2}}=2\cos{\frac{\pi}{2}}-1=-1
        \end{equation}
        \begin{equation}
            \boxed{\chi_{\mathbb{R}^3}((a,b,c,d))=-1}
        \end{equation}
    \end{enumerate}
    \item[в)] Выпишем следы матриц $\rho_4$ для всех вращений, сохраняющих куб:
    \begin{enumerate}
        \item $e$ (единичный, 1 шт.) -- тождественное движение (ничего не делает с кубом).
        \begin{equation}
            \text{tr}R_0=2\cos 0+1=3
        \end{equation}
        \begin{equation}
            \boxed{\chi_{\mathbb{R}^3}(e)=3}
        \end{equation}
        \item $(a,b)$ (транспозиции, 6 шт.) -- повороты на $\pi$ вокруг прямой $l_1$, проходящей через центры противоположных рёбер.
        \begin{equation}
            \text{tr}R_\pi=2\cos \pi+1=-1
        \end{equation}
        \begin{equation}
            \boxed{\chi_{\mathbb{R}^3}((a,b))=-1}
        \end{equation}
        \item $(a,b)(c,d)$ (произведение транспозиций, 3 шт.) -- повороты на $\pi$ вокруг прямой $l_2$, проходящей через центры противоположных граней.
        \begin{equation}
            \text{tr}R_\pi=2\cos \pi+1=-1
        \end{equation}
        \begin{equation}
            \boxed{\chi_{\mathbb{R}^3}((a,b)(c,d))=-1}
        \end{equation}
        \item $(a,b,c)$ (цикл длины 3, 8 шт.) -- поворот на $\frac{2\pi}{3}$, $\frac{4\pi}{3}$ вокруг прямой $l_3$, проходящей через диагональ куба.
        \begin{equation}
            \text{tr}R_{\frac{2\pi}{3}}=\text{tr}R_{\frac{4\pi}{3}}=2\cos\frac{2\pi}{3}+1=0
        \end{equation}
        \begin{equation}
            \boxed{\chi_{\mathbb{R}^3}((a,b,c))=0}
        \end{equation}
        \item $(a,b,c,d)$ (цикл длины 4, 6 шт.) -- повороты на $\frac{\pi}{2}$, $\frac{3\pi}{2}$ вокруг прямой $l_2$, проходящей через центры противоположных граней.
        \begin{equation}
            \text{tr}R_{\frac{\pi}{2}}=\text{tr}R_{\frac{3\pi}{2}}=2\cos\frac{\pi}{2}+1=1
        \end{equation}
        \begin{equation}
            \boxed{\chi_{\mathbb{R}^3}((a,b,c,d))=1}
        \end{equation}
    \end{enumerate}
\end{itemize}
\end{zad}
\begin{zad}
\begin{itemize}
    \item[а)]
    \begin{predl}
        Для любой группы $G$ коммутант является нормальной подгруппой.
    \end{predl}
    \begin{proof}
        Коммутант группы порождён коммутаторами её элементов $[x,y]$. А значит любой элемент коммутанта можно представить в виде $[x_1,y_1]...[x_l,y_l]$.
        \begin{equation}
            \forall g,x,y\in G,\quad g(xy)g^{-1}=(gxg^{-1})(gyg^{-1})
        \end{equation}
        \begin{equation*}
            \forall g,x,y\in G,\quad g[x,y]g^{-1}=gxyx^{-1}y^{-1}g^{-1}=gxg^{-1}gyg^{-1}gx^{-1}g^{-1}gy^{-1}g^{-1}=[gxg^{-1},gyg^{-1}]
        \end{equation*}
        Используем выведенные выше равенства:
        \begin{equation*}
            g([x_1,y_1]...[x_l,y_l])g^{-1}=(g[x_1,y_1]g^{-1})...(g[x_l,y_l]g^{-1})=[gx_1g^{-1},gy_1g^{-1}]...[gx_lg^{-1},gy_lg^{-1}]\in[G,G]
        \end{equation*}
        Значит, $[G,G]$ является нормальной подгруппой.
    \end{proof}
    \item[б)*]
    \begin{predl}
        Для любой группы $G$ фактор по коммутанту является коммутативной группой.
    \end{predl}
    \begin{proof}
        Пусть $x[G,G]$, $y[G,G] \in G/[G,G]$.
        \begin{equation}
            (x[G,G])(y[G,G])=xy[G,G]=xy[y^{-1},x^{-1}][G,G]=yx[G,G]=(y[G,G])(x[G,G])
        \end{equation}
        Значит, $G/[G,G]$ является коммуативной группой.
    \end{proof}
\end{itemize}
\end{zad}
\begin{zad}
\begin{itemize}
    \item[а)]
    \begin{predl}
        Единственные одномерные представления симметрической группы - это тривиальное представление и знаковое представление.
    \end{predl}
    \begin{proof}
        Пусть $\rho$ -- одномерное представление $S_n$. Любая транспозиция имеет порядок 2, значит должна переходить либо в $1$, либо в $-1$. Рассмотрим 2 случая:
        \begin{enumerate}
            \item $\exists$ транспозиция $\alpha:$ $\rho(\alpha)=1$. Все транспозиции лежат в одном классе сопряжённости, значит $\forall\beta\in S_n$ и $\gamma\in S_n$ можно написать $\alpha=\gamma\beta\gamma^{-1}$.
            \begin{equation}
                \rho(\alpha)=\rho(\gamma\beta\gamma^{-1})=\rho(\gamma)\rho(\beta)\rho(\gamma^{-1})=\rho(\gamma)\rho(\gamma^{-1})\rho(\beta)=\rho(\gamma\gamma^{-1}\beta)=\rho(\beta)
            \end{equation}
            Значит любая транспозиция переходит в 1. Любой элемент $S_n$ может быть представлен в виде произведения транспозиций. Значит любой элемент $S_n$ переходит в 1 и $\rho$ -- тривиальное представление.
            \item $\exists$ транспозиция $\alpha:$ $\rho(\alpha)=-1$. Значит любая транспозиция переходит в $-1$ (по аналогии с предыдущим пунктом). Любой элемент $S_n$ может быть представлен в виде произведения $m$ транспозиций (чётность элемента равна чётности их количества $m$ по предложению \ref{pr2}). Значит любой элемент $S_n$ переходит в свою чётность $(-1)^m$ и $\rho$ -- знаковое представление.
        \end{enumerate}
    \end{proof}
    \item[б)*]
    \begin{predl}
        Коммутант группы $S_n$ -- это $A_n$.
    \end{predl}
    \begin{proof}
        Любой тройной цикл является коммутатором:
        \begin{equation}
            (a,b,c)=(b,c)(a,b)(b,c)(a,b)=[(b,c),(a,b)]
        \end{equation}
        Значит, любой тройной цикл лежит в коммутанте. Коммутатор любых двух перестановок чётен, поэтому $[S_n,S_n]\subseteq A_n$. Рассмотрим случаи произведения двух транспозиций:
        \begin{enumerate}
            \item \begin{equation}
                (a,b)(a,b)=e=(a,b,c)(c,b,a)
            \end{equation}
            \item \begin{equation}
                (a,b)(a,c)=(a,c,b),\quad b\neq c
            \end{equation}
            \item \begin{equation}
                (a,b)(c,d)=(a,d,b)(a,d,c)
            \end{equation}
        \end{enumerate}
        Таким образом, произведение транспозиций -- цикл длины 3 или произведение циклов длины 3. Т.к. любой элемент $A_n$ раскладывается в чётное число транспозиций, то, разбив транспозиции на пары и заменив на циклы длины 3, любой элемент $A_n$ выражается через циклы длины 3. Значит $A_n$ порождена циклами длины 3 и $A_n\subseteq [S_n,S_n]$. Таким образом, $[S_n,S_n]= A_n$.
    \end{proof}
\end{itemize}
\end{zad}
\section{Унитарность. Характеры представлений.}
\begin{upr}
\item[а)] 
\begin{predl}
    Вектор $e_1\otimes f_1+e_2\otimes f_2$ не является разложимым.
\end{predl}
\begin{proof}
    Докажем от противного. Предположим, что $\exists$ $v=\sum\limits_i a^ie_i$, $u=\sum\limits_i b^jf_j$: \begin{equation}
        v\otimes u=e_1\otimes f_1+e_2\otimes f_2
    \end{equation}
    Подставим $v$ и $u$:
    \begin{equation}
        a^1b^1e_1\otimes f_1+a^2b^1 e_2\otimes f_1+a^2b^1 e_2\otimes f_1+a^2b^2 e_2\otimes f_2=e_1\otimes f_1+e_2\otimes f_2
    \end{equation}
    \begin{equation}
        a^1b^1=a^2b^2=1, \quad a^1b^2=a^2b^1=0
    \end{equation}
    Из правых уравнений следует, что в каждой паре из $a^1, b^2$ и $a^2, b^1$ есть хотя бы 1 нуль. Противоречие с левыми уравнениями.
\end{proof}
\item[б)] Пусть $A=\left(
\begin{array}{cccc}
a_{11} & a_{12}\\
a_{21} & a_{22}\\
\end{array}
\right)$, $B=\left(
\begin{array}{cccc}
b_{11} & b_{12}\\
b_{21} & b_{22}\\
\end{array}
\right)$.
\begin{equation}
    \boxed{A\otimes B=\left(
\begin{array}{cc}
a_{11}B & a_{12}B\\
a_{21}B & a_{22}B\\
\end{array}
\right)=\left(
\begin{array}{cccc}
a_{11}b_{11} & a_{11}b_{12} & a_{12}b_{11} & a_{12}b_{12}\\
a_{11}b_{21} & a_{11}b_{22} & a_{12}b_{21} & a_{12}b_{22}\\
a_{21}b_{11} & a_{21}b_{12} & a_{22}b_{11} & a_{22}b_{12}\\
a_{21}b_{12} & a_{21}b_{22} & a_{22}b_{21} & a_{22}b_{22}\\
\end{array}
\right)}
\end{equation}
\begin{equation}
    \text{tr}A\otimes B=a_{11}b_{11}+a_{11}b_{22}+a_{22}b_{11}+a_{22}b_{22},\quad \text{tr}A=a_{11}+a_{22},\quad \text{tr}B=b_{11}+b_{22}
\end{equation}
\begin{equation}
    \boxed{\text{tr}A\otimes B=\text{tr}A\cdot \text{tr}B}
\end{equation}
\end{upr}
\begin{zad}
\begin{itemize}
\item[а)] 
Классы сопряжённости групп $D_n$ описаны в задаче 4.3, п. а. В $D_7$ 5 классов сопряжённости: $\{e\}$, $\{r^{-1},r^1\}$, $\{r^2,r^{-2}\}$, $\{r^3,r^{-3}\}$, $\{sr^b\}$. Значит и число неприводимых представлений равно 5. Найдём их размерности: $|D_7|=d_1^2+d_2^2+d_3^2+d_4^2+d_5^2=14$. Единственный возможный вариант: $d_1=d_2=1$, $d_3=d_4=d_5=2$.\\
Найдём все одномерные представления (их всего 2). Коммутанты $D_n$ описаны в задаче 4.3, п. б. $[D_7,D_7]=\{e,r,r^2,...,r^{6}\}$. Факторгруппа $D_7/[D_7,D_7]\simeq\mathbb{Z}_2$, т.е. $|D_7/[D_7,D_7]|=|\mathbb{Z}_2|=2$ (ещё раз получаем, что всего 2 одномерных неприводимых представления). Любое одномерное неприводимое представление -- представление фактора по коммутанту. Первое одномерное неприводимое представление -- \textit{тривиальное}: все элементы $D_7$ переходят в 1.
\begin{equation}
    \boxed{\rho_1(r)=\rho_1(s)=1}
\end{equation}
Для отражений верно, что $(sr^b)^2=sr^bsr^b=r^{-b}r^b=e$. Поэтому $\rho(sr^b)^2=\rho((sr^b)^2)=\rho(e)=1$ и $\rho(sr^b)=\pm 1$. $\rho(sr^b)=1$ соответствует тривиальному представлению, а $\rho(sr^b)=-1$ -- второму одномерному представлению. Т.е. во втором одномерном представлении все повороты перейдут в $1$, а отражения -- в $-1$.
\begin{equation}
    \boxed{\rho_2(r)=1,\quad \rho_2(s)=-1}
\end{equation}
\item[б)] Найдём 3 оставшихся двумерных неприводимых представления. Для них должны выполняться тождества: $\rho(r^b)^7=\rho((r^b)^7)=\rho(e)=1$ (матрицы поворота), $\rho(sr^b)^2=\rho((sr^b)^2)=\rho(e)=1$ (матрица отражения), $\rho(r)\rho(s)\rho(r)\rho(s)=\rho(rsrs)=\rho(e)=1$.
\begin{equation}
    \boxed{\rho_3(r)=\left(
    \begin{array}{cc}
    \cos\frac{2\pi}{7} & -\sin\frac{2\pi}{7}\\
    \sin\frac{2\pi}{7} & \cos\frac{2\pi}{7}\\
    \end{array}
    \right),\quad \rho_3(s)=\left(
    \begin{array}{cc}
    1 & 0\\
    0 & -1\\
    \end{array}
    \right)}
\end{equation}
\begin{equation}
    \boxed{\rho_4(r)=\left(
    \begin{array}{cc}
    \cos\frac{4\pi}{7} & -\sin\frac{4\pi}{7}\\
    \sin\frac{4\pi}{7} & \cos\frac{4\pi}{7}\\
    \end{array}
    \right),\quad \rho_4(s)=\left(
    \begin{array}{cc}
    1 & 0\\
    0 & -1\\
    \end{array}
    \right)}
\end{equation}
\begin{equation}
    \boxed{\rho_5(r)=\left(
    \begin{array}{cc}
    \cos\frac{6\pi}{7} & -\sin\frac{6\pi}{7}\\
    \sin\frac{6\pi}{7} & \cos\frac{6\pi}{7}\\
    \end{array}
    \right),\quad \rho_5(s)=\left(
    \begin{array}{cc}
    1 & 0\\
    0 & -1\\
    \end{array}
    \right)}
\end{equation}
\begin{table}[h!]
\centering
\begin{tabular}{|l|l|l|l|l|l|}
\hline
 & $e$ $^1$ & $r^1$, $r^{-1}$ $^2$ & $r^2$, $r^{-2}$ $^2$ & $r^3$, $r^{-3}$ $^2$ & $sr^b$ $^7$ \\ \hline
$\chi^{(1)}$ & $1$ & $1$ & $1$ & $1$ & $1$ \\ \hline
$\chi^{(2)}$ & $1$ & $1$ & $1$ & $1$ & $-1$ \\ \hline
$\chi^{(3)}$ & $2$ & $2\cos\frac{2\pi}{7}$ & $2\cos\frac{4\pi}{7}$ & $2\cos\frac{6\pi}{7}$ & $0$ \\ \hline
$\chi^{(4)}$ & $2$ & $2\cos\frac{4\pi}{7}$ & $2\cos\frac{6\pi}{7}$ & $2\cos\frac{2\pi}{7}$ & $0$ \\ \hline
$\chi^{(5)}$ & $2$ &  $2\cos\frac{6\pi}{7}$ & $2\cos\frac{2\pi}{7}$ & $2\cos\frac{4\pi}{7}$ & $0$  \\ \hline
\end{tabular}
\caption{Таблица характеров группы $D_7$}
\end{table}\\
Ещё раз проверим, что все найденные представления являются неприводимыми. Для этого воспользуемся критерием $\braket{\chi^{(i)},\chi^{(i)}}=1$:
\begin{equation}
    \braket{\chi^{(1)},\chi^{(1)}}=\frac{1}{14}(1\cdot1+2\cdot1+2\cdot 1+2\cdot 1+7\cdot1)=1
\end{equation}
\begin{equation}
    \braket{\chi^{(2)},\chi^{(2)}}=\frac{1}{14}(1\cdot1+2\cdot1+2\cdot 1+2\cdot 1+7\cdot1)=1
\end{equation}
\begin{equation}
    \braket{\chi^{(3)},\chi^{(3)}}=\frac{1}{14}(1\cdot4+2\cdot4\cos^2\frac{2\pi}{7}+2\cdot4\cos^2\frac{4\pi}{7}+1\cdot4\cos^2\frac{6\pi}{7}+7\cdot0)=1
\end{equation}
\begin{equation}
    \braket{\chi^{(4)},\chi^{(4)}}=\frac{1}{14}(1\cdot4+2\cdot4\cos^2\frac{4\pi}{7}+2\cdot4\cos^2\frac{6\pi}{7}+1\cdot4\cos^2\frac{2\pi}{7}+7\cdot0)=1
\end{equation}
\begin{equation}
    \braket{\chi^{(5)},\chi^{(5)}}=\frac{1}{14}(1\cdot4+2\cdot4\cos^2\frac{6\pi}{7}+2\cdot4\cos^2\frac{2\pi}{7}+1\cdot4\cos^2\frac{4\pi}{7}+7\cdot0)=1
\end{equation}
В приведённых выше равенствах использовано, что:
\begin{multline}
    \cos^2\frac{2\pi}{7}+\cos^2\frac{4\pi}{7}+\cos^2\frac{6\pi}{7}=\frac{(e^{i\frac{2\pi}{7}}+e^{-i\frac{2\pi}{7}})^2}{4}+\frac{(e^{i\frac{4\pi}{7}}+e^{-i\frac{4\pi}{7}})^2}{4}+\frac{(e^{i\frac{6\pi}{7}}+e^{-i\frac{6\pi}{7}})^2}{4}=\\=\frac{1}{4}(e^{i\frac{4\pi}{7}}+e^{-i\frac{4\pi}{7}}+2+e^{i\frac{8\pi}{7}}+e^{-i\frac{8\pi}{7}}+2+e^{i\frac{12\pi}{7}}+e^{-i\frac{12\pi}{7}}+2)=\frac{1}{4}(5+e^{-i\frac{12\pi}{7}}(1+e^{i\frac{4\pi}{7}}+e^{i\frac{8\pi}{7}}+...+e^{i\frac{24\pi}{7}}))=\\=\frac{5}{4}+\frac{e^{-i\frac{12\pi}{7}}}{4}\frac{e^{4\pi i}-1}{e^{i\frac{4\pi}{7}}-1}=\frac{5}{4}
\end{multline}
где в предпоследнем равенстве использована формула для суммы геометрической прогрессии.
\end{itemize}
\end{zad}
\begin{zad}
Классы сопряжённости $S_4$ описаны в задаче 3.2, п. а и в задаче 4.4, п. а и б. Всего классов сопряжённости 5, значит и неприводимых представлений 5. Найдём их размерности: $|S_4|=d_1^2+d_2^2+d_3^2+d_4^2+d_5^2=24$. У $S_n$ всего 2 одномерных неприводимых представления: тривиальное и знаковое (см. задачу 5.5, п.а), значит $d_1=d_2=1$, $d_3,d_4,d_5>1$. Найдём размерности остальных. Единственный возможный вариант: $d_3=d_4=3$, $d_5=2$.\\
Трёхмерные представления $S_4$ рассмотрены в задаче 5.3, п. б и в: $\rho_3$ получено из изоморфизма $S_4$ и группы симметрий тетраэдра, $\rho_4$ -- из изоморфизма $S_4$ и группы вращений куба.\\
Характеры двумерного произведения $\rho_5$ можно найти из следствия 1 предложения 9 лекции 6:
\begin{equation}
    \chi_\text{reg}=d_1\chi^{(1)}+d_2\chi^{(2)}+d_3\chi^{(3)}+d_4\chi^{(4)}+d_5\chi^{(5)}\rightarrow \chi^{(5)}=\frac{1}{2}(\chi_\text{reg}-\chi^{(1)}-\chi^{(2)}-3\chi^{(3)}-3\chi^{(4)})
\end{equation}
Характер регулярного произведения можно найти из предложения 8 лекции 6:
\begin{equation}
    \chi_\text{reg}(e)=|S_4|=24, \quad \chi_\text{reg}(g)=0, \; g\neq e
\end{equation}
\begin{equation}
    \chi^{(5)}(e)=\frac{1}{2}(24-1-1-3\cdot 3-3\cdot 3)=2=d_5
\end{equation}
\begin{equation}
    \chi^{(5)}((a,b))=\frac{1}{2}(0-1+1-3\cdot 1-3\cdot (-1))=0
\end{equation}
\begin{equation}
    \chi^{(5)}((a,b)(c,d))=\frac{1}{2}(0-1-1-3\cdot (-1)-3\cdot (-1))=2
\end{equation}
\begin{equation}
    \chi^{(5)}((a,b,c))=\frac{1}{2}(0-1-1-3\cdot 0-3\cdot 0)=-1
\end{equation}
\begin{equation}
    \chi^{(5)}((a,b,c,d))=\frac{1}{2}(0-1+1-3\cdot(-1)-3\cdot 1)=0
\end{equation}
\begin{table}[h!]
\centering
\begin{tabular}{|l|l|l|l|l|l|}
\hline
 & $e$ $^1$ & $(a,b)$ $^6$ & $(a,b)(c,d)$ $^3$ & $(a,b,c)$ $^8$ & $(a,b,c,d)$ $^6$ \\ \hline
$\chi^{(1)}$ & $1$ & $1$ & $1$ & $1$ & $1$ \\ \hline
$\chi^{(2)}$ & $1$ & $-1$ & $1$ & $1$ & $-1$ \\ \hline
$\chi^{(3)}$ & $3$ & $1$ & $-1$ & $0$ & $-1$ \\ \hline
$\chi^{(4)}$ & $3$ & $-1$ & $-1$ & $0$ & $1$ \\ \hline
$\chi^{(5)}$ & $2$ &  $0$ & $2$ & $-1$ & $0$  \\ \hline
\end{tabular}
\caption{Таблица характеров группы $S_4$}
\end{table}\\
Проверим соотношения ортогональности между характерами:
\begin{equation}
    \braket{\chi^{(1)},\chi^{(1)}}=\frac{1}{24}(1+6\cdot1+3\cdot 1+8\cdot 1+6\cdot1)=1
\end{equation}
\begin{equation}
    \braket{\chi^{(1)},\chi^{(2)}}=\frac{1}{24}(1-6\cdot1+3\cdot 1+8\cdot 1-6\cdot1)=0
\end{equation}
\begin{equation}
    \braket{\chi^{(1)},\chi^{(3)}}=\frac{1}{24}(1\cdot3+6\cdot1-3\cdot 1+8\cdot0-6\cdot1)=0
\end{equation}
\begin{equation}
    \braket{\chi^{(1)},\chi^{(4)}}=\frac{1}{24}(1\cdot3-6\cdot1-3\cdot 1+8\cdot 0+6\cdot1)=0
\end{equation}
\begin{equation}
    \braket{\chi^{(1)},\chi^{(5)}}=\frac{1}{24}(1\cdot2+6\cdot0+3\cdot2-8\cdot 1+6\cdot0)=0
\end{equation}

\begin{equation}
    \braket{\chi^{(2)},\chi^{(2)}}=\frac{1}{24}(1+6\cdot1+3\cdot 1+8\cdot 1+6\cdot1)=1
\end{equation}
\begin{equation}
    \braket{\chi^{(2)},\chi^{(3)}}=\frac{1}{24}(1\cdot3-6\cdot1-3\cdot 1+8\cdot0+6\cdot1)=0
\end{equation}
\begin{equation}
    \braket{\chi^{(2)},\chi^{(4)}}=\frac{1}{24}(1\cdot3+6\cdot1-3\cdot 1+8\cdot 0-6\cdot1)=0
\end{equation}
\begin{equation}
    \braket{\chi^{(2)},\chi^{(5)}}=\frac{1}{24}(1\cdot2+6\cdot0+3\cdot2-8\cdot 1+6\cdot0)=0
\end{equation}

\begin{equation}
    \braket{\chi^{(3)},\chi^{(3)}}=\frac{1}{24}(1\cdot9+6\cdot1+3\cdot 1+8\cdot0+6\cdot1)=1
\end{equation}
\begin{equation}
    \braket{\chi^{(3)},\chi^{(4)}}=\frac{1}{24}(1\cdot9-6\cdot1+3\cdot 1+8\cdot 0-6\cdot1)=0
\end{equation}
\begin{equation}
    \braket{\chi^{(3)},\chi^{(5)}}=\frac{1}{24}(1\cdot6+6\cdot0-3\cdot2-8\cdot0+6\cdot0)=0
\end{equation}

\begin{equation}
    \braket{\chi^{(4)},\chi^{(4)}}=\frac{1}{24}(1\cdot9+6\cdot1+3\cdot 1+8\cdot 0+6\cdot1)=1
\end{equation}
\begin{equation}
    \braket{\chi^{(4)},\chi^{(5)}}=\frac{1}{24}(1\cdot6-6\cdot0-3\cdot2-8\cdot0+6\cdot0)=0
\end{equation}

\begin{equation}
    \braket{\chi^{(5)},\chi^{(5)}}=\frac{1}{24}(1\cdot4+6\cdot0+3\cdot4+8\cdot1+6\cdot0)=1
\end{equation}
Рассмотрим $\rho_6=\rho_3\otimes\rho_4$. По предложению 5 лекции 6 получим:
\begin{equation}
    \chi^{(6)}=\chi^{(4)}\cdot \chi^{(5)}
\end{equation}
\begin{table}[h!]
\centering
\begin{tabular}{|l|l|l|l|l|l|}
\hline
 & $e$ $^1$ & $(a,b)$ $^6$ & $(a,b)(c,d)$ $^3$ & $(a,b,c)$ $^8$ & $(a,b,c,d)$ $^6$ \\ \hline
$\chi^{(6)}$ & $9$ & $-1$ & $1$ & $0$ & $-1$ \\ \hline
\end{tabular}
\end{table}\\
Воспользуемся алгоритом разложения на неприводимые:
\begin{equation}
    \chi^{(6)}=\sum\limits_{i=1}^5 a_i\chi^{(i)},\quad a_i=\braket{\chi^{(i)},\chi^{(6)}}
\end{equation}
\begin{equation}
    a_1=\frac{1}{24}(1\cdot9-6\cdot1+3\cdot1+8\cdot0-6\cdot1)=0
\end{equation}
\begin{equation}
    a_2=\frac{1}{24}(1\cdot9+6\cdot1+3\cdot1+8\cdot0+6\cdot1)=1
\end{equation}
\begin{equation}
    a_3=\frac{1}{24}(1\cdot27-6\cdot1-3\cdot1+8\cdot0+6\cdot1)=1
\end{equation}
\begin{equation}
    a_4=\frac{1}{24}(1\cdot27+6\cdot1-3\cdot1+8\cdot0-6\cdot1)=1
\end{equation}
\begin{equation}
    a_5=\frac{1}{24}(1\cdot18+6\cdot0+3\cdot2+8\cdot0-6\cdot0)=1
\end{equation}
Таким образом, получаем разложение:
\begin{equation}
    \boxed{V_6=V_2\oplus V_3\oplus V_4\oplus V_5}
\end{equation}
\end{zad}
\begin{zad}
\begin{predl}[\textbf{Второе соотношение ортогональности для характеров}]
    Для любых двух классов сопряжённости $C_i$, $C_j$, где $i\leq i,j\leq k$ верно:
    \begin{equation}
        \sum\limits_{\alpha=1}^k\chi^{(\alpha)}(h_i)\overline{\chi^{(\alpha)}(h_j)}=\delta_{i,j}\frac{|G|}{|C_i|}
    \end{equation}
    \begin{proof}
        Разложим $\gamma_i$ по базису $\chi^{(\alpha)}$:
        \begin{equation}\label{eq2}
            \gamma_i=\sum\limits_{\alpha=1}^k\gamma_i^{(\alpha)}\chi^{(\alpha)},
        \end{equation} 
        где $\gamma_i^\alpha$ можно найти по формуле
        \begin{equation}
            \gamma_i^{(\alpha)}=\braket{\chi^{(\alpha)},\gamma_i}=\frac{1}{|G|}\sum\limits_{j=1}^k\overline{\chi^{(\alpha)}(h_j)}\gamma_i(h_j)
        \end{equation}
        Воспользуемся определением характеристической функции $\gamma_i$:
        \begin{equation}
            \gamma_i(h_j)=\delta_{i,j}
        \end{equation}
        Тогда в сумме останутся $|C_i|$ слагаемых $\chi^{(\alpha)}(h_i)$:
        \begin{equation}
            \gamma_i^{(\alpha)}=\frac{|C_i|}{|G|}\overline{\chi^{(\alpha)}(h_i)}
        \end{equation}
        Подставим в (\ref{eq2}):
        \begin{equation}
            \gamma_i=\sum\limits_{\alpha=1}^k\frac{|C_i|}{|G|}\overline{\chi^{(\alpha)}(h_i)}\chi^{(\alpha)}=\frac{|C_i|}{|G|}\sum\limits_{\alpha=1}^k\overline{\chi^{(\alpha)}(h_i)}\chi^{(\alpha)}
        \end{equation}
        \begin{equation}
            \gamma_i(h_j)=\frac{|C_i|}{|G|}\sum\limits_{\alpha=1}^k\overline{\chi^{(\alpha)}(h_i)}\chi^{(\alpha)}(h_j)=\delta_{i,j}
        \end{equation}
        \begin{equation}
            \sum\limits_{\alpha=1}^k\overline{\chi^{(\alpha)}(h_i)}\chi^{(\alpha)}(h_j)=\delta_{i,j}\frac{|G|}{|C_j|}
        \end{equation}
        Поменяем индексы $i$ и $j$ местами:
        \begin{equation}
            \sum\limits_{\alpha=1}^k\chi^{(\alpha)}(h_i)\overline{\chi^{(\alpha)}(h_j)}=\delta_{i,j}\frac{|G|}{|C_i|}
        \end{equation}
    \end{proof}
\end{predl}
\end{zad}
\section{Разные конструкции. Группа $SO(2)$.}
\begin{upr}[*]
\begin{predl}
    Пусть $A$ -- матрица, состоящая из одного жорданова блока размера $n\times n$, $n>1$ с собственным значением $\lambda\in\mathbb{C}$. Матрица $\exp(\alpha A)$ не будет диагональной при любом $\alpha\in\mathbb{C}$, $\alpha\neq 0$. При $\lambda\neq 0$ матрица $A^k$ не будет диагональной при любом $k\in\mathbb{N}$.
\end{predl}
\begin{proof}
    Жорданова клетка $A$:
    \begin{equation}
    A = \left(
    \begin{array}{cccccc}
    \lambda & 1 & 0 & \ldots & 0 & 0\\
    0 & \lambda & 1 & \ldots & 0 & 0\\
    0 & 0 & \lambda & \ldots & 0 & 0\\
    \vdots & \vdots & \vdots &\ddots & \vdots & \vdots\\
    0 & 0 & 0 & \ldots & \lambda & 1\\
    0 & 0 & 0 & \ldots & 0 & \lambda
    \end{array}
    \right)
    \end{equation}
    \begin{equation}
        A=\lambda E+B
    \end{equation}
    где $B$ -- нильпонентная матрица: $B^n=0$. Домножение $B^i$ на $B$ сдвигает <<диагональ единиц>> на одну позицию вверх. Найдём экспоненту матрицы $B$:
    \begin{equation}
        e^B=E+\sum\limits_{i=1}^{n-1}\frac{B^i}{i!}=\left(
    \begin{array}{cccccc}
    1 & 1 & \frac{1}{2} & \ldots & \frac{1}{(n-2)!} & \frac{1}{(n-1)!}\\
    0 & 1 & 1 & \ldots & \frac{1}{(n-3)!} & \frac{1}{(n-2)!}\\
    0 & 0 & 1 & \ldots & \frac{1}{(n-4)!} & \frac{1}{(n-3)!}\\
    \vdots & \vdots & \vdots &\ddots & \vdots & \vdots\\
    0 & 0 & 0 & \ldots & 1 & 1\\
    0 & 0 & 0 & \ldots & 0 & 1
    \end{array}
    \right)
    \end{equation}
    \begin{equation}
        e^{\alpha B}=E+\sum\limits_{i=1}^{n-1}\frac{(\alpha B)^i}{i!}=\left(
    \begin{array}{cccccc}
    1 & \alpha & \frac{\alpha^2}{2} & \ldots & \frac{\alpha^{n-2}}{(n-2)!} & \frac{\alpha^{n-1}}{(n-1)!}\\
    0 & 1 & \alpha & \ldots & \frac{\alpha^{n-3}}{(n-3)!} & \frac{\alpha^{n-2}}{(n-2)!}\\
    0 & 0 & 1 & \ldots & \frac{\alpha^{n-4}}{(n-4)!} & \frac{\alpha^{n-3}}{(n-3)!}\\
    \vdots & \vdots & \vdots &\ddots & \vdots & \vdots\\
    0 & 0 & 0 & \ldots & 1 & \alpha\\
    0 & 0 & 0 & \ldots & 0 & 1
    \end{array}
    \right)
    \end{equation}
    \begin{equation}
        e^{\alpha A}=e^{\alpha \lambda E}e^{\alpha B}=e^{\alpha\lambda}e^{\alpha B}=e^{\alpha\lambda}\left(
    \begin{array}{cccccc}
    1 & \alpha & \frac{\alpha^2}{2} & \ldots & \frac{\alpha^{n-2}}{(n-2)!} & \frac{\alpha^{n-1}}{(n-1)!}\\
    0 & 1 & \alpha & \ldots & \frac{\alpha^{n-3}}{(n-3)!} & \frac{\alpha^{n-2}}{(n-2)!}\\
    0 & 0 & 1 & \ldots & \frac{\alpha^{n-4}}{(n-4)!} & \frac{\alpha^{n-3}}{(n-3)!}\\
    \vdots & \vdots & \vdots &\ddots & \vdots & \vdots\\
    0 & 0 & 0 & \ldots & 1 & \alpha\\
    0 & 0 & 0 & \ldots & 0 & 1
    \end{array}
    \right)
    \end{equation}
    Данная матрица не является диагональной при $\alpha\neq0$.\\
    \begin{equation*}
        A^k=(\lambda E+B)^k=\sum\limits_{i=1}^{\text{min}(k,n-1)}C_k^i\lambda^{n-i}B^i=\left(
    \begin{array}{cccccc}
    \lambda^k & C_k^1\lambda^{k-1} & C_k^2\lambda^{k-2} & \ldots & C_k^{n-2}\lambda^{k-n+2} & C_k^{n-1}\lambda^{k-n+1}\\
    0 & \lambda^k & C_k^1\lambda^{k-1} & \ldots & C_k^{n-3}\lambda^{k-n+3} & C_k^{n-2}\lambda^{k-n+2}\\
    0 & 0 & \lambda^k & \ldots & C_k^{n-4}\lambda^{k-n+4} & C_k^{n-3}\lambda^{k-n+3}\\
    \vdots & \vdots & \vdots &\ddots & \vdots & \vdots\\
    0 & 0 & 0 & \ldots & \lambda^k & C_k^1\lambda^{k-1}\\
    0 & 0 & 0 & \ldots & 0 & \lambda^k
    \end{array}
    \right)
    \end{equation*}
    где подразумевается, что $C^i_k=0$ при $i>k$. При $k\geq 1$ и $\lambda\neq 0$ матрица $A^k$ не диагональна.
\end{proof}
\end{upr}
\begin{zad}
\begin{itemize}
\item[а)] 
Классы сопряжённости групп $D_n$ описаны в задаче 4.3, п. а. В $D_6$ 6 классов сопряжённости: $\{e\}$, $\{r^{-1},r^1\}$, $\{r^2,r^{-2}\}$, $\{r^3\}$, $\{sr^{2b}\}$, $\{sr^{2b+1}\}$. Значит и число неприводимых представлений равно 6. Найдём их размерности: $|D_6|=d_1^2+d_2^2+d_3^2+d_4^2+d_5^2+d_6^2=12$. Единственный возможный вариант: $d_1=d_2=d_3=d_4=1$, $d_5=d_6=2$.\\
Найдём все одномерные представления (их всего 4). Коммутанты $D_n$ описаны в задаче 4.3, п. б. $[D_6,D_6]=\{e,r^2,r^4\}$. Факторгруппа $D_6/[D_6,D_6]\simeq C_2\times C_2$, т.е. $|D_6/[D_6,D_6]|=|C_2\times C_2|=4$ (ещё раз получаем, что всего 4 одномерных неприводимых представления). Любое одномерное неприводимое представление -- представление фактора по коммутанту. Первое одномерное неприводимое представление -- \textit{тривиальное}: все элементы $D_6$ переходят в 1.
\begin{equation}
    \boxed{\rho_1(r)=\rho_1(s)=1}
\end{equation}
Для отражений верно, что $(sr^b)^2=sr^bsr^b=r^{-b}r^b=e$. Поэтому $\rho(sr^b)^2=\rho((sr^b)^2)=\rho(e)=1$ и $\rho(sr^b)=\pm 1$. $\rho(sr^b)=1$ соответствует тривиальному представлению, а $\rho(sr^b)=-1$ -- второму одномерному представлению. Т.е. во втором одномерном представлении все повороты перейдут в $1$, а отражения -- в $-1$.
\begin{equation}
    \boxed{\rho_2(r)=1,\quad \rho_2(s)=-1}
\end{equation}
Произведение двух нечётных поворотов -- чётный поворот. Поэтому все чётные повороты можно перевести в $1$, а нечётные -- в $-1$. Отражения $sr^{2b}$ можно перевести в $1$, а $sr^{2b+1}$ -- в $-1$ (третье представление) или наоборот (четвёртое представление). При этом будут выполняться необходимые тождества: $\rho(r^ar^b)=\rho(r^{a+b})$, $\rho(r^asr^b)=\rho(sr^{b-a})=\rho(sr^{a+b})=\rho(sr^ar^b)$, $\rho(sr^asr^b)=\rho(r^{b-a})$.
\begin{equation}
    \boxed{\rho_3(r^2)=1,\quad \rho_3(r)=-1,\quad \rho_3(sr^2)=1,\quad \rho_3(sr)=-1}
\end{equation}
\begin{equation}
    \boxed{\rho_4(r^2)=1,\quad \rho_4(r)=-1,\quad \rho_4(sr^2)=-1,\quad \rho_4(sr)=1}
\end{equation}
Найдём 2 оставшихся двумерных неприводимых представления. Для них должны выполняться тождества: $\rho(r^b)^6=\rho((r^b)^6)=\rho(e)=1$ (матрицы поворота), $\rho(sr^b)^2=\rho((sr^b)^2)=\rho(e)=1$ (матрица отражения), $\rho(r)\rho(s)\rho(r)\rho(s)=\rho(rsrs)=\rho(e)=1$.
\begin{equation}
    \boxed{\rho_5(r)=\left(
    \begin{array}{cc}
    \cos\frac{2\pi}{6} & -\sin\frac{2\pi}{6}\\
    \sin\frac{2\pi}{6} & \cos\frac{2\pi}{6}\\
    \end{array}
    \right),\quad \rho_5(s)=\left(
    \begin{array}{cc}
    1 & 0\\
    0 & -1\\
    \end{array}
    \right)}
\end{equation}
\begin{equation}
    \boxed{\rho_6(r)=\left(
    \begin{array}{cc}
    \cos\frac{4\pi}{6} & -\sin\frac{4\pi}{6}\\
    \sin\frac{4\pi}{6} & \cos\frac{4\pi}{6}\\
    \end{array}
    \right),\quad \rho_6(s)=\left(
    \begin{array}{cc}
    1 & 0\\
    0 & -1\\
    \end{array}
    \right)}
\end{equation}
\begin{table}[h!]
\centering
\begin{tabular}{|l|l|l|l|l|l|l|}
\hline
 & $e$ $^1$ & $r^1$, $r^{-1}$ $^2$ & $r^2$, $r^{-2}$ $^2$ & $r^3$ $^1$ & $sr^{2b}$ $^3$ & $sr^{2b+1}$ $^3$ \\ \hline
$\chi^{(1)}$ & $1$ & $1$ & $1$ & $1$ & $1$ & $1$ \\ \hline
$\chi^{(2)}$ & $1$ & $1$ & $1$ & $1$ & $-1$ & $-1$\\ \hline
$\chi^{(3)}$ & $1$ & $-1$ & $1$ & $-1$ & $1$ & $-1$\\ \hline
$\chi^{(4)}$ & $1$ & $-1$ & $1$ & $-1$ & $-1$ & $1$\\ \hline
$\chi^{(5)}$ & $2$ & $1$ & $-1$ & $-2$ & $0$ & $0$ \\ \hline
$\chi^{(6)}$ & $2$ & $-1$ & $-1$ & $2$ & $0$ & $0$ \\ \hline
\end{tabular}
\caption{Таблица характеров группы $D_6$}
\end{table}\\
Ещё раз проверим, что все найденные представления являются неприводимыми. Для этого воспользуемся критерием $\braket{\chi^{(i)},\chi^{(i)}}=1$:
\begin{equation}
    \braket{\chi^{(1)},\chi^{(1)}}=\frac{1}{12}(1\cdot1+2\cdot1+2\cdot 1+1\cdot 1+3\cdot1+3\cdot1)=1
\end{equation}
\begin{equation}
    \braket{\chi^{(2)},\chi^{(2)}}=\frac{1}{12}(1\cdot1+2\cdot1+2\cdot 1+1\cdot 1+3\cdot1+3\cdot1)=1
\end{equation}
\begin{equation}
    \braket{\chi^{(3)},\chi^{(3)}}=\frac{1}{12}(1\cdot1+2\cdot1+2\cdot 1+1\cdot 1+3\cdot1+3\cdot1)=1
\end{equation}
\begin{equation}
    \braket{\chi^{(4)},\chi^{(4)}}=\frac{1}{12}(1\cdot1+2\cdot1+2\cdot 1+1\cdot 1+3\cdot1+3\cdot1)=1
\end{equation}
\begin{equation}
    \braket{\chi^{(5)},\chi^{(5)}}=\frac{1}{12}(1\cdot4+2\cdot1+2\cdot 1+1\cdot 4+3\cdot0+3\cdot0)=1
\end{equation}
\begin{equation}
    \braket{\chi^{(6)},\chi^{(6)}}=\frac{1}{12}(1\cdot4+2\cdot1+2\cdot 1+1\cdot 4+3\cdot0+3\cdot0)=1
\end{equation}
\item[б)] Группа $D_{nh}$ рассмотрена в задаче 3.3, п. а. $D_{nh}=D_n\times C_2$. По теореме 1 лекции 7 все неприводимые представления $D_{6h}$ можно получить как $\rho^{D_6}\boxtimes\rho^{C_2}$, где $\rho^{D_6}$ и $\rho^{C_2}$ -- неприводимые представления $D_6$ и $C_2$ соответственно. В группе $C_2$ всего 2 элемента: $e$ и $r$, каждый из которых является классом сопряжённости. Значит всего 2 неприводимых представления. Найдём их размерности: $|C_2|=d_1^2+d_2^2=2$. Единственный возможный вариант: $d_1=d_2=1$.
\begin{equation}
    \rho_1(e)=1,\quad\rho_1(r)=1
\end{equation}
\begin{equation}
    \rho_2(e)=1,\quad\rho_2(r)=-1
\end{equation}
\begin{table}[h!]
\centering
\begin{tabular}{|l|l|l|}
\hline
 & $e$ $^1$ & $r$, $^1$ \\ \hline
$\chi^{(1)}$ & $1$ & $1$ \\ \hline
$\chi^{(2)}$ & $1$ & $-1$ \\ \hline
\end{tabular}
\caption{Таблица характеров группы $C_2$}
\end{table}
\begin{equation}
\text{dim}(V\otimes U)=\text{dim}\;V\cdot\text{dim}\;U
\end{equation}
Значит для получения двумерных неприводимых представлений нужно перемножить два двумерных представления $D_6$ с двумя одномерными представлениями $C_2$. Получится всего 4 двумерных подпространства $D_{6h}$.\\
Их характеры можно найти по формуле
\begin{equation}
    \chi_{\rho^{D_6}\boxtimes\rho^{C_2}}(g^{D_6},g^{C_2})=\chi_{\rho^{D_6}}(g^{D_6})\cdot\chi_{\rho^{C_2}}(g^{C_2}),
\end{equation}
где $g^{D_6}\in D_6$ и $g^{C_2}\in C_2$.
\begin{table}[h!]
\centering
\begin{tabular}{|l|l|l|l|l|l|l|}
\hline
 & $(e,e)$ $^1$ & $(r^1,e)$, $(r^{-1},e)$ $^2$ & $(r^2,e)$, $(r^{-2},e)$ $^2$ & $(r^3,e)$ $^1$ & $(sr^{2b},e)$ $^3$ & $(sr^{2b+1},e)$ $^3$ \\ \hline
$\chi_{\rho_5^{D_6}\boxtimes\rho_1^{C_2}}$ & $2$ & $1$ & $-1$ & $-2$ & $0$ & $0$ \\ \hline
$\chi_{\rho_6^{D_6}\boxtimes\rho_1^{C_2}}$ & $2$ & $-1$ & $-1$ & $2$ & $0$ & $0$ \\ \hline
 & $(e,r)$ $^1$ & $(r^1,r)$, $(r^{-1},r)$ $^2$ & $(r^2,r)$, $(r^{-2},r)$ $^2$ & $(r^3,r)$ $^1$ & $(sr^{2b},r)$ $^3$ & $(sr^{2b+1},r)$ $^3$ \\ \hline
$\chi_{\rho_5^{D_6}\boxtimes\rho_1^{C_2}}$ & $2$ & $1$ & $-1$ & $-2$ & $0$ & $0$ \\ \hline
$\chi_{\rho_6^{D_6}\boxtimes\rho_1^{C_2}}$ & $2$ & $-1$ & $-1$ & $2$ & $0$ & $0$ \\ \hline
 & $(e,e)$ $^1$ & $(r^1,e)$, $(r^{-1},e)$ $^2$ & $(r^2,e)$, $(r^{-2},e)$ $^2$ & $(r^3,e)$ $^1$ & $(sr^{2b},e)$ $^3$ & $(sr^{2b+1},e)$ $^3$ \\ \hline
$\chi_{\rho_5^{D_6}\boxtimes\rho_2^{C_2}}$ & $2$ & $1$ & $-1$ & $-2$ & $0$ & $0$ \\ \hline
$\chi_{\rho_6^{D_6}\boxtimes\rho_2^{C_2}}$ & $2$ & $-1$ & $-1$ & $2$ & $0$ & $0$ \\ \hline
 & $(e,r)$ $^1$ & $(r^1,r)$, $(r^{-1},r)$ $^2$ & $(r^2,r)$, $(r^{-2},r)$ $^2$ & $(r^3,r)$ $^1$ & $(sr^{2b},r)$ $^3$ & $(sr^{2b+1},r)$ $^3$ \\ \hline
$\chi_{\rho_5^{D_6}\boxtimes\rho_2^{C_2}}$ & $-2$ & $-1$ & $1$ & $2$ & $0$ & $0$ \\ \hline
$\chi_{\rho_6^{D_6}\boxtimes\rho_2^{C_2}}$ & $-2$ & $1$ & $1$ & $-2$ & $0$ & $0$ \\ \hline
\end{tabular}
\caption{Двумерные неприводимые представления $D_{6h}$}
\end{table}
\end{itemize}
\end{zad}
\begin{zad}
Классы сопряжённости $A_4$ описаны в задаче 3.2, п. б. Всего классов сопряжённости 4, значит и неприводимых представлений 4. Найдём их размерности: $|A_4|=d_1^2+d_2^2+d_3^2+d_4^2=12$. Единственный возможный вариант: $d_1=d_2=d_3=1$, $d_4=3$.\\
Найдём все одномерные представления. Для этого найдём коммутант $A_4$. В $A_4$ всего 4 коммутатора, они порождают коммутант: $[A_4,A_4]=\{e, (1,2)(3,4), (1,3)(2,4), (1,4)(2,3)\}$. Факторгруппа $A_4/[A_4,A_4]\simeq\mathbb{Z}_3$, т.е. $|A_4/[A_4,A_4]|=|\mathbb{Z}_3|=3$ (ещё раз получаем, что всего 2 одномерных неприводимых представления). Любое одномерное неприводимое представление -- представление фактора по коммутанту. Первое одномерное неприводимое представление -- \textit{тривиальное}: все элементы $A_4$ переходят в 1.
\begin{equation}
    \rho_1(g)=1, \quad \forall g\in A_4
\end{equation}
Второе одномерное неприводимое представление -- циклы из группы $(1,2,3)$ переходят в корень 3 степени из 1 ($e^{i\frac{2\pi}{3}}$), циклы из группы $(1,3,2)$ -- в другой корень 3 степени из 1 ($e^{i\frac{4\pi}{3}}$) (квадрат цикла из группы $(1,2,3)$ равен циклу из группы $(1,3,2)$ и наоборот). Произведения транспозиций $(a,b)(c,d)$ перейдут в 1, поскольку произведение циклов длины 3 из разных группы равно произведению транспозиций и $\rho((a,b)(c,d))=e^{i\frac{2\pi}{3}}e^{i\frac{4\pi}{3}}=1$.
\begin{equation}
    \rho_2((a,b)(c,d))=1,\quad \rho_2((1,2,3))=e^{i\frac{2\pi i}{3}},\quad \rho_2((1,3,2))=e^{i\frac{4\pi i}{3}}
\end{equation}
Третье одномерное неприводимое представление -- циклы из группы $(1,2,3)$ переходят в корень 3 степени из 1 ($e^{i\frac{4\pi}{3}}$), циклы из группы $(1,3,2)$ -- в другой корень 3 степени из 1 ($e^{i\frac{2\pi}{3}}$) (квадрат цикла из группы $(1,2,3)$ равен циклу из группы $(1,3,2)$ и наоборот). Произведения транспозиций $(a,b)(c,d)$ перейдут в 1, поскольку произведение циклов длины 3 из разных группы равно произведению транспозиций и $\rho((a,b)(c,d))=e^{i\frac{2\pi}{3}}e^{i\frac{4\pi}{3}}=1$.
\begin{equation}
    \rho_3((a,b)(c,d))=1,\quad \rho_2((1,2,3))=e^{i\frac{4\pi i}{3}},\quad \rho_2((1,3,2))=e^{i\frac{2\pi i}{3}}
\end{equation}
где через $(1,2,3)$ обозначены все циклы длины 3, содержащиеся в одном с $(1,2,3)$ классе сопряжённости, через $(1,3,2)$ аналогично.\\
Трёхмерное неприводимое представление $A_4$ можно получить как ограничение трёхмерных неприводимых представлений $S_4$ на $A_4$ (они сольются в одно).
\begin{table}[h!]
\centering
\begin{tabular}{|l|l|l|l|l|}
\hline
 & $e$ $^1$ & $(a,b)(c,d)$ $^3$ & $(1,2,3)$ $^4$ & $(1,3,2)$ $^4$ \\ \hline
$\chi^{(1)}$ & $1$ & $1$ & $1$ & $1$ \\ \hline
$\chi^{(2)}$ & $1$ & $1$ & $e^{i\frac{2\pi}{3}}$ & $e^{i\frac{4\pi}{3}}$ \\ \hline
$\chi^{(3)}$ & $1$ & $1$ & $e^{i\frac{4\pi}{3}}$ & $e^{i\frac{2\pi}{3}}$ \\ \hline
$\chi^{(4)}$ & $3$ & $-1$ & $0$ & $0$ \\ \hline
\end{tabular}
\caption{Таблица характеров группы $A_4$}
\end{table}\\
Ещё раз проверим, что все найденные представления являются неприводимыми. Для этого воспользуемся критерием $\braket{\chi^{(i)},\chi^{(i)}}=1$:
\begin{equation}
    \braket{\chi^{(1)},\chi^{(1)}}=\frac{1}{12}(1\cdot1+3\cdot1+4\cdot 1+4\cdot 1)=1
\end{equation}
\begin{equation}
    \braket{\chi^{(2)},\chi^{(2)}}=\frac{1}{12}(1\cdot1+3\cdot1+4\cdot e^{i\frac{4\pi}{3}}+4\cdot e^{i\frac{2\pi}{3}})=1
\end{equation}
\begin{equation}
    \braket{\chi^{(3)},\chi^{(3)}}=\frac{1}{12}(1\cdot1+3\cdot1+4\cdot e^{i\frac{2\pi}{3}}+4\cdot e^{i\frac{4\pi}{3}})=1
\end{equation}
\begin{equation}
    \braket{\chi^{(4)},\chi^{(4)}}=\frac{1}{12}(1\cdot9+3\cdot1+4\cdot0+4\cdot0)=1
\end{equation}
Разложим ограничения неприводимых представлений $S_4$ на $A_4$:
\begin{table}[h!]
\centering
\begin{tabular}{|l|l|l|l|l|}
\hline
 & $e$ $^1$ & $(a,b)(c,d)$ $^3$ & $(1,2,3)$ $^4$ & $(1,2,3)$ $^4$ \\ \hline
$\bar{\chi}^{(1)}$ & $1$ & $1$ & $1$ & $1$ \\ \hline
$\bar{\chi}^{(2)}$ & $1$ & $1$ & $1$ & $1$ \\ \hline
$\bar{\chi}^{(3)}$ & $3$ & $-1$ & $0$ & $0$ \\ \hline
$\bar{\chi}^{(4)}$ & $3$ & $-1$ & $0$ & $0$ \\ \hline
$\bar{\chi}^{(5)}$ & $2$ & $2$ & $-1$ & $-1$ \\ \hline
\end{tabular}
\caption{Ограничение неприводимых представлений $S_4$ на $A_4$}
\end{table}\\
Как видно,
\begin{equation}
    \boxed{\bar{\chi}^{(1)}=\bar{\chi}^{(2)}=\chi^{(1)}}
\end{equation}
\begin{equation}
    \boxed{\bar{\chi}^{(3)}=\bar{\chi}^{(4)}=\chi^{(4)}}
\end{equation}
Разложим $\bar{\chi}^{(5)}$ при помощи алгоритма разложения на неприводимые:
\begin{equation}
    \bar{\chi}^{(5)}=\sum\limits_{i=1}^5 a_i\chi^{(i)},\quad a_i=\braket{\chi^{(i)},\chi^{(5)}}
\end{equation}
\begin{equation}
    a_1=\frac{1}{12}(1\cdot2+3\cdot2-4\cdot1-4\cdot1)=0
\end{equation}
\begin{equation}
    a_2=\frac{1}{12}(1\cdot2+3\cdot2-4\cdot e^{i\frac{2\pi}{3}}-4\cdot e^{i\frac{4\pi}{3}})=1
\end{equation}
\begin{equation}
    a_3=\frac{1}{12}(1\cdot2+3\cdot2-4\cdot e^{i\frac{4\pi}{3}}-4\cdot e^{i\frac{2\pi}{3}})=1
\end{equation}
\begin{equation}
    a_4=\frac{1}{12}(1\cdot6-3\cdot2-4\cdot0-4\cdot0)=0
\end{equation}
Таким образом, двумерное пространство оказалось приводимым:
\begin{equation}
    \boxed{\bar{\chi}^{(5)}=\chi^{(2)}+\chi^{(3)}}
\end{equation}
\end{zad}
\begin{zad}
\begin{itemize}
    \item[а)]
    \begin{predl}
        Порядок группы вращений додекаэдра $G_0$ равен 60.
    \end{predl}
    \begin{proof}
    Пусть $x$ -- произвольная грань додекаэдра. Симметриями она может быть переведена в любую другую, а значит $|Gx|=12$ (в додекаэдре 12 граней). Вращения, переводящие грань додекаэдра (правильный пятиугольник) в себя, образуют группу $C_5$. $|G_x|=|C_5|=5$. Таким образом,
    \begin{equation}
        \boxed{|G_0|=|G_x||Gx|=60}
    \end{equation}
    \end{proof}
    \item[б)] Различные собственными движения, сохраняющие додекаэдр:
    \begin{enumerate}
        \item $\text{Id}$ (1 шт.) -- тождественное движение (ничего не делает с додекаэдром).
        \item $r_1$ (12 шт., порядок 5) -- повороты на $\pm\frac{2\pi}{5}$ вокруг прямых $l_1$, проходящих через центры противоположных граней.
        \item $r_2$ (12 шт., порядок 5) -- повороты на $\pm\frac{4\pi}{5}$ вокруг прямых $l_1$, проходящих через центры противоположных граней.
        \item $r_3$ (20 шт., порядок 3) -- повороты на $\frac{2\pi}{3}$, $\frac{4\pi}{3}$ вокруг прямых $l_2$, проходящих через противоположные вершины.
        \item $r_4$ (15 шт., порядок 2) -- повороты на $\pi$ вокруг прямых $l_3$, проходящих через середины противоположных рёбер.
    \end{enumerate}
    \item[в)]
    \begin{predl}
        $G_0$ изоморфна группе чётных перестановок $A_5$.
    \end{predl}
    \begin{proof}
    Для доказательства изоморфизма выпишем соответствие между перестановками различных циклических типов в $A_5$ 5 тетраэдров и различными движениями, сохраняющими додекаэдр:
    \begin{enumerate}
        \item $e$ (единичный, 1 шт.) -- $\text{Id}$.
        \item $(1,2,3,4,5)$, $(1,2,4,5,3)$, $(1,2,5,3,4)$, $(1,3,2,5,4)$, $(1,3,4,2,5)$, $(1,3,5,4,2)$, $(1,4,2,3,5)$, $(1,4,3,5,2)$, $(1,4,5,2,3)$, $(1,5,2,4,3)$, $(1,5,3,2,4)$, $(1,5,4,3,2)$\\ (циклы длины 5, 12 шт.) -- $r_1$.
        \item $(1,2,3,5,4)$, $(1,2,4,3,5)$, $(1,2,5,4,3)$, $(1,3,2,4,5)$, $(1,3,4,5,2)$, $(1,3,5,2,4)$, $(1,4,2,5,3)$, $(1,4,3,2,5)$, $(1,4,5,3,2)$, $(1,5,2,3,4)$, $(1,5,3,4,2)$, $(1,5,4,2,3)$\\ (циклы длины 5, 12 шт.) -- $r_2$.
        \item $(a,b,c)$ (циклы длины 3, 20 шт.) -- $r_3$.
        \item $(a,b)(c,d)$ (произведение транспозиций, 15 шт.) -- $r_4$.
    \end{enumerate}
    Таким образом, построено однозначное соответствие между перестановками и движениями, сохраняющее групповые операции, а значит изоморфизм групп построен.
    \end{proof}
    \item[г)]
    \begin{predl}
        Группа всех симметрий додекаэдра $G$ изоморфна группе $C_2\times A_5$.
    \end{predl}
    \begin{proof}
        В $G$ тождественное движение и центральная симметрия образуют $C_2$.
        Все движения из $G$ можно представить в виде произведения собственных движений из $G_0$ и элементов $C_2$. Подгруппы $G_0$ и $C_2$ коммутируюет и пересекаются только по $e$. По предложению 7 лекции 4 предложение доказано.
    \end{proof}
    \item[д)]
    \begin{predl}
        Группы $S_5$ и $C_2\times A_5$ не изоморфны.
    \end{predl}
    \begin{proof}
        В группе $S_5$ максимальный порядок среди всех элементов 6 (у элементов $(a,b,c)(d,e)$. В группе $C_2\times A_5$ есть элемент порядка 10 (у элементов $((a,b,c,d,e),s)$, где $s$ -- центальная симметрия). Значит, эти группы изоморфными быть не могут.
    \end{proof}
    \end{itemize}
\end{zad}
\begin{zad}
    По предложению 7 лекции 3 перестановки сопряжены тогда и только тогда, когда имеют одинаковую циклическую структуру. Значит, в $S_5$ всего 7 классов сопряжённости: $\{e,(a,b),(a,b,c),(a,b,c,d),(a,b,c,d,e),(a,b)(c,d),(a,b,c)(d,e)\}$.\\
    Рассмотрим перестановочное представление $S_5$. Оно имеет два нетривиальных подпредставления
    \begin{equation}
        U_1=\{\sum\limits_{i=1}^5 x_ie_i|x_1=x_2=...=x_5\},\quad U_2=\{\sum\limits_{i=1}^5 x_ie_i|\sum\limits_{i=1}^5x_i=0\}
    \end{equation}
    \begin{equation}
        \mathbb{C}^5=U_1\oplus U_2
    \end{equation}
    Любое одномерное представление неприводимо. Докажем, что $U_2$ также является неприводимым. Для этого воспользуемся базисом:
    \begin{equation}
        e'=e_1-e_2=\left(
    \begin{array}{c}
    1\\
    -1\\
    0\\
    0\\
    0\\
    \end{array}
    \right),\quad e''=e_1-e_3=\left(
    \begin{array}{c}
    1\\
    0\\
    -1\\
    0\\
    0\\
    \end{array}
    \right)
    \end{equation}
    \begin{equation}
    e'''=e_1-e_4=\left(
    \begin{array}{c}
    1\\
    0\\
    0\\
    -1\\
    0\\
    \end{array}
    \right),\quad e''''=e_1-e_5=\left(
    \begin{array}{c}
    1\\
    0\\
    0\\
    0\\
    -1\\
    \end{array}
    \right)
    \end{equation}
    Любой вектор может быть выражен через базисные вектора:
    \begin{equation}
        \left(
    \begin{array}{c}
    x_1\\
    x_2\\
    x_3\\
    x_4\\
    x_5\\
    \end{array}
    \right)=-x_2\left(
    \begin{array}{c}
    1\\
    -1\\
    0\\
    0\\
    0\\
    \end{array}
    \right)-x_3\left(
    \begin{array}{c}
    1\\
    0\\
    -1\\
    0\\
    0\\
    \end{array}
    \right)-x_4\left(
    \begin{array}{c}
    1\\
    0\\
    0\\
    -1\\
    0\\
    \end{array}
    \right)-x_5\left(
    \begin{array}{c}
    1\\
    0\\
    0\\
    0\\
    -1\\
    \end{array}
    \right)
    \end{equation}
    \begin{equation}
        \left(
    \begin{array}{c}
    x_1\\
    x_2\\
    x_3\\
    x_4\\
    x_5\\
    \end{array}
    \right)=-x_2e'-x_3e''-x_4e'''-x_5e''''
    \end{equation}
    Выпишем, во что будут переходить базисные векторы при домножении слева на $\rho(g)$ (из перестановочного представления): $\rho(e)e'=(1,-1,0,0,0)^T=e'$, $\rho(e)e''=(1,0,-1,0,0)^T=e''$, $\rho(e)e'''=(1,0,0,-1,0)^T=e'''$, $\rho(e)e''''=(1,0,0,0,-1)^T=e''''$, $\rho((1,2))e'=(-1,1,0,0,0)^T=-e'$, $\rho((1,2))e''=(0,1,-1,0,0)^T=e''-e'$, $\rho((1,2))e'''=(0,1,0,-1,0)^T=e'''-e'$, $\rho((1,2))e''''=(0,1,0,0-1)^T=e''''-e'$, $\rho((1,2,3))e'=(0,1,-1,0,0)^T=e''-e'$, $\rho((1,2,3))e''=(-1,1,0,0,0)^T=-e'$, $\rho((1,2,3))e'''=(0,1,0,-1,0)^T=e'''-e'$,  $\rho((1,2,3))e''''=(0,1,0,0,-1)^T=e''''-e'$, $\rho((1,2,3,4))e'=(0,1,-1,0,0)^T=e''-e'$, $\rho((1,2,3,4))e''=(0,1,0,-1,0)^T=e'''-e'$, $\rho((1,2,3,4))e'''= (-1,1,0,0,0)^T=-e'$,  $\rho((1,2,3,4))e''''=(0,1,0,0,-1)^T=e''''-e'$, $\rho((1,2,3,4,5))e'=(0,1,-1,0,0)^T=e''-e'$, $\rho((1,2,3,4,5))e''=(0,1,0,-1,0)^T=e'''-e'$, $\rho((1,2,3,4,5))e'''=(0,1,0,0,-1)^T=e''''-e'$,  $\rho((1,2,3,4,5))e''''=(-1,1,0,0,0)^T=-e'$, $\rho((1,2)(3,4))e'=(-1,1,0,0,0)^T=-e'$, $\rho((1,2)(3,4))e''=(0,1,0,-1,0)^T=e'''-e'$, $\rho((1,2)(3,4))e'''=(0,1,-1,0,0)^T=e''-e'$, $\rho((1,2)(3,4))e''''=(0,1,0,0,-1)^T=e''''-e'$, $\rho((1,2,3)(4,5))e'=(0,1,-1,0,0)^T=e''-e'$, $\rho((1,2,3)(4,5))e''=(-1,1,0,0,0)^T=-e'$, $\rho((1,2,3)(4,5))e'''=(0,1,0,0,-1)^T=e''''-e'$, $\rho((1,2,3)(4,5))e''''=(0,1,0,-1,0)^T=e'''-e'$.\\
    Представление $U_2$:
    \begin{equation*}
        \boxed{e\rightarrow\left(
    \begin{array}{cccc}
    1 & 0 & 0 & 0\\
    0 & 1 & 0 & 0\\
    0 & 0 & 1 & 0\\
    0 & 0 & 0 & 1\\
    \end{array}
    \right),\quad (1,2)\rightarrow\left(
    \begin{array}{cccc}
    -1 & -1 & -1 & -1\\
    0 & 1 & 0 & 0\\
    0 & 0 & 1 & 0\\
    0 & 0 & 0 & 1\\
    \end{array}
    \right),\quad (1,2,3)\rightarrow\left(
    \begin{array}{cccc}
    -1 & -1 & -1 & -1\\
    1 & 0 & 0 & 0\\
    0 & 0 & 1 & 0\\
    0 & 0 & 0 & 1\\
    \end{array}
    \right)}
    \end{equation*}
    \begin{equation}
    \boxed{(1,2,3,4)\rightarrow\left(
    \begin{array}{cccc}
    -1 & -1 & -1 & -1\\
    1 & 0 & 0 & 0\\
    0 & 1 & 0 & 0\\
    0 & 0 & 0 & 1\\
    \end{array}
    \right),\quad (1,2,3,4,5)\rightarrow\left(
    \begin{array}{cccc}
    -1 & -1 & -1 & -1\\
    1 & 0 & 0 & 0\\
    0 & 1 & 0 & 0\\
    0 & 0 & 1 & 0\\
    \end{array}
    \right)}
    \end{equation}
    \begin{equation}
        \boxed{(1,2)(3,4)\rightarrow\left(
        \begin{array}{cccc}
        -1 & -1 & -1 & -1\\
        0 & 0 & 1 & 0\\
        0 & 1 & 0 & 0\\
        0 & 0 & 0 & 1\\
        \end{array}
        \right),\quad (1,2,3)(4,5)\rightarrow\left(
        \begin{array}{cccc}
        -1 & -1 & -1 & -1\\
        1 & 0 & 0 & 0\\
        0 & 0 & 0 & 1\\
        0 & 0 & 1 & 0\\
        \end{array}
        \right)}
    \end{equation}
\end{zad}
Характер представления $U_2$:
\begin{table}[h!]
\centering
\begin{tabular}{|l|l|l|l|l|l|l|l|}
\hline
 & $e$ $^1$ & $(a,b)$ $^{10}$ & $(a,b,c)$ $^{20}$ & $(a,b,c,d)$ $^{30}$ & $(a,b,c,d,e)$ $^{24}$ & $(a,b)(c,d)$ $^{15}$ & $(a,b,c)(d,e)$ $^{20}$ \\ \hline
$\chi^{(4)}$ & $4$ & $2$ & $1$ & $0$ & $-1$ & $0$ & $-1$\\ \hline
\end{tabular}
\end{table}\\
Проверим, что все представление $U_2$ являются неприводимыми. Для этого воспользуемся критерием $\braket{\chi^{(i)},\chi^{(i)}}=1$:
\begin{equation}
    \braket{\chi^{(4)},\chi^{(4)}}=\frac{1}{120}(1\cdot16+10\cdot4+20\cdot1+30\cdot0+24\cdot1+15\cdot0+20\cdot1)=1
\end{equation}
Таким образом, пятимерное перестановочное произведение разложено в прямую сумму двух неприводимых:
\begin{equation}
    \boxed{\mathbb{C}^5=U_1\oplus U_2}
\end{equation}
\begin{zad}
\begin{itemize}
    \item[а)$^*$] Проверим, какие классы сопряжённости из $S_5$ перейдут в $A_5$. Конечно, класс $\{e\}$ сохранится. 2 класса, состоящие из произведений транспозиций и циклов длины 3, сохранятся также.\\
    Рассмотрим циклы длины 5. $|A_5|=60$, а значит класс сопряжённости (орбита) не может состоять из 24 элементов (60 на 24 не делится). В $A_5$ невозможно получить каждый цикл длины 5 из каждого, т.е. циклы длины 5 создадут 2 класса сопряжённости в $A_5$, которые соответствуют поворотам на $\pm\frac{2\pi}{5}$ и $\pm\frac{4\pi}{5}$ вокруг прямых, проходящих через центры противоположных граней додекаэдра (по 12 в каждом классе): $\{(1,2,3,4,5), (1,2,4,5,3), (1,2,5,3,4), (1,3,2,5,4), (1,3,4,2,5), (1,3,5,4,2), (1,4,2,3,5),\\ (1,4,3,5,2), (1,4,5,2,3), (1,5,2,4,3), (1,5,3,2,4), (1,5,4,3,2)\}$;\\ $\{(1,2,3,5,4), (1,2,4,3,5), (1,2,5,4,3), (1,3,2,4,5), (1,3,4,5,2), (1,3,5,2,4), (1,4,2,5,3),\\ (1,4,3,2,5), (1,4,5,3,2), (1,5,2,3,4), (1,5,3,4,2), (1,5,4,2,3)\}$.\\
    Таким образом, всего будет 5 классов сопряженности в $A_5$.
    \item[б)] 3 различных неприводимых представления группы $A_5$:
    \begin{enumerate}
        \item Тривиальное одномерное преставление. Все элементы $A_5$ переходят в 1.
        \item Первое трёхмерное представление, соответствующее вращениям додекаэдра:
        \begin{equation*}
        e\rightarrow
        \left(
    \begin{array}{ccc}
    1 & 0 & 0\\
    0 & 1 & 0\\
    0 & 0 & 1\\
    \end{array}
    \right),\quad (a,b,c)\rightarrow
        \left(
    \begin{array}{ccc}
    \cos\frac{2\pi}{3} & -\sin\frac{2\pi}{3} & 0\\
    \sin\frac{2\pi}{3} & \cos\frac{2\pi}{3} & 0\\
    0 & 0 & 1\\
    \end{array}
    \right),\quad (a,b)(c,d)\rightarrow
        \left(
    \begin{array}{ccc}
    \cos\pi & -\sin\pi & 0\\
    \sin\pi & \cos\pi & 0\\
    0 & 0 & 1\\
    \end{array}
    \right)
    \end{equation*}
    \begin{equation*}
        (1,2,3,4,5)\rightarrow
        \left(
    \begin{array}{ccc}
    \cos\frac{2\pi}{5} & -\sin\frac{2\pi}{5} & 0\\
    \sin\frac{2\pi}{5} & \cos\frac{2\pi}{5} & 0\\
    0 & 0 & 1\\
    \end{array}
    \right),\quad (1,2,3,5,4)\rightarrow
        \left(
    \begin{array}{ccc}
    \cos\frac{4\pi}{5} & -\sin\frac{4\pi}{5} & 0\\
    \sin\frac{4\pi}{5} & \cos\frac{4\pi}{5} & 0\\
    0 & 0 & 1\\
    \end{array}
    \right)
    \end{equation*}
    \item Второе трёхмерное представление, соответствующее вращениям додекаэдра (соответствует другому способу нумерования тетраэдров):
        \begin{equation*}
        e\rightarrow
        \left(
    \begin{array}{ccc}
    1 & 0 & 0\\
    0 & 1 & 0\\
    0 & 0 & 1\\
    \end{array}
    \right),\quad (a,b,c)\rightarrow
        \left(
    \begin{array}{ccc}
    \cos\frac{2\pi}{3} & -\sin\frac{2\pi}{3} & 0\\
    \sin\frac{2\pi}{3} & \cos\frac{2\pi}{3} & 0\\
    0 & 0 & 1\\
    \end{array}
    \right),\quad (a,b)(c,d)\rightarrow
        \left(
    \begin{array}{ccc}
    \cos\pi & -\sin\pi & 0\\
    \sin\pi & \cos\pi & 0\\
    0 & 0 & 1\\
    \end{array}
    \right)
    \end{equation*}
    \begin{equation*}
        (1,2,3,4,5)\rightarrow
        \left(
    \begin{array}{ccc}
    \cos\frac{4\pi}{5} & -\sin\frac{4\pi}{5} & 0\\
    \sin\frac{4\pi}{5} & \cos\frac{4\pi}{5} & 0\\
    0 & 0 & 1\\
    \end{array}
    \right),\quad (1,2,3,5,4)\rightarrow
        \left(
    \begin{array}{ccc}
    \cos\frac{2\pi}{5} & -\sin\frac{2\pi}{5} & 0\\
    \sin\frac{2\pi}{5} & \cos\frac{2\pi}{5} & 0\\
    0 & 0 & 1\\
    \end{array}
    \right)
    \end{equation*}
    \end{enumerate}
    \item[в)] Классы сопряжённости $A_5$ описаны в п. а. Всего классов сопряжённости 5, значит и неприводимых представлений 5. Найдём их размерности: $|A_5|=d_1^2+d_2^2+d_3^2+d_4^2+d_5^2=60$. Единственный возможный вариант: $d_1=1$, $d_2=d_3=3$, $d_4=4$, $d_5=5$.\\
    Единственное одномерное представление $\rho_1$ -- тривиальное. Оба трёхмерных представления $\rho_2$ и $\rho_3$ $A_5$ рассмотрены в п. б. Четырёхмерное представление в $A_5$ $\rho_4$ -- ограничение представления $U_2$ (см. задачу 7.5) на $A_5$.\\
    Характеры пятимерного произведения $\rho_5$ можно найти из следствия 1 предложения 9 лекции 6:
    \begin{equation}
        \chi_\text{reg}=d_1\chi^{(1)}+d_2\chi^{(2)}+d_3\chi^{(3)}+d_4\chi^{(4)}+d_5\chi^{(5)}\rightarrow \chi^{(5)}=\frac{1}{5}(\chi_\text{reg}-\chi^{(1)}-3\chi^{(2)}-3\chi^{(3)}-4\chi^{(4)})
    \end{equation}
    Характер регулярного произведения можно найти из предложения 8 лекции 6:
    \begin{equation}
        \chi_\text{reg}(e)=|A_5|=60, \quad \chi_\text{reg}(g)=0, \; g\neq e
    \end{equation}
    \begin{equation}
        \chi^{(5)}(e)=\frac{1}{5}(60-1-3\cdot 3-3\cdot 3-4\cdot4)=5=d_5
    \end{equation}
    \begin{equation}
        \chi^{(5)}((a,b,c))=\frac{1}{5}(0-1-3\cdot 0-3\cdot0-4\cdot1)=-1
    \end{equation}
    \begin{equation}
        \chi^{(5)}((a,b)(c,d))=\frac{1}{5}(0-1-3\cdot (-1)-3\cdot (-1)-4\cdot0)=1
    \end{equation}
    \begin{equation}
        \chi^{(5)}((1,2,3,4,5))=\frac{1}{5}(0-1-3\cdot (1+2\cos\frac{2\pi}{5})-3\cdot (1+2\cos\frac{4\pi}{5})-4\cdot(-1))=0
    \end{equation}
    \begin{equation}
        \chi^{(5)}((1,2,3,5,4))=\frac{1}{5}(0-1-3\cdot (1+2\cos\frac{4\pi}{5})-3\cdot (1+2\cos\frac{2\pi}{5})-4\cdot(-1))=0
    \end{equation}
    В последних 2 уравнениях использовано равенство:
    \begin{equation}
        \cos\frac{2\pi}{5}+\cos\frac{4\pi}{5}=\frac{e^{i\frac{2\pi}{5}}+e^{-i\frac{2\pi}{5}}}{2}+\frac{e^{i\frac{4\pi}{5}}+e^{-i\frac{4\pi}{5}}}{2}=\frac{1}{2}(e^{i\frac{2\pi}{5}}+e^{-i\frac{2\pi}{5}}+e^{i\frac{4\pi}{5}}+e^{-i\frac{4\pi}{5}})=\frac{e^{-i\frac{4\pi}{5}}}{2}\frac{e^{i\frac{8\pi}{5}}-1}{e^{i\frac{2\pi}{5}}-1}=-\frac{1}{2}
    \end{equation}
    Таблица характеров $A_5$:
    \begin{table}[h!]
    \centering
    \begin{tabular}{|l|l|l|l|l|l|}
    \hline
     & $e$ $^1$ & $(a,b,c)$ $^{20}$ & $(1,2,3,4,5)$ $^{12}$ & $(1,2,3,5,4)$ $^{12}$ & $(a,b)(c,d)$ $^{15}$    \\ \hline
    $\chi^{(1)}$ & $1$ & $1$ & $1$ & $1$ & $1$ \\ \hline
    $\chi^{(2)}$ & $3$ & $0$ & $1+2\cos\frac{2\pi}{5}$ & $1+2\cos\frac{4\pi}{5}$ & $-1$ \\ \hline
    $\chi^{(3)}$ & $3$ & $0$ & $1+2\cos\frac{4\pi}{5}$ & $1+2\cos\frac{2\pi}{5}$ & $-1$ \\ \hline
    $\chi^{(4)}$ & $4$ & $1$ & $-1$ & $-1$ & $0$ \\ \hline
    $\chi^{(5)}$ & $5$ & $-1$ & $0$ & $0$ & $1$  \\ \hline
    \end{tabular}
    \caption{Таблица характеров группы $A_5$}
    \end{table}\\
Проверим представления на неприводимость, для этого воспользуемся критерием $\braket{\chi^{(i)},\chi^{(i)}}=1$:
\begin{equation}
    \braket{\chi^{(1)},\chi^{(1)}}=\frac{1}{60}(1\cdot1+20\cdot1+12\cdot1+12\cdot1+15\cdot1)=1
\end{equation}
\begin{equation}
    \braket{\chi^{(2)},\chi^{(2)}}=\frac{1}{60}\left(1\cdot9+20\cdot0+12\cdot\left(1+2\cos\frac{2\pi}{5}\right)^2+12\cdot\left(1+2\cos\frac{4\pi}{5}\right)^2+15\cdot1\right)=1
\end{equation}
\begin{equation}
    \braket{\chi^{(3)},\chi^{(3)}}=\frac{1}{60}\left(1\cdot9+20\cdot0+12\cdot\left(1+2\cos\frac{4\pi}{5}\right)^2+12\cdot\left(1+2\cos\frac{2\pi}{5}\right)^2+15\cdot1\right)=1
\end{equation}
Неприводимость $\chi^{(4)}$ проверена в задаче 7.5.
\begin{equation}
    \braket{\chi^{(5)},\chi^{(5)}}=\frac{1}{60}(1\cdot25+20\cdot1+12\cdot0+12\cdot0+15\cdot1)=1
\end{equation}
\end{itemize}
\end{zad}
\section{Группы Ли, алгебры Ли.}
\begin{upr}
Перемножим матрицы:
\begin{equation*}
    \left(
    \begin{array}{cc}
    e^{i\alpha} & 0\\
    0 & e^{-i\alpha}\\
    \end{array}
    \right)\left(
    \begin{array}{cc}
    a & b\\
    c & d\\
    \end{array}
    \right)\left(
    \begin{array}{cc}
    e^{-i\alpha} & 0\\
    0 & e^{i\alpha}\\
    \end{array}
    \right)=\left(
    \begin{array}{cc}
    e^{i\alpha a} & e^{i\alpha b}\\
    e^{-i\alpha c} & e^{i\alpha d}\\
    \end{array}
    \right)\left(
    \begin{array}{cc}
    e^{-i\alpha} & 0\\
    0 & e^{i\alpha}\\
    \end{array}
    \right)=\left(
    \begin{array}{cc}
    a & e^{2i\alpha}b\\
    e^{-2i\alpha}c & d\\
    \end{array}
    \right)
    \end{equation*}
    Рассматривая $a$, $b$, $c$, $d$ как координаты в четырёхмерном пространстве, получим
    \begin{equation*}
        \left(
    \begin{array}{c}
    1\\
    0\\
    0\\
    0\\
    \end{array}
    \right)\rightarrow\left(
    \begin{array}{c}
    1\\
    0\\
    0\\
    0\\
    \end{array}
    \right),\quad \left(
    \begin{array}{c}
    0\\
    1\\
    0\\
    0\\
    \end{array}
    \right)\rightarrow e^{2i\alpha}\left(
    \begin{array}{c}
    0\\
    1\\
    0\\
    0\\
    \end{array}
    \right),\quad \left(
    \begin{array}{c}
    0\\
    0\\
    1\\
    0\\
    \end{array}
    \right)\rightarrow e^{-2i\alpha}\left(
    \begin{array}{c}
    0\\
    0\\
    1\\
    0\\
    \end{array}
    \right),\quad \left(
    \begin{array}{c}
    0\\
    0\\
    0\\
    1\\
    \end{array}
    \right)\rightarrow \left(
    \begin{array}{c}
    0\\
    0\\
    0\\
    1\\
    \end{array}
    \right)
    \end{equation*}
    Тогда для $e^{i\alpha}\in U(1)$ верно, что
    \begin{equation}
        \rho(e^{i\alpha})=\left(
    \begin{array}{cccc}
    1 & 0 & 0 & 0\\
    0 & e^{2i\alpha} & 0 & 0\\
    0 & 0 & e^{-2i\alpha} & 0\\
    0 & 0 & 0 & 1\\
    \end{array}
    \right)
    \end{equation}
    Характер этого перестановочного представления:
    \begin{equation}
        \chi(\alpha)=2+e^{2i\alpha}+e^{-2i\alpha}
    \end{equation}
    \begin{equation}
        \boxed{\chi(\alpha)=2(1+\cos2\alpha)}
    \end{equation}
    Все неприводимые представления абелевых групп одномерны. Неприводимые представления $U(1)$ были рассмотрены на лекции 7:
    \begin{equation}
        \chi^{(k)}(\alpha)=e^{ik\alpha}
    \end{equation}
    Разложение характера в ряд Фурье:
    \begin{equation}
        \chi=\sum\limits_{k=-\infty}^\infty c_k\chi^{(k)},\quad c_k=\frac{1}{2\pi}\int\limits_0^{2\pi}\chi(\alpha)\overline{\chi^{(k)}(\alpha)}d\alpha
    \end{equation}
    \begin{equation}
        c_k=\frac{1}{2\pi}\int\limits_0^{2\pi}(2+e^{2i\alpha}+e^{-2i\alpha})e^{-ik\alpha}d\alpha
    \end{equation}
    \begin{equation}
        c_0=\frac{1}{2\pi}\int\limits_0^{2\pi}(2+e^{2i\alpha}+e^{-2i\alpha})d\alpha=\frac{1}{2\pi}(2\cdot2\pi)=2
    \end{equation}
    \begin{equation}
        c_{\pm 2}=\frac{1}{2\pi}\int\limits_0^{2\pi}(2+e^{2i\alpha}+e^{-2i\alpha})e^{\mp 2i\alpha}d\alpha=\frac{1}{2\pi}(2\pi)=1
    \end{equation}
    Все остальные $c_k=0$ как интегралы по целому числу периодов от синусов и косинусов.
    \begin{equation}
        \boxed{\chi=2\chi^{(0)}+\chi^{(-2)}+\chi^{(2)}}
    \end{equation}
\end{upr}
\begin{zad}
Группа унитарных матриц $U(n)$ состоит из таких матриц $g_{n\times n}\in U(n)$, что $g^*g=gg^*=E$. $g=(g_1, g_2,..., g_n)$. Распишем условие $gg^*=E$ через столбцы $g_i$: $g_i\bar{g_j}^T=\delta_{ij}$. При $i=j$: $g_i\bar{g_i}^T=|g_i|^2=1$ -- всего $n$ вещественных уравнений; при $i\neq j$: $g_i\bar{g_j}^T=0$ -- всего $\frac{n(n-1)}{2}$ комплексных или $n(n-1)$ вещественных уравнений. Всего вещественных уравнений:
\begin{equation}
    \boxed{n+n(n-1)=n^2}
\end{equation}
Рассмотрим все возможные гладкие кривые $g(t)\in U(n)$, где $g(0)=E$. При малых $t$ эта кривая имеет вид $g(t)=E+At+o(t)$, где $A=g'(0)$. $gg^*=(E+At+o(t))(E+A^*t+o(t))=E+(A+A^*)t+o(t)=E$, значит $A+A^*=0$. Всё $T_EU(n)$ состоит из таких $A$, значит $T_EU(n)$ -- пространство антиэрмитовых матриц размера $n\times n$.
\begin{predl}
    $T_EU(n)$ замкнуто относительно коммутатора.
\end{predl}
\begin{proof}
    Пусть $A,B\in T_EU(n)$, тогда $A^*=-A$, $B^*=-B$.
    \begin{equation*}
        [A,B]+[A,B]^*=AB-BA+(AB-BA)^*=AB-BA+B^*A^*-A^*B^*=AB-BA+BA-AB=0
    \end{equation*}
    Значит, $[A,B]\in T_EU(n)$ и $T_EU(n)$ замкнуто относительно коммутатора.
\end{proof}
\end{zad}
\begin{zad}
\begin{itemize}
    \item[а)]
    \begin{predl}
        Алгебра Ли $\mathfrak{so}(3,\mathbb{R})$ изоморфна алгебре векторов $\mathbb{R}^3$.
    \end{predl}
    \begin{proof}
        Для матриц $[A,B]=AB-BA$. Алгебра Ли $\mathfrak{so}(3,\mathbb{R})$ состоит из всех кососимметричных матриц размера $3\times3$ (пример 1, п. в лекции 8). Базис в $\mathfrak{so}(3,\mathbb{R})$:
        \begin{equation}
            I_1=\left(
        \begin{array}{ccc}
        0 & 1 & 0\\
        -1 & 0 & 0\\
        0 & 0 & 0\\
        \end{array}
        \right),\quad I_2=\left(
        \begin{array}{ccc}
        0 & 0 & -1\\
        0 & 0 & 0\\
        1 & 0 & 0\\
        \end{array}
        \right),\quad I_3=\left(
        \begin{array}{ccc}
        0 & 0 & 0\\
        0 & 0 & 1\\
        0 & -1 & 0\\
        \end{array}
        \right)
        \end{equation}
        \begin{equation}
            [I_1,I_2]=I_3,\quad [I_1,I_3]=-I_2,\quad [I_2,I_3]=I_1
        \end{equation}
        Структурные константы:
        \begin{equation}
            c_{ij}^k=\epsilon_{ijk}
        \end{equation}
        Для векторов в $\mathbb{R}^3$ коммутатор -- векторное произведение. Пусть $e_1$, $e_2$, $e_3$ -- базис в $\mathbb{R}^3$.
        \begin{equation}
            [e_i,e_j]=\epsilon_{ijk}e_k
        \end{equation}
        Структурные константы:
        \begin{equation}
            c_{ij}^k=\epsilon_{ijk}
        \end{equation}
    \end{proof}
    \item[б)]
    \begin{predl}
        Алгебра Ли $\mathfrak{su}(2)$ изоморфна алгебре $\mathfrak{so}(3,\mathbb{R})$.
    \end{predl}
    \begin{proof}
        Пусть $g\in SU(2)$, тогда $gg^*=E$ и $\det g=1$. Рассмотрим малое приращение
        \begin{equation}
            g=E+t\delta g+o(t),\quad t\in\mathbb{R}
        \end{equation}
        Подставим это в уравнение на $g$:
        \begin{equation*}
            (E+t\delta g+o(t))(E+t\delta g+o(t))^*=(E+t\delta g+o(t))(E+t\delta g^*+o(t))=E+t(\delta g+\delta g^*)+o(t)=E
        \end{equation*}
        \begin{equation}
            \det g=1+t\;\text{tr}\delta g+o(t)=1
        \end{equation}
        \begin{equation}
            \delta g=-\delta g^*,\quad \text{tr\;}\delta g=0
        \end{equation}
        Таким образом, алгебра Ли $\mathfrak{su}(2)$ состоит из антиэрмитовых матриц с нулевым следом. Пусть $A\in \mathfrak{su}(2)$, тогда общий вид такой матрицы
        \begin{equation}
            A=\left(
        \begin{array}{cc}
        ia_1 & a_2+ia_3\\
        -a_2+ia_3 & -ia_1\\
        \end{array}
        \right)=a_1\left(
        \begin{array}{cc}
        i & 0\\
        0 & -i\\
        \end{array}
        \right)+a_2\left(
        \begin{array}{cc}
        0 & 1\\
        -1 & 0\\
        \end{array}
        \right)+a_3\left(
        \begin{array}{cc}
        0 & i\\
        i & 0\\
        \end{array}
        \right)
        \end{equation}
        \begin{equation}
            \left(
        \begin{array}{cc}
        i & 0\\
        0 & -i\\
        \end{array}
        \right)\left(
        \begin{array}{cc}
        0 & 1\\
        -1 & 0\\
        \end{array}
        \right)=\left(
        \begin{array}{cc}
        0 & 2i\\
        2i & 0\\
        \end{array}
        \right)
        \end{equation}
        Базис в $\mathfrak{su}(2)$:
        \begin{equation}
            I_1=\frac{1}{2}\left(
        \begin{array}{cc}
        i & 0\\
        0 & -i\\
        \end{array}
        \right),\quad I_2=\frac{1}{2}\left(
        \begin{array}{cc}
        0 & 1\\
        -1 & 0\\
        \end{array}
        \right),\quad I_3=\frac{1}{2}\left(
        \begin{array}{cc}
        0 & i\\
        i & 0\\
        \end{array}
        \right)
        \end{equation}
        \begin{equation}
            [I_1,I_2]=I_3,\quad [I_1,I_3]=-I_2,\quad [I_2,I_3]=I_1
        \end{equation}
        Структурные константы:
        \begin{equation}
            c_{ij}^k=\epsilon_{ijk}
        \end{equation}
    \end{proof}
    \item[в)$^*$]
    \begin{predl}
        Алгебра Ли $\mathfrak{sl}(2,\mathbb{R})$ не изоморфна алгебре $\mathfrak{so}(3,\mathbb{R})$.
    \end{predl}
    \begin{proof}
        Алгебра Ли $\mathfrak{sl}(2,\mathbb{R})$ состоит из всех матриц размера $2\times2$ с нулевым следом. Базис в $\mathfrak{sl}(2,\mathbb{R})$:
        \begin{equation}
            I_1=\left(
        \begin{array}{cc}
        1 & 0\\
        0 & -1\\
        \end{array}
        \right),\quad I_2=\left(
        \begin{array}{cc}
        0 & 1\\
        0 & 0\\
        \end{array}
        \right),\quad I_3=\left(
        \begin{array}{cc}
        0 & 0\\
        1 & 0\\
        \end{array}
        \right)
        \end{equation}
        \begin{equation}
            [I_1,I_2]=\left(
        \begin{array}{cc}
        1 & 0\\
        0 & -1\\
        \end{array}
        \right)\left(
        \begin{array}{cc}
        0 & 1\\
        0 & 0\\
        \end{array}
        \right)-\left(
        \begin{array}{cc}
        0 & 1\\
        0 & 0\\
        \end{array}
        \right)\left(
        \begin{array}{cc}
        1 & 0\\
        0 & -1\\
        \end{array}
        \right)=\left(
        \begin{array}{cc}
        0 & 2\\
        0 & 0\\
        \end{array}
        \right)
        \end{equation}
        Т.е. коммутатор $[I_1,I_2]$ пропорционален $I_2$, чего точно не может быть в векторном произведении.
    \end{proof}
\end{itemize}
\end{zad}
\section{Симметричные тензоры. Группы Ли $SO(3)$, $SU(2)$.}
\begin{upr}
\begin{equation}
    A=\left(
    \begin{array}{cc}
    a_{11} & a_{12}\\
    a_{21} & a_{22}\\ 
    \end{array}
    \right),\quad A\otimes A=\left(
    \begin{array}{cc}
    a_{11}A & a_{12}A\\
    a_{21}A & a_{22}A\\ 
    \end{array}
    \right)=\left(
    \begin{array}{cccc}
    a_{11}^2 & a_{11}a_{12} & a_{11}a_{12} & a_{22}^2\\
    a_{11}a_{21} & a_{11}a_{22} & a_{12}a_{21} & a_{12}a_{22}\\
    a_{11}a_{21} & a_{12}a_{21} & a_{11}a_{22} & a_{12}a_{22}\\
    a_{21}^2 & a_{21}a_{22} & a_{21}a_{22} & a_{22}^2\\
    \end{array}
    \right)
\end{equation}
Пусть $e_1$, $e_2$ -- базис в векторном пространстве $V$. Базисные векторы пространства $V\otimes V$:
\begin{equation}
    e_1\otimes e_1 = \left(
    \begin{array}{c}
    1\\
    0\\
    0\\
    0\\
    \end{array}
    \right),\quad e_1\otimes e_2=\left(
    \begin{array}{c}
    0\\
    1\\
    0\\
    0\\
    \end{array}
    \right),\quad e_2\otimes e_1=\left(
    \begin{array}{c}
    0\\
    0\\
    1\\
    0\\
    \end{array}
    \right),\quad e_2\otimes e_2=\left(
    \begin{array}{c}
    0\\
    0\\
    0\\
    1\\
    \end{array}
    \right)
\end{equation}
\begin{equation}
    S^2V=\braket{e_1\otimes e_1, e_1\otimes e_2+e_2\otimes e_1, e_2\otimes e_2}
\end{equation}
\begin{equation}
    A\otimes A(e_1\otimes e_1)=\left(
    \begin{array}{c}
    a_{11}^2\\
    a_{11}a_{21}\\
    a_{11}a_{21}\\
    a_{21}^2\\
    \end{array}
    \right)=a_{11}^2e_1\otimes e_1+a_{11}a_{21}(e_1\otimes e_2+e_2\otimes e_1)+a_{21}^2e_2\otimes e_2
\end{equation}
\begin{equation}
    A\otimes A(e_1\otimes e_2+e_2\otimes e_1)=\left(
    \begin{array}{c}
    2a_{11}a_{12}\\
    a_{11}a_{22}+a_{12}a_{21}\\
    a_{11}a_{22}+a_{12}a_{21}\\
    2a_{21}a_{22}\\
    \end{array}
    \right)=2a_{11}a_{12}e_1\otimes e_1+(a_{11}a_{22}+a_{12}a_{21})(e_1\otimes e_2+e_2\otimes e_1)+2a_{21}a_{22}e_2\otimes e_2
\end{equation}
\begin{equation}
    A\otimes A(e_2\otimes e_2)=\left(
    \begin{array}{c}
    a_{12}^2\\
    a_{12}a_{22}\\
    a_{12}a_{22}\\
    a_{22}^2\\
    \end{array}
    \right)=a_{12}^2e_1\otimes e_1+a_{12}a_{22}(e_1\otimes e_2+e_2\otimes e_1)+a_{22}^2e_2\otimes e_2
\end{equation}
Таким образом, $S^2A$ можно задать матрицей:
\begin{equation}
    \boxed{S^2A=
    \left(
    \begin{array}{ccc}
    a_{11}^2 & 2a_{11}a_{12} & a_{12}^2\\
    a_{11}a_{21} & a_{11}a_{22}+a_{12}a_{21} & a_{12}a_{22}\\
    a_{22}^2 & 2a_{21}a_{22} & a_{22}^2\\
    \end{array}
    \right)}
\end{equation}
\begin{equation}
    \boxed{\text{tr}\; S^2A=a_{11}^2+a_{11}a_{22}+a_{12}a_{21}+a_{22}^2}
\end{equation}
\begin{equation}
    \Lambda^2V=\braket{e_1\otimes e_2-e_2\otimes e_1}
\end{equation}
\begin{equation}
    A\otimes A(e_1\otimes e_2-e_2\otimes e_1)=\left(
    \begin{array}{c}
    0\\
    a_{11}a_{22}-a_{12}a_{21}\\
    a_{12}a_{21}-a_{11}a_{22}\\
    0\\
    \end{array}
    \right)=(a_{11}a_{22}-a_{12}a_{21})(e_1\otimes e_2-e_2\otimes e_1)
\end{equation}
Таким образом, $\Lambda^2A$ можно задать матрицей:
\begin{equation}
    \boxed{\Lambda^2A=(a_{11}a_{22}-a_{12}a_{21})}
\end{equation}
\begin{equation}
    \boxed{\text{tr}\; \Lambda^2A=a_{11}a_{22}-a_{12}a_{21}}
\end{equation}
Как видно, предложение 3 лекции 9 в данном упражнении выполняется ($\text{tr} A=a_{11}+a_{22}$, $\text{tr} A^2=a^2_{11}+2a_{12}a_{21}+a^2_{22}$):
\begin{equation}
    \text{tr}\; S^2A=\frac{1}{2}((\text{tr}A)^2+\text{tr} A^2),\quad \text{tr}\; \Lambda^2A=\frac{1}{2}((\text{tr}A)^2-\text{tr} A^2)
\end{equation}
\end{upr}
\begin{upr}
Пусть $g\in SU(2)$, тогда
\begin{equation}
    g=\left(
    \begin{array}{cc}
    a & b\\
    -\bar{b} & \bar{a}\\ 
    \end{array}
    \right),\quad |a|^2+|b|^2=1
\end{equation}
\begin{equation}
    \left\{
    \begin{array}{l}
    a=a_0+ia_3,\\
    b=a_2+ia_1.\\
    \end{array}
    \right.\rightarrow g=a_0E+ia_1\sigma_1+ia_2\sigma_2+ia_3\sigma_3
\end{equation}
где $\sigma_1=\left(
    \begin{array}{cc}
    0 & 1\\
    1 & 0\\ 
    \end{array}
    \right)$, $\sigma_2=\left(
    \begin{array}{cc}
    0 & -i\\
    i & 0\\ 
    \end{array}
    \right)$, $\sigma_3=\left(
    \begin{array}{cc}
    1 & 0\\
    0 & -1\\ 
    \end{array}
    \right)$ -- матрицы Паули.\\
Пусть $a_0=\cos\alpha$ и $a_1^2+a_2^2+a_3^2=\sin^2\alpha$. Введём $n_j$ так, что $a_j=n_j\sin\alpha$ и $\vec{n}=(n_1,n_2,n_3)$.
\begin{equation}
    g=E\cos\alpha+i\sin\alpha(\vec{n},\vec{\sigma})
\end{equation}
Разложим $\exp(i\alpha(\vec{n},\vec{\sigma}))$ в ряд Тейлора:
\begin{equation}\label{eq3}
    \exp(i\alpha(\vec{n},\vec{\sigma}))=\sum\limits_{n=0}^\infty\frac{(i\alpha(\vec{n},\vec{\sigma}))^n}{n!}=\sum\limits_{n=0}^\infty\frac{(-1)^n\alpha^{2n}(\vec{n},\vec{\sigma})^{2n}}{(2n)!}+i\sum\limits_{n=0}^\infty\frac{(-1)^n\alpha^{2n+1}(\vec{n},\vec{\sigma})^{2n+1}}{(2n+1)!}
\end{equation}
\begin{equation}
    (\vec{n},\vec{\sigma})^2=n_1^2\sigma_1^2+n_2^2\sigma_2^2+n_3^2\sigma_3^2+n_1n_2(\sigma_1\sigma_2+\sigma_2\sigma_1)+n_1n_3(\sigma_1\sigma_3+\sigma_3\sigma_1)+n_2n_3(\sigma_2\sigma_3+\sigma_3\sigma_2)
\end{equation}
\begin{equation}
    \sigma_i\sigma_j+\sigma_j\sigma_i=2\delta_{ij}\rightarrow (\vec{n},\vec{\sigma})^2=n_1^2\sigma_1^2+n_2^2\sigma_2^2+n_3^2\sigma_3^2=(n_1^2+n_2^2+n_3^2)E=E
\end{equation}
\begin{equation}\label{eq4}
    (\vec{n},\vec{\sigma})^{2n}=E,\quad (\vec{n},\vec{\sigma})^{2n+1}=(\vec{n},\vec{\sigma})
\end{equation}
\end{upr}
Подставим (\ref{eq4}) в (\ref{eq3}):
\begin{equation}
    \exp(i\alpha(\vec{n},\vec{\sigma}))=\sum\limits_{n=0}^\infty\frac{(-1)^n\alpha^{2n}E}{(2n)!}+i\sum\limits_{n=0}^\infty\frac{(-1)^n\alpha^{2n+1}(\vec{n},\vec{\sigma})}{(2n+1)!}=E\cos\alpha+i\sin\alpha(\vec{n},\vec{\sigma})
\end{equation}
Таким образом,
\begin{equation}
    \boxed{g=\exp(i\alpha(\vec{n},\vec{\sigma})),\quad \forall g\in SU(2)}
\end{equation}
\begin{zad}
\begin{itemize}
    \item[а)]
    \begin{equation}
        g=E\cos\alpha+i\sin\alpha(\vec{n},\vec{\sigma})
    \end{equation}
    \begin{equation}
        g^{-1}=g^*=(E\cos\alpha+i\sin\alpha(\vec{n},\vec{\sigma}))^*=E\cos\alpha-i\sin\alpha(\vec{n},\vec{\sigma}^*)
    \end{equation}
    \begin{equation}
        \boxed{g^{-1}=E\cos\alpha-i\sin\alpha(\vec{n},\vec{\sigma})}
    \end{equation}
    \item[б)] 
    %\begin{equation}
    %    g=\exp(i\alpha\sigma_3)=\left(
    %\begin{array}{cc}
    %e^{i\alpha} & 0\\
    %0 & e^{-i\alpha}\\ 
    %\end{array}
    %\right),\quad g^{-1}=\left(
    %\begin{array}{cc}
    %e^{-i\alpha} & 0\\
    %0 & e^{i\alpha}\\ 
    %\end{array}
    %\right)
    %\end{equation}
    \begin{equation}
        g=\exp(i\alpha\sigma_3)=E\cos\alpha+i\sin\alpha\sigma_3,\quad g^{-1}=E\cos\alpha-i\sin\alpha\sigma_3
    \end{equation}
    Рассмотрим действие $g$ на базисе $i\sigma_1$, $i\sigma_2$, $i\sigma_3$ алгебры Ли $\mathfrak{su}(2)$.
    \begin{multline*}
        gi\sigma_1g^{-1}=(E\cos\alpha+i\sin\alpha\sigma_3)i\sigma_1(E\cos\alpha-i\sin\alpha\sigma_3)=i\sigma_1\cos^2\alpha+i\sigma_3
        \sigma_1\sigma_3\sin^2\alpha-\\-\sin\alpha\cos\alpha\sigma_3\sigma_1+\sin\alpha\cos\alpha\sigma_1\sigma_3=i\sigma_1(\cos^2\alpha-\sin^2\alpha)-2i\sigma_2\sin\alpha\cos\alpha=i\sigma_1\cos2\alpha-i\sigma_2\sin 2\alpha
    \end{multline*}
    \begin{multline*}
        gi\sigma_2g^{-1}=(E\cos\alpha+i\sin\alpha\sigma_3)i\sigma_2(E\cos\alpha-i\sin\alpha\sigma_3)=i\sigma_2\cos^2\alpha+i\sigma_3
        \sigma_2\sigma_3\sin^2\alpha-\\-\sin\alpha\cos\alpha\sigma_3\sigma_2+\sin\alpha\cos\alpha\sigma_2\sigma_3=i\sigma_2(\cos^2\alpha-\sin^2\alpha)-2i\sigma_1\sin\alpha\cos\alpha=i\sigma_2\cos2\alpha+i\sigma_1\sin 2\alpha
    \end{multline*}
    \begin{multline*}
        gi\sigma_3g^{-1}=(E\cos\alpha+i\sin\alpha\sigma_3)i\sigma_3(E\cos\alpha-i\sin\alpha\sigma_3)=i\sigma_3\cos^2\alpha+i\sigma_3^3\sin^2\alpha-\\-\sin\alpha\cos\alpha+\sin\alpha\cos\alpha=i\sigma_3
    \end{multline*}
    Таким образом, матрица присоединённого действия
    \begin{equation}
        \boxed{A=\left(
    \begin{array}{ccc}
    \cos 2\alpha & \sin2\alpha & 0\\
    -\sin 2\alpha & \cos2\alpha & 0\\
    0 & 0 & 1\\
    \end{array}
    \right)}
    \end{equation}
    соответствует повороту на $2\alpha$ вокруг оси, проходящей через $i\sigma_3$.
    Это преообразование является ортогональным:
    \begin{equation}
        AA^T=E
    \end{equation}
    \item[в)] Покажем, что присоединённое представление сохраняет скалярное произведение: $(A,B)=c\text{Tr}AB$, где $c=\text{const}$:
    \begin{equation}
        (gAg^{-1},gBg^{-1})=c\text{Tr}(gAg^{-1}gBg^{-1})=c\text{Tr}(gABg^{-1})=c\text{Tr}(AB)=(A,B)
    \end{equation}
    Найдём $c$. Вычислим скалярные произведения базисных векторов: $(i\sigma_j,i\sigma_k)=-2c\delta_{jk}=1$. Тогда $c=-\frac{1}{2}$.\\
    Поскольку преобразование сохраняет скалярное произведение, то оно является ортогональным.
    \item[г)]
    \begin{predl}
        Полученный гомоморфизм $\varphi$ из группы $SU(2)$ в $SO(3)$ является сюръективным.
    \end{predl}
    \begin{proof}
    По аналогии с п. а, действия $g=\exp(i\alpha\sigma_j)$ являются поворотами на $2\alpha$ вокруг оси, проходящей через $i\sigma_j$. Из таких поворотов состоит любой элемент $SO(3)$, а значит гомоморфизм $\varphi$ сюръективен.
    \end{proof}
    \begin{equation}
        \text{Ker}\;\varphi=\{g\in SU(2):gi\sigma_jg^{-1}=i\sigma_j\}
    \end{equation}
    \begin{equation}
        \cos 2\alpha = 1\rightarrow\alpha=\pi n, n\in\mathbb{Z}\rightarrow g=\cos\pi n=(-1)^nE
    \end{equation}
    \begin{equation}
        \boxed{\text{Ker}\;\varphi=\{E,-E\}\simeq C_2}
    \end{equation}
\end{itemize}
\end{zad}
\begin{zad}
\begin{itemize}
    \item[а)] Естественный базис алгебры Ли $\mathfrak{so}(n)$:
    \begin{equation}
        J_{ab}=E_{ab}-E_{ba}
    \end{equation}
    \begin{multline}
        [J_{ab},J_{cd}]=[E_{ab}-E_{ba},E_{cd}-E_{dc}]=(E_{ab}-E_{ba})(E_{cd}-E_{dc})-(E_{cd}-E_{dc})(E_{ab}-E_{ba})=\\=E_{ab}E_{cd}-E_{ba}E_{cd}-E_{ab}E_{dc}+E_{ba}E_{dc}-E_{cd}E_{ab}+E_{dc}E_{ab}+E_{cd}E_{ba}-E_{dc}E_{ba}=\\=\delta_{bc}E_{ad}-\delta_{ac}E_{bd}-\delta_{bd}E_{ac}+\delta_{ad}E_{bc}-\delta_{ad}E_{cb}+\delta_{ac}E_{db}+\delta_{bd}E_{ca}-\delta_{bc}E_{da}=\\=\delta_{bc}(E_{ad}-E_{da})+\delta_{ad}(E_{bc}-E_{cb})+\delta_{ac}(E_{db}-E_{bd})+\delta_{bd}(E_{ca}-E_{ac})
    \end{multline}
    \begin{equation}
        \boxed{[J_{ab},J_{cd}]=\delta_{bc}J_{ad}+\delta_{ad}J_{bc}+\delta_{ac}J_{db}+\delta_{bd}J_{ca}}
    \end{equation}
    \item[б)$^*$] 
    \begin{predl}
        Алгебра Ли $\mathfrak{so}(4)$ изоморфна прямой сумме $\mathfrak{so}(3)\oplus\mathfrak{so}(3)$.
    \end{predl}
    \begin{proof}
        Естественный базис алгебры Ли $\mathfrak{so}(4)$: $J_{12}, J_{13}, J_{14}, J_{23}, J_{24}, J_{34}$. Пусть
        \begin{equation}
            J_1=\frac{1}{2}(J_{12}+J_{34}),\quad J_2=\frac{1}{2}(J_{14}+J_{23}),\quad J_3=\frac{1}{2}(J_{13}-J_{24})
        \end{equation}
        \begin{equation}
            J'_1=\frac{1}{2}(J_{12}-J_{34}),\quad J'_2=\frac{1}{2}(J_{13}+J_{24}),\quad J'_3=\frac{1}{2}(J_{14}-J_{23})
        \end{equation}
        \begin{equation*}
            [J_1,J_2]=\frac{1}{4}([J_{12},J_{14}]+[J_{34},J_{14}]+[J_{12},J_{23}]+[J_{34},J_{23}])=\frac{1}{4}(J_{42}+J_{13}+J_{13}+J_{42})=\frac{1}{2}(J_{13}-J_{24})=J_3
        \end{equation*}
        \begin{equation*}
            [J_1,J_3]=\frac{1}{4}([J_{12},J_{13}]+[J_{34},J_{13}]-[J_{12},J_{24}]-[J_{34},J_{24}])=\frac{1}{4}(J_{32}+J_{41}-J_{14}-J_{23})=-\frac{1}{2}(J_{14}-J_{23})=-J_2
        \end{equation*}
        \begin{equation*}
            [J_2,J_3]=\frac{1}{4}([J_{14},J_{13}]+[J_{23},J_{13}]-[J_{14},J_{24}]-[J_{23},J_{24}])=\frac{1}{4}(J_{34}+J_{12}+J_{12}+J_{34})=\frac{1}{2}(J_{12}+J_{34})=J_1
        \end{equation*}
        Следовательно,
        \begin{equation}
            [J_i,J_j]=\epsilon_{ijk}J_k\rightarrow \braket{J_1,J_2,J_3}\simeq\mathfrak{so}(3)
        \end{equation}
        \begin{equation*}
            [J'_1,J'_2]=\frac{1}{4}([J_{12},J_{13}]+[J_{12},J_{24}]-[J_{34},J_{13}]-[J_{34},J_{24}])=\frac{1}{4}(-J_{23}+J_{14}+J_{14}-J_{23})=\frac{1}{2}(J_{14}-J_{23})=J'_3
        \end{equation*}
        \begin{equation*}
            [J'_1,J'_3]=\frac{1}{4}([J_{12},J_{14}]-[J_{34},J_{14}]-[J_{12},J_{23}]+[J_{34},J_{23}])=\frac{1}{4}(-J_{24}-J_{13}-J_{13}-J_{24})=-\frac{1}{2}(J_{13}+J_{24})=-J'_2
        \end{equation*}
        \begin{equation*}
            [J'_2,J'_3]=\frac{1}{4}([J_{13},J_{14}]+[J_{24},J_{14}]-[J_{13},J_{23}]-[J_{24},J_{23}])=\frac{1}{4}(-J_{34}+J_{12}+J_{12}-J_{34})=\frac{1}{2}(J_{12}-J_{34})=J'_1
        \end{equation*}
        Следовательно,
        \begin{equation}
            [J'_i,J'_j]=\epsilon_{ijk}J'_k\rightarrow \braket{J'_1,J'_2,J'_3}\simeq\mathfrak{so}(3)
        \end{equation}
        \begin{equation*}
            [J_1,J'_1]=\frac{1}{4}([J_{12},J_{12}]+[J_{34},J_{12}]-[J_{12},J_{34}]-[J_{34},J_{34}])=0
        \end{equation*}
        \begin{equation*}
            [J_1,J'_2]=\frac{1}{4}([J_{12},J_{13}]+[J_{34},J_{13}]+[J_{12},J_{24}]+[J_{34},J_{24}])=\frac{1}{4}(-J_{23}-J_{14}+J_{14}+J_{23})=0
        \end{equation*}
        \begin{equation*}
            [J_1,J'_3]=\frac{1}{4}([J_{12},J_{14}]+[J_{34},J_{14}]-[J_{12},J_{23}]-[J_{34},J_{23}])=\frac{1}{4}(-J_{24}+J_{13}-J_{13}+J_{24})=0
        \end{equation*}
        \begin{equation*}
            [J_2,J'_1]=\frac{1}{4}([J_{14},J_{12}]+[J_{23},J_{12}]-[J_{14},J_{34}]-[J_{23},J_{34}])=\frac{1}{4}(J_{24}-J_{13}-J_{13}-J_{24})=0
        \end{equation*}
        \begin{equation*}
            [J_2,J'_2]=\frac{1}{4}([J_{14},J_{13}]+[J_{23},J_{13}]+[J_{14},J_{24}]+[J_{23},J_{24}])=\frac{1}{4}(J_{34}+J_{12}-J_{12}-J_{34})=0
        \end{equation*}
        \begin{equation*}
            [J_2,J'_3]=\frac{1}{4}([J_{14},J_{14}]+[J_{23},J_{14}]-[J_{14},J_{23}]-[J_{23},J_{23}])=0
        \end{equation*}
        \begin{equation*}
            [J_3,J'_1]=\frac{1}{4}([J_{13},J_{12}]-[J_{24},J_{12}]-[J_{13},J_{34}]+[J_{24},J_{34}])=\frac{1}{4}(J_{23}+J_{14}-J_{14}-J_{23})=0
        \end{equation*}
        \begin{equation*}
            [J_3,J'_2]=\frac{1}{4}([J_{13},J_{13}]-[J_{24},J_{13}]+[J_{13},J_{24}]-[J_{24},J_{24}])=0
        \end{equation*}
        \begin{equation*}
            [J_3,J'_3]=\frac{1}{4}([J_{13},J_{14}]-[J_{24},J_{14}]-[J_{13},J_{23}]+[J_{24},J_{23}])=\frac{1}{4}(-J_{34}-J_{12}+J_{12}+J_{34})=0
        \end{equation*}
        Следовательно,
        \begin{equation}
            [J_i,J'_j]=0
        \end{equation}
        Таким образом,
        \begin{equation}
            \boxed{\mathfrak{so}(4)\simeq\mathfrak{so}(3)\oplus\mathfrak{so}(3)}
        \end{equation}
    \end{proof}
\end{itemize}
\end{zad}
\section{Представления алгебры $\mathfrak{su}(2)$.}
\begin{zad}
\begin{itemize}
    \item[а)]
    \begin{predl}
        $J_+J_-=-J^2-J_3^2-iJ_3$.
    \end{predl}
    \begin{proof}
        \begin{equation}
            J^2=-J_1^2-J_2^2-J_3^2,\quad J_+=J_1+iJ_2,\quad J_-=J_1-iJ_2
        \end{equation}
        \begin{equation}
            J_+J_-=(J_1+iJ_2)(J_1-iJ_2)=J_1^2-iJ_1J_2+iJ_2J_1+J_2^2=J_1^2+J_2^2-i[J_1,J_2]
        \end{equation}
        \begin{equation}
            \boxed{J_+J_-=-J^2-J_3^2-iJ_3}
        \end{equation}
    \end{proof}
    \item[б)]
    \begin{equation}
        J_+J_-=\text{diag}(a_{j-1}b_j,a_{j-2}b_{j-1},...,a_{-j}b_{1-j},0)
    \end{equation}
    Подставим доказанное в п. а равенством $J_+J_-=-J^2-J_3^2-iJ_3$ и $b_{m+1}=-\bar{a}_m$:
    \begin{equation}
        \text{diag}(a_{j-1}b_j,a_{j-2}b_{j-1},...,a_{-j}b_{1-j},0)=-J^2-J_3^2-iJ_3
    \end{equation}
    \begin{equation}
        \text{diag}(-|a_{j-1}|^2,-|a_{j-2}|^2,...,-|a_{-j}|^2,0)=\text{diag}(-\lambda-j^2-j,...,-\lambda+j^2+j=0)
    \end{equation}
    \begin{equation}
        |a_{j-1-k}|^2=\lambda+j-k-(j-k)^2=2j-k-k^2+2jk=2j(k+1)-k(k+1)
    \end{equation}
    \begin{equation}
        |a_{j-1-k}|=\sqrt{(2j-k)(k+1)}
    \end{equation}
    Домножим векторы ортонормированного базиса на фазу, чтобы $a_m^2\in\mathbb{R}$.
    \begin{equation}
        \boxed{a_{j-1-k}=\pm\sqrt{(2j-k)(k+1)}}
    \end{equation}
\end{itemize}
\end{zad}
\begin{zad}
По предложению 6 лекции 10 характеры неприводимых представлений $\pi_j$ группы $SU(2)$ равны:
\begin{equation}
    \chi_j(\varphi)=\frac{\sin((2j+1)\varphi)}{\sin\varphi}
\end{equation}
\begin{equation*}
    \chi_{\pi_\frac{1}{2}\otimes\pi_\frac{1}{2}}(\varphi)=\chi^2_{\frac{1}{2}}(\varphi)=\frac{\sin^22\varphi}{\sin^2\varphi}=\frac{2(1+\cos 2\varphi)\sin\varphi}{\sin\varphi}=\frac{2\sin\varphi+\sin 3\varphi-\sin\varphi}{\sin\varphi}=\frac{\sin\varphi+\sin3\varphi}{\sin\varphi}
\end{equation*}
\begin{equation}
    \chi_{\pi_\frac{1}{2}\otimes\pi_\frac{1}{2}}(\varphi)=\chi_0(\varphi)+\chi_1(\varphi)
\end{equation}
Таким образом, $\pi_{\frac{1}{2}}\otimes\pi_{\frac{1}{2}}$ раскладывается как
\begin{equation}
    \boxed{\pi_{\frac{1}{2}}\otimes\pi_{\frac{1}{2}}=\pi_0\oplus \pi_1}
\end{equation}
\end{zad}
\begin{zad}
\begin{predl}
Представление $\pi_j$, в котором действие генераторов $iJ_3$, $J_+$, $J_-$ задано формулами:
\begin{equation}
    iJ_3\rightarrow\left(
    \begin{array}{ccccc}
    j & 0 & \hdots & 0 & 0\\
    0 & j-1 & \hdots & 0 & 0\\
    \hdots & \hdots & \hdots & \hdots & \hdots\\
    0 & 0 & \hdots & -j+1 & 0\\
    0 & 0 & \hdots & 0 & -j\\
    \end{array}
    \right),
    \end{equation}
    \begin{equation}
    J_+\rightarrow\left(
    \begin{array}{ccccc}
    0 & a_{j-1} & \hdots & 0 & 0\\
    0 & 0 & \hdots & 0 & 0\\
    \hdots & \hdots & \hdots & \hdots & \hdots\\
    0 & 0 & \hdots & 0 & a_{-j}\\
    0 & 0 & \hdots & 0 & 0\\
    \end{array}
    \right),\quad J_-\rightarrow\left(
    \begin{array}{ccccc}
    0 & 0 & \hdots & 0 & 0\\
    b_j & 0 & \hdots & 0 & 0\\
    \hdots & \hdots & \hdots & \hdots & \hdots\\
    0 & 0 & \hdots & 0 & 0\\
    0 & 0 & \hdots & b_{1-j} & 0\\
    \end{array}
    \right),
\end{equation}
является неприводимым.
\end{predl}
\begin{proof}
    Пусть $U\subset V$ -- инвариантное относительно $iJ_3$, $J_+$, $J_-$ подпространство, $u_1,...,u_t$ -- его базис. Разложим векторы $u_i$ по базису $v_{\lambda,k}$:
    \begin{equation}
        u_i=\sum_{k=-j}^jh_kv_{\lambda,k}
    \end{equation}
    В лекции 10 показано, что $J_+$ нильпотентный, поэтому $\exists$ минимальное $n$: $J_+^nu_i=0$.
    \begin{equation}
        J_+^{n-1}u_j=hv_{\lambda,j}\in U\rightarrow v_{\lambda,j}\in U
    \end{equation}
    Применением $J_-$ из $v_{\lambda,j}$ можно получить все $v_{\lambda,k}$. Таким образом, $U=V$ и в $V$ нет инвариантных подпространств и представление $\pi_j$ является неприводимым.
\end{proof}
\end{zad}
\section{Представления групп $SO(3)$ и $SU(2)$.}
\begin{upr}
\begin{itemize}
    \item[а)] В задаче 10.2 показано, что
    \begin{equation}
        \pi_\frac{1}{2}\otimes\pi_\frac{1}{2}=\pi_0\oplus \pi_1
    \end{equation}
    По предложению 3 лекции 11 $\pi_\frac{1}{2}\otimes(\pi_0\oplus\pi_1)=\pi_\frac{1}{2}\oplus\pi_\frac{1}{2}\oplus\pi_\frac{3}{2}$. Таким образом,
    \begin{equation}
        \boxed{\pi_\frac{1}{2}\otimes\pi_\frac{1}{2}\otimes\pi_\frac{1}{2}=\pi_\frac{1}{2}\oplus\pi_\frac{1}{2}\oplus\pi_\frac{3}{2}}
    \end{equation}
    \item[б)] По предложению 3 лекции 11
    \begin{equation}
        \pi_0\otimes\pi_j=\pi_j,\quad\pi_\frac{1}{2}\otimes\pi_j=\pi_{j+\frac{1}{2}}\oplus\pi_{j-\frac{1}{2}},\quad j>0
    \end{equation}
    При тензорном умножении $\pi_\frac{1}{2}$ на себя образуется $\pi_0$. Далее при умножении на $\pi_\frac{1}{2}$ $\pi_0$ пропадает (превращается в $\pi_\frac{1}{2}$). После этого $\pi_0$ вновь образуются из произведений $\pi_\frac{1}{2}\otimes\pi_\frac{1}{2}$. Затем они пропадут и т.д. Т.е. наличие $\pi_0$ определяется чётностью степени $\pi_\frac{1}{2}$: при нечётных степенях кратность равна 0. Поскольку 101 -- нечётное число, то кратность вхождения тривиального представления в $\pi^{101}_\frac{1}{2}$ равна 0.
\end{itemize}
\end{upr}
\begin{zad}
\begin{itemize}
    \item[а)] Произвольный многочлен $\mathbb{C}[x_1,x_2,x_3]_0$:
    \begin{equation}
        P_0(x_1,x_2,x_3)=a_1
    \end{equation}
    \begin{equation}
        \Delta P_0=0\rightarrow H_0=\mathbb{C}[x_1,x_2,x_3]_0, H_0=\braket{1}
    \end{equation}
    \begin{equation}
        \boxed{\dim H_0=1}
    \end{equation}
    Произвольный многочлен $\mathbb{C}[x_1,x_2,x_3]_1$:
    \begin{equation}
        P_1(x_1,x_2,x_3)=a_1x_1+a_2x_2+a_3x_3
    \end{equation}
    \begin{equation}
        \Delta P_1=0\rightarrow H_1=\mathbb{C}[x_1,x_2,x_3]_0, H_1=\braket{x_1,x_2,x_3}
    \end{equation}
    \begin{equation}
        \boxed{\dim H_1=3}
    \end{equation}
    Произвольный многочлен $\mathbb{C}[x_1,x_2,x_3]_2$:
    \begin{equation}
        P_2(x_1,x_2,x_3)=a_1x_1^2+a_2x_2^2+a_3x_3^2+a_4x_1x_2+a_5x_1x_3+a_6x_2x_3
    \end{equation}
    \begin{equation}
        \Delta P_2=2a_1+2a_2+2a_3
    \end{equation}
    \begin{equation}
    H_2=\mathbb{C}[x_1,x_2,x_3]_2 \text{ при } a_1+a_2+a_3=0
    \end{equation}
    \begin{equation}
         H_2=\braket{x_1^2-x_2^2,x_1^2-x_3^2,x_1x_2,x_2x_3,x_1x_3}
    \end{equation}
    \begin{equation}
        \boxed{\dim H_2=5}
    \end{equation}
    Произвольный многочлен $\mathbb{C}[x_1,x_2,x_3]_3$:
    \begin{equation*}
        P_3(x_1,x_2,x_3)=a_1x_1^3+a_2x_2^3+a_3x_3^3+a_4x_1^2x_2+a_5x_1x_2^2+a_6x_1^2x_3+a_7x_1x_3^2+a_8x_2^2x_3+a_9x_2x_3^2+a_{10}x_1x_2x_3
    \end{equation*}
    \begin{equation}
        \Delta P_3=6a_1x_1+6a_2x_2+6a_3x_3+2a_4x_2+2a_5x_1+2a_6x_3+2a_7x_1+2a_8x_3+2a_9x_2
    \end{equation}
    \begin{equation}
    H_3=\mathbb{C}[x_1,x_2,x_3]_3 \text{ при } \left\{
    \begin{array}{l}
    3a_1+a_5+a_7=0,\\
    3a_2+a_4+a_9=0,\\
    3a_3+a_6+a_8=0.\\
    \end{array}
    \right.
    \end{equation}
    \begin{equation}
         H_3=\braket{x_1(x_2^2-x_3^2),x_2(x_1^2-x_3^2),x_3(x_1^2-x_2^2),x_1(x_1^2-3x_2^2),x_2(x_2^2-3x_1^2),x_3(x_3^2-x_1^2),x_1x_2x_3}
    \end{equation}
    \begin{equation}
        \boxed{\dim H_3=7}
    \end{equation}
    \item[б)] \begin{predl}
    Оператор Лапласа $\Delta$ является инвариантным относительно действия группы $SO(3)$.
    \end{predl}
    \begin{proof}
        Группа $SO(3)$ действует на пространстве $\mathbb{C}[x_1,x_2,x_3]_n$ по формуле
        \begin{equation}
            P(x_1,x_2,x_3)\rightarrow P((x_1,x_2,x_3)g)
        \end{equation}
        Это можно переписать в виде:
        \begin{equation}
            x'_j=\sum\limits_{i=1}^3x_ig_{ij},\quad \sum\limits_{j=1}^3g_{ij}(g_{jk})^T=\sum\limits_{j=1}^3g_{ij}g_{kj}=\delta_{ik}
        \end{equation}
        \begin{equation}
            \frac{\partial^2}{\partial x_i^2}=\frac{\partial}{\partial x_i}\left(\sum\limits_{j=1}^3\frac{\partial x'_j}{\partial x_i}\frac{\partial}{\partial x'_j}\right)=\sum\limits_{j,k=1}^3\frac{\partial x'_k}{\partial x_i}\frac{\partial x'_j}{\partial x_i}\frac{\partial^2}{\partial x'_j\partial x'_k}=g_{ik}g_{ij}\frac{\partial^2}{\partial x'_j\partial x'_k}
        \end{equation}
        \begin{equation}
            \Delta=\sum\limits_{i=1}^3\frac{\partial^2}{\partial x_i^2}=\sum\limits_{i=1}^3g_{ik}g_{ij}\sum\limits_{j,k=1}^3\frac{\partial^2}{\partial x'_j\partial x'_k}=\delta_{jk}\sum\limits_{j,k=1}^3\frac{\partial^2}{\partial x'_j\partial x'_k}=\sum\limits_{j}^3\frac{\partial^2}{\partial x'^2_j}=\Delta'
        \end{equation}
        Таким образом, оператор Лапласа $\Delta$ является инвариантным относительно действия группы $SO(3)$.
    \end{proof}
    \begin{sled}
    Пространства $H_n$ являются инвариантными подпространствами относительно группы $SO(3)$.
    \end{sled}
    \begin{proof}
        Пусть $P(x_1,x_2,x_3)\in H_n$, тогда $\Delta P=0$. По предложению 33: $\Delta' P=\Delta P=0$, значит $P(x'_1,x'_2,x'_3)\in H_n$ и пространства $H_n$ являются инвариантными подпространствами относительно группы $SO(3)$.
    \end{proof}
    \item[в)]
    \begin{predl}
    Отображение
    \begin{equation}
        t_{i_1...i_n}\rightarrow P(x)=t_{i_1...i_n}x_{i_1}...x_{i_n}
    \end{equation}
    осуществляет изоморфизм между пространством симметричных бесследовых тензоров в $S^n\mathbb{C}^3$ и пространством гармонических полиномов степени $n$.
    \end{predl}
    \begin{proof}
        Поскольку все операции линейные, то отображение является гомоморфизмом. Рассмотрим многочлен
        \begin{equation}
            P(x_1,x_2,x_3)=\sum\limits_{i_1,...,i_n=1}^3t_{i_1,...,i_n}x_{i_1}...x_{i_n}
        \end{equation}
        \begin{equation}
            \Delta P(x_1,x_2,x_3)=\sum\limits_{i_1,...,i_n=1}^3t_{i_1,...,i_n}\sum\limits_{i=1}^3\sum_{j=1}^{n-1}\sum_{k=j+1}^nx_{i_1}...\bar{x}_{i_j}...\bar{x}_{i_k}...x_{i_n}\delta_{i,i_j}\delta_{i,i_k},
        \end{equation}
        где $\bar{x}_{i_j}$ обозначает, что множителя $x_{i_j}$ нет.
        \begin{equation}
            \Delta P(x_1,x_2,x_3)=\sum\limits_{i_1,...,i_n=1}^3t_{i_1,...,i_n}\sum_{j=1}^{n-1}\sum_{k=j+1}^nx_{i_1}...\bar{x}_{i_j}...\bar{x}_{i_k}...x_{i_n}\delta_{i_j,i_k}
        \end{equation}
        \begin{equation}
        \Delta P(x_1,x_2,x_3)=0\Leftrightarrow \sum\limits_{i_j,i_k=1}^3t_{i_1,...,i_n}\delta_{i_j,i_k}=0\;\forall j,k\in\{1,...,n\}
        \end{equation}
        Условие справа означает бесследовость тензора $t_{i_1,...,i_n}$. Тензор $t_{i_1,...,i_n}$ симметричен, поскольку в $P$ от перемены мест множителей произведение не меняется. Отображение между пространством симметричных бесследовых тензоров в $S^n\mathbb{C}^3$ и пространством гармонических полиномов степени $n$ взаимно-однозначно, поскольку по коэффициентам $t_{i_1,...i_n}$ многочлен восстанавливается единственным образом. И наоборот, по многочлену легко выписать тензор $t_{i_1,...,i_n}$. Таким образом, показан изоморфизм между пространством симметричных бесследовых тензоров в $S^n\mathbb{C}^3$ и пространством гармонических полиномов степени $n$.
    \end{proof}
    \item[г)$^*$]
    \begin{equation}
        \mathbb{C}[x_1,x_2,x_3]_n=\{x_1^{b_1}x_2^{b_2}x_3^{b_3}\},\quad b_1+b_2+b_3=n
    \end{equation}
    Найдём число таких $x_1^{b_1}x_2^{b_2}x_3^{b_3}$. Оно равно количество способов поставить 2 перегородки между $n+3$ элементами. Всего мест между элементами для перегородок равно $n+2$. Следовательно,
    \begin{equation}
        \boxed{\dim \mathbb{C}[x_1,x_2,x_3]_n=C_{n+2}^2=\frac{(n+1)(n+2)}{2}}
    \end{equation}
    \begin{predl}
    Оператор Лапласа $\Delta:\mathbb{C}[x_1,x_2,x_3]_n\rightarrow\mathbb{C}[x_1,x_2,x_3]_{n-2}$ является cюръективным отображением.
    \end{predl}
    \begin{proof}
        Произвольный многочлен $P_n$:
        \begin{equation}
            P_n(x_1,x_2,x_3)=\sum\limits_{b_1+b_2+b_3=n}c_{b_1,b_2,b_3}x_1^{b_1}x_2^{b_2}x_3^{b_3}
        \end{equation}
        Пусть $m=\max(b_1,b_2,b_3)$. Пусть без ограничения общности $m=b_1$. Докажем утверждение методом математической индукции:
        \begin{enumerate}
            \item Проверим, что оно верно для $m=n$. Произвольный многочлен $\mathbb{C}[x_1,x_2,x_3]_n$:
            \begin{equation}
                P_n(x_1,x_2,x_3)=a_1x_1^n
            \end{equation}
            Соответствующий ему многочлен $P_{n+2}$ из $\mathbb{C}[x_1,x_2,x_3]_{n+2}$: $\Delta P_{n+2}=P_n$:
            \begin{equation}
                P_{n+2}(x_1,x_2,x_3)=\frac{a_1x_1^{n+2}}{(n+2)(n+1)}
            \end{equation}
            Значит, утверждение верно при $m=n$.
            \item Предположим, что утверждение верно для $m\geq k$:
            \begin{equation}
                \forall P_n(x_1,x_2,x_3): \max(b_1,b_2,b_3)\geq k \;\exists\; P_{n+2}(x_1,x_2,x_3): \Delta P_{n+2}=P_n
            \end{equation}
            \item Проверим, что оно верно для $m=k-1$.
            \begin{equation}
                P_{n}(x_1,x_2,x_3)=\sum\limits_{b_1+b_2+b_3=n}c_{b_1,b_2,b_3}x_1^{b_1}x_2^{b_2}x_3^{b_3}, \quad \max(b_1,b_2,b_3)=k-1
            \end{equation}
            $b_1=k-1$, $b_2\leq k-1$, $b_3\leq k-1$.
            \begin{equation*}
                \Delta (x_1^{k+1}x_2^{b_2}x_3^{b_3})=(k+1)kx_1^{k-1}x_2^{b_2}x_3^{b_3}+b_2(b_2-1)x_1^{k+1}x_2^{b_2-2}x_3^{b_3}+b_3(b_3-1)x_1^{k+1}x_2^{b_2}x_3^{b_3-2}
            \end{equation*}
            По предположению индукции $\exists P'_{n+2}(x_1,x_2,x_3): \Delta P'_{n+2}=b_2(b_2-1)x_1^{k+1}x_2^{b_2-2}x_3^{b_3}+b_3(b_3-1)x_1^{k+1}x_2^{b_2}x_3^{b_3-2}$, поскольку $\max(k+1,b_2-2,b_3)=\max(k+1,b_2,b_3-2)=k+1\geq k$.
            \begin{equation}
                x_1^{k-1}x_2^{b_2}x_3^{b_3}=\Delta\left(\frac{ x_1^{k+1}x_2^{b_2}x_3^{b_3}}{(k+1)k}-P'_{n+2}\right)
            \end{equation}
            А значит и
            \begin{equation}
                \forall P_n(x_1,x_2,x_3): \max(b_1,b_2,b_3)=k-1 \;\exists\; P_{n+2}(x_1,x_2,x_3): \Delta P_{n+2}=P_n
            \end{equation}
        \end{enumerate}
    \end{proof}
    В п. в показан изоморфизм между пространством симметричных бесследовых тензоров в $S^n\mathbb{C}^3$ и пространством $H_n$. По предложению 1 лекции 9 размерность пространства симметричных тензоров в $S^n\mathbb{C}^3$:
    \begin{equation}
        \dim S^n\mathbb{C}^3=C_{n+2}^2
    \end{equation}
    Условие бесследовости:
    \begin{equation}
        \delta_{i_ji_k}t_{i_1...i_j...i_k...i_n}=0
    \end{equation}
    Число уравнений равно числу способов выбрать 2 из $n$ элементов: $C^2_n$.
    \begin{equation}
        \dim H_n=\dim S^n\mathbb{C}^3-C^2_n=\frac{(n+1)(n+2)}{2}-\frac{n(n-1)}{2}
    \end{equation}
    \begin{equation}
        \boxed{\dim H_n=2n+1}
    \end{equation}
\end{itemize}
\end{zad}
\begin{zad}
\begin{itemize}
    \item[а)] 
    \begin{equation}
        V=\braket{e_+,e_-},\quad S^2V=\braket{e_+\otimes e_+,\frac{1}{\sqrt{2}}(e_+\otimes e_-+e_-\otimes e_+),e_-\otimes e_-}
    \end{equation}
    \begin{multline*}
        \braket{\pi_1\otimes\pi_\frac{1}{2}\simeq S^2V\otimes V=\bra e_+\otimes e_+\otimes e_+, e_+\otimes e_+\otimes e_-,\frac{1}{\sqrt{2}}(e_+\otimes e_-\otimes e_++e_-\otimes e_+\otimes e_+),\\\frac{1}{\sqrt{2}}(e_+\otimes e_-\otimes e_-+e_-\otimes e_+\otimes e_-),e_-\otimes e_-\otimes e_+,e_-\otimes e_-\otimes e_-}
    \end{multline*}
    Пусть $e_1=e_+\otimes e_+\otimes e_+$, $e_2=e_+\otimes e_+\otimes e_-$, $e_3=\frac{1}{\sqrt{2}}(e_+\otimes e_-\otimes e_++e_-\otimes e_+\otimes e_+)$, $e_4=\frac{1}{\sqrt{2}}(e_+\otimes e_-\otimes e_-+e_-\otimes e_+\otimes e_-)$, $e_5=e_-\otimes e_-\otimes e_+$, $e_6=e_-\otimes e_-\otimes e_-$.
    \begin{equation}
        J_+(e_+\otimes e_+\otimes e_+)=0,\quad J_+(e_+\otimes e_+\otimes e_-)=e_+\otimes e_+\otimes e_+
    \end{equation}
    \begin{equation}
        J_+\left(\frac{1}{\sqrt{2}}(e_+\otimes e_-\otimes e_++e_-\otimes e_+\otimes e_+)\right)=\sqrt{2}(e_+\otimes e_+\otimes e_+)
    \end{equation}
    \begin{equation*}
        J_+\left(\frac{1}{\sqrt{2}}(e_+\otimes e_-\otimes e_-+e_-\otimes e_+\otimes e_-)\right)=\sqrt{2}(e_+\otimes e_+\otimes e_-)+\frac{1}{\sqrt{2}}(e_+\otimes e_-\otimes e_++e_-\otimes e_+\otimes e_+)
    \end{equation*}
    \begin{equation*}
        J_+(e_-\otimes e_-\otimes e_+)=e_+\otimes e_-\otimes e_++e_-\otimes e_+\otimes e_+,\quad J_+(e_-\otimes e_-\otimes e_-)=e_+\otimes e_-\otimes e_-+e_-\otimes e_+\otimes e_-+e_-\otimes e_-\otimes e_+
    \end{equation*}
    \begin{equation}
        \boxed{J_+=\left(
    \begin{array}{cccccc}
    0 & 1 & \sqrt{2} & 0 & 0 & 0\\
    0 & 0 & 0 & \sqrt{2} & 0 & 0\\
    0 & 0 & 0 & 1 & \sqrt{2} & 0\\
    0 & 0 & 0 & 0 & 0 & \sqrt{2}\\
    0 & 0 & 0 & 0 & 0 & 1\\
    0 & 0 & 0 & 0 & 0 & 0\\
    \end{array}
    \right)}
    \end{equation}
    
    \begin{equation*}
        J_-(e_+\otimes e_+\otimes e_+)=-e_-\otimes e_+\otimes e_+-e_+\otimes e_-\otimes e_+-e_+\otimes e_+\otimes e_-
    \end{equation*}
    \begin{equation}
        J_-(e_+\otimes e_+\otimes e_-)=-e_+\otimes e_-\otimes e_--e_-\otimes e_+\otimes e_-
    \end{equation}
    \begin{equation*}
        J_-\left(\frac{1}{\sqrt{2}}(e_+\otimes e_-\otimes e_++e_-\otimes e_+\otimes e_+)\right)=-\frac{1}{\sqrt{2}}(e_+\otimes e_-\otimes e_-+e_-\otimes e_+\otimes e_-)-\sqrt{2}e_-\otimes e_-\otimes e_+
    \end{equation*}
    \begin{equation}
        J_-\left(\frac{1}{\sqrt{2}}(e_+\otimes e_-\otimes e_-+e_-\otimes e_+\otimes e_-)\right)=-\sqrt{2}e_-\otimes e_-\otimes e_-
    \end{equation}
    \begin{equation}
        J_+(e_-\otimes e_-\otimes e_+)=-e_-\otimes e_-\otimes e_-,\quad J_-(e_-\otimes e_-\otimes e_-)=0
    \end{equation}
    \begin{equation}
        \boxed{J_-=\left(
    \begin{array}{cccccc}
    0 & 0 & 0 & 0 & 0 & 0\\
    -1 & 0 & 0 & 0 & 0 & 0\\
    -\sqrt{2} & 0 & 0 & 0 & 0 & 0\\
    0 & -\sqrt{2} & -1 & 0 & 0 & 0\\
    0 & 0 & -\sqrt{2} & 0 & 0 & 0\\
    0 & 0 & 0 & -\sqrt{2} & -1 & 0\\
    \end{array}
    \right)}
    \end{equation}
    
    \begin{equation}
        iJ_3(e_+\otimes e_+\otimes e_+)=\frac{3}{2}(e_+\otimes e_+\otimes e_+),\quad iJ_3(e_+\otimes e_+\otimes e_-)=\frac{1}{2}(e_+\otimes e_+\otimes e_-)
    \end{equation}
    \begin{equation}
        iJ_3(\frac{1}{\sqrt{2}}(e_+\otimes e_-\otimes e_++e_-\otimes e_+\otimes e_+))=\frac{1}{2\sqrt{2}}(e_+\otimes e_-\otimes e_++e_-\otimes e_+\otimes e_+)
    \end{equation}
    \begin{equation}
        iJ_3(\frac{1}{\sqrt{2}}(e_+\otimes e_-\otimes e_-+e_-\otimes e_+\otimes e_-))=-\frac{1}{2\sqrt{2}}(e_+\otimes e_-\otimes e_-+e_-\otimes e_+\otimes e_-)
    \end{equation}
    \begin{equation}
        iJ_3(e_-\otimes e_-\otimes e_+)=-\frac{1}{2}(e_-\otimes e_-\otimes e_+),\quad iJ_3(e_-\otimes e_-\otimes e_-)=-\frac{3}{2}(e_-\otimes e_-\otimes e_-)
    \end{equation}
    \begin{equation}
        \boxed{iJ_3=\left(
    \begin{array}{cccccc}
    \frac{3}{2} & 0 & 0 & 0 & 0 & 0\\
    0 & \frac{1}{2} & 0 & 0 & 0 & 0\\
    0 & 0 & \frac{1}{2} & 0 & 0 & 0\\
    0 & 0 & 0 & -\frac{1}{2} & 0 & 0\\
    0 & 0 & 0 & 0 & -\frac{1}{2} & 0\\
    0 & 0 & 0 & 0 & 0 & -\frac{3}{2}\\
    \end{array}
    \right)}
    \end{equation}
\end{itemize}
\item[б)]
По предложению 3 лекции 11:
\begin{equation}
\pi_1\otimes\pi_\frac{1}{2}=\pi_\frac{1}{2}\oplus\pi_\frac{3}{2}\simeq S^3V\oplus V
\end{equation}
Значит, $V$ и $S^3V$ являются инвариантными подпространствами.
\begin{multline*}
S^3V=\braket{e_+\otimes e_+\otimes e_+,\frac{1}{\sqrt{3}}(e_+\otimes e_+\otimes e_-+e_+\otimes e_-\otimes e_++e_-\otimes e_+\otimes e_+),\\ \frac{1}{\sqrt{3}}(e_+\otimes e_-\otimes e_-+e_-\otimes e_+\otimes e_-+e_-\otimes e_-\otimes e_+),e_-\otimes e_-\otimes e_-}
\end{multline*}
Выразим базисные векторы в $S^3V$ через базисные векторы в $S^2V\otimes V$ из п. а:
\begin{equation}
e'_1=e_+\otimes e_+\otimes e_+=e_1
\end{equation}
\begin{equation}
e'_2=\frac{1}{\sqrt{3}}(e_+\otimes e_+\otimes e_-+e_+\otimes e_-\otimes e_++e_-\otimes e_+\otimes e_+)=\frac{e_2}{\sqrt{3}}+\sqrt{\frac{2}{3}}e_3
\end{equation}
\begin{equation}
e'_3=\frac{1}{\sqrt{3}}(e_+\otimes e_-\otimes e_-+e_-\otimes e_+\otimes e_-+e_-\otimes e_-\otimes e_+)=\sqrt{\frac{2}{3}}e_4+\frac{e_5}{\sqrt{3}}
\end{equation}
\begin{equation}
e'_4=e_-\otimes e_-\otimes e_-=e_6
\end{equation}
\begin{equation}
\boxed{S^3V=\braket{e_1,\frac{e_2}{\sqrt{3}}+\sqrt{\frac{2}{3}}e_3,\sqrt{\frac{2}{3}}e_4+\frac{e_5}{\sqrt{3}},e_6}}
\end{equation}
Дополним $e'_1$, $e'_2$, $e'_3$' $e'_4$ до базиса векторами $e_2$ и $e_5$ при помощи ортогонализации Грама-Шмидта. Вектор $e_2$ выбран, т.к. он ортогонален всем векторам, кроме $e'_2$:
\begin{equation}
    \tilde e'_5=e_2-\frac{(e_2,e'_2)}{(e'_2,e'_2)}e'_2=e_2-\frac{1}{\sqrt{3}}\left(\frac{e_2}{\sqrt{3}}+\sqrt{\frac{2}{3}}e_3\right)=\frac{2e_2-\sqrt{2}e_3}{3}
\end{equation}
\begin{equation}
    e'_5=\frac{\tilde e'_5}{(\tilde e'_5,\tilde e'_5)}=\sqrt{\frac{2}{3}}e_2-\frac{e_3}{\sqrt{3}}
\end{equation}
Вектор $e_5$ выбран, т.к. он оротонален всем векторам, кроме $e'_3$:
\begin{equation}
    \tilde e'_6=e_5-\frac{(e_5,e'_3)}{(e'_3,e'_3)}e'_3=e_5-\frac{1}{\sqrt{3}}\left(\sqrt{\frac{2}{3}}e_4+\frac{e_5}{\sqrt{3}}\right)=\frac{2e_5-\sqrt{2}e_4}{3}
\end{equation}
\begin{equation}
    e'_6=\frac{\tilde e'_6}{(\tilde e'_6,\tilde e'_6)}=\sqrt{\frac{2}{3}}e_5-\frac{e_4}{\sqrt{3}}
\end{equation}
Рассмотрим действие генераторов $J_+$, $J_-$, $iJ_3$ в новом базисе.
    \begin{equation}
        J_+(e'_1)=J_+(e_1)=0,\quad J_+(e'_2)=J_+\left(\frac{e_2}{\sqrt{3}}+\sqrt{\frac{2}{3}}e_3\right)=\sqrt{3}e_1=\sqrt{3}e'_1
    \end{equation}
    \begin{equation*}
        J_+(e'_3)=J_+\left(\sqrt{\frac{2}{3}}e_4+\frac{e_5}{\sqrt{3}}\right)=2\sqrt{\frac{2}{3}}e_3+\frac{2}{\sqrt{3}}e_2=2e'_2,\quad J_+(e'_4)=J_+(e_6)=e_5+\sqrt{2}e_4=\sqrt{3}e'_3
    \end{equation*}
    \begin{equation}
        J_+(e'_5)=J_+\left(\sqrt{\frac{2}{3}}e_2-\frac{e_3}{\sqrt{3}}\right)=\sqrt{\frac{2}{3}}e_1-\sqrt{\frac{2}{3}}e_1=0
    \end{equation}
    \begin{equation*}
        J_+(e'_6)=J_+\left(\sqrt{\frac{2}{3}}e_5-\frac{e_4}{\sqrt{3}}\right)=\frac{e_3}{\sqrt{3}}-\sqrt{\frac{2}{3}}e_2=-e'_5
    \end{equation*}
    Как видно, $J_+(e'_6)=-e'_5$. Нужно, чтобы $J_+(e'_6)=e'_5$, поэтому поменяем знак $e'_6$:
    \begin{equation}
        e'_6=\frac{e_4}{\sqrt{3}}-\sqrt{\frac{2}{3}}e_5
    \end{equation}
    \begin{equation}
        J_+=\left(
    \begin{array}{cccccc}
    0 & \sqrt{3} & 0 & 0 & 0 & 0\\
    0 & 0 & 2 & 0 & 0 & 0\\
    0 & 0 & 0 & \sqrt{3} & 0 & 0\\
    0 & 0 & 0 & 0 & 0 & 0\\
    0 & 0 & 0 & 0 & 0 & 1\\
    0 & 0 & 0 & 0 & 0 & 0\\
    \end{array}
    \right)
    \end{equation}
    
    \begin{equation}
        J_-(e'_1)=J_-(e_1)=-\sqrt{3}e'_2,\quad J_-(e'_2)=J_-\left(\frac{e_2}{\sqrt{3}}+\sqrt{\frac{2}{3}}e_3\right)=-2e'_3
    \end{equation}
    \begin{equation}
        J_-(e'_3)=J_-\left(\sqrt{\frac{2}{3}}e_4+\frac{e_5}{\sqrt{3}}\right)=-\sqrt{3}e_4,\quad J_-(e'_4)=J_-(e_6)=0
    \end{equation}
    \begin{equation}
        J_-(e'_5)=J_-\left(\sqrt{\frac{2}{3}}e_2-\frac{e_3}{\sqrt{3}}\right)=-e'_6,\quad J_-(e'_6)=J_-\left(\sqrt{\frac{2}{3}}e_5-\frac{e_4}{\sqrt{3}}\right)=0
    \end{equation}
    \begin{equation}
        J_-=\left(
    \begin{array}{cccccc}
    0 & 0 & 0 & 0 & 0 & 0\\
    -\sqrt{3} & 0 & 0 & 0 & 0 & 0\\
    0 & -2 & 0 & 0 & 0 & 0\\
    0 & 0 & -\sqrt{3} & 0 & 0 & 0\\
    0 & 0 & 0 & 0 & 0 & 0\\
    0 & 0 & 0 & 0 & -1 & 0\\
    \end{array}
    \right)
    \end{equation}
    
    \begin{equation}
        iJ_3(e'_1)=iJ_3(e_1)=\frac{3}{2}e'_2,\quad iJ_3(e'_2)=iJ_3\left(\frac{e_2}{\sqrt{3}}+\sqrt{\frac{2}{3}}e_3\right)=\frac{1}{2}e'_2
    \end{equation}
    \begin{equation}
        iJ_3(e'_3)=iJ_3\left(\sqrt{\frac{2}{3}}e_4+\frac{e_5}{\sqrt{3}}\right)=-\frac{1}{2}e_4,\quad iJ_3(e'_4)=iJ_3(e_6)=-\frac{3}{2}e_4
    \end{equation}
    \begin{equation}
        iJ_3(e'_5)=iJ_3\left(\sqrt{\frac{2}{3}}e_2-\frac{e_3}{\sqrt{3}}\right)=\frac{1}{2}e'_5,\quad iJ_3(e'_6)=iJ_3\left(\sqrt{\frac{2}{3}}e_5-\frac{e_4}{\sqrt{3}}\right)=-\frac{1}{2}e'_6
    \end{equation}
    \begin{equation}
        iJ_3=\left(
    \begin{array}{cccccc}
    \frac{3}{2} & 0 & 0 & 0 & 0 & 0\\
    0 & \frac{1}{2} & 0 & 0 & 0 & 0\\
    0 & 0 & -\frac{1}{2} & 0 & 0 & 0\\
    0 & 0 & 0 & -\frac{3}{2} & 0 & 0\\
    0 & 0 & 0 & 0 & \frac{1}{2} & 0\\
    0 & 0 & 0 & 0 & 0 & -\frac{1}{2}\\
    \end{array}
    \right)
    \end{equation}
    \begin{equation}
        \boxed{V=\braket{\sqrt{\frac{2}{3}}e_2-\frac{e_3}{\sqrt{3}},\frac{e_4}{\sqrt{3}}-\sqrt{\frac{2}{3}}e_5}}
    \end{equation}
    \item[в)$^*$] $3j$ символы Вигнера -- коэффиценты разложения одного базиса по другому:
    \begin{equation}
        v_{j,m}=\sum C_{j,m}^{j_1,m_1,j_2,m_2}v_{j_1,m_1}\otimes v_{j_2,m_2},\quad j_1=1, j_2=\frac{1}{2}
    \end{equation}
    \begin{equation}
        v_{\frac{3}{2},\frac{3}{2}}=e'_1=e_1=e_+\otimes e_+\otimes e_+=v_{1,1}\otimes v_{\frac{1}{2},\frac{1}{2}}
    \end{equation}
    \begin{equation}
        \boxed{C_{\frac{3}{2},\frac{3}{2}}^{1,1,\frac{1}{2},\frac{1}{2}}=1}
    \end{equation}
    
    \begin{equation}
        v_{\frac{3}{2},\frac{1}{2}}=e'_2=\frac{e_2}{\sqrt{3}}+\sqrt{\frac{2}{3}}e_3=\frac{1}{\sqrt{3}}e_+\otimes e_+\otimes e_-+\sqrt{\frac{2}{3}}\frac{1}{\sqrt{2}}(e_+\otimes e_-+e_-\otimes e_+)\otimes e_+
    \end{equation}
    \begin{equation}
        v_{\frac{3}{2},\frac{1}{2}}=\frac{1}{\sqrt{3}}v_{1,1}\otimes v_{\frac{1}{2},-\frac{1}{2}}+\sqrt{\frac{2}{3}}v_{1,0}\otimes v_{\frac{1}{2},\frac{1}{2}}
    \end{equation}
    \begin{equation}
        \boxed{C_{\frac{3}{2},\frac{1}{2}}^{1,1,\frac{1}{2},-\frac{1}{2}}=\frac{1}{\sqrt{3}},\quad C_{\frac{3}{2},\frac{1}{2}}^{1,0,\frac{1}{2},\frac{1}{2}}=\sqrt{\frac{2}{3}}}
    \end{equation}
    
    \begin{equation}
        v_{\frac{3}{2},-\frac{1}{2}}=e'_3=\sqrt{\frac{2}{3}}e_4+\frac{e_5}{\sqrt{3}}=\frac{1}{\sqrt{3}}e_-\otimes e_-\otimes e_++\sqrt{\frac{2}{3}}\frac{1}{\sqrt{2}}(e_+\otimes e_-+e_-\otimes e_+)\otimes e_-
    \end{equation}
    \begin{equation}
        v_{\frac{3}{2},-\frac{1}{2}}=\frac{1}{\sqrt{3}}v_{1,-1}\otimes v_{\frac{1}{2},\frac{1}{2}}+\sqrt{\frac{2}{3}}v_{1,0}\otimes v_{\frac{1}{2},-\frac{1}{2}}
    \end{equation}
    \begin{equation}
        \boxed{C_{\frac{3}{2},\frac{1}{2}}^{1,-1,\frac{1}{2},\frac{1}{2}}=\frac{1}{\sqrt{3}},\quad C_{\frac{3}{2},\frac{1}{2}}^{1,0,\frac{1}{2},-\frac{1}{2}}=\sqrt{\frac{2}{3}}}
    \end{equation}
    
    \begin{equation}
        v_{\frac{3}{2},-\frac{3}{2}}=e'_4=e_6=e_-\otimes e_-\otimes e_-=v_{1,-1}\otimes v_{\frac{1}{2},-\frac{1}{2}}
    \end{equation}
    \begin{equation}
        \boxed{C_{\frac{3}{2},-\frac{3}{2}}^{1,-1,\frac{1}{2},-\frac{1}{2}}=1}
    \end{equation}
    
    \begin{equation}
        v_{\frac{3}{2},\frac{1}{2}}=e'_5=\sqrt{\frac{2}{3}}e_2-\frac{e_3}{\sqrt{3}}=\sqrt{\frac{2}{3}}e_+\otimes e_+\otimes e_--\frac{1}{\sqrt{3}}\frac{1}{\sqrt{2}}(e_+\otimes e_-+e_-\otimes e_+)\otimes e_+
    \end{equation}
    \begin{equation}
        v_{\frac{3}{2},\frac{1}{2}}=\sqrt{\frac{2}{3}}v_{1,1}\otimes v_{\frac{1}{2},-\frac{1}{2}}-\frac{1}{\sqrt{3}}v_{1,0}\otimes v_{\frac{1}{2},\frac{1}{2}}
    \end{equation}
    \begin{equation}
        \boxed{C_{\frac{3}{2},\frac{1}{2}}^{1,1,\frac{1}{2},-\frac{1}{2}}=\sqrt{\frac{2}{3}},\quad C_{\frac{3}{2},\frac{1}{2}}^{1,0,\frac{1}{2},\frac{1}{2}}=-\frac{1}{\sqrt{3}}}
    \end{equation}
    
    \begin{equation}
        v_{\frac{3}{2},-\frac{1}{2}}=e'_6=\frac{e_4}{\sqrt{3}}-\sqrt{\frac{2}{3}}e_5=-\sqrt{\frac{2}{3}}e_-\otimes e_-\otimes e_++\frac{1}{\sqrt{3}}\frac{1}{\sqrt{2}}(e_+\otimes e_-+e_-\otimes e_+)\otimes e_-
    \end{equation}
    \begin{equation}
        v_{\frac{3}{2},-\frac{1}{2}}=-\sqrt{\frac{2}{3}}v_{1,-1}\otimes v_{\frac{1}{2},\frac{1}{2}}+\frac{1}{\sqrt{3}}v_{1,0}\otimes v_{\frac{1}{2},-\frac{1}{2}}
    \end{equation}
    \begin{equation}
        \boxed{C_{\frac{3}{2},-\frac{1}{2}}^{1,-1,\frac{1}{2},\frac{1}{2}}=-\sqrt{\frac{2}{3}},\quad C_{\frac{3}{2},-\frac{1}{2}}^{1,0,\frac{1}{2},-\frac{1}{2}}=\frac{1}{\sqrt{3}}}
    \end{equation}
\end{zad}
\begin{zad}[$^*$]
\begin{itemize}
    \item[а)] Соответствие между перестановками различных циклических типов в $S_4$ и различными собственными движениями, сохраняющими куб (см. задачу 4.4):
    \begin{enumerate}
        \item $e$ (единичный, 1 шт.) -- тождественное движение (ничего не делает с кубом).
        \item $(a,b)$ (транспозиции, 6 шт.) -- повороты на $\pi$ вокруг прямой $l_1$, проходящей через центры противоположных рёбер.
        \item $(a,b)(c,d)$ (произведение транспозиций, 3 шт.) -- повороты на $\pi$ вокруг прямой $l_2$, проходящей через центры противоположных граней.
        \item $(a,b,c)$ (цикл длины 3, 8 шт.) -- поворот на $\frac{2\pi}{3}$, $\frac{4\pi}{3}$ вокруг прямой $l_3$, проходящей через диагональ куба.
        \item $(a,b,c,d)$ (цикл длины 4, 6 шт.) -- повороты на $\frac{\pi}{2}$, $\frac{3\pi}{2}$ вокруг прямой $l_2$, проходящей через центры противоположных граней.
    \end{enumerate}
    Таблица характеров группы $S_4$ (см. задачу 6.3):
    \begin{table}[h!]
    \centering
    \begin{tabular}{|l|l|l|l|l|l|}
    \hline
     & $e$ $^1$ & $(a,b)$ $^6$ & $(a,b)(c,d)$ $^3$ & $(a,b,c)$ $^8$ & $(a,b,c,d)$ $^6$ \\ \hline
    $\chi^{(1)}$ & $1$ & $1$ & $1$ & $1$ & $1$ \\ \hline
    $\chi^{(2)}$ & $1$ & $-1$ & $1$ & $1$ & $-1$ \\ \hline
    $\chi^{(3)}$ & $3$ & $1$ & $-1$ & $0$ & $-1$ \\ \hline
    $\chi^{(4)}$ & $3$ & $-1$ & $-1$ & $0$ & $1$ \\ \hline
    $\chi^{(5)}$ & $2$ &  $0$ & $2$ & $-1$ & $0$  \\ \hline
    \end{tabular}
    \end{table}
    \begin{equation}
        \chi_{\pi_2}(R(\alpha))=e^{-2i\alpha}+e^{-i\alpha}+1+e^{i\alpha}+e^{2i\alpha}=1+2\cos\alpha+2\cos 2\alpha
    \end{equation}
    \begin{equation}
        \chi_{\pi_2}(R(0))=1+2+2=5
    \end{equation}
    \begin{equation}
        \chi_{\pi_2}(R(\pi))=1-2+2=1
    \end{equation}
    \begin{equation}
        \chi_{\pi_2}\left(R\left(\frac{2\pi}{3}\right)\right)=1-1-1=-1
    \end{equation}
    \begin{equation}
        \chi_{\pi_2}\left(R\left(\frac{\pi}{2}\right)\right)=1+0-2=-1
    \end{equation}
    \begin{table}[h!]
    \centering
    \begin{tabular}{|l|l|l|l|l|l|}
    \hline
     & $e$ $^1$ & $(a,b)$ $^6$ & $(a,b)(c,d)$ $^3$ & $(a,b,c)$ $^8$ & $(a,b,c,d)$ $^6$ \\ \hline
    $\chi_{\pi_2}$ & $5$ & $1$ & $1$ & $-1$ & $-1$ \\ \hline
    \end{tabular}
    \end{table}\\
    Разложим $\chi_{\pi_2}$ при помощи алгоритма разложения на неприводимые:
    \begin{equation}
        \chi_{\pi_2}=\sum\limits_{i=1}^5 a_i\chi^{(i)},\quad a_i=\braket{\chi^{(i)},\chi_{\pi_2}}
    \end{equation}
    \begin{equation}
        a_1=\frac{1}{24}(1\cdot5+6\cdot1+3\cdot1+8\cdot(-1)+6\cdot(-1))=0
    \end{equation}
    \begin{equation}
        a_2=\frac{1}{24}(1\cdot5+6\cdot(-1)+3\cdot 1+8\cdot(-1)+6\cdot1)=0
    \end{equation}
    \begin{equation}
        a_3=\frac{1}{24}(1\cdot15+6\cdot1+3\cdot(-1)+8\cdot0+6\cdot1)=1
    \end{equation}
    \begin{equation}
        a_4=\frac{1}{24}(1\cdot15+6\cdot(-1)+3\cdot(-1)+8\cdot0+6\cdot(-1))=0
    \end{equation}
    \begin{equation}
        a_5=\frac{1}{24}(1\cdot10+6\cdot0+3\cdot2+8\cdot1+6\cdot0)=1
    \end{equation}
    \begin{equation}
        \boxed{\chi_{\pi_2}=\chi^{(3)}+\chi^{(5)}}
    \end{equation}
\item[б)] 
\begin{equation*}
    \chi_{\pi_n}(R(\alpha))=\sum\limits_{j=-n}^ne^{ij\alpha}=\frac{e^{-in\alpha}(e^{i(2n+1)\alpha}-1)}{e^{i\alpha}-1}=\frac{e^{i(n+1)\alpha}-e^{-in\alpha}}{e^{i\alpha}-1}=\frac{e^{i(n+\frac{1}{2})\alpha}-e^{-i(n+\frac{1}{2})\alpha}}{e^{\frac{i\alpha}{2}}-e^{-\frac{i\alpha}{2}}}
\end{equation*}
\begin{equation}
    \chi_{\pi_n}(R(\alpha))=\frac{\sin(n+\frac{1}{2})\alpha}{\sin\frac{\alpha}{2}}=\sin n\alpha\frac{\cos\frac{\alpha}{2}}{\sin\frac{\alpha}{2}}+\cos n\alpha
\end{equation}
\begin{equation}
    \chi_{\pi_n}(R(0))=2n+1
\end{equation}
\begin{equation}
    \chi_{\pi_n}(R(\pi))=\left\{
        \begin{array}{l}
        1,\;\;\;\quad n=2k\\
        -1,\quad n=2k+1\\
        \end{array}
        \right.\quad k\in\mathbb{Z}
\end{equation}
\begin{equation}
    \chi_{\pi_n}\left(R\left(\frac{2\pi}{3}\right)\right)=\left\{
        \begin{array}{l}
        1,\quad\quad n=3k\\
        0,\quad\quad n=3k+1\\
        -1,\;\quad n=3k+2\\
        \end{array}
        \right.\quad k\in\mathbb{Z}
\end{equation}
\begin{equation}
    \chi_{\pi_n}\left(R\left(\frac{\pi}{2}\right)\right)=\left\{
        \begin{array}{l}
        1,\quad\quad n=4k, 4k+1\\
        -1,\;\quad n=4k+2, 4k+3\\
        \end{array}
        \right.\quad k\in\mathbb{Z}
\end{equation}
Рассмотрим все случаи:
\begin{enumerate}
    \item $n=12k,k\in\mathbb{Z}$.
    \begin{table}[h!]
    \centering
    \begin{tabular}{|l|l|l|l|l|l|}
    \hline
     & $e$ $^1$ & $(a,b)$ $^6$ & $(a,b)(c,d)$ $^3$ & $(a,b,c)$ $^8$ & $(a,b,c,d)$ $^6$ \\ \hline
    $\chi_{\pi_n}$ & $2n+1$ & $1$ & $1$ & $1$ & $1$ \\ \hline
    \end{tabular}
    \end{table}\\
    Разложим $\chi_{\pi_n}$ при помощи алгоритма разложения на неприводимые:
    \begin{equation}
        \chi_{\pi_n}=\sum\limits_{i=1}^5 a_i\chi^{(i)},\quad a_i=\braket{\chi^{(i)},\chi_{\pi_n}}
    \end{equation}
    \begin{equation}
        a_1=\frac{1}{24}(1\cdot(2n+1)+6\cdot1+3\cdot1+8\cdot1+6\cdot1)=\frac{n}{12}+1
    \end{equation}
    \begin{equation}
        a_2=\frac{1}{24}(1\cdot(2n+1)+6\cdot(-1)+3\cdot1+8\cdot1+6\cdot(-1))=\frac{n}{12}
    \end{equation}
    \begin{equation}
        a_3=\frac{1}{24}(1\cdot(6n+3)+6\cdot1+3\cdot(-1)+8\cdot0+6\cdot(-1))=\frac{n}{4}
    \end{equation}
    \begin{equation}
        a_4=\frac{1}{24}(1\cdot(6n+3)+6\cdot(-1)+3\cdot(-1)+8\cdot0+6\cdot1)=\frac{n}{4}
    \end{equation}
    \begin{equation}
        a_5=\frac{1}{24}(1\cdot(4n+2)+6\cdot0+3\cdot2+8\cdot(-1)+6\cdot0)=\frac{n}{6}
    \end{equation}
    \begin{equation}
        \boxed{\chi_{\pi_n}=\chi^{(1)}\left(\frac{n}{12}+1\right)+\chi^{(2)}\frac{n}{12}+(\chi^{(3)}+\chi^{(4)})\frac{n}{4}+\chi^{(5)}\frac{n}{6}}
    \end{equation}
    \item $n=12k+1,k\in\mathbb{Z}$.
    \begin{table}[h!]
    \centering
    \begin{tabular}{|l|l|l|l|l|l|}
    \hline
     & $e$ $^1$ & $(a,b)$ $^6$ & $(a,b)(c,d)$ $^3$ & $(a,b,c)$ $^8$ & $(a,b,c,d)$ $^6$ \\ \hline
    $\chi_{\pi_n}$ & $2n+1$ & $-1$ & $-1$ & $0$ & $1$ \\ \hline
    \end{tabular}
    \end{table}\\
    Разложим $\chi_{\pi_n}$ при помощи алгоритма разложения на неприводимые:
    \begin{equation}
        \chi_{\pi_n}=\sum\limits_{i=1}^5 a_i\chi^{(i)},\quad a_i=\braket{\chi^{(i)},\chi_{\pi_n}}
    \end{equation}
    \begin{equation}
        a_1=\frac{1}{24}(1\cdot(2n+1)+6\cdot(-1)+3\cdot(-1)+8\cdot0+6\cdot1)=\frac{n-1}{12}
    \end{equation}
    \begin{equation}
        a_2=\frac{1}{24}(1\cdot(2n+1)+6\cdot1+3\cdot(-1)+8\cdot0+6\cdot(-1))=\frac{n-1}{12}
    \end{equation}
    \begin{equation}
        a_3=\frac{1}{24}(1\cdot(6n+3)+6\cdot(-1)+3\cdot1+8\cdot0+6\cdot(-1))=\frac{n-1}{4}
    \end{equation}
    \begin{equation}
        a_4=\frac{1}{24}(1\cdot(6n+3)+6\cdot1+3\cdot1+8\cdot0+6\cdot1)=\frac{n+3}{4}
    \end{equation}
    \begin{equation}
        a_5=\frac{1}{24}(1\cdot(4n+2)+6\cdot0+3\cdot(-2)+8\cdot0+6\cdot0)=\frac{n-1}{6}
    \end{equation}
    \begin{equation}
        \boxed{\chi_{\pi_n}=\chi^{(1)}\frac{n-1}{12}+\chi^{(2)}\frac{n-1}{12}+\chi^{(3)}\frac{n+3}{4}+\chi^{(4)}\frac{n+3}{4}+\chi^{(5)}\frac{n-1}{6}}
    \end{equation}
    
    \item $n=12k+2,k\in\mathbb{Z}$.
    \begin{table}[h!]
    \centering
    \begin{tabular}{|l|l|l|l|l|l|}
    \hline
     & $e$ $^1$ & $(a,b)$ $^6$ & $(a,b)(c,d)$ $^3$ & $(a,b,c)$ $^8$ & $(a,b,c,d)$ $^6$ \\ \hline
    $\chi_{\pi_n}$ & $2n+1$ & $1$ & $1$ & $-1$ & $-1$ \\ \hline
    \end{tabular}
    \end{table}\\
    Разложим $\chi_{\pi_n}$ при помощи алгоритма разложения на неприводимые:
    \begin{equation}
        \chi_{\pi_n}=\sum\limits_{i=1}^5 a_i\chi^{(i)},\quad a_i=\braket{\chi^{(i)},\chi_{\pi_n}}
    \end{equation}
    \begin{equation}
        a_1=\frac{1}{24}(1\cdot(2n+1)+6\cdot1+3\cdot1+8\cdot(-1)+6\cdot(-1))=\frac{n-2}{12}
    \end{equation}
    \begin{equation}
        a_2=\frac{1}{24}(1\cdot(2n+1)+6\cdot(-1)+3\cdot1+8\cdot(-1)+6\cdot1)=\frac{n-2}{12}
    \end{equation}
    \begin{equation}
        a_3=\frac{1}{24}(1\cdot(6n+3)+6\cdot1+3\cdot(-1)+8\cdot0+6\cdot1)=\frac{n+2}{4}
    \end{equation}
    \begin{equation}
        a_4=\frac{1}{24}(1\cdot(6n+3)+6\cdot(-1)+3\cdot(-1)+8\cdot0+6\cdot(-1))=\frac{n-2}{4}
    \end{equation}
    \begin{equation}
        a_5=\frac{1}{24}(1\cdot(4n+2)+6\cdot0+3\cdot2+8\cdot1+6\cdot0)=\frac{n+4}{6}
    \end{equation}
    \begin{equation}
        \boxed{\chi_{\pi_n}=\chi^{(1)}\frac{n-2}{12}+\chi^{(2)}\frac{n+2}{12}+\chi^{(3)}\frac{n+2}{4}+\chi^{(4)}\frac{n-2}{4}+\chi^{(5)}\frac{n+4}{6}}
    \end{equation}
    
    \item $n=12k+3,k\in\mathbb{Z}$.
    \begin{table}[h!]
    \centering
    \begin{tabular}{|l|l|l|l|l|l|}
    \hline
     & $e$ $^1$ & $(a,b)$ $^6$ & $(a,b)(c,d)$ $^3$ & $(a,b,c)$ $^8$ & $(a,b,c,d)$ $^6$ \\ \hline
    $\chi_{\pi_n}$ & $2n+1$ & $-1$ & $-1$ & $1$ & $-1$ \\ \hline
    \end{tabular}
    \end{table}\\
    Разложим $\chi_{\pi_n}$ при помощи алгоритма разложения на неприводимые:
    \begin{equation}
        \chi_{\pi_n}=\sum\limits_{i=1}^5 a_i\chi^{(i)},\quad a_i=\braket{\chi^{(i)},\chi_{\pi_n}}
    \end{equation}
    \begin{equation}
        a_1=\frac{1}{24}(1\cdot(2n+1)+6\cdot(-1)+3\cdot(-1)+8\cdot1+6\cdot(-1))=\frac{n-3}{12}
    \end{equation}
    \begin{equation}
        a_2=\frac{1}{24}(1\cdot(2n+1)+6\cdot1+3\cdot(-1)+8\cdot1+6\cdot1)=\frac{n+9}{12}
    \end{equation}
    \begin{equation}
        a_3=\frac{1}{24}(1\cdot(6n+3)+6\cdot(-1)+3\cdot1+8\cdot0+6\cdot1)=\frac{n+1}{4}
    \end{equation}
    \begin{equation}
        a_4=\frac{1}{24}(1\cdot(6n+3)+6\cdot1+3\cdot1+8\cdot0+6\cdot(-1))=\frac{n+1}{4}
    \end{equation}
    \begin{equation}
        a_5=\frac{1}{24}(1\cdot(4n+2)+6\cdot0+3\cdot(-2)+8\cdot(-1)+6\cdot0)=\frac{n-3}{6}
    \end{equation}
    \begin{equation}
        \boxed{\chi_{\pi_n}=\chi^{(1)}\frac{n-3}{12}+\chi^{(2)}\frac{n+9}{12}+\chi^{(3)}\frac{n+1}{4}+\chi^{(4)}\frac{n+1}{4}+\chi^{(5)}\frac{n-3}{6}}
    \end{equation}
    
    \item $n=12k+4,k\in\mathbb{Z}$.
    \begin{table}[h!]
    \centering
    \begin{tabular}{|l|l|l|l|l|l|}
    \hline
     & $e$ $^1$ & $(a,b)$ $^6$ & $(a,b)(c,d)$ $^3$ & $(a,b,c)$ $^8$ & $(a,b,c,d)$ $^6$ \\ \hline
    $\chi_{\pi_n}$ & $2n+1$ & $1$ & $1$ & $0$ & $1$ \\ \hline
    \end{tabular}
    \end{table}\\
    Разложим $\chi_{\pi_n}$ при помощи алгоритма разложения на неприводимые:
    \begin{equation}
        \chi_{\pi_n}=\sum\limits_{i=1}^5 a_i\chi^{(i)},\quad a_i=\braket{\chi^{(i)},\chi_{\pi_n}}
    \end{equation}
    \begin{equation}
        a_1=\frac{1}{24}(1\cdot(2n+1)+6\cdot1+3\cdot1+8\cdot0+6\cdot1)=\frac{n+8}{12}
    \end{equation}
    \begin{equation}
        a_2=\frac{1}{24}(1\cdot(2n+1)+6\cdot(-1)+3\cdot1+8\cdot0+6\cdot(-1))=\frac{n-4}{12}
    \end{equation}
    \begin{equation}
        a_3=\frac{1}{24}(1\cdot(6n+3)+6\cdot1+3\cdot(-1)+8\cdot0+6\cdot(-1))=\frac{n}{4}
    \end{equation}
    \begin{equation}
        a_4=\frac{1}{24}(1\cdot(6n+3)+6\cdot(-1)+3\cdot(-1)+8\cdot0+6\cdot1)=\frac{n}{4}
    \end{equation}
    \begin{equation}
        a_5=\frac{1}{24}(1\cdot(4n+2)+6\cdot0+3\cdot2+8\cdot0+6\cdot0)=\frac{n+2}{6}
    \end{equation}
    \begin{equation}
        \boxed{\chi_{\pi_n}=\chi^{(1)}\frac{n+8}{12}+\chi^{(2)}\frac{n-4}{12}+\chi^{(3)}\frac{n}{4}+\chi^{(4)}\frac{n}{4}+\chi^{(5)}\frac{n+2}{6}}
    \end{equation}
    
    \item $n=12k+5,k\in\mathbb{Z}$.
    \begin{table}[h!]
    \centering
    \begin{tabular}{|l|l|l|l|l|l|}
    \hline
     & $e$ $^1$ & $(a,b)$ $^6$ & $(a,b)(c,d)$ $^3$ & $(a,b,c)$ $^8$ & $(a,b,c,d)$ $^6$ \\ \hline
    $\chi_{\pi_n}$ & $2n+1$ & $-1$ & $-1$ & $-1$ & $1$ \\ \hline
    \end{tabular}
    \end{table}\\
    Разложим $\chi_{\pi_n}$ при помощи алгоритма разложения на неприводимые:
    \begin{equation}
        \chi_{\pi_n}=\sum\limits_{i=1}^5 a_i\chi^{(i)},\quad a_i=\braket{\chi^{(i)},\chi_{\pi_n}}
    \end{equation}
    \begin{equation}
        a_1=\frac{1}{24}(1\cdot(2n+1)+6\cdot(-1)+3\cdot(-1)+8\cdot(-1)+6\cdot1)=\frac{n-5}{12}
    \end{equation}
    \begin{equation}
        a_2=\frac{1}{24}(1\cdot(2n+1)+6\cdot1+3\cdot(-1)+8\cdot(-1)+6\cdot(-1))=\frac{n-5}{12}
    \end{equation}
    \begin{equation}
        a_3=\frac{1}{24}(1\cdot(6n+3)+6\cdot(-1)+3\cdot1+8\cdot0+6\cdot(-1))=\frac{n-1}{4}
    \end{equation}
    \begin{equation}
        a_4=\frac{1}{24}(1\cdot(6n+3)+6\cdot1+3\cdot1+8\cdot0+6\cdot1)=\frac{n+3}{4}
    \end{equation}
    \begin{equation}
        a_5=\frac{1}{24}(1\cdot(4n+2)+6\cdot0+3\cdot(-2)+8\cdot1+6\cdot0)=\frac{n+1}{6}
    \end{equation}
    \begin{equation}
        \boxed{\chi_{\pi_n}=(\chi^{(1)}+\chi^{(2)})\frac{n-5}{12}+\chi^{(3)}\frac{n-1}{4}+\chi^{(4)}\frac{n+3}{4}+\chi^{(5)}\frac{n+1}{6}}
    \end{equation}
    
    \item $n=12k+6,k\in\mathbb{Z}$.
    \begin{table}[h!]
    \centering
    \begin{tabular}{|l|l|l|l|l|l|}
    \hline
     & $e$ $^1$ & $(a,b)$ $^6$ & $(a,b)(c,d)$ $^3$ & $(a,b,c)$ $^8$ & $(a,b,c,d)$ $^6$ \\ \hline
    $\chi_{\pi_n}$ & $2n+1$ & $1$ & $1$ & $1$ & $-1$ \\ \hline
    \end{tabular}
    \end{table}\\
    Разложим $\chi_{\pi_n}$ при помощи алгоритма разложения на неприводимые:
    \begin{equation}
        \chi_{\pi_n}=\sum\limits_{i=1}^5 a_i\chi^{(i)},\quad a_i=\braket{\chi^{(i)},\chi_{\pi_n}}
    \end{equation}
    \begin{equation}
        a_1=\frac{1}{24}(1\cdot(2n+1)+6\cdot1+3\cdot1+8\cdot1+6\cdot(-1))=\frac{n+6}{12}
    \end{equation}
    \begin{equation}
        a_2=\frac{1}{24}(1\cdot(2n+1)+6\cdot(-1)+3\cdot1+8\cdot1+6\cdot1)=\frac{n+6}{12}
    \end{equation}
    \begin{equation}
        a_3=\frac{1}{24}(1\cdot(6n+3)+6\cdot1+3\cdot(-1)+8\cdot0+6\cdot1)=\frac{n+2}{4}
    \end{equation}
    \begin{equation}
        a_4=\frac{1}{24}(1\cdot(6n+3)+6\cdot(-1)+3\cdot(-1)+8\cdot0+6\cdot(-1))=\frac{n-2}{4}
    \end{equation}
    \begin{equation}
        a_5=\frac{1}{24}(1\cdot(4n+2)+6\cdot0+3\cdot2+8\cdot(-1)+6\cdot0)=\frac{n}{6}
    \end{equation}
    \begin{equation}
        \boxed{\chi_{\pi_n}=(\chi^{(1)}+\chi^{(2)})\frac{n+6}{12}+\chi^{(3)}\frac{n+2}{4}+\chi^{(4)}\frac{n-2}{4}+\chi^{(5)}\frac{n}{6}}
    \end{equation}
    
    \item $n=12k+7,k\in\mathbb{Z}$.
    \begin{table}[h!]
    \centering
    \begin{tabular}{|l|l|l|l|l|l|}
    \hline
     & $e$ $^1$ & $(a,b)$ $^6$ & $(a,b)(c,d)$ $^3$ & $(a,b,c)$ $^8$ & $(a,b,c,d)$ $^6$ \\ \hline
    $\chi_{\pi_n}$ & $2n+1$ & $-1$ & $-1$ & $0$ & $-1$ \\ \hline
    \end{tabular}
    \end{table}\\
    Разложим $\chi_{\pi_n}$ при помощи алгоритма разложения на неприводимые:
    \begin{equation}
        \chi_{\pi_n}=\sum\limits_{i=1}^5 a_i\chi^{(i)},\quad a_i=\braket{\chi^{(i)},\chi_{\pi_n}}
    \end{equation}
    \begin{equation}
        a_1=\frac{1}{24}(1\cdot(2n+1)+6\cdot(-1)+3\cdot(-1)+8\cdot0+6\cdot(-1))=\frac{n-7}{12}
    \end{equation}
    \begin{equation}
        a_2=\frac{1}{24}(1\cdot(2n+1)+6\cdot1+3\cdot(-1)+8\cdot0+6\cdot1)=\frac{n+5}{12}
    \end{equation}
    \begin{equation}
        a_3=\frac{1}{24}(1\cdot(6n+3)+6\cdot(-1)+3\cdot1+8\cdot0+6\cdot1)=\frac{n+1}{4}
    \end{equation}
    \begin{equation}
        a_4=\frac{1}{24}(1\cdot(6n+3)+6\cdot1+3\cdot1+8\cdot0+6\cdot(-1))=\frac{n+1}{4}
    \end{equation}
    \begin{equation}
        a_5=\frac{1}{24}(1\cdot(4n+2)+6\cdot0+3\cdot(-2)+8\cdot0+6\cdot0)=\frac{n-1}{6}
    \end{equation}
    \begin{equation}
        \boxed{\chi_{\pi_n}=\chi^{(1)}\frac{n-7}{12}+\chi^{(2)}\frac{n+5}{12}+(\chi^{(3)}+\chi^{(4)})\frac{n+1}{4}+\chi^{(5)}\frac{n-1}{6}}
    \end{equation}
    
    \item $n=12k+8,k\in\mathbb{Z}$.
    \begin{table}[h!]
    \centering
    \begin{tabular}{|l|l|l|l|l|l|}
    \hline
     & $e$ $^1$ & $(a,b)$ $^6$ & $(a,b)(c,d)$ $^3$ & $(a,b,c)$ $^8$ & $(a,b,c,d)$ $^6$ \\ \hline
    $\chi_{\pi_n}$ & $2n+1$ & $1$ & $1$ & $-1$ & $1$ \\ \hline
    \end{tabular}
    \end{table}\\
    Разложим $\chi_{\pi_n}$ при помощи алгоритма разложения на неприводимые:
    \begin{equation}
        \chi_{\pi_n}=\sum\limits_{i=1}^5 a_i\chi^{(i)},\quad a_i=\braket{\chi^{(i)},\chi_{\pi_n}}
    \end{equation}
    \begin{equation}
        a_1=\frac{1}{24}(1\cdot(2n+1)+6\cdot1+3\cdot1+8\cdot(-1)+6\cdot1)=\frac{n+4}{12}
    \end{equation}
    \begin{equation}
        a_2=\frac{1}{24}(1\cdot(2n+1)+6\cdot(-1)+3\cdot1+8\cdot(-1)+6\cdot(-1))=\frac{n-8}{12}
    \end{equation}
    \begin{equation}
        a_3=\frac{1}{24}(1\cdot(6n+3)+6\cdot1+3\cdot(-1)+8\cdot0+6\cdot(-1))=\frac{n}{4}
    \end{equation}
    \begin{equation}
        a_4=\frac{1}{24}(1\cdot(6n+3)+6\cdot(-1)+3\cdot(-1)+8\cdot0+6\cdot1)=\frac{n}{4}
    \end{equation}
    \begin{equation}
        a_5=\frac{1}{24}(1\cdot(4n+2)+6\cdot0+3\cdot2+8\cdot1+6\cdot0)=\frac{n+4}{6}
    \end{equation}
    \begin{equation}
        \boxed{\chi_{\pi_n}=\chi^{(1)}\frac{n+4}{12}+\chi^{(2)}\frac{n-8}{12}+\chi^{(3)}\frac{n}{4}+\chi^{(4)}\frac{n}{4}+\chi^{(5)}\frac{n+4}{6}}
    \end{equation}
    
    \item $n=12k+9,k\in\mathbb{Z}$.
    \begin{table}[h!]
    \centering
    \begin{tabular}{|l|l|l|l|l|l|}
    \hline
     & $e$ $^1$ & $(a,b)$ $^6$ & $(a,b)(c,d)$ $^3$ & $(a,b,c)$ $^8$ & $(a,b,c,d)$ $^6$ \\ \hline
    $\chi_{\pi_n}$ & $2n+1$ & $-1$ & $-1$ & $1$ & $1$ \\ \hline
    \end{tabular}
    \end{table}\\
    Разложим $\chi_{\pi_n}$ при помощи алгоритма разложения на неприводимые:
    \begin{equation}
        \chi_{\pi_n}=\sum\limits_{i=1}^5 a_i\chi^{(i)},\quad a_i=\braket{\chi^{(i)},\chi_{\pi_n}}
    \end{equation}
    \begin{equation}
        a_1=\frac{1}{24}(1\cdot(2n+1)+6\cdot(-1)+3\cdot(-1)+8\cdot1+6\cdot1)=\frac{n+3}{12}
    \end{equation}
    \begin{equation}
        a_2=\frac{1}{24}(1\cdot(2n+1)+6\cdot1+3\cdot(-1)+8\cdot1+6\cdot(-1))=\frac{n+3}{12}
    \end{equation}
    \begin{equation}
        a_3=\frac{1}{24}(1\cdot(6n+3)+6\cdot(-1)+3\cdot1+8\cdot0+6\cdot(-1))=\frac{n-1}{4}
    \end{equation}
    \begin{equation}
        a_4=\frac{1}{24}(1\cdot(6n+3)+6\cdot1+3\cdot1+8\cdot0+6\cdot1)=\frac{n+3}{4}
    \end{equation}
    \begin{equation}
        a_5=\frac{1}{24}(1\cdot(4n+2)+6\cdot0+3\cdot(-2)+8\cdot(-1)+6\cdot0)=\frac{n-3}{6}
    \end{equation}
    \begin{equation}
        \boxed{\chi_{\pi_n}=(\chi^{(1)}+\chi^{(2)})\frac{n+3}{12}+\chi^{(3)}\frac{n-1}{4}+\chi^{(4)}\frac{n+3}{4}+\chi^{(5)}\frac{n-3}{6}}
    \end{equation}
    
    \item $n=12k+10,k\in\mathbb{Z}$.
    \begin{table}[h!]
    \centering
    \begin{tabular}{|l|l|l|l|l|l|}
    \hline
     & $e$ $^1$ & $(a,b)$ $^6$ & $(a,b)(c,d)$ $^3$ & $(a,b,c)$ $^8$ & $(a,b,c,d)$ $^6$ \\ \hline
    $\chi_{\pi_n}$ & $2n+1$ & $1$ & $1$ & $0$ & $-1$ \\ \hline
    \end{tabular}
    \end{table}\\
    Разложим $\chi_{\pi_n}$ при помощи алгоритма разложения на неприводимые:
    \begin{equation}
        \chi_{\pi_n}=\sum\limits_{i=1}^5 a_i\chi^{(i)},\quad a_i=\braket{\chi^{(i)},\chi_{\pi_n}}
    \end{equation}
    \begin{equation}
        a_1=\frac{1}{24}(1\cdot(2n+1)+6\cdot1+3\cdot1+8\cdot0+6\cdot(-1))=\frac{n+2}{12}
    \end{equation}
    \begin{equation}
        a_2=\frac{1}{24}(1\cdot(2n+1)+6\cdot(-1)+3\cdot1+8\cdot0+6\cdot1)=\frac{n+2}{12}
    \end{equation}
    \begin{equation}
        a_3=\frac{1}{24}(1\cdot(6n+3)+6\cdot1+3\cdot(-1)+8\cdot0+6\cdot1)=\frac{n+2}{4}
    \end{equation}
    \begin{equation}
        a_4=\frac{1}{24}(1\cdot(6n+3)+6\cdot(-1)+3\cdot(-1)+8\cdot0+6\cdot(-1))=\frac{n-2}{4}
    \end{equation}
    \begin{equation}
        a_5=\frac{1}{24}(1\cdot(4n+2)+6\cdot0+3\cdot2+8\cdot0+6\cdot0)=\frac{n+2}{6}
    \end{equation}
    \begin{equation}
        \boxed{\chi_{\pi_n}=(\chi^{(1)}+\chi^{(2)})\frac{n+2}{12}+\chi^{(3)}\frac{n+2}{4}+\chi^{(4)}\frac{n-2}{4}+\chi^{(5)}\frac{n+2}{6}}
    \end{equation}
    
    \item $n=12k+11,k\in\mathbb{Z}$.
    \begin{table}[h!]
    \centering
    \begin{tabular}{|l|l|l|l|l|l|}
    \hline
     & $e$ $^1$ & $(a,b)$ $^6$ & $(a,b)(c,d)$ $^3$ & $(a,b,c)$ $^8$ & $(a,b,c,d)$ $^6$ \\ \hline
    $\chi_{\pi_n}$ & $2n+1$ & $-1$ & $-1$ & $-1$ & $-1$ \\ \hline
    \end{tabular}
    \end{table}\\
    Разложим $\chi_{\pi_n}$ при помощи алгоритма разложения на неприводимые:
    \begin{equation}
        \chi_{\pi_n}=\sum\limits_{i=1}^5 a_i\chi^{(i)},\quad a_i=\braket{\chi^{(i)},\chi_{\pi_n}}
    \end{equation}
    \begin{equation}
        a_1=\frac{1}{24}(1\cdot(2n+1)+6\cdot(-1)+3\cdot(-1)+8\cdot(-1)+6\cdot(-1))=\frac{n-11}{12}
    \end{equation}
    \begin{equation}
        a_2=\frac{1}{24}(1\cdot(2n+1)+6\cdot1+3\cdot(-1)+8\cdot(-1)+6\cdot1)=\frac{n+1}{12}
    \end{equation}
    \begin{equation}
        a_3=\frac{1}{24}(1\cdot(6n+3)+6\cdot(-1)+3\cdot1+8\cdot0+6\cdot1)=\frac{n+1}{4}
    \end{equation}
    \begin{equation}
        a_4=\frac{1}{24}(1\cdot(6n+3)+6\cdot1+3\cdot1+8\cdot0+6\cdot(-1))=\frac{n+1}{4}
    \end{equation}
    \begin{equation}
        a_5=\frac{1}{24}(1\cdot(4n+2)+6\cdot0+3\cdot(-2)+8\cdot1+6\cdot0)=\frac{n+1}{6}
    \end{equation}
    \begin{equation}
        \boxed{\chi_{\pi_n}=\chi^{(1)}\frac{n-11}{12}+\chi^{(2)}\frac{n+1}{12}+(\chi^{(3)}+\chi^{(4)})\frac{n+1}{4}+\chi^{(5)}\frac{n+1}{6}}
    \end{equation}
\end{enumerate}
\end{itemize}
\end{zad}
\section{Представления более общих групп Ли.}
\begin{zad}
\begin{itemize}
    \item[а)]
    \begin{equation}
        g=\left(
    \begin{array}{cccccc}
    e^{i\varphi_1} & 0 & 0\\
    0 & e^{-i\varphi_1+i\varphi_2} & 0\\
    0 & 0 & e^{-i\varphi_2}\\
    \end{array}
    \right)
    \end{equation}
    \begin{equation}
        \chi_V(g)=e^{i\varphi_1}+e^{-i\varphi_1+i\varphi_2}+e^{-i\varphi_2}
    \end{equation}
    Из лекции 12:
    \begin{equation}
        \chi_{\Lambda^2V}(g)=e^{-i\varphi_1}+e^{i\varphi_1-i\varphi_2}+e^{i\varphi_2}
    \end{equation}
    \begin{equation}
        \chi_{V\otimes\Lambda^2V}=\chi_V\chi_{\Lambda^2V}=(e^{i\varphi_1}+e^{-i\varphi_1+i\varphi_2}+e^{-i\varphi_2})(e^{-i\varphi_1}+e^{i\varphi_1-i\varphi_2}+e^{i\varphi_2})
    \end{equation}
    \begin{equation}
        \chi_{V\otimes\Lambda^2V}=3+e^{2i\varphi_1-i\varphi_2}+e^{i(\varphi_1+\varphi_2)}+e^{-2i\varphi_1+i\varphi_2}+e^{-i\varphi_1+2i\varphi_2}+e^{-i(\varphi_1+\varphi_2)}+e^{i\varphi_1-2i\varphi_2}
    \end{equation}
    \begin{equation}
        \boxed{\chi_{V\otimes\Lambda^2V}=3+2\cos(2\varphi_1-\varphi_2)+2\cos(\varphi_1-2\varphi_2)+2\cos(\varphi_1+\varphi_2)}
    \end{equation}
    \item[б)] Генераторами группы $SU(3)$ являются \textit{матрицы Гелл-Манна}:
    \begin{equation}
        \lambda^1=\left(
    \begin{array}{ccc}
    0 & 1 & 0\\
    1 & 0 & 0\\
    0 & 0 & 0\\
    \end{array}
    \right),\quad \lambda^2=\left(
    \begin{array}{ccc}
    0 & -i & 0\\
    i & 0 & 0\\
    0 & 0 & 0\\
    \end{array}
    \right),\quad \lambda^3=\left(
    \begin{array}{ccc}
    1 & 0 & 0\\
    0 & -1 & 0\\
    0 & 0 & 0\\
    \end{array}
    \right)
    \end{equation}
    \begin{equation}
        \lambda^4=\left(
    \begin{array}{ccc}
    0 & 0 & 1\\
    0 & 0 & 0\\
    1 & 0 & 0\\
    \end{array}
    \right),\quad \lambda^5=\left(
    \begin{array}{ccc}
    0 & 0 & -i\\
    0 & 0 & 0\\
    i & 0 & 0\\
    \end{array}
    \right),\quad \lambda^6=\left(
    \begin{array}{ccc}
    0 & 0 & 0\\
    0 & 0 & 1\\
    0 & 1 & 0\\
    \end{array}
    \right)
    \end{equation}
    \begin{equation}
        \lambda^7=\left(
    \begin{array}{ccc}
    0 & 0 & 0\\
    0 & 0 & -i\\
    0 & i & 0\\
    \end{array}
    \right),\quad\lambda^8=\frac{1}{\sqrt{3}}\left(
    \begin{array}{ccc}
    1 & 0 & 0\\
    0 & 1 & 0\\
    0 & 0 & -2\\
    \end{array}
    \right)
    \end{equation}
    \begin{equation*}
    g\lambda^1g^{-1}=\left(
    \begin{array}{ccc}
        0 & e^{i(2\varphi_1-\varphi_2)} & 0\\
        e^{i(-2\varphi_1+\varphi_2)} & 0 & 0\\
        0 & 0 & 0\\
    \end{array}
    \right),\quad g\lambda^2g^{-1}=\left(
    \begin{array}{ccc}
        0 & -ie^{i(2\varphi_1-\varphi_2)} & 0\\
        ie^{i(-2\varphi_1+\varphi_2)} & 0 & 0\\
        0 & 0 & 0\\
    \end{array}
    \right)
    \end{equation*}
    \begin{equation*}
    g\lambda^3g^{-1}=\left(
    \begin{array}{ccc}
        1 & 0 & 0\\
        0 & -1 & 0\\
        0 & 0 & 0\\
    \end{array}
    \right),\quad g\lambda^4g^{-1}=\left(
    \begin{array}{ccc}
        0 & 0 & e^{i(\varphi_1+\varphi_2)}\\
        0 & 0 & 0\\
        e^{-i(\varphi_1+\varphi_2)} & 0 & 0\\
    \end{array}
    \right)
    \end{equation*}
    \begin{equation*}
    g\lambda^5g^{-1}=\left(
    \begin{array}{ccc}
        0 & 0 & -ie^{i(\varphi_1+\varphi_2)}\\
        0 & 0 & 0\\
        ie^{-i(\varphi_1+\varphi_2)} & 0 & 0\\
    \end{array}
    \right),\quad g\lambda^6g^{-1}=\left(
    \begin{array}{ccc}
        0 & 0 & 0\\
        0 & 0 & e^{-i(\varphi_1-2\varphi_2)}\\
        0 & e^{i(\varphi_1-2\varphi_2)} & 0\\
    \end{array}
    \right)
    \end{equation*}
    \begin{equation*}
    g\lambda^7g^{-1}=\left(
    \begin{array}{ccc}
        0 & 0 & 0\\
        0 & 0 & -ie^{-i(\varphi_1-2\varphi_2)}\\
        0 & ie^{i(\varphi_1-2\varphi_2)} & 0\\
    \end{array}
    \right),\quad g\lambda^8g^{-1}=\frac{1}{\sqrt{3}}\left(
    \begin{array}{ccc}
        1 & 0 & 0\\
        0 & 1 & 0\\
        0 & 0 & -2\\
    \end{array}
    \right)
    \end{equation*}
    \begin{equation}
        g\lambda^1g^{-1}=\cos(2\varphi_1-\varphi_2)\lambda^1-\sin(2\varphi_1-\varphi_2)\lambda^2
    \end{equation}
    \begin{equation}
        g\lambda^2g^{-1}=\sin(2\varphi_1-\varphi_2)\lambda^1+\cos(2\varphi_1-\varphi_2)\lambda^2
    \end{equation}
    \begin{equation}
        g\lambda^3g^{-1}=\lambda^3
    \end{equation}
    \begin{equation}
        g\lambda^4g^{-1}=\cos(\varphi_1+\varphi_2)\lambda^4-\sin(\varphi_1+\varphi_2)\lambda^5
    \end{equation}
    \begin{equation}
        g\lambda^5g^{-1}=\sin(\varphi_1+\varphi_2)\lambda^4+\cos(\varphi_1+\varphi_2)\lambda^5
    \end{equation}
    \begin{equation}
        g\lambda^6g^{-1}=\cos(2\varphi_2-\varphi_1)\lambda^6-\sin(2\varphi_2-\varphi_1)\lambda^7
    \end{equation}
    \begin{equation}
        g\lambda^7g^{-1}=\sin(2\varphi_2-\varphi_1)\lambda^6+\cos(2\varphi_2-\varphi_1)\lambda^7
    \end{equation}
    \begin{equation}
        g\lambda^8g^{-1}=\lambda^8
    \end{equation}
    Присоединённое представление:
    \begin{equation}
        Ad_g=\tiny\left(
    \begin{array}{cccccccc}
    \cos(2\varphi_1-\varphi_2) & \sin(2\varphi_1-\varphi_2) & 0 & 0 & 0 & 0 & 0 & 0\\
    -\sin(2\varphi_1-\varphi_2) & \cos(2\varphi_1-\varphi_2) & 0 & 0 & 0 & 0 & 0 & 0\\
    0 & 0 & 1 & 0 & 0 & 0 & 0 & 0\\
    0 & 0 & 0 & \cos(\varphi_1+\varphi_2) & \sin(\varphi_1+\varphi_2) & 0 & 0 & 0\\
    0 & 0 & 0 & -\sin(\varphi_1+\varphi_2) & +\cos(\varphi_1+\varphi_2) & 0 & 0 & 0\\
    0 & 0 & 0 & 0 & 0 & \cos(2\varphi_2-\varphi_1) & \sin(2\varphi_2-\varphi_1) & 0\\
    0 & 0 & 0 & 0 & 0 & -\sin(2\varphi_2-\varphi_1) & \cos(2\varphi_2-\varphi_1) & 0\\
    0 & 0 & 0 & 0 & 0 & 0 & 0 & 1\\
    \end{array}
    \right)
    \end{equation}
    Характер присоединённого представления $SU(3)$:
    \begin{equation}
        \boxed{\chi_{Ad_g}=2(1+\cos(2\varphi_1-\varphi_2)+\cos(\varphi_1-2\varphi_2)+\cos(\varphi_1+\varphi_2))}
    \end{equation}
    \item[в)$^*$] Как видно из п. а и б ($\chi_\text{triv}=1$ -- характер тривиального):
    \begin{equation}
        \boxed{\chi_{V\otimes\Lambda^2V}=\chi_{Ad_g}+\chi_\text{triv}}
    \end{equation}
    \begin{zad}
    \begin{itemize}
        \item[а)] Базис в алгебре Ли $\mathfrak{so}(3,1)$:
        \begin{equation*}
            J_1=i\left(
    \begin{array}{cccc}
    0 & 0 & 0 & 0\\
    0 & 0 & 0 & 0\\
    0 & 0 & 0 & -1\\
    0 & 0 & 1 & 0\\
    \end{array}
    \right),\quad J_2=i\left(
    \begin{array}{cccc}
    0 & 0 & 0 & 0\\
    0 & 0 & 0 & 1\\
    0 & 0 & 0 & 0\\
    0 & -1 & 0 & 0\\
    \end{array}
    \right),\quad J_3=i\left(
    \begin{array}{cccc}
    0 & 0 & 0 & 0\\
    0 & 0 & -1 & 0\\
    0 & 1 & 0 & 0\\
    0 & 0 & 0 & 0\\
    \end{array}
    \right)
        \end{equation*}
        \begin{equation*}
            K_1=i\left(
    \begin{array}{cccc}
    0 & 1 & 0 & 0\\
    1 & 0 & 0 & 0\\
    0 & 0 & 0 & 0\\
    0 & 0 & 0 & 0\\
    \end{array}
    \right),\quad K_2=i\left(
    \begin{array}{cccc}
    0 & 0 & 1 & 0\\
    0 & 0 & 0 & 0\\
    1 & 0 & 0 & 0\\
    0 & 0 & 0 & 0\\
    \end{array}
    \right),\quad K_3=i\left(
    \begin{array}{cccc}
    0 & 0 & 0 & 1\\
    0 & 0 & 0 & 0\\
    0 & 0 & 0 & 0\\
    1 & 0 & 0 & 0\\
    \end{array}
    \right)
        \end{equation*}
    Найдём структурные константы в этом базисе:
    \begin{equation}
        [J_i,J_j]=\sum\limits_ka_{ij}^kJ_k,\quad [K_i,K_j]=\sum\limits_kb_{ij}^kJ_k,\quad [J_i,K_j]=\sum\limits_kb_{ij}^kK_k
    \end{equation}
    \begin{equation}
        [J_1,J_2]=J_1J_2-J_2J_1=\left(
    \begin{array}{cccc}
    0 & 0 & 0 & 0\\
    0 & 0 & 1 & 0\\
    0 & -1 & 0 & 0\\
    0 & 0 & 0 & 0\\
    \end{array}
    \right)=iJ_3
    \end{equation}
    \begin{equation}
        [J_1,J_3]=J_1J_3-J_3J_1=\left(
    \begin{array}{cccc}
    0 & 0 & 0 & 0\\
    0 & 0 & 0 & 1\\
    0 & 0 & 0 & 0\\
    0 & -1 & 0 & 0\\
    \end{array}
    \right)=-iJ_2
    \end{equation}
    \begin{equation}
        [J_2,J_3]=J_2J_3-J_3J_2=\left(
    \begin{array}{cccc}
    0 & 0 & 0 & 0\\
    0 & 0 & 0 & 0\\
    0 & 0 & 0 & -1\\
    0 & 0 & 1 & 0\\
    \end{array}
    \right)=iJ_1
    \end{equation}
    \begin{equation}
        \boxed{a_{ij}^k=i\epsilon_{ijk}}
    \end{equation}
    
    \begin{equation}
        [K_1,K_2]=K_1K_2-K_2K_1=\left(
    \begin{array}{cccc}
    0 & 0 & 0 & 1\\
    0 & 0 & 0 & 0\\
    0 & 0 & 0 & 0\\
    1 & 0 & 0 & 0\\
    \end{array}
    \right)=-iK_3
    \end{equation}
    \begin{equation}
        [K_1,K_3]=K_1K_3-K_3K_1=\left(
    \begin{array}{cccc}
    0 & 0 & -1 & 0\\
    0 & 0 & 0 & 0\\
    -1 & 0 & 0 & 0\\
    0 & 0 & 0 & 0\\
    \end{array}
    \right)=iK_2
    \end{equation}
    \begin{equation}
        [K_2,K_3]=K_2K_3-K_3K_2=\left(
    \begin{array}{cccc}
    0 & 1 & 0 & 0\\
    1 & 0 & 0 & 0\\
    0 & 0 & 0 & 0\\
    0 & 0 & 0 & 0\\
    \end{array}
    \right)=-iK_1
    \end{equation}
    \begin{equation}
        \boxed{b_{ij}^k=-i\epsilon_{ijk}}
    \end{equation}
    
    \begin{equation}
        [J_1,K_2]=J_1K_2-K_2J_1=\left(
    \begin{array}{cccc}
    0 & 0 & 0 & -1\\
    0 & 0 & 0 & 0\\
    0 & 0 & 0 & 0\\
    -1 & 0 & 0 & 0\\
    \end{array}
    \right)=iK_3
    \end{equation}
    \begin{equation}
        [J_1,K_3]=J_1K_3-K_3J_1=\left(
    \begin{array}{cccc}
    0 & 0 & 1 & 0\\
    0 & 0 & 0 & 0\\
    1 & 0 & 0 & 0\\
    0 & 0 & 0 & 0\\
    \end{array}
    \right)=-iK_2
    \end{equation}
    \begin{equation}
        [J_2,K_3]=J_2K_3-K_3J_2=\left(
    \begin{array}{cccc}
    0 & -1 & 0 & 0\\
    -1 & 0 & 0 & 0\\
    0 & 0 & 0 & 0\\
    0 & 0 & 0 & 0\\
    \end{array}
    \right)=iK_1
    \end{equation}
    \begin{equation}
        \boxed{c_{ij}^k=i\epsilon_{ijk}}
    \end{equation}
    \item[б)]
    \begin{predl}
    Алгебры $\mathfrak{so}(3,1)$ и $\mathfrak{sl}(2,\mathbb{C})$ изоморфны.
    \end{predl}
    \begin{proof}
        Для доказательства изоморфизма проверим совпадение структурных констант в $\mathfrak{so}(3,1)$ и $\mathfrak{sl}(2,\mathbb{C})$. Структурные константы в $\mathfrak{so}(3,1)$ (см. п. а):
        \begin{equation}
            [J_i,J_j]=i\epsilon_{ijk}J_k,\quad [K_i,K_j]=-i\epsilon_{ijk}K_k,\quad [J_i,K_j]=i\epsilon_{ijk}K_k
        \end{equation}
        Естественный базис в $\mathfrak{sl}(2,\mathbb{C})$: $e,h,f,ie,if,ih$, где
        \begin{equation}
            e=\left(
        \begin{array}{cc}
        0 & 1\\
        0 & 0\\
        \end{array}
        \right),\quad h=\left(
        \begin{array}{cc}
        1 & 0\\
        0 & -1\\
        \end{array}
        \right),\quad f=\left(
        \begin{array}{cc}
        0 & 0\\
        1 & 0\\
        \end{array}
        \right)
        \end{equation}
        Выберем в $\mathfrak{sl}(2,\mathbb{C})$ базис из
        \begin{equation*}
            k=\frac{1}{2}(e+f)=\left(
        \begin{array}{cc}
        0 & 1\\
        1 & 0\\
        \end{array}
        \right),\quad l=\frac{h}{2}=\frac{1}{2}\left(
        \begin{array}{cc}
        1 & 0\\
        0 & -1\\
        \end{array}
        \right),\quad m=\frac{1}{2}(e-f)=\frac{1}{2}\left(
        \begin{array}{cc}
        0 & 1\\
        -1 & 0\\
        \end{array}
        \right)
        \end{equation*}
        \begin{equation}
            [k,l]=-m,\quad [k,m]=-l,\quad [l,m]=k
        \end{equation}
        Соответствие между базисами: $k\rightarrow iJ_1$, $l\rightarrow iJ_2$, $m\rightarrow iJ_3$, $ik\rightarrow K_1$, $il\rightarrow K_2$, $im\rightarrow K_3$.
    \end{proof}
    \end{itemize}
    \end{zad}
    \begin{zad}
    $\gamma_1,\gamma_2,...,\gamma_N$ -- образующие, которые удовлетворяют соотношениям:
    \begin{equation}
        \gamma_a\gamma_b+\gamma_b\gamma_a=\delta_{a,b}
    \end{equation}
    \begin{equation}
        J_{ab}=\gamma_a\gamma_b,\quad a\neq b
    \end{equation}
    \begin{multline}
        [J_{ab},J_{cd}]=\gamma_a\gamma_b\gamma_c\gamma_d-\gamma_c\gamma_d\gamma_a\gamma_b=\gamma_a\gamma_b\gamma_c\gamma_d-\gamma_c(\delta_{ad}-\gamma_a\gamma_d)\gamma_b=\gamma_a\gamma_b\gamma_c\gamma_d-\gamma_c\delta_{ad}\gamma_b+\\+\gamma_c\gamma_a(\delta_{bd}-\gamma_b\gamma_d)=(\gamma_a\gamma_b\gamma_c-\gamma_c\gamma_a\gamma_b)\gamma_d+J_{ca}\delta_{bd}-\gamma_c\gamma_b\delta_{ad}=(\gamma_a\gamma_b\gamma_c-(\delta_{ca}-\gamma_a\gamma_c)\gamma_b)\gamma_d+\\+J_{ca}\delta_{bd}+J_{bc}\delta_{ad}
    \end{multline}
    \begin{equation}
        \boxed{[J_{ab},J_{cd}]=J_{ad}\delta_{bc}+J_{db}\delta_{ac}+J_{ca}\delta_{bd}+J_{bc}\delta_{ad}}
    \end{equation}
    \end{zad}
    \begin{zad}
    \begin{itemize}
        \item[а)] $V$ -- четырёхмерное представление алгебры $\mathfrak{so}(4)$. $\mathfrak{so}(4)\simeq \mathfrak{su}(2)\oplus\mathfrak{su}(2)$, значит
        \begin{equation}
            \chi_V=\chi_{\mathfrak{su}(2)}+\chi_{\mathfrak{su}(2)}
        \end{equation}
        \begin{equation}
            \chi_V=e^{i\varphi_1}+e^{-i\varphi_1}+e^{i\varphi_2}+e^{-i\varphi_2}
        \end{equation}
        \begin{equation}
            \boxed{\chi_V=2\cos\varphi_1+2\cos\varphi_2}
        \end{equation}
    \end{itemize}
    \end{zad}
\end{itemize}
\end{zad}
\end{document} 
