\documentclass[12pt]{article}

% report, book
%  Русский язык

%\usepackage{bookmark}

\usepackage[T2A]{fontenc}			% кодировка
\usepackage[utf8]{inputenc}			% кодировка исходного текста
\usepackage[english,russian]{babel}	% локализация и переносы
\usepackage[title,toc,page,header]{appendix}
\usepackage{amsfonts}
\usepackage{hyperref,bookmark}
\usepackage{bm}


% Математика
\usepackage{amsmath,amsfonts,amssymb,amsthm,mathtools} 
%%% Дополнительная работа с математикой
%\usepackage{icomma} % "Умная" запятая: $0,2$ --- число, $0, 2$ --- перечисление

\usepackage{cancel}%зачёркивание
\usepackage{braket}
%% Шрифты
\usepackage{euscript}	 % Шрифт Евклид
\usepackage{mathrsfs} % Красивый матшрифт


\usepackage[left=2cm,right=2cm,top=1cm,bottom=2cm,bindingoffset=0cm]{geometry}
\usepackage{wasysym}
\usepackage{tikz-cd} 


\usepackage{etoolbox}
\usepackage{dynkin-diagrams}
\def\row#1/#2!{#1_{\IfStrEq{#2}{}{n}{#2}} & \dynkin{#1}{#2}\\}
\newcommand{\tble}[1]{
   \renewcommand*\do[1]{\row##1!}
   \[
      \begin{array}{ll}\docsvlist{#1}\end{array}
   \]
}
%\tble{A/{},B/{},C/{},D/{},E/6,E/7,E/8,F/4,G/2}


%размеры
\renewcommand{\appendixtocname}{Приложения}
\renewcommand{\appendixpagename}{Приложения}
\renewcommand{\appendixname}{Приложение}
\makeatletter
\let\oriAlph\Alph
\let\orialph\alph
\renewcommand{\@resets@pp}{\par
  \@ppsavesec
  \stepcounter{@pps}
  \setcounter{subsection}{0}%
  \if@chapter@pp
    \setcounter{chapter}{0}%
    \renewcommand\@chapapp{\appendixname}%
    \renewcommand\thechapter{\@Alph\c@chapter}%
  \else
    \setcounter{subsubsection}{0}%
    \renewcommand\thesubsection{\@Alph\c@subsection}%
  \fi
  \if@pphyper
    \if@chapter@pp
      \renewcommand{\theHchapter}{\theH@pps.\oriAlph{chapter}}%
    \else
      \renewcommand{\theHsubsection}{\theH@pps.\oriAlph{subsection}}%
    \fi
    \def\Hy@chapapp{appendix}%
  \fi
  \restoreapp
}
\makeatother
\newtheorem{theorem}{Theorem}[]
\newtheorem{prop}[theorem]{Proposition}
\newtheorem{sled}[theorem]{Следствие}

\theoremstyle{definition}
\newtheorem{zad}{Задача}[section]
\newtheorem{upr}[zad]{Упражнение}
\newtheorem{defin}{Definition}[]

\title{Problem solutions\\
EP "Quantum field theory, string theory and\\
mathematical physics"\\[2cm]
Group-theoretical approach in integrable systems\\ (I.A. Sechin, М.А. Vasilev)}
\author{Andrew Kotsevich, B02-920s}
\date{7 semester, 2022}


\begin{document}
\maketitle
\newpage
\tableofcontents
\newpage
\section*{Theoretical minimum}
\subsection*{Poisson manifold, symplectic form, integrable systems and Lax pair.}
\begin{itemize}
    \item Lagrangian mechanics:
    \begin{equation}
        S=\int L(q,\dot{q},t)dt,\quad \delta S=0
    \end{equation}
    Equations of motion:
    \begin{equation}
        \frac{d}{dt}\frac{\partial L}{\partial\dot{q}_i}-\frac{\partial L}{\partial q_i}=0
    \end{equation}
    \item Hamiltonian mechanics:
    \begin{equation}
        p_i=\frac{\partial L}{\partial\dot{q}_i},\quad H(p,q,t)=\sum\limits_{i=1}^np_i\dot{q}_i-L(\dot{q}(q,p,t),q,t)
    \end{equation}
    Equations of motion:
    \begin{equation}
        \begin{cases}
            \dot{p}_i=-\frac{\partial H}{\partial q_i},\\
            \dot{q}_i=\frac{\partial H}{\partial p_i}.
        \end{cases}
    \end{equation}
\end{itemize}
\begin{defin}
    \textit{Poisson bracket} $\{\bm{\cdot},\bm{\cdot}\}$ is a map $C^\infty(M)\times C^\infty(M)\rightarrow C^\infty(M)$, which satisfies
    \begin{enumerate}
        \item Anticommutativity: $\{f,g\}=-\{g,f\}$.
        \item Bilinearity: $\{\alpha f+\beta g,h\}=\alpha\{f,h\}+\beta\{g,h\}$, $\{f,\alpha g+\beta h\}=\alpha\{f,g\}+\beta\{f,h\}$.
        \item Jacobi identity: $\{\{f,g\},h\}+\{\{h,f\},g\}+\{\{g,h\},f\}=0$.
        \item Leibnitz rule: $\{fg,h\}=f\{g,h\}+\{f,h\}g$.
    \end{enumerate}
\end{defin}
\begin{defin}
    Smooth manifold $M$, on which the Poisson bracket is given, is called \textit{a Poisson manifold}.
\end{defin}
In canonical coordinates $(q_i,p_i)$:
\begin{equation}
    \{f,g\}=\sum\limits_{i=1}^n\left(\frac{\partial f}{\partial p_i}\frac{\partial g}{\partial q_i}-\frac{\partial f}{\partial q_i}\frac{\partial g}{\partial p_i}\right)
\end{equation}
\begin{equation}
    \{p_i,q_j\}=\delta_{ij},\quad\{q_i,q_j\}=\{p_i,p_j\}=0
\end{equation}
\begin{equation}
    \frac{df(p,q,t)}{dt}=\frac{\partial f}{\partial p}\frac{dp}{dt}+\frac{\partial f}{\partial q}\frac{dq}{dt}+\frac{\partial f}{\partial t}=-\frac{\partial f}{\partial p}\frac{\partial H}{\partial q}+\frac{\partial f}{\partial q}\frac{\partial H}{\partial p}+\frac{\partial f}{\partial t}=\{H,f\}+\frac{\partial f}{\partial t}
\end{equation}
\begin{equation}
    \dot{f}(p,q)=\{H,f\}
\end{equation}
\begin{defin}
    \textit{Symplectic form} is a differential form $\omega\in\Omega^2(M)$, such that
    \begin{itemize}
        \item $\omega$ is closed $d\omega=0$.
        \item $\omega$ is non-degenerate in every point of $M$ ($\forall x\in M\;\forall\xi\neq0\hookrightarrow\exists\eta:\omega(\xi,\eta)\neq0$).
    \end{itemize}
\end{defin}
\begin{defin}
    \textit{Poisson bracket} $\{\bm{\cdot},\bm{\cdot}\}$ on $(M,\omega)$ is a bilinear operation on differentiable functions, such that
    \begin{equation}
        \{f,g\}=\omega(v_f,v_g)
    \end{equation}
\end{defin}
\begin{defin}
    Integrals of motion $F_i$ and $F_j$ are \textit{integrals in involution}, if $\{F_i,F_j\}$.
\end{defin}
\begin{defin}
    \textit{Integrable hamiltonian system} on M ($\dim=2n$) is a collection of $n$ independent integrals of motion in involution.
\end{defin}
\textbf{Examples:}
\begin{enumerate}
    \item Oscillator $H=\frac{p^2}{2m}+\frac{m\omega^2q^2}{2}$.
    \begin{equation}
        \begin{cases}
            \dot{p}=-m\omega^2q,\\
            \dot{q}=\frac{p}{m}.
        \end{cases}\rightarrow\begin{cases}
            \ddot{q}+\omega^2q=0,\\
            \dot{q}=\frac{p}{m}.
        \end{cases}\rightarrow\begin{cases}
            q(t)=A\sin\omega t+B\cos\omega t,\\
            p(t)=m\omega(A\cos\omega t-B\sin\omega t).
        \end{cases}
    \end{equation}
    \item Central field (Kepler problem) $H=\sum\limits_{i=1}^3\frac{p_i^2}{2m}+V(r)$.
    \begin{equation}
        \begin{cases}
            \dot{p}_i=-\frac{\partial V}{\partial q_i},\\
            \dot{q}_i=\frac{p_i}{m}.
        \end{cases}
    \end{equation}
    Spherical coordinates:
    \begin{equation}
    \begin{cases}
        x_1=r\sin\theta\cos\varphi,\\
        x_2=r\sin\theta\sin\varphi,\\
        x_3=r\cos\theta.
    \end{cases} 
    \end{equation}
    Angular momentum $J_{ij}=q_ip_j-q_jp_i$. Integrals of motion:
    \begin{equation}
        \begin{cases}
            H=\frac{1}{2}\left(p_r^2+\frac{p_\theta^2}{r^2}+\frac{p_\varphi^2}{r^2\sin^2\theta}\right)+V(r),\\
            J^2=J_{12}^2+J_{13}^2+J_{23}^2=p_\theta^2+\frac{p_\varphi^2}{\sin^2\theta},\\
            J_{12}=p_\varphi.
        \end{cases}
    \end{equation}
\end{enumerate}
%ДОПИСАТЬ!
\begin{theorem}[Liouville]
    The solution for hamiltonian integrable system is obtained by "quadrature".
    \begin{proof}[Proof]
        %ДОПИСАТЬ!
    \end{proof}
\end{theorem}
Suppose $M_f$ is connected and compact, $M_f\simeq T^n=S^1\times...\times S^1$.

Action variables $I_j=\frac{1}{2\pi}\int\limits_{C_j}\alpha$, angle $\theta_k=\frac{\partial S}{\partial I_k}$.
\begin{multline}
    \int\limits_{C_j}d\theta_k=\frac{\partial}{\partial I_k}\int\limits_{C_j}dS=\frac{\partial}{\partial I_k}\int\limits_{C_j}\sum\limits_{i=1}^n\left(\frac{\partial S}{\partial q_i}dq_i+\frac{\partial S}{\partial I_i}dI_i\right)=\frac{\partial}{\partial I_k}\int\limits_{C_j}\sum\limits_{i=1}^n\frac{\partial S}{\partial q_i}dq_i=\\=\frac{\partial}{\partial I_k}\int\limits_{C_j}\alpha=2\pi\delta_{jk}
\end{multline}
How to find such $n$ integrals? There isn't algorithm, but suppose that equations of motion could be written in form
\begin{equation}
    \dot{L}=[L,M],
\end{equation}
where $L,M\in\text{Mat}_{n\times n}$ -- \textit{Lax pair}.
\begin{prop}
    If equations of motion are $\dot{L}=[L,M]$, then integrals $I_k=\frac{1}{k}\text{Tr}L^k$ are conserved.
    \begin{proof}[Proof.]
        \begin{multline}
            \frac{d}{dt}I_k=\frac{1}{k}\text{Tr}(\dot{L}L^{k-1}+...+L^{k-1}\dot{L}_k)=\text{Tr}(\dot{L}L^{k-1})=\text{Tr}(LML^{k-1}-ML^k)=\\=\text{Tr}(ML^k-ML^k)=0
        \end{multline}
    \end{proof}
\end{prop}
\textbf{Example:}\\
Calogero-Moser system of interacting particles on a line
\begin{equation}
    H=\frac{1}{2}\sum\limits_{i=1}^np_i^2-\frac{\nu^2}{2}\sum\limits_{i\neq j}\frac{1}{(q_i-q_j)^2}
\end{equation}
Poisson brackets are canonical. Equations of motion:
\begin{equation}
    \begin{cases}
        \dot{p}_i=-2\nu^2\sum\limits_{i\neq j}\frac{1}{(q_i-q_j)^2},\\
        \dot{q}_i=p_i
    \end{cases}
\end{equation}
Lax matrices can be chosen in the form:
\begin{equation}
    L_{ii}=p_i,\quad L_{ij}=\frac{\nu}{q_i-q_j},\quad i\neq j
\end{equation}
\begin{equation}
    M_{ii}=-\nu\sum\limits_{k\neq i}\frac{1}{(q_i-q_k)^2},\quad M_{ij}=-\frac{\nu}{(q_i-q_j)^2},\quad i\neq j
\end{equation}
So Calogero system has additional integrals of motion
\begin{equation}
    \text{Tr} L=\sum\limits_ip_i=P,\quad\frac{1}{2}\text{Tr}L^2=\frac{1}{2}\sum\limits_{i=1}^np_i^2-\frac{\nu^2}{2}\sum\limits_{i\neq j}\frac{1}{(q_i-q_j)^2}=H
\end{equation}
\begin{equation}
    \frac{1}{3}\text{Tr}L^3=\frac{1}{3}\sum\limits_{i=1}^np_i^3-\nu^2\sum\limits_{i\neq j}\frac{p_i}{(q_i-q_j)^2}
\end{equation}
The last integral is nontrivial. How to construct such Lax representations and understend if a system is integrable?\\
Idea: use symmetries to get integrals of motion.
\subsection*{Symplectic geometry, Hamiltonian approach to symmetry.}
\begin{defin}
    \textit{Symplectic manifold} is a pair $(M,\omega)$, such that
    \begin{itemize}
        \item $M$ -- smooth manifold.
        \item $\omega\in\Omega^2(M)$ -- symplectic form.
    \end{itemize}
\end{defin}
Dimension of symplectic manifold is even. Symplectic manifold is Poisson, but not vice versa.
\textbf{Examples}:
%WRITE!!!

Let $M$ be a Poisson manifold, then from bilinearity and Leibntz rule in local coordinates
\begin{equation}
    \{f,g\}(x)=\sum\limits_{i,j}\pi_{ij}(x)\frac{\partial f}{\partial x_i}\frac{\partial g}{\partial x_j}
\end{equation}
\begin{itemize}
    \item Anticommutativity: $\pi_{ij}(x)=-\pi_{ji}(x)$.
    \item Jacobi identity: $\pi_{ik}(x)\frac{\partial}{\partial x_k}\pi_{jl}(x)+\pi_{lk}(x)\frac{\partial}{\partial x_k}\pi_{ij}(x)+\pi_{jk}(x)\frac{\partial}{\partial x_k}\pi_{li}(x)=0$.
\end{itemize}
Assume $\pi_{ij}$ be an invertible matrix. One can define a symplectic form $\omega=-\sum\limits_{i\neq j}(\pi^{-1})_{ij}dx_i\wedge dx_j$. However, if the Poisson brackets have nontrivial kernel, i.e. there exists a function $f$: $\{f,\cdot\}=0$, then this Poisson manifold isn't symplectic.\\
One can fix the levels of all functions in the kernel of the Poisson brackets and define symplectic structures on these level manifolds called \textit{the symplectic leaves.}\\
Consider canonical symplectic structure on a cotangent bundle $T^*M$. Let $\pi:T^*M\rightarrow M$ be a projection map
\begin{equation}
    \pi(x,\beta)=x,\quad x\in M,\quad\beta\in T^*_xM
\end{equation}
Choose a point $x\in M$ and a chart $U\subset M$: $x\in U$. Choose local coordinates $q_1,...,q_n(U)$.\\ $(dq_1)_x$, ..., $(dq_n)_x$ -- basis in $T^*_xM$, then any $\beta\in T^*_xM$ has a form
\begin{equation}
    \beta=\sum\limits_{i=1}^np_i(x,\beta)(dq_i)_x
\end{equation}
So, $p_1,...,p_n, q_1, ..., q_n$ -- basis in $\pi^{-1}(U)$. Using this coordinates, one can write Liouville 1-form %ПОДУМАТЬ, НУЖНО ЛИ ПИСАТЬ КУРСИВОМ ФОРМУ ЛИУВИЛЛЯ
\begin{equation}
    \alpha=\sum\limits_{i=1}^np_idq_i\rightarrow\omega=d\alpha,\quad d\omega=d^2\alpha=0
\end{equation}
\begin{theorem}[Darboux]
    Let $(M,\omega)$ -- symplectic manifold and $x\in M$, then one can introduce locally around $x\in M$ a system of local coordinates $(p_i,q_i)$, such that $\omega=\sum\limits_{i=1}^ndp_i\wedge dq_i$.
\end{theorem}
Symplectic form -- non-degenerate 2-form, so there is $1:1$ mapping $\Omega^1(M)\leftrightarrow\text{Vect}(M)$ (vector fields on $M$). Contaction operation
\begin{equation}
    \lambda=(\omega,\cdot)=i_v\omega
\end{equation}
Vector field $v\in\text{Vect}(M)$ defines a local one-parameter group of diffeomorphims
\begin{equation}
    \exp(vt):\mathbb{R}\times M\rightarrow M,\quad t\in\mathbb{R},\quad x\in M
\end{equation}
\begin{equation}
    \begin{cases}
        \exp(v0)(x)=x,\\
        \frac{d}{dt}(\exp(vt)(x))=v(\exp(vt)(x))
    \end{cases}
\end{equation}
Group properties:
\begin{itemize}
    \item $\exp(v(t+s))=\exp(vt)\exp(vs)$.
    \item $\exp(v(-t))=(\exp vt)^{-1}$.
\end{itemize}
\begin{defin}
    Let $v\in\text{Vect}(M)$, then $\forall\lambda\in\Omega^{\bullet}(M)$ \textit{Lie derivative} $L_v$ is
    \begin{equation}
        L_v\lambda=\frac{d}{dt}\left(\exp(vt)_*\lambda\right)|_{t=0}
    \end{equation}
\end{defin}
Properies of the Lie derivative:
\begin{itemize}
    \item Cartan formula: $L_v=di_v+i_vd$.
    \item $L_{[v,u]}=[L_v,L_u]$.
    \item $[L_v,i_u]=i_{[v,u]}$.
    \item $L_v\omega(v_1,...,v_k)=(L_v\omega)(v_1,...,v_k)+\sum\limits_{i=1}^k\omega(v_1,...,[v,v_i],...,v_k)$.
\end{itemize}
\subsection*{Hamiltonian and symplectic vector fields. Lie groups acting on manifolds}
\begin{defin}
    Let ($M,\omega$) -- a symplectic manifold. Vector field $v_H$ is \textit{hamiltonian} if
    \begin{equation}
        i_{v_H}\omega=-dH
    \end{equation}
\end{defin}
\begin{defin}
    Vector field $v$ is \textit{symplectic} if
    \begin{equation}
        L_v\omega=0
    \end{equation}
    Using Cartan formula
    \begin{equation}
        L_v\omega=di_v\omega+i_vd\omega=d(i_v\omega)=0
    \end{equation}
    So, $i_v\omega$ is closed form.
\end{defin}
\begin{prop}
    Any hamiltonian vector field is symplectic.
    \begin{proof}[Proof.]
        Let $v_f$ is hamiltonian field, then
        \begin{equation}
            i_{v_f}\omega=-df
        \end{equation}
        $i_{v_f}\omega$ is exact form, then $i_{v_f}\omega$ is closed form
        \begin{equation}
            di_{v_f}\omega=-d^2f=0
        \end{equation}
    \end{proof}
\end{prop}
Example of symplectic but not hamiltonian vector field:\\
Symplectic manifold ($M,\omega$): $M=T^2=S^1\times S^1$, $\omega=d\varphi_1\wedge d\varphi_2$.
Symplectic vector field: $v=\frac{\partial}{\partial\varphi_1}$.
\begin{equation}
    i_v\omega=d\varphi_2\rightarrow d(i_v\omega)=0
\end{equation}
$\varphi_2$ isn't a function on $M$, so $H\neq\varphi_2$ and $v$ isn't hamiltonian vector field.\\
If $H^1(M)=0$, then any symplectic vector field is hamiltonian.
\begin{prop}
    If $v$,$u$ are symplectic vector fields, then their commutator $[v,u]$ is a Hamiltonian vector with hamiltonian $\omega(v,u)$.
    \begin{proof}[Proof.]
        \begin{equation}
            i_{[v,u]}\omega=L_vi_u\omega-i_uL_v\omega=(di_v+i_vd)i_u\omega=d(i_vi_u\omega)+i_vd(i_u\omega)=d(\omega(u,v))=-d(\omega(v,u))
        \end{equation}
        \begin{equation}
            i_{[v,u]}\omega=-dH,\quad H=\omega(v,u)
        \end{equation}
    \end{proof}
\end{prop}
Therefore, if $f,g\in\Omega^0(M)$ and $v_f,v_g$ -- coresponding vector fields then $[v_f,v_g]=v_{\{f,g\}}$.\\
Properties of symplectic vector field:
\begin{itemize}
    \item $\omega([v_1,v_2],v_3)+\omega([v_2,v_3],v_1)+\omega([v_3,v_1],v_2)=0$.
    \item $L_{v_1}\omega(v_2,v_3)+L_{v_2}\omega(v_3,v_1)+L_{v_3}\omega(v_1,v_2)=0$.
    \item $\omega([v_1,v_2],v_3)=L_{v_3}\omega(v_1,v_2)$.
\end{itemize}
Let $M$ is a smooth manifold, then $T^*M$ is symplectic. Let $v$ -- a vector field on $M$, then there exists a unique vector field $\Tilde{v}$ on $T^*M$, which lifts the flow of $v$. This field is Hamiltonian with
\begin{equation}
    H=i_{\tilde{v}}\alpha=\sum p_iv_i(q)
\end{equation}
Consider Lie groups acting on manifolds. Let $G$ be a Lie group and $M$ -- smooth manifold. Action:
\begin{equation}
    .:G\times M\rightarrow M,\quad(g,x)\rightarrow g.x
\end{equation}
Let $\mathfrak{g}=\text{Lie}(G)$ -- Lie algebra of group G.\\
Consider an element $\xi\in\mathfrak{g}$ and one-parametric subgroup in $G$, generated by this element $\{e^{\xi t},t\in\mathbb{R}\}$. This allows to construct \textit{a fundamental vector field} on $M$:
\begin{equation}
    v_\xi(x)=\frac{d}{dt}(e^{\xi t}.x)|_{t=0}
\end{equation}
Any Lie group can act on itself:
\begin{itemize}
    \item Left action (left multiplication):
    \begin{equation}
        (g,h)\rightarrow g.h=gh
    \end{equation}
    \item Right action (right multiplication):
    \begin{equation}
        (g,h)\rightarrow g.h=hg^{-1}
    \end{equation}
    \item Conjugation:
    \begin{equation}
        (g,h)\rightarrow g.h=ghg^{-1}
    \end{equation}
\end{itemize}
\begin{defin}
    The derivative at the identity element of $G$ gives an invertible linear map
    \begin{equation}
        \text{Ad}_g:\mathfrak{g}\rightarrow\mathfrak{g},\quad\text{Ad}_g(\xi)=\frac{d}{dt}(ge^{\xi t}g^{-1})|_{t=0}
    \end{equation}
    and defines \textit{the adjoint representation}
    \begin{equation}
        \text{Ad}:G\rightarrow\text{End}(\mathfrak{g}),\quad\text{Ad}(g)=\text{Ad}_g
    \end{equation}
\end{defin}
\begin{defin}
    One can also define \textit{the coadjoint representation} of $G$ $\text{Ad}^*$ on dual to its Lie algebra $\mathfrak{g}^*$:
    \begin{equation}
        \text{Ad}^*_g:\mathfrak{g}^*\rightarrow\mathfrak{g}^*,\quad\braket{\text{Ad}^*_g(\varphi),\xi}=\braket{\varphi,\text{Ad}_{g^{-1}}(\xi)}
    \end{equation}
    $g^{-1}$ here to have homomorphisms:
    \begin{equation}
        \text{Ad}^*_g\text{Ad}^*_h=\text{Ad}^*_{gh},\quad \text{Ad}_g\text{Ad}_h=\text{Ad}_{gh}
    \end{equation}
    \begin{equation}
        \text{Ad}^*:G\rightarrow\text{End}(\mathfrak{g}^*),\quad\text{Ad}^*(g)=\text{Ad}^*_g
    \end{equation}
\end{defin}
\begin{defin}
    \textit{Adjoint and coadjoint representations of Lie algebra} $\mathfrak{g}$ can be defined as the infinitesimal versions:
    \begin{equation}
        \text{ad}_\xi:\mathfrak{g}\rightarrow\mathfrak{g},\quad\text{ad}_\xi(\eta)=[\xi,\eta]
    \end{equation}
    \begin{equation}
        \text{ad}:\mathfrak{g}\rightarrow\text{End}(\mathfrak{g}),\quad\text{ad}(\xi)=\text{ad}_\xi
    \end{equation}
    \begin{equation}
        \text{ad}^*_\xi:\mathfrak{g}^*\rightarrow\mathfrak{g}^*,\quad\braket{\text{ad}^*_\xi(\varphi),\eta}=\braket{\varphi,-\text{ad}_{\xi}(\eta)}=-\braket{\varphi,[\xi,\eta]}
    \end{equation}
\end{defin}
These operations are also homomorphisms:
\begin{equation}
    [\text{ad}_\xi,\text{ad}_\eta]=\text{ad}_{[\xi,\eta]},\quad[\text{ad}^*_\xi,\text{ad}^*_\eta]=\text{ad}^*_{[\xi,\eta]}
\end{equation}
Consider a Lie algebra $\mathfrak{g}$ and its dual $\mathfrak{g}^*$. $(\mathfrak{g}^*)^*=\mathfrak{g}$, so linear functions on $\mathfrak{g}^*$ are elements of $\mathfrak{g}$ and one can naturally define Poisson brackets on $\mathfrak{g}^*$:
\begin{equation}
    \{f_\xi,f_\eta\}(\varphi)=\braket{\varphi,[\xi,\eta]}
\end{equation}
Also we should claim Leibniz rule and define Poisson bracket for polynomials on $\mathfrak{g}^*$. Thus, $\mathfrak{g}^*$ is a Poisson manifold.\\
For two functions $f,g:\mathfrak{g}^*\rightarrow\mathbb{R}$
\begin{equation}
    \{f,g\}(\varphi)=\braket{\varphi,[df,dg]}
\end{equation}
\textit{Example:} $\mathfrak{g}=\mathfrak{so}(3)$.\\
Commutators:
\begin{equation}
    [S_i,S_j]=\epsilon_{ijk}S_k
\end{equation}
Poisson brackets on $\mathfrak{so}(3)^*\simeq\mathbb{R}^3$:
\begin{equation}
    \{S_i,S_j\}=\epsilon_{ijk}S_k
\end{equation}
$\mathfrak{g}^*$ isn't a symplectic manifold, Poisson bracket is degenerate.
\begin{equation}
    C=S_1^2+S_2^2+S_3^2,\quad\{C,S_i\}=0
\end{equation}
\begin{prop}
    Kernel of the Poisson bracket is the set of $\text{Ad}^*$-invariant functions
    \begin{equation}
        f:\mathfrak{g}^*\rightarrow\mathbb{R},\quad f(\text{Ad}^*_g(\varphi))=f(\varphi),\quad\forall g\in G,\varphi\in\mathfrak{g}^*
    \end{equation}
    \begin{proof}[Proof.]
        Let $g=e^{\xi t}$, then
        \begin{equation}
            \text{Ad}^*_g(\varphi)=\varphi+t\text{ad}^*_\xi(\varphi),\quad t\rightarrow 0
        \end{equation}
        \begin{equation}
            f(\text{Ad}^*_g(\varphi))=f(\varphi)+t\braket{\text{ad}^*_\xi(\varphi),df}\rightarrow\braket{\text{ad}^*_\xi(\varphi),df}=0
        \end{equation}
        \begin{equation}
            \braket{\text{ad}^*_\xi(\varphi),df}=-\braket{\varphi,[\xi,df]}=0
        \end{equation}
        Then for all linear functions $f$ on $\mathfrak{g}^*$
        \begin{equation}
            \{f,\cdot\}=0
        \end{equation}
    \end{proof}
\end{prop}    
In order to construct a symplectic manifold, one needs to fix these $\text{Ad}^*$-invariant functions. Consider a coadjoint orbit of $G$ ($\text{Ad}^*$-invariant functions are constants on the coadjoint orbits).\\
Coadjoint orbit of an element $\varphi\in\mathfrak{g}^*:\mathcal{O}_\varphi\equiv\text{Ad}^*_G=\{\text{Ad}^*_g(\varphi)|g\in G\}$.
\subsection*{Integrable systems related to semisimple Lie algebras}
\begin{defin}
    \textit{Lie algebra} $\mathfrak{g}$ -- a vector space with the commutation operation $[\cdot,\cdot]:\mathfrak{g}\times\mathfrak{g}\rightarrow\mathfrak{g}:\forall\xi,\eta,\lambda\in\mathfrak{g}$ $\forall a,b\in\mathbb{C}\hookrightarrow$
    \begin{itemize}
        \item linear: $[a\xi+b\eta,\lambda]=a[\xi,\lambda]+b[\eta,\lambda]$.
        \item skew-symmetric: $[\xi,\eta]=-[\eta,\xi]$.
        \item Jacobi identity: $[\xi,[\eta,\lambda]]+[\eta,[\lambda,\xi]]+[\lambda,[\xi,\eta]]=0$.
    \end{itemize}   
\end{defin}
Denote the basic elements in $\mathfrak{g}$ as $t_a$, $a=1,...,\text{dim}\mathfrak{g}$ and Lie brackets as $[t_a,t_b]=\sum\limits_cf_{ab}^ct_c$, $f_{ab}^c$ -- structure constants.\\
The adjoint representation of $\mathfrak{g}$ on $\mathfrak{g}$:
\begin{equation}
    \text{ad}:\mathfrak{g}\rightarrow\text{End}(\mathfrak{g}),\quad \text{ad}:\xi\rightarrow\text{ad}_\xi
\end{equation}
\begin{equation}
    \text{ad}_\xi(\eta)=[\xi,\eta]
\end{equation}
The matrix elements of the adjoint representation are $f_{ab}^c$:
\begin{equation}
    \text{ad}_{t_a}(t_b)=[t_a,t_b]=\sum\limits_cf_{ab}^ct_c\rightarrow(\text{ad}_{t_a})^c_b=f_{ab}^c
\end{equation}
One can use the adjoint representation to define a natural bilinear product on $\mathfrak{g}$ -- Killing form $(\cdot,\cdot)$:
\begin{equation}
    (\xi,\eta)=\text{Tr}(\text{ad}_\xi\text{ad}_\eta)
\end{equation}
In this basis
\begin{equation}
    (\xi,\eta)=\sum\limits_{a,b}\xi^a\eta^b\text{Tr}(\text{ad}_{t_a}\text{ad}_{t_b})=\sum\limits_{a,b}\xi^a\eta^b\sum\limits_{c,d}f_{ac}^df_{bd}^c
\end{equation}
The Killing form is invariant: $(\xi,[\eta,\lambda])=([\xi,\eta],\lambda)$:
\begin{equation}
    (\xi,[\eta,\lambda])=\text{Tr}(\text{ad}_\xi\text{ad}_{[\eta,\lambda]})=\text{Tr}(\text{ad}_\xi[\text{ad}_\eta,\text{ad}_\lambda])=\text{Tr}([\text{ad}_\xi,\text{ad}_\eta]\text{ad}_\lambda)=([\xi,\eta],\lambda)
\end{equation}
\begin{defin}
    A subspace $I\subset\mathfrak{g}$ is called \textit{an ideal} if
    \begin{equation}
        \forall\xi\in I,\eta\in\mathfrak{g}\hookrightarrow[\xi,\eta]\in I
    \end{equation}
\end{defin}
\begin{defin}
    An ideal $I$ is called \textit{abelian} if $\forall\xi,\eta\in I\hookrightarrow[\xi,\eta]=0$.
\end{defin}
\begin{defin}
    A Lie algebra $\mathfrak{g}$ is \textit{semisimple} if it doesn't contain any nontrivial abelian ideal.
\end{defin}
\begin{defin}
    A Lie algebra $\mathfrak{g}$ is \textit{simple} if it's nonabelian and its ideals are only $\{0\}$ and $\mathfrak{g}$.
\end{defin}
A semisimple Lie algebra is a direct sum of simple ones.
\begin{theorem}[Cartan criterion]
    $\mathfrak{g}$ is semisimple $\Leftrightarrow$ Killing form is nondegenerate.
\end{theorem}
For any simesimple $\mathfrak{g}$ one has $\mathfrak{g}=[\mathfrak{g},\mathfrak{g}]$ (it's important to get examples of not semisimple).\\
Examples of semisimple Lie algebras: classical Lie algebras $\mathfrak{sl}_n$, $\mathfrak{so}_n$.\\
Not semisimple Lie algebras: $\{p,q,c:[p,q]=c,[p,c]=[q,c]=0\}$, $\mathfrak{gl}_n$: $[1,\cdot]=0$ and $\nexists x,y:[x,y]=1$. These algebras have nontrivial abelian ideals.
\begin{defin}
    Let $\mathfrak{g}$ be a semisimple Lie algebra, $\xi\in\mathfrak{g}$ is \textit{a semisimple} element if the matrix $\text{ad}_\xi$ can be diagonalized.
\end{defin}
\begin{defin}
    Let $\mathfrak{g}$ be a semisimple Lie algebra. \textit{A Cartan subalgebra} $\mathfrak{h}\subset\mathfrak{g}$ is a maximal abelian subalgebra: $\forall\xi\in\mathfrak{h}$ is semisimple.\\
    The dimension of $\mathfrak{h}$ is called the rank of $\mathfrak{g}$: $\text{rk}\mathfrak{g}=\dim\mathfrak{h}$.
\end{defin}
\begin{defin}
    Denote the basis in $\mathfrak{h}$ as $h_1,...,h_r$, $r=\text{rk}\;\mathfrak{g}$. All these elements are semisimple and commute with each other, so $\text{ad}_h$ can be diagonalized simultaneously and have a basis of common eigenvectors $e_\alpha\in\mathfrak{g}$:
    \begin{equation}
        \text{ad}_he_\alpha=\alpha(h)e_\alpha,\quad\forall h\in\mathfrak{h}
    \end{equation}
    This defines a map $\alpha:\mathfrak{h}\rightarrow\mathbb{C}$, $\alpha\in\mathfrak{h}^*$ -- this linear form is called \textit{the root} of the Lie algebra $\mathfrak{g}$. Denote the set of all roots as $\Delta$. 
\end{defin}
If $\alpha$ is a root, then $-\alpha$ is also a root. and if $\alpha\neq0$, then the eigenspace is one-dimensional. This provides the Cartan decomposition of $\mathfrak{g}$: $\{h_i\}$ -- basis in $\mathfrak{h}$, then $\{h_i,e_\alpha\}$ form a basis in $\mathfrak{g}$:
\begin{equation}
    \mathfrak{g}=\mathfrak{h}\oplus\underset{\alpha\in\Delta}{\oplus}\mathfrak{g}_\alpha
\end{equation}
\textit{The basic example:} $\mathfrak{g}=\mathfrak{sl}_n$.\\
$\mathfrak{h}$ is the Cartan subalgebra of traceless diagonal matrices. Denote $\lambda_i\in\mathfrak{h}^*$:
\begin{equation}
    \lambda_i(\text{diag}(a_1,a_2,...,a_n))=a_i
\end{equation}
Then the space of roots is $\Delta=\{\lambda_i-\lambda_j|1\leq i\leq n,1\leq j\leq n, i\neq j\}$ and the decomposition is
\begin{equation}
    \mathfrak{g}=\mathfrak{h}\oplus\underset{i\neq j}{\oplus}\mathfrak{g}_{\lambda_i-\lambda_j},\quad \mathfrak{g}_{\lambda_i-\lambda_j}=\braket{E_{ij}}
\end{equation}
\section{Integrable systems and Lax pairs. Symplectic manifolds.}
\begin{enumerate}
    \item \textbf{Lax pair for oscillators.}\\
    Find a Lax pair representation $\dot{L}=[L,M]$ for a one-dimensional harmonic oscillator
    \begin{equation}
        H=\frac{p^2}{2}+\frac{\omega^2x^2}{2}
    \end{equation}
    Use this ansatz for L-operator
    \begin{equation}
        L=\begin{pmatrix}
            p & f(q)\\
            f(q) & -p
        \end{pmatrix}
    \end{equation}
    How does the answer change if the anharmonic oscillator is considered instead of the harmonic one?\\
    \textbf{Solution.}\\
    Equations of motion:
    \begin{equation}\label{eq2}
        \begin{cases}
            \dot{p}=-\frac{\partial H}{\partial q}=-\omega^2 q,\\
            \dot{q}=\frac{\partial H}{\partial p}=p
        \end{cases}\rightarrow\dot{L}=\begin{pmatrix}
            \dot{p} & \frac{\partial f}{\partial q}\dot{q}\\
            \frac{\partial f}{\partial q}\dot{q} & -\dot{p}
        \end{pmatrix}=\begin{pmatrix}
            -\omega^2q & \frac{\partial f}{\partial q}p\\
            \frac{\partial f}{\partial q}p & \omega^2q
        \end{pmatrix}
    \end{equation}
    Let $M=\begin{pmatrix}
    a & b\\
    c & d
    \end{pmatrix}$, then
    \begin{equation}\label{eq3}
        [L,M]=LM-ML=\begin{pmatrix}
            (c-b)f(q) & (d-a)f(q)+2bp\\
            (a-d)f(q)-2cp & (b-c)f(q)
        \end{pmatrix}
    \end{equation}
    Comparing (\ref{eq2}) and (\ref{eq3}), we obtain
    \begin{equation}
        a=d=0,\quad b=-c=\frac{1}{2}\frac{\partial f}{\partial q}=\frac{\omega^2q}{2f(q)}
    \end{equation}
    \begin{equation}
        \frac{\partial f}{\partial q}=\frac{\omega^2q}{f(q)}\rightarrow f(q)=\pm\sqrt{\omega^2q^2+C}
    \end{equation}
    Let $C=0$, then
    \begin{equation}
        f(q)=\omega q\rightarrow b=-c=\frac{\omega}{2}
    \end{equation}
    Lax pair:
    \begin{equation}
        \boxed{L=\begin{pmatrix}
            p & \omega q\\
            \omega q & -p
        \end{pmatrix},\quad M=\begin{pmatrix}
            0 & \frac{\omega}{2}\\
            -\frac{\omega}{2} & 0
        \end{pmatrix}}
    \end{equation}
    Anharmonic oscillator:
    \begin{equation}
        H=\frac{p^2}{2}+V(q)
    \end{equation}
    Equations of motion:
    \begin{equation}\label{eq4}
        \begin{cases}
            \dot{p}=-\frac{\partial H}{\partial q}=-V'(q),\\
            \dot{q}=\frac{\partial H}{\partial p}=p
        \end{cases}\rightarrow\dot{L}=\begin{pmatrix}
            \dot{p} & \frac{\partial f}{\partial q}\dot{q}\\
            \frac{\partial f}{\partial q}\dot{q} & -\dot{p}
        \end{pmatrix}=\begin{pmatrix}
            -V'(q) & \frac{\partial f}{\partial q}p\\
            \frac{\partial f}{\partial q}p & V'(q)
        \end{pmatrix}
    \end{equation}
    Comparing (\ref{eq3}) and (\ref{eq4}), we obtain
    \begin{equation}
        a=d=0,\quad b=-c=\frac{1}{2}\frac{\partial f}{\partial q}=\frac{\omega^2q}{2f(q)}
    \end{equation}
    \begin{equation}
        \frac{\partial f}{\partial q}=\frac{V'(q)}{f(q)}\rightarrow f(q)=\pm\sqrt{2V(q)+C}
    \end{equation}
    Let $C=0$, then
    \begin{equation}
        f(q)=\sqrt{2V(q)}\rightarrow b=-c=\frac{V'(q)}{2\sqrt{2V(q)}}
    \end{equation}
    Lax pair:
    \begin{equation}
        \boxed{L=\begin{pmatrix}
            p & \sqrt{2V(q)}\\
            \sqrt{2V(q)} & -p
        \end{pmatrix},\quad M=\begin{pmatrix}
            0 & \frac{V'(q)}{2\sqrt{2V(q)}}\\
            -\frac{V'(q)}{2\sqrt{2V(q)}} & 0
        \end{pmatrix}}
    \end{equation}
    \item \textbf{Rational Ruijsenaars-Schneider system.}\\
    Consider a many-body system on a line in coordinates $\{q_i,p_i\}, 1\leq i\leq n$ with standard Poisson brackets
    \begin{equation}
        \{p_i,q_j\}=\delta_{ij},\quad\{q_i,q_j\}=\{p_i,p_j\}=0
    \end{equation}
    and Hamilton function
    \begin{equation}
        H=\sum\limits_{i=1}^n e^{p_i/c}\prod\limits_{k\neq i}\frac{q_i-q_k+\eta}{q_i-q_k},
    \end{equation}
    where $c$ and $\eta$ are constants.
    \begin{itemize}
        \item Write the equations of motion for the system in the form $\Ddot{q}_i=...$ (It can be useful to find the expression for $\dot{q}_i$ firstly, and express other quantities via $\dot{q}_i$).
        \item Show that Hamiltonian equations of motion can be presented in the Lax form
        \begin{equation}
            \dot{L}=[L,M],
        \end{equation}
        where matrices $L$ and $M$ are
        \begin{equation}
            L_{ij}=\frac{e^{p_j/c}}{q_i-q_j+\eta}\prod_{k\neq j}\frac{q_j-q_k+\eta}{q_j-q_k}
        \end{equation}
        \begin{equation}
            M_{ij}=-\frac{\dot{q}_j}{q_i-q_j},\quad i\neq j,\quad 
        \end{equation}
        \begin{equation}
            M_{ii}=-\frac{\dot{q}_i}{\eta}+\sum\limits_{k\neq i}\frac{\eta\dot{q}_k}{(q_i-q_k+\eta)(q_i-q_k)}=-\frac{\dot{q}_i}{\eta}+\sum\limits_{k\neq i}\left(\frac{\dot{q}_k}{q_i-q_k}-\frac{\dot{q}_k}{q_i-q_k+\eta}\right)
        \end{equation}
        \item Let $\eta=\frac{\nu}{c}$. Investigate the limit $c\rightarrow\infty$ while $\nu$ remains constant.
    \end{itemize}
    \textbf{Solution.}
    \begin{itemize}
        \item 
        \begin{equation}
            \{f(p),q_i\}=\frac{\partial f}{\partial p_i},\quad \{f(q),p_i\}=-\frac{\partial f}{\partial q_i}
        \end{equation}
        \begin{equation}
            \dot{q}_i=\{H,q_i\}=\{e^{p_i/c},q_i\}\prod\limits_{k\neq i}\frac{q_i-q_k+\eta}{q_i-q_k}=\frac{e^{p_i/c}}{c}\prod\limits_{k\neq i}\frac{q_i-q_k+\eta}{q_i-q_k}
        \end{equation}
        \begin{equation}
            H=c\sum\limits_{i=1}^n\dot{q}_i
        \end{equation}
        \begin{equation}
            \dot{p}_i=\{H,p_i\}=e^{p_i/c}\left\{\prod\limits_{k\neq i}\frac{q_i-q_k+\eta}{q_i-q_k},p_i\right\}+\sum\limits_{j\neq i}e^{p_j/c}\left\{\prod\limits_{k\neq j}\frac{q_j-q_k+\eta}{q_j-q_k},p_i\right\}
        \end{equation}
        \begin{multline}
            \left\{\prod\limits_{k\neq i}\frac{q_i-q_k+\eta}{q_i-q_k},p_i\right\}=\prod\limits_{k\neq i,j}\frac{q_i-q_k+\eta}{q_i-q_k}\sum\limits_{j\neq i}\left\{\frac{q_i-q_j+\eta}{q_i-q_j},p_i\right\}=\\=\prod\limits_{k\neq i,j}\frac{q_i-q_k+\eta}{q_i-q_k}\sum\limits_{j\neq i}\frac{\eta}{(q_i-q_j)^2}=\prod\limits_{k\neq i}\frac{q_i-q_k+\eta}{q_i-q_k}\sum\limits_{j\neq i}\left(\frac{1}{q_i-q_j}-\frac{1}{q_i-q_j+\eta}\right)
        \end{multline}
        \begin{multline}
            \left\{\prod\limits_{k\neq j}\frac{q_j-q_k+\eta}{q_j-q_k},p_i\right\}=\prod\limits_{k\neq i,j}\frac{q_j-q_k+\eta}{q_j-q_k}\left\{\frac{q_j-q_i+\eta}{q_j-q_i},p_i\right\}=\\=-\prod\limits_{k\neq i,j}\frac{q_j-q_k+\eta}{q_j-q_k}\frac{\eta}{(q_i-q_j)^2}=-\prod\limits_{k\neq j}\frac{q_j-q_k+\eta}{q_j-q_k}\left(\frac{1}{q_j-q_i}-\frac{1}{q_j-q_i+\eta}\right)
        \end{multline}
        \begin{multline}
            \dot{p}_i=e^{p_i/c}\prod\limits_{k\neq i}\frac{q_i-q_k+\eta}{q_i-q_k}\sum\limits_{j\neq i}\left(\frac{1}{q_i-q_j}-\frac{1}{q_i-q_j+\eta}\right)-\\-\sum\limits_{j\neq i}e^{p_j/c}\prod\limits_{k\neq j}\frac{q_j-q_k+\eta}{q_j-q_k}\left(\frac{1}{q_j-q_i}-\frac{1}{q_j-q_i+\eta}\right)=\\=\sum\limits_{j\neq i}\left(c\dot{q}_i\left(\frac{1}{q_i-q_j}-\frac{1}{q_i-q_j+\eta}\right)+c\dot{q}_j\left(\frac{1}{q_i-q_j}-\frac{1}{q_i-q_j-\eta}\right)\right)
        \end{multline}
        \begin{equation}
            \dot{p}_i=\eta c\left(\frac{1}{q_i-q_j}\left(\frac{\dot{q}_i}{q_i-q_j+\eta}+\frac{\dot{q}_j}{q_j-q_i+\eta}\right)\right)
        \end{equation}
        \begin{multline}
            \ddot{q}_i=\frac{d}{dt}\left(\frac{e^{p_i/c}}{c}\prod\limits_{k\neq i}\frac{q_i-q_k+\eta}{q_i-q_k}\right)=\dot{p}_i\frac{e^{p_i/c}}{c^2}\prod\limits_{k\neq i}\frac{q_i-q_k+\eta}{q_i-q_k}+\frac{e^{p_i/c}}{c}\prod\limits_{k\neq i}\frac{q_i-q_k+\eta}{q_i-q_k}\times\\\times\sum\limits_{k\neq i}\left(\frac{\dot{q}_i-\dot{q}_k}{q_i-q_k+\eta}-\frac{\dot{q}_i-\dot{q}_k}{q_i-q_k}\right)=\frac{\dot{p}_i\dot{q}_i}{c}+\sum\limits_{k\neq i}\dot{q}_i(\dot{q}_i-\dot{q}_k)\left(\frac{1}{q_i-q_k+\eta}-\frac{1}{q_i-q_k}\right)
        \end{multline}
        \begin{equation}
            \boxed{\ddot{q}_i=\sum\limits_{k\neq i}\dot{q}_i\dot{q}_j\left(\frac{2}{q_i-q_k}-\frac{1}{q_i-q_k+\eta}-\frac{1}{q_i-q_k+\eta}\right)}
        \end{equation}
        \item
        \begin{equation}
            L_{ij}=\frac{e^{p_j/c}}{q_i-q_j+\eta}\prod_{k\neq j}\frac{q_j-q_k+\eta}{q_j-q_k}=\frac{c\dot{q}_j}{q_i-q_j+\eta},\quad L_{ii}=\frac{\dot{q}_i}{\eta}
        \end{equation}
        \begin{multline}
            \dot{L}_{ij}=\frac{c\ddot{q}_j}{q_i-q_j+\eta}-\frac{c\dot{q}_j(\dot{q}_i-\dot{q}_j)}{(q_i-q_j+\eta)^2}=\frac{c\dot{q}_i\dot{q}_j}{q_i-q_j+\eta}\left(\frac{2}{q_j-q_i}-\frac{1}{q_j-q_i+\eta}-\frac{1}{q_j-q_i-\eta}\right)+\\+\sum_{k\neq i,j}\frac{c\dot{q}_j\dot{q}_k}{q_i-q_j+\eta}\left(\frac{2}{q_j-q_k}-\frac{1}{q_j-q_k+\eta}-\frac{1}{q_j-q_k-\eta}\right)-\frac{c\dot{q}_j(\dot{q}_i-\dot{q}_j)}{(q_i-q_j+\eta)^2}
        \end{multline}
        \begin{multline}
            [L,M]_{ij}=(L_{ii}-L_{jj})M_{ij}+(M_{jj}-M_{ii})L_{ij}+\sum\limits_{k=1}^n(L_{ik}M_{kj}-M_{ik}L_{kj})=-\frac{\dot{q}_i-\dot{q}_j}{\eta}\frac{c\dot{q}_j}{q_i-q_j}+\\+\frac{c\dot{q}_j}{q_i-q_j+\eta}\frac{\dot{q}_i-\dot{q}_j}{\eta}+\frac{c\dot{q}_j}{q_i-q_j+\eta}\left(\frac{\dot{q}_i}{q_j-q_i}-\frac{\dot{q}_i}{q_j-q_i+\eta}-\frac{\dot{q}_j}{q_j-q_i}+\frac{\dot{q}_j}{q_i-q_j+\eta}\right)+\\+\sum\limits_{k\neq i,j}\frac{c\dot{q}_j\dot{q}_k}{q_i-q_j+\eta}\left(\frac{1}{q_j-q_k}-\frac{1}{q_j-q_k+\eta}-\frac{1}{q_i-q_k}+\frac{1}{q_i-q_k+\eta}\right)+\\+\sum\limits_{k\neq i,j}\left(\frac{c\dot{q}_j\dot{q}_k}{(q_i-q_k)(q_k-q_j+\eta)}-\frac{c\dot{q}_j\dot{q}_k}{(q_k-q_j)(q_i-q_k+\eta)}\right)
        \end{multline}
        \begin{equation}
            \frac{1}{(q_i-q_k)(q_k-q_j+\eta)}=\left(\frac{1}{q_i-q_k}+\frac{1}{q_k-q_j+\eta}\right)\frac{1}{q_i-q_j+\eta}
        \end{equation}
        \begin{equation}
            \frac{1}{(q_k-q_j)(q_i-q_k+\eta)}=\left(\frac{1}{q_k-q_j}+\frac{1}{q_i-q_k+\eta}\right)\frac{1}{q_i-q_j+\eta}
        \end{equation}
        \begin{multline}
            [L,M]_{ij}=\frac{c\dot{q}_j^2}{\eta(q_i-q_j)}-\frac{c\dot{q}_i\dot{q}_j}{\eta(q_i-q_j)}+\frac{c\dot{q}_i\dot{q}_j}{\eta(q_i-q_j+\eta)}-\frac{c\dot{q}_j^2}{\eta(q_i-q_j+\eta)}+\frac{c\dot{q}_i\dot{q}_j}{(q_i-q_j+\eta)(q_i-q_j+\eta)}-\\-\frac{c\dot{q}_j^2}{(q_i-q_j+\eta)(q_i-q_j)}-\frac{c\dot{q}_i\dot{q}_j}{(q_i-q_j+\eta)(q_i-q_j)}+\frac{c\dot{q}_j^2}{(q_i-q_j+\eta)^2}+\\+\sum\limits_{k\neq i,j}\frac{c\dot{q}_j\dot{q}_k}{q_i-q_j+\eta}\left(\frac{2}{q_j-q_k}-\frac{1}{q_j-q_k+\eta}-\frac{1}{q_j-q_k-\eta}\right)
        \end{multline}
        \begin{equation}
            \frac{c\dot{q}_j^2}{\eta(q_i-q_j)}-\frac{c\dot{q}_j^2}{\eta(q_i-q_j+\eta)}=\frac{c\dot{q}_j^2}{(q_i-q_j)(q_i-q_j+\eta)}
        \end{equation}
        \begin{equation}
            \frac{c\dot{q}_i\dot{q}_j}{\eta(q_i-q_j+\eta)}-\frac{c\dot{q}_i\dot{q}_j}{\eta(q_i-q_j)}=-\frac{c\dot{q}_i\dot{q}_j}{(q_i-q_j)(q_i-q_j+\eta)}
        \end{equation}
        \begin{multline}
            [L,M]_{ij}=\frac{c\dot{q}_j^2}{(q_i-q_j+\eta)^2}+\frac{c\dot{q}_i\dot{q}_j}{q_i-q_j+\eta}\left(\frac{1}{q_i-q_j+\eta}-\frac{2}{q_i-q_j}\right)+\\+\sum\limits_{k\neq i,j}\frac{c\dot{q}_j\dot{q}_k}{q_i-q_j+\eta}\left(\frac{2}{q_j-q_k}-\frac{1}{q_j-q_k+\eta}-\frac{1}{q_j-q_k-\eta}\right)
        \end{multline}
        As seen,
        \begin{equation}
            \dot{L}_{ij}=[L,M]_{ij}
        \end{equation}
    %\item
    %\begin{equation}
        %\dot{q}_i=\frac{e^{p_i/c}}{c}\prod\limits_{k\neq i}\frac{q_i-q_k+\frac{\nu}{c}}{q_i-q_k}\rightarrow0,\quad\ddot{q}_i\rightarrow0
    %\end{equation}
    %\begin{equation}
        %\dot{p}_i=\nu\sum\limits_{k\neq i}\frac{1}{q_i-q_k}\left(\frac{\dot{q}_i}{q_i-q_k+\frac{\nu}{c}}+\frac{\dot{q}_k}{q_k-q_i+\frac{\nu}{c}}\right)\rightarrow\nu\sum\limits_{k\neq i}\frac{\dot{q}_i-\dot{q}_j}{(q_i-q_k)^2}
    %\end{equation}
    \end{itemize}
    \item \textbf{Classical Sklyanin algebra.}\\
    Consider a four-dimensional space with coordinates $(S_0, S_1, S_2, S_3)$. Let $J_1, J_2, J_3$ be appropriate constants. Check that the operations defined on linear functions as
    \begin{equation}
        \{S_0,S_i\}=-\{S_i,S_0\}=\epsilon_{ijk}S_jS_k(J_j-J_k),\quad \{S_i,S_j\}=\epsilon_{ijk}S_0S_k,\quad i,j,k\in\{1,2,3\}
    \end{equation}
    and on polynomials via the Leibniz rule define Poisson brackets on this space.\\
    Show that these Poisson brackets are degenerate. Namely, find Casimir functions of the form
    \begin{equation}
        C_1=\alpha S_0^2+\beta(S_1^2+S_2^2+S_3^2)
    \end{equation}
    \begin{equation}
        C_2=\gamma S_0^2+\delta(J_1S_1^2+J_2S_2^2+J_3S_3^2)
    \end{equation}
    \textbf{Solution.}\\
    Check 3 axioms:
    \begin{itemize}
    \item Antisymmetry of Poisson brackets $\{f,g\}=-\{g,f\}$ follows from the definition.
    \item Define $\{\{S_\mu,S_\nu\},S_\lambda\}$ in such a way that the Leibniz identity $\{f,gh\}=g\{f,h\}+\{f,g\}h$ holds.
    \item Jacobi identity.
        \begin{multline}
            \{\{S_0,S_1\},S_2\}=2\{\epsilon_{123}S_2S_3(J_2-J_3),S_2\}=2\epsilon_{123}S_2(J_2-J_3)\{S_3,S_2\}=\\=2\epsilon_{123}S_2(J_2-J_3)\epsilon_{321}S_0S_1=-2S_2S_0S_1(J_2-J_3)
        \end{multline}
        \begin{multline}
            \{\{S_2,S_0\},S_1\}=-2\{\epsilon_{231}S_3S_1(J_3-J_1),S_1\}=-2\epsilon_{231}S_1(J_3-J_1)\{S_3,S_1\}=\\=-2\epsilon_{231}S_1(J_3-J_1)\epsilon_{312}S_0S_2=-2S_1S_0S_2(J_3-J_1)
        \end{multline}
        \begin{multline}
            \{\{S_1,S_2\},S_0\}=\{\epsilon_{123}S_0S_3,S_0\}=-2\epsilon_{123}S_0\epsilon_{312}S_1S_2(J_1-J_2)=-2S_0S_1S_2(J_1-J_2)
        \end{multline}
        \begin{equation}
            \{\{S_0,S_1\},S_2\}+\{\{S_2,S_0\},S_1\}+\{\{S_1,S_2\},S_0\}=0
        \end{equation}
        Analogically, for other pairs $(0,2,3)$ or $(0,1,3)$.
        \begin{multline}
            \{\{S_0,S_i\},S_i\}+\{\{S_i,S_i\},S_0\}+\{\{S_i,S_0\},S_i\}=\{\{S_0,S_i\},S_i\}+0-\{\{S_0,S_i\},S_i\}=0
        \end{multline}
        \begin{multline}
            \{\{S_0,S_0\},S_i\}+\{\{S_i,S_0\},S_0\}+\{\{S_0,S_i\},S_i\}=\\=0+\{\{S_i,S_0\},S_0\}-\{\{S_i,S_0\},S_i\}=0
        \end{multline}
        \begin{equation}
            \{\{S_1,S_2\},S_3\}=\{\epsilon_{123}S_0S_3,S_3\}=2\epsilon_{123}\epsilon_{312}S_1S_2(J_1-J_2)S_3=2S_1S_2S_3(J_1-J_2)
        \end{equation}
        \begin{equation}
            \{\{S_3,S_1\},S_2\}=\{\epsilon_{312}S_0S_2,S_2\}=2\epsilon_{312}\epsilon_{213}S_1S_3(J_1-J_3)S_2=-2S_1S_2S_3(J_1-J_3)
        \end{equation}
        \begin{equation}
            \{\{S_2,S_3\},S_1\}=\{\epsilon_{231}S_0S_1,S_1\}=2\epsilon_{231}\epsilon_{123}S_2S_3(J_2-J_3)S_1=2S_1S_2S_3(J_2-J_3)
        \end{equation}
        \begin{equation}
            \{\{S_1,S_2\},S_3\}+\{\{S_3,S_1\},S_2\}+\{\{S_2,S_3\},S_1\}=0
        \end{equation}
    \end{itemize}
    Find Casimir elements:
    \begin{multline}
        \{S_0,C_1\}=\{S_0,\alpha S_0^2+\beta(S_1^2+S_2^2+S_3^2)\}=\{S_0,\beta(S_1^2+S_2^2+S_3^2)\}=\\=2\beta(\{S_0,S_1\}S_1+\{S_0,S_2\}S_2+\{S_0,S_3\}S_3)=\\=4\beta S_1S_2S_3(J_2-J_3-(J_1-J_3)+J_1-J_2)=0
    \end{multline}
    \begin{multline}
        \{S_1,C_1\}=\{S_1,\alpha S_0^2+\beta(S_1^2+S_2^2+S_3^2)\}=\{S_1,\alpha S_0^2+\beta(S_2^2+S_3^2)\}=\\=2\alpha\{S_1,S_0\}S_0+2\beta(\{S_1,S_2\}S_2+\{S_1,S_3\}S_3)=\\=-4\alpha S_0S_2S_3(J_2-J_3)+4\beta S_0S_2S_3(1-1)=-4\alpha S_0S_2S_3(J_2-J_3)
    \end{multline}
    \begin{equation}
        \boxed{\alpha=0\rightarrow C_1=\beta(S_1^2+S_2^2+S_3^2)}
    \end{equation}
    $\{S_2,C_1\}$ and $\{S_3,C_1\}$ give the same condition.
    \begin{multline}
        \{S_0,C_2\}=\{S_0,\gamma S_0^2+\delta(J_1S_1^2+J_2S_2^2+J_3S_3^2)\}=\{S_0,\delta(J_1S_1^2+J_2S_2^2+J_3S_3^2)\}=\\=2\delta(J_1\{S_0,S_1\}S_1+J_2\{S_0,S_2\}S_2+J_3\{S_0,S_3\}S_3)=\\=4\delta S_1S_2S_3(J_1(J_2-J_3)-J_2(J_1-J_3)+J_3(J_1-J_2))=0
    \end{multline}
    \begin{multline}
        \{S_1,C_2\}=\{S_1,\gamma S_0^2+\delta(J_1S_1^2+J_2S_2^2+J_3S_3^2)\}=\{S_1,\gamma S_0^2+\delta(J_2S_2^2+J_3S_3^2)\}=\\=2\gamma\{S_1,S_0\}S_0+2\delta(J_2\{S_1,S_2\}S_2+J_3\{S_1,S_3\}S_3)=\\=-4\gamma S_0S_2S_3(J_2-J_3)+4\delta S_0S_2S_3(J_2-J_3)=4(\delta-\gamma) S_0S_2S_3(J_2-J_3)
    \end{multline}
    \begin{equation}
        \boxed{\delta=\gamma\rightarrow C_2=\gamma(S_0^2+J_1S_1^2+J_2S_2^2+J_3S_3^2)}
    \end{equation}
    \item \textbf{Universal Poisson brackets.}\\
    Consider a system with Hamilton function
    \begin{equation}
        H=\frac{p_1^2+p_2^2+p_3^2}{2}
    \end{equation}
    and Poisson brackets
    \begin{equation}\label{eq1}
        \{p_i,x_j\}=\delta_{ij},\quad\{x_i,x_j\}=0,\quad \{p_i,p_j\}=F_{ij}(x)
    \end{equation}
    \begin{itemize}
        \item Write the equations of motion for the system. What is the physical meaning of these equations?
        \item Which conditions should be imposed on $F_{ij}(x)$ to make (\ref{eq1}) a Poisson bracket.
        \item Write down the corresponding symplectic form. Check if it is non-degenerate.
    \end{itemize}
    \textbf{Solution.}
    \begin{itemize}
        \item Equations of motion:
        \begin{equation}
            \begin{cases}
                \frac{dp_i}{dt}=\{H,p_i\},\\
                \frac{dx_i}{dt}=\{H,x_i\}.
            \end{cases}
        \end{equation}
        \begin{align}
            \{H,p_i\}=\frac{1}{2}\left\{p_1^2+p_2^2+p_3^2,p_i\right\}=(p_1\{p_1,p_i\}+p_2\{p_2,p_i\}+p_3\{p_3,p_i\})=\\=p_1F_{1i}(x)+p_2F_{2i}(x)+p_3F_{3i}(x)
        \end{align}
        \begin{align}
            \{H,x_i\}=\frac{1}{2}\{p_1^2+p_2^2+p_3^2,x_i\}=(p_1\{p_1,x_i\}+p_2\{p_1,x_i\}+p_3\{p_3,x_i\})=\\=p_1\delta_{1i}+p_2\delta_{2i}+p_3\delta_{3i}=p_i
        \end{align}
        \begin{equation}
            \boxed{\begin{cases}
            \frac{dp_i}{dt}=p_1F_{1i}(x)+p_2F_{2i}(x)+p_3F_{3i}(x),\\
            \frac{dx_i}{dt}=p_i.
            \end{cases}}
        \end{equation}
        Equations of motion are similar to the motion in magnetic field:
        \begin{equation}
            \ddot{x}_i=\dot{p}_i=p_j F_{ji}=\dot{x}_jF_{ji}
        \end{equation}
        Let
        \begin{equation}
            B_k=\frac{1}{2}\epsilon_{kij}F_{ij}\rightarrow F_{ij}=\epsilon_{ijk}B_k
        \end{equation}
        \begin{equation}
            \ddot{q}_i=\epsilon_{ijk}\dot{x}_jB_k
        \end{equation}
    \item Using the anticommutative property of Jacobi bracket, we get
    \begin{equation}
        F_{ij}(x)=\{p_i,p_j\}=-\{p_j,p_i\}=-F_{ji}(x)\rightarrow\boxed{F_{ij}(x)=-F_{ji}(x)}
    \end{equation}
    Using Jacobi identity, we get
    \begin{equation}
        \{\{p_i,p_j\},p_k\}+\{\{p_k,p_i\},p_j\}+\{\{p_j,p_k\},p_i\}=0
    \end{equation}
    \begin{equation}
        \{F_{ij}(x),p_k\}+\{F_{ki}(x),p_j\}+\{F_{jk}(x),p_i\}=0
    \end{equation}
    This identity is equal
    \begin{equation}
        \boxed{\frac{\partial F_{ij}(x)}{\partial x^k}+\frac{\partial F_{jk}(x)}{\partial x^i}+\frac{\partial F_{ki}(x)}{\partial x^j}=0\leftrightarrow dF=0}
    \end{equation}
    We get Bianchi identity.
    \item Let $x=(\bm{x},\bm{p})$ -- vector in the phase space. Poisson bracket $\{f,g\}(x)=\pi_{ij}(x)\frac{\partial f}{\partial x_i}\frac{\partial g}{\partial x_j}$ corresponds symplectic form $\omega=A(\pi^{-1})_{ij}dx_i\wedge dx_j$ with non-zero $A$ (for example, $A=-\frac{1}{2}$).
    \begin{equation*}
        \pi=\begin{pmatrix}
            0 & 0 & 0 & 1 & 0 & 0\\
            0 & 0 & 0 & 0 & 1 & 0\\
            0 & 0 & 0 & 0 & 0 & 1\\
            -1 & 0 & 0 & 0 & F_{12} & F_{13}\\
            0 & -1 & 0 & -F_{12} & 0 & F_{23}\\
            0 & 0 & -1 & -F_{13} & -F_{23} & 0\\
        \end{pmatrix}\rightarrow\pi^{-1}=\begin{pmatrix}
            0 & F_{12} & F_{13} & -1 & 0 & 0\\
            -F_{12} & 0 & F_{23} & 0 & -1 & 0\\
            -F_{13} & -F_{23} & 0 & 0 & 0 & -1\\
            1 & 0 & 0 & 0 & 0 & 0\\
            0 & 1 & 0 & 0 & 0 & 0\\
            0 & 0 & 1 & 0 & 0 & 0\\
        \end{pmatrix}
    \end{equation*}
        \begin{equation}
            \boxed{\omega=\sum\limits_{i=1}^3dp_i\wedge dx_i-F_{12}(x)dx_1\wedge dx_2-F_{23}(x)dx_2\wedge dx_3-F_{31}(x)dx_3\wedge dx_1}
        \end{equation} 
        Consider arbitrary non-zero vector $\xi=\xi_i(x,p)\frac{\partial}{\partial{x_i}}+\Tilde{\xi_i}(x,p) \frac{\partial}{\partial{p_i}}$.
        \[\omega(\xi,\cdot)=\sum_{i=1}^{3}\left(\tilde{\xi}_i+\sum_{j\neq i}\limits F_{ij}\xi_j\right)dx_i-\xi_i dp_i.\]
        According to the definition of nondegeneracy, we need to find some vector $\eta$:
        \begin{equation}
            \omega(\xi,\eta)\neq0
        \end{equation}
        Let $\eta=\frac{\partial}{\partial{x_i}}$, then $\omega(\xi,\eta)=\tilde{\xi}_i+\sum_{j\neq i}\limits F_{ij}\xi_j$. Let $\eta=\frac{\partial}{\partial{p_i}}$, then $\omega(\xi,\eta)=-\xi_i$. $\xi_i(x,p)$ and $\tilde{\xi}_i(x,p)$ can't be 0 together, because $\xi\neq 0$, so $\omega(\xi,\eta)\neq0$.
    \end{itemize}
    \item \textbf{Lagrange top.}\\
    Consider a system with a Hamilton function $H(\Vec{S},\Vec{P})$ defined on a six-dimensional space with coordinates $S_1$, $S_2$, $S_3$, $P_1$, $P_2$, $P_3$ and Poisson brackets
    \begin{equation}\label{eq5}
        \{S_i,S_j\}=\epsilon_{ijk}S_k,\quad \{S_i,P_j\}=\epsilon_{ijk}P_k,\quad \{P_i,P_j\}=0
    \end{equation}
    Denote $\omega_i=\frac{\partial H}{\partial S_i}$ and $h_i=-\frac{\partial H}{\partial P_i}$.
    \begin{itemize}
        \item Write down the equations of motion for this system (in components and in vector form).
        \item Consider a special form of Hamilton function
        \begin{equation}
            H=\frac{1}{2}(J_1S_1^2+J_2S_2^2+J_3S_3^2)-(h_1P_1+h_2P_2+h_3P_3)
        \end{equation}
        Which physical system is described by this Hamilton function?\\
        Let $J_1=J_2=a$, $J_3=b$, $h_1=h_2=0$ and $h_3=h$. Show that in this case a scalar product ($\Vec{S}\cdot\Vec{h}$) is in involution with Hamilton function $H$, i.e. defines the additional conservation law.
        \item Show that the Poisson brackets (\ref{eq5}) are degenerate and check that there are two Casimir functions
        \begin{equation}
            C_1=P_1^2+P_2^2+P_3^2,\quad C_2=S_1P_1+S_2P_2+S_3P_3
        \end{equation}
        Are the Poisson brackets (\ref{eq5}) related to any Lie algebra? Describe this Lie algebra.
        \item Fix the level surface of Casimir functions $C_1=p_1^2$, $C_2=ps$, where $p\neq0$ and $s$ are constants. Show that a change of variables $S_i\hookrightarrow y_i=S_i-\frac{s}{p}p_i$ defines an isomorphism of the level surface with the cotangent bundle $T^*S^2$.
        \item Consider another change of variables on the level surface
        \begin{equation}
            p_1=p\cos\theta\cos\phi,\quad p_2=p\cos\theta\sin\phi,\quad p_3=p\sin\theta
        \end{equation}
        \begin{equation}
            y_1=p_\phi\tan\theta\cos\phi,\quad y_2=p_\phi\tan\theta\sin\phi+p_\theta\cos\phi,\quad y_3=-p_\phi
        \end{equation}
        \begin{equation}
            -\frac{\pi}{2}\leq\theta\leq\frac{\pi}{2},\quad0\leq\phi\leq2\pi
        \end{equation}
        Compute the Poisson brackets between coordinates $\theta$, $\phi$, $p_\theta$, $p_\phi$ and show that they are non-degenerate. Find a symplectic form $\omega$, corresponding to these Poisson brackets and calculate the integral of $\omega$ over the sphere $\theta$, $\phi$. How could these results be interpreted?
    \end{itemize}
    \item \textbf{Two oscillators.}\\
    Consider a system of two independent oscillators with Hamilton function
    \begin{equation}
        H=H_1+H_2=\frac{\omega_1}{2}(p_1^2+q_1^2)+\frac{\omega_2}{2}(p_2^2+q_2^2)
    \end{equation}
    and standard Poisson brackets
    \begin{equation}
        \{p_i,q_j\}=\delta_{ij},\quad \{p_i,p_j\}=\{q_i,q_j\}=0
    \end{equation}
    \begin{itemize}
        \item Describe geometrically the level manifold of Hamilton function $H=E$ for different values of $E$.
        \item Describe geometrically the level manifold of functions $H_1=E_1$ and $H_2=E_2$ for different values of $E_1$, $E_2$. How the level manifolds from the previous point are made up of these level manifolds?
        \item Let $\omega_1=\omega_2=\omega$. Prove that in this case there are three independent conserved quantities
        \begin{equation}
            I_1=q_1q_2+p_1p_2,\quad I_2=p_1q_2-p_2q_1,\quad I_3=\frac{1}{2}(p_1^2+q_1^2-p_2^2-q_2^2)
        \end{equation}
        \item Check that these quantities satisfy the condition
        \begin{equation}
            \omega^2(I_1^2+I_2^2+I_3^2)=H^2
        \end{equation}
        and show that their level manifolds define the Hopf fibration $S^1\hookrightarrow S^3\twoheadrightarrow S^2$.
    \end{itemize}
    \textbf{Solution.}
    \begin{itemize}
        \item The level manifold:
        \begin{equation}
            H=\frac{\omega_1}{2}(p_1^2+q_1^2)+\frac{\omega_2}{2}(p_2^2+q_2^2)=E
        \end{equation}
        In coordinates $\Tilde{p}_1=p_1\sqrt{\frac{\omega_1}{2}}$, $\Tilde{q}_1=q_1\sqrt{\frac{\omega_1}{2}}$, $\Tilde{p}_2=p_2\sqrt{\frac{\omega_2}{2}}$, $\Tilde{p}_1=p_2\sqrt{\frac{\omega_2}{2}}$:
        \begin{equation}
            \boxed{\Tilde{p}_1^2+\Tilde{q}_1^2+\Tilde{p}_2^2+\Tilde{q}_2^2=E}
        \end{equation}
        We get the equation of a sphere $S^3$ with radius $\sqrt{E}$.
        \item The level manifold of function $H_1$:
        \begin{equation}
            H_1=\frac{\omega_1}{2}(p_1^2+q_1^2)=E_1
        \end{equation}
        In coordinates $\Tilde{p}_1=p_1\sqrt{\frac{\omega_1}{2}}$, $\Tilde{q}_1=q_1\sqrt{\frac{\omega_1}{2}}$:
        \begin{equation}
            \boxed{\Tilde{p}_1^2+\Tilde{q}_1^2=E_1}
        \end{equation}
        We get the equation of a circle $S^1$ with radius $\sqrt{E_1}$.\\
        In coordinates $\Tilde{p}_2=p_2\sqrt{\frac{\omega_2}{2}}$, $\Tilde{q}_2=q_2\sqrt{\frac{\omega_2}{2}}$:
        \begin{equation}
            \boxed{\Tilde{p}_2^2+\Tilde{q}_2^2=E_2}
        \end{equation}
        We get the equation of a circle $S^1$ with radius $\sqrt{E_2}$.\\
        Summarizing, we get a torus $S^1\times S^1$.
        \item 
        \begin{equation}
            H=\frac{\omega}{2}(p_1^2+q_1^2+p_2^2+q_2^2)
        \end{equation}
        \begin{equation}
            \text{rk}\begin{pmatrix}
                \frac{\partial I_1}{\partial q_1} & \frac{\partial I_1}{\partial p_1} & \frac{\partial I_1}{\partial q_2} & \frac{\partial I_1}{\partial p_2}\\
                \frac{\partial I_2}{\partial q_1} & \frac{\partial I_2}{\partial p_1} & \frac{\partial I_2}{\partial q_2} & \frac{\partial I_2}{\partial p_2}\\
                \frac{\partial I_3}{\partial q_1} & \frac{\partial I_3}{\partial p_1} & \frac{\partial I_3}{\partial q_2} & \frac{\partial I_3}{\partial p_2}
            \end{pmatrix}=\text{rk}\begin{pmatrix}
                q_2 & p_2 & q_1 & p_1\\
                -p_2 & q_2 & p_1 & -q_1\\
                q_1 & p_1 & -q_2 & -p_2
            \end{pmatrix}=3
        \end{equation}
        So, $I_1$, $I_2$, $I_3$ are independent quantities.
        \begin{multline}
            \dot{I}_1=\{H,I_1\}=\frac{\omega}{2}(\{p_1^2,q_1q_2\}+\{q_1^2,p_1p_2\}+\{p_2^2,q_1q_2\}+\{q_2^2,p_1p_2\})=\\=\frac{\omega}{2}(2p_1\{p_1,q_1\}q_2+2q_1\{q_1,p_1\}p_2+2q_1p_2\{p_2,q_2\}+2p_1q_2\{q_2,p_2\})=\\=\omega(p_1q_2-q_1p_2+q_1p_2-p_1q_2)=0
        \end{multline}
        \begin{multline}
            \dot{I}_2=\{H,I_2\}=\frac{\omega}{2}(\{p_1^2,-p_2q_1\}+\{q_1^2,p_1q_2\}+\{p_2^2,p_1q_2\}+\{q_2^2,-p_2q_1\})=\\=\frac{\omega}{2}(-2p_1p_2\{p_1,q_1\}+2q_1\{q_1,p_1\}q_2+2p_1p_2\{p_2,q_2\}-2q_2\{q_2,p_2\}q_1)=\\=\omega(-p_1p_2+q_1q_2+p_1p_2-q_1q_2)=0
        \end{multline}
        \begin{multline}
            \dot{I}_3=\{H,I_3\}=\frac{\omega}{4}(\{p_1^2,q_1^2\}+\{q_1^2,p_1^2\}+\{p_2^2,-q_2^2\}+\{q_2^2,-p_2^2\})=\\=\frac{\omega}{2}(4p_1q_1\{p_1,q_1\}+4q_1p_1\{q_1,p_1\}-4p_2q_2\{p_2,q_2\}-4q_2p_2\{q_2,p_2\})=\\=\omega(p_1q_1-q_1p_1-p_2q_2+q_2p_2)=0
        \end{multline}
        So, $I_1$, $I_2$, $I_3$ are conserved quantities.
        \item
        \begin{multline}
            \omega^2(I_1^2+I_2^2+I_3^2)=\omega^2\left((q_1q_2+p_1p_2)^2+(p_1q_2-p_2q_1)^2+\frac{1}{4}(p_1^2+q_1^2-p_2^2-q_2^2)^2\right)=\\=\omega^2\left(q_1^2q_2^2+p_1^2p_2^2+2q_1q_2p_1p_2+p_1^2q_2^2+p_2^2q_1^2-2p_1q_2p_2q_1+\right.\\\left.+\frac{1}{4}(p_1^4+q_1^4+p_2^4+q_2^4+2p_1^2q_1^2-2p_1^2p_2^2-2p_1^2q_2^2-2q_1^2p_2^2-2q_1^2q_2^2+2p_2^2q_2^2)\right)=\\=\omega^2\left(\frac{1}{4}(p_1^4+q_1^4+p_2^4+q_2^4)+\frac{1}{2}(p_1^2q_1^2+p_1^2p_2^2+p_1^2q_2^2+p_2^2q_1^2+q_1^2q_2^2+p_2^2q_2^2)\right)=\\=\frac{\omega^2}{4}(p_1^2+q_1^2+p_2^2+q_2^2)^2=H^2
        \end{multline}
        \begin{equation}
            \boxed{H^2=\omega^2(I_1^2+I_2^2+I_3^2)}
        \end{equation}
    \end{itemize}
\end{enumerate}
\section{Vector fields, Lie groups actions, coadjoint orbits.}
\begin{enumerate}
    \item \textbf{Lie derivative.}\\
    Consider a vector field $v$ on a smooth manifold $M$ and define two operations on differential forms on $M$: contraction
    \begin{equation}
        i_v:\Omega^{n}(M)\rightarrow\Omega^{n-1}(M),\; i_v\lambda=\lambda(v,\cdot,\cdot,...)
    \end{equation}
    and Lie derivative
    \begin{equation}
        \mathcal{L}_v:\Omega^n(M)\rightarrow \Omega^{n}(M), \; \mathcal{L}_v\lambda=\frac{d}{dt}(\exp(vt)^*\lambda)\big|_{t=0}
    \end{equation}
\begin{itemize}
    \item  Show that these operations satisfy the following properties:
    \begin{equation}
        \mathcal{L}_v=di_v+i_vd,
    \end{equation}
    \begin{equation}
        \mathcal{L}_{[v,u]} = [\mathcal{L}_{v},\mathcal{L}_{u}],
    \end{equation}
    \begin{equation}
        [\mathcal{L}_v, i_u] =i_{[v,u]}.
    \end{equation}
    \item Show also that Lie derivative is a derivation with respect to contraction, i.e.
    \begin{equation}
        \mathcal{L}_{v}\lambda(v_1, . . . ,v_k) = (\mathcal{L}_{v}\lambda)(v_1, . . . , v_k) +\sum_{i=1}^k\limits\lambda(v_1, . . . , [v,v_i], . . . , v_k)
    \end{equation}
    \item Let $v_1, v_2, v_3$ be symplectic vector fields which conserve the symplectic form $\omega$. Show that
    \begin{equation}
        \omega([v_1,v_2], v_3) = -\mathcal{L}_{v_3} \omega(v_1, v_2),
    \end{equation}
    \begin{equation}
        \omega([v_1, v_2],v_3) + \omega([v_2, v_3], v_1) +\omega([v_3,v_1],v_2) = 0,
    \end{equation}
    \begin{equation}
        \mathcal{L}_{v_1} \omega(v_2, v_3) + \mathcal{L}_{v_2} \omega(v_3, v_1) + \mathcal{L}_{v_3} \omega(v_1, v_2) = 0.
    \end{equation}
    \end{itemize}
    \textbf{Solution.}\\
    Lie derivative:
    \begin{multline}
        (\mathcal{L}_vT)^{j_1...j_s}_{i_1...i_q}=\xi^k\frac{\partial T^{j_1...j_s}_{i_1...i_q}}{\partial x^k}-T^{k j_2...j_s}_{i_1...i_q}\;\frac{\partial \xi^{j_1}}{\partial x^k}-...-T^{j_1...j_{s-1}k}_{i_1...i_q}\;\frac{\partial \xi^{j_s}}{\partial x^{k}}+T^{j_1...j_s}_{k i_2...i_q}\;\frac{\partial \xi^{k}}{\partial x^{i_1}}+...+\\+T^{j_1...j_{s}}_{i_1...i_{q-1}k}\;\frac{\partial \xi^{k}}{\partial x^{i_q}}
    \end{multline}  
    \begin{itemize}
        \item
        \begin{itemize}
        \item Show Cartan identity by induction.\\
        Let $\lambda=f\in C^\infty(M)$ and $v=\xi^i\frac{\partial}{\partial x^i}$, then
        \begin{equation}
            \mathcal{L}_v f =\xi^i \frac{\partial f}{\partial x^i} =v(f)=df(v) =i_vdf
        \end{equation}
        Let $\lambda=dx^i$. Lie derivative commutates with external differential:
        \begin{equation}
            d\mathcal{L}_v=\mathcal{L}_vd\rightarrow \mathcal{L}_v\lambda=\mathcal{L}_vdx^i=d\mathcal{L}_vx^i=di_vdx^i=di_v\lambda
        \end{equation}
        \begin{equation}
            d\lambda=d^2x^i=0
        \end{equation}
        Show that if Cartan identity is true for differential forms $\alpha\in\Omega^s(M)$ and $\beta\in\Omega^p(M)$:
        \begin{equation}
            \mathcal{L}_v\alpha=i_vd\alpha+di_v\alpha,\quad\mathcal{L}_v\beta=i_vd\beta+di_v\beta,
        \end{equation}
        then this identity is true for $\omega=\alpha\wedge\beta$.
        Leibniz rules for external differential and contraction:
        \begin{equation}
            d\omega=d\alpha\wedge\beta+(-1)^s\alpha\wedge d\beta,\quad i_v\omega=i_v\alpha\wedge\beta+(-1)^s\alpha\wedge i_v\beta
        \end{equation}
        \begin{multline}
            i_v d\omega=i_v(d\alpha\wedge\beta+(-1)^s\alpha\wedge d\beta)=i_vd\alpha\wedge\beta + (-1)^{s-1}d\alpha\wedge i_v \beta+\\+ (-1)^s i_v\alpha \wedge d\beta+\alpha \wedge i_vd\beta
        \end{multline}
        \begin{multline}
            di_v \omega=d(i_v\alpha\wedge\beta+(-1)^s\alpha\wedge i_v\beta)=di_v\alpha\wedge\beta+(-1)^{s-1}i_v\alpha\wedge d\beta+\\+(-1)^sd\alpha\wedge i_v\beta+\alpha\wedge di_v\beta
        \end{multline}
        \begin{equation}
            i_vd\omega+di_v\omega=i_vd\alpha\wedge\beta+di_v\alpha\wedge\beta+\alpha \wedge i_vd\beta+\alpha\wedge di_v\beta=\mathcal{L}_v\alpha\wedge\beta+\alpha\wedge\mathcal{L}_v\beta
        \end{equation}
        \begin{equation}
            i_vd\omega+di_v\omega=\mathcal{L}_v\omega
        \end{equation}
        Thus, Cartan identity is true for $\lambda\in\Omega^n(M)$:
        \begin{equation}
            \boxed{\mathcal{L}_v=di_v+i_vd}
        \end{equation}
        \item Show, that $\mathcal{L}_{[v,u]} = [\mathcal{L}_{v},\mathcal{L}_{u}]$ by induction.\\
        Let $\lambda=f\in C^\infty(M)$, then $\mathcal{L}_vf=v(f)$.
        \begin{equation}
            \mathcal{L}_{[v,u]}f=[v,u](f)=v(u(f))-u(v(f))
        \end{equation}
        \begin{equation}
            [\mathcal{L}_{v},\mathcal{L}_{u}]f=(\mathcal{L}_{v}\mathcal{L}_{u}-\mathcal{L}_{u}\mathcal{L}_{v})f=\mathcal{L}_vu(f)-\mathcal{L}_uv(f)=v(u(f))-u(v(f))
        \end{equation}
        \begin{equation}
            \mathcal{L}_{[v,u]}f=[\mathcal{L}_{v},\mathcal{L}_{u}]f
        \end{equation}
        Let $\omega=df\in\Omega^1(M)$, then
        \begin{equation}
            \mathcal{L}_{[v,u]}df=d\mathcal{L}_{[v,u]}f=d([v,u](f))=d(v(u(f)))-d(u(v(f)))
        \end{equation}
        \begin{multline}
            [\mathcal{L}_v,\mathcal{L}_u]df=(\mathcal{L}_v\mathcal{L}_u-\mathcal{L}_u\mathcal{L}_v)df=\mathcal{L}_vd(\mathcal{L}_uf)-\mathcal{L}_ud(\mathcal{L}_vf)=\mathcal{L}_vd(u(f))-\mathcal{L}_ud(v(f))=\\=d\mathcal{L}_v(u(f))-d\mathcal{L}_u(v(f))=d(v(u(f)))-d(u(v(f)))
        \end{multline}
        \begin{equation}
            \mathcal{L}_{[v,u]}df=[\mathcal{L}_v,\mathcal{L}_u]df
        \end{equation}
        Let $w$ is a vector field, then
        \begin{equation}
            \mathcal{L}_vw=[v,w]
        \end{equation}
        \begin{multline}
            [\mathcal{L}_u,\mathcal{L}_v]w=(\mathcal{L}_u\mathcal{L}_v-\mathcal{L}_v\mathcal{L}_u)w=\mathcal{L}_u[v,w]-\mathcal{L}_v[u,w]=[u,[v,w]]-[v,[u,w]]=\\=[u,[v,w]]+[v,[w,u]]=-[w,[u,v]]=[[u,v],w]=\mathcal{L}_{[u,v]}w
        \end{multline}
        Show, that identity is true for 2 arbitrary tensors $\alpha$ and $\beta$:
        \begin{equation}
            \mathcal{L}_{[u,v]}\alpha =[\mathcal{L}_u,\mathcal{L}_v]\alpha,\quad \mathcal{L}_{[u,v]}\beta =[\mathcal{L}_u,\mathcal{L}_v]\beta,
        \end{equation}
        then this identity is true for $\omega=\alpha\otimes\beta$.
        We will now show that our formula is correct for the tensor $\alpha\otimes\beta$. Use Leibniz rule:
        \begin{multline}
            \mathcal{L}_u\mathcal{L}_v (\alpha \otimes \beta)=\mathcal{L}_u(\mathcal{L}_v \alpha \otimes \beta+\alpha \otimes \mathcal{L}_v \beta)=\\=\mathcal{L}_u\mathcal{L}_v \alpha \otimes \beta+\mathcal{L}_u\alpha\otimes \mathcal{L}_v\beta+\mathcal{L}_v\alpha\otimes \mathcal{L}_u\beta+\alpha\otimes\mathcal{L}_u\mathcal{L}_v\beta
        \end{multline}
        \begin{multline}
            [\mathcal{L}_u,\mathcal{L}_v](\alpha\otimes\beta)=[\mathcal{L}_u,\mathcal{L}_v]\alpha\otimes\beta+\alpha\otimes[\mathcal{L}_u,\mathcal{L}_v]\beta=\mathcal{L}_{[u,v]}\alpha\otimes\beta+\alpha\otimes\mathcal{L}_{[u,v]}\beta=\\=\mathcal{L}_{[u,v]}(\alpha\otimes\beta)
        \end{multline}
        Thus, this identity is true for all tensors:
        \begin{equation}
            \boxed{\mathcal{L}_{[v,u]} = [\mathcal{L}_{v},\mathcal{L}_{u}]}
        \end{equation}
        \item Show, that $[\mathcal{L}_v, i_u] =i_{[v,u]}$ by induction.\\
        For a function $f\in C^1(M,\mathbb{R})$:
        \begin{equation}
            [\mathcal{L}_v, i_u]f =\mathcal{L}_v i_u f-i_u\mathcal{L}_v f=0,\quad i_{[v,u]}f=0
        \end{equation}
        For a $1$-form $\alpha$:
        \begin{equation}
            \mathcal{L}_vi_u\alpha=\mathcal{L}_v(\alpha(u))=v(\alpha(u)),\;i_u\mathcal{L}_v\alpha=i_u di_v\alpha+i_ui_vd\alpha=u(\alpha(v))+d\alpha(v,u)
        \end{equation}
        \begin{equation}
            d\alpha(v,u)=v(\alpha(u))-u(\alpha(v))-\alpha([v,u])
        \end{equation}
        \begin{equation}
            [\mathcal{L}_v,i_u]\alpha=\alpha([v,u])
        \end{equation}
        \begin{equation}
            i_{[v,u]}\alpha=\alpha([v,u])
        \end{equation}
        \begin{equation}
            [\mathcal{L}_v,i_u](\omega\wedge \eta)=[\mathcal{L}_v,i_u]\omega\wedge\eta+(-1)^k\omega\wedge[\mathcal{L}_v,i_u]\eta,
        \end{equation}
        where $\omega$ is assumed to be a $k$-form, and $\eta$ is an arbitrary form (follows from Leibniz's rules for the Lie derivative and the contraction).
        \begin{equation}
            \boxed{[\mathcal{L}_v, i_u] =i_{[v,u]}}
        \end{equation}
    \end{itemize}
    \item Show, that
    \begin{equation}
        \mathcal{L}_{v}\lambda(v_1, . . . ,v_k) = (\mathcal{L}_{v}\lambda)(v_1, . . . , v_k) +\sum_{i=1}^k\limits\lambda(v_1, . . . , [v,v_i], . . . , v_k)
    \end{equation}
    \begin{multline}
        \mathcal{L}_{v}\lambda(v_1,...,v_k)=\mathcal{L}_{v}(i_{v_1}...i_{v_k}\lambda)=\\=(i_{[v,v_1]}+i_{v_1}\mathcal{L}_{v})(i_{v_2}...i_{v_k}\lambda)=(i_{[v,v_1]}i_{v_2}...i_{v_k}\lambda)+i_{v_1}(\mathcal{L}_{v})(i_{v_2}...i_{v_k}\lambda)=\\=(i_{[v,v_1]}i_{v_2}...i_{v_k}\lambda)+i_{v_1}(i_{[v,v_2]}+i_{v_2}\mathcal{L}_{v})(i_{v_3}...i_{v_k}\lambda)=\\=...=(i_{v_1},...,i_{v_k})(\mathcal{L}_{v}\lambda)+\sum_{i=1}^k\limits(i_{v_1}...i_{[v,v_j]}...i_{v_k}\lambda)=\\=(\mathcal{L}_{v}\lambda)(v_1, ... , v_k)+\sum_{i=1}^k\limits\lambda(v_1, ... , [v,v_i], ... , v_k)
    \end{multline}
    \item
    \begin{itemize}
        \item Show, that $\omega([v_1,v_2], v_3) = -\mathcal{L}_{v_3} \omega(v_1, v_2)$ for symplectic $\omega$.
        \begin{equation}
            \mathcal{L}_{v_3}\omega=0
        \end{equation}
        \begin{multline}
            \mathcal{L}_{v_3} \omega(v_1, v_2)=(\mathcal{L}_{v_3}\omega)(v_1, v_2) +\omega([v_3,v_1],v_2)+\omega(v_1,[v_3,v_2])=\\=\omega([v_3,v_1],v_2)+\omega([v_2,v_3],v_1)
        \end{multline}
        Now we use identity from the next --:
        \begin{equation}
            \omega([v_1, v_2],v_3) + \omega([v_2, v_3], v_1) +\omega([v_3,v_1],v_2) = 0
        \end{equation}
        \begin{equation}
            \boxed{\mathcal{L}_{v_3} \omega(v_1, v_2)=-\omega([v_1,v_2], v_3)}
        \end{equation}
        \item $v_i$ is a symplectic vector field, so
        \begin{equation}
            \mathcal{L}_{v_i}\omega=0
        \end{equation}
        $\omega$ is a symplectic form, so $d\omega=0$.
        \begin{multline}
            d\omega(v_1,v_2,v_3)=v_1(\omega(v_2,v_3))-v_2(\omega(v_1,v_3))+v_3(\omega(v_1,v_2))-\\-\omega([v_1,v_2],v_3)+\omega([v_1,v_3],v_2)-\omega([v_2,v_3],v_1)=0
        \end{multline}
        \begin{multline}
            v_1(\omega(v_2,v_3))=\mathcal{L}_{v_1}(\omega(v_2,v_3))=(\mathcal{L}_{v_1}\omega)(v_2,v_3)+\omega([v_1,v_2],v_3)+\omega(v_2,[v_1,v_3])=\\=\omega([v_1,v_2],v_3)+\omega([v_3,v_1],v_2)
        \end{multline}
        \begin{multline}
            v_2(\omega(v_1,v_3))=\mathcal{L}_{v_2}(\omega(v_1,v_3))=(\mathcal{L}_{v_2}\omega)(v_1,v_3)+\omega([v_2,v_1],v_3)+\omega(v_1,[v_2,v_3])=\\=-\omega([v_1,v_2],v_3)-\omega([v_2,v_3],v_1)
        \end{multline}
        \begin{multline}
            v_3(\omega(v_1,v_2))=\mathcal{L}_{v_3}(\omega(v_1,v_2))=(\mathcal{L}_{v_3}\omega)(v_1,v_2)+\omega([v_3,v_1],v_2)+\omega(v_1,[v_3,v_2])=\\=\omega([v_3,v_1],v_2)+\omega([v_2,v_3],v_1)
        \end{multline}
        \begin{equation}
            \boxed{\omega([v_1, v_2],v_3) + \omega([v_2, v_3], v_1) +\omega([v_3,v_1],v_2) = 0}
        \end{equation}
        \item
        \begin{multline}
            \mathcal{L}_{v_1} \omega(v_2, v_3) + \mathcal{L}_{v_2} \omega(v_3, v_1) + \mathcal{L}_{v_3} \omega(v_1, v_2) =\\= -\omega([v_2,v_3], v_1)-\omega([v_1,v_3], v_2)-\omega([v_1,v_2], v_3)
        \end{multline}
        \begin{equation}
            \boxed{\mathcal{L}_{v_1} \omega(v_2, v_3) + \mathcal{L}_{v_2} \omega(v_3, v_1) + \mathcal{L}_{v_3} \omega(v_1, v_2)=0}
        \end{equation}
    \end{itemize}
    \end{itemize}
    \item Let $M$ be a smooth manifold and $T^*M$ -- its cotangent bundle equipped with canonical symplectic form $\omega =d\alpha.$ Let $v$ be a vector field on M, and $\Tilde{v}$ a vector field on $T^*M$ which lifts the flow of $v$
    \begin{equation}
        \exp(\Tilde{v}t)(x,\beta) = (\exp(vt)x, \exp(vt)_{*}\beta)
    \end{equation}
    \begin{itemize}
        \item Find the expression for the vector field $\Tilde{v}$ in local coordinates $p, q$ for the vector field $v =\sum_{i}\limits v_i\frac{\partial}{\partial q_i}.$
        \item Show that the flow of the vector field $\Tilde{v}$ preserves the Liouville 1-form and the symplectic form
        \[\mathcal{L}_{\Tilde{v}}\alpha = 0,\;\mathcal{L}_{\Tilde{v}}\omega= 0.\]
        \item Show that this vector field is Hamiltonian with $H=i_{\Tilde{v}} \alpha=\sum_{i}p_iv_i(x)$.
        \item Consider a Lie group $G$ acting on the manifold $M$ as
        \[G\times M \rightarrow M: (g, x) \mapsto  g.x\]
        this action can be naturally lifted to the action of $G$ on the cotangent bundle $T^*M$
        \[G \times T^*M \rightarrow T^*M: (g,(x,\beta)) \mapsto (g.x, g_*\beta).\]
        Show that the lifted action is Hamiltonian and find the momentum map.
    \end{itemize}
    \textbf{Solution.}
    \begin{itemize}
        \item Infinitesimal form:
        \begin{equation}
            \Tilde{v}(q,p_idq^i)=(q(0)+t v(x)+\mathcal{O}(t), p_i(0) d(q^i(0)-tv^i(x)+\mathcal{O}(t)))
        \end{equation}
        \begin{equation}
            q^i(t)\approx q^i(0)+tv^i(x) \rightarrow \frac{\partial q^i(t)}{\partial t}=v^i(x)
        \end{equation}
        \begin{equation}
            p_i d\left(q^i(0)-tv^i(x)+\mathcal{O}(t)\right)\approx \left(p_i(0)-t p_k\frac{\partial v^k(x)}{\partial q^i}\right) dq^i
        \end{equation}
        \begin{equation}
            p_i(t)=p_i(0)-t p_k(0)\frac{\partial v^k(x)}{\partial q^i}\rightarrow \frac{\partial p_i(t)}{\partial t}=-p_k(0)\frac{\partial v^k(x)}{\partial q^i}
        \end{equation}
        \begin{multline}
            \Tilde{v} (f)=\frac{df(x,\beta)}{dt}=\frac{\partial f}{\partial q^i}\frac{\partial q^i(t)}{\partial t}+\frac{\partial f}{\partial p_i}\frac{\partial p_i(t)}{\partial t}=\frac{\partial f}{\partial q^i}\frac{\partial q^i(t)}{\partial t}+\frac{\partial f}{\partial p_i}\frac{\partial p_i(t)}{\partial t}=\\=\left(v^i(x)\frac{\partial}{\partial q^i}+\left(-p_k(0)\frac{\partial v^k(x)}{\partial q^i}\right)\frac{\partial}{\partial p_i}\right)f=\left(\Tilde{v}^i\frac{\partial}{\partial q^i}+\Tilde{v}_{i+n}\frac{\partial}{\partial p_i}\right)f
        \end{multline}
        \begin{equation}
            \boxed{\Tilde{v}=v^i \frac{\partial}{\partial q^i}-p_j \frac{\partial v^j}{\partial q^i} \frac{\partial}{\partial p_i}}
        \end{equation}
    \item Liouville 1-form:
    \begin{equation}
        \alpha=p_i dq^i\rightarrow\omega = d p_i \wedge d q^i
    \end{equation}
    \begin{multline}
        \mathcal{L}_{\Tilde{v}}\alpha =(d i_{\Tilde{v}}+i_{\Tilde{v}}d)\alpha=d (p_j v^i \delta^j_i)-v^j \delta_j^i dp_i + \left(-p^j \frac{\partial v_j}{\partial q^i}\right)dq^i=\\=v^i dp_i + p_i \frac{\partial v^i}{\partial q^j}   dq^j-v^i dp_i - p^j \frac{\partial v_j}{\partial q^i} dq^i=0
    \end{multline}
    \begin{multline}
        \mathcal{L}_{\Tilde{v}}\omega=(d i_{\Tilde{v}}+i_{\Tilde{v}}d)\omega=d i_{\Tilde{v}}\omega = di_{\Tilde{v}}(dp_i\wedge dq^i) =\\= d\left(-v^i dp_i+\left(-p^j \frac{\partial v_j}{\partial q^i}\right)dq^i\right)=-dv^i \wedge dp_i - \frac{\partial v_j}{\partial q^i} dp^j\wedge dq^i-p^jd\left( \frac{\partial v_j}{\partial q^i} dq^i\right)=\\=-dv^i \wedge dp_i + dv_j\wedge dp^j-p^jd(dv_j) =0
    \end{multline}
    \[\;.\]
    \item
    \begin{equation}
        i_{\Tilde{v}}\omega= -v^i dp_i+\left(-p^j \frac{\partial v_j}{\partial q^i}\right)dq^i
    \end{equation}
    \begin{equation}
        dH=v^i dp_i+p_i \frac{\partial v^i}{\partial q^j}dq^j=v^i dp_i+p^j \frac{\partial v_j}{\partial q^i}dq^i
    \end{equation}
    \begin{equation}
        i_{\Tilde{v}}\omega=-dH
    \end{equation}
    As seen, vector field $\tilde{v}$  is Hamiltonian with $H=i_{\Tilde{v}} \alpha=\sum_{i}p_iv_i(q).$
\end{itemize}
\item
\item \textbf{Coadjoint orbits of $GL(N)$.}\\
Consider the coadjoint orbits of $GL(N)$ passing through the diagonal element
\begin{equation}
    S=g^{-1}\Lambda g,\quad\Lambda=(\lambda_1,...,\lambda_N)
\end{equation}
\begin{itemize}
    \item Find the dimension of the coadjoint orbit for $\lambda_i\neq\lambda_j$ for all $i\neq j$.
    \item Find the dimension of the coadjoint orbit for the diagonal element
    \begin{equation}
        \Lambda=\text{diag}(\underbrace{\mu_1, ... , \mu_1}_{n_1},...,\underbrace{\mu_k, ... , \mu_k}_{n_k})
    \end{equation}
    Which nontrivial orbit has minimal dimension?
    \item  Deduce that from canonical Poisson brackets
    \begin{equation}
        \{\xi_{i\alpha},\eta_{j\beta}\} = \delta_{ij}\delta_{\alpha\beta},\; \{\xi_{i\alpha},\xi_{j\beta}\} = 0, \;\{\eta_{i\alpha},\eta_{j\beta}\} = 0, i,j = 1,..., N,\;\alpha,\beta = 1, ..., K
    \end{equation}
    follows the Poisson–Lie brackets for $\mathfrak{gl}^*(N)$ between elements
    \begin{equation}
        f_{S_{ij}}=\sum_{\alpha=1}^{K}\limits \xi_{i\alpha}\eta_{j\alpha}
    \end{equation}
\end{itemize}
\textbf{Solution.}
\begin{itemize}
    \item Coadjoint orbit:
    \begin{equation}
        O_\Lambda=\{\text{Ad}^*_g(\Lambda)=g^{-1}\Lambda g|g\in GL(N)\}
    \end{equation}
    Let find the stabilizer of matrices with $\lambda_i\neq\lambda_j$ for all $i\neq j$. We should find matices $g$:
    \begin{equation}
        \Lambda=g^{-1}\Lambda g\rightarrow g\Lambda=\Lambda g
    \end{equation}
    Commutating matrices have one set of the eigenvectors. Eigenvectors of the $\Lambda$:\\
    $\{(1,0,...,0)^T,(0,1,...,0)^T,...,(0,0,...,1)^T\}$. So, $g$ is a diagonal matrix. The dimension of diagonal matrices:
    \begin{equation}
        \text{dim}(Stab)=N
    \end{equation}
    Dimension of $GL(N)$:
    \begin{equation}
        \text{dim}(GL(N))=N^2
    \end{equation}
    Dimension of orbit:
    \begin{equation}
        \boxed{\text{dim}(Orb)=N^2-N}
    \end{equation}
    \item The eigenspace associated with a block with $\mu_i$ is a vector space of dimension $n_i$. After any action on $\lambda$ we must not leave the corresponding vector space. This is true for any $i$. Therefore we have that the stabilizer $\lambda$ consists only of block matrices, where the blocks have the corresponding size
    \begin{equation}
        g = \left(\begin{array}{@{}cccc@{}}
		\cline{1-1}
		\multicolumn{1}{|c|}{G_1} & 0 & \dots & 0\\
		\cline{1-2}
		0 & \multicolumn{1}{|c|}{G_2} & \dots & 0\\
		\cline{2-2}
		\vdots & \vdots & \ddots & \vdots\\
		\cline{4-4}
		0 & 0 & \dots & \multicolumn{1}{|c|}{G_k}\\
		\cline{4-4}
	\end{array}\right),
    \end{equation}
    where $G_i$ is a matrix of size $n_i \times n_i:\sum\limits_{i=1}^kn_i=N$. The dimension of such matrices is 
    \begin{equation}
        \text{dim}(Stab) = \sum_{i=1}^{k}n_i^2
    \end{equation}
    \begin{equation}
        \boxed{\text{dim}(Orb)=N^2-\sum_{i=1}^k\limits n_i^2}
    \end{equation}
    In case $k=1$:
    \begin{equation}
        \sum\limits_{i=1}^1n_i^2=n_1^2,\quad \sum\limits_{i=1}^1n_i=n_1=N\rightarrow\text{dim}(Orb)=0
    \end{equation}
    So, the trivial orbit, consisting from the identity matrix, has 0 dimension. To find a nontrivial orbit ($k\geq2$) with minimal dimension we should maximize $\sum\limits_{i=1}^kn_i^2$ with $\sum\limits_{i=1}^kn_i=N$.\\
    Let $S_k=\sum\limits_{i=1}^kn_i^2$. Compare $S_k$ with $S_{k-1}$
    \begin{equation}
        S_k=\sum_{i=1}^{k}n_i^2,\quad S_{k-1}=\sum_{i=1}^{k-1}m_i^2
    \end{equation}
    Let $m_i=n_i$ for $i<k-1$ and $m_{k-1}=n_{k-1}+n_{k},$ than
    \begin{equation}
        S_{k-1}-S_{k} = 2n_{k-1}n_{k}>0
    \end{equation}
    For any set $\{n_1,...n_k\}$ find a set $\{m_1,...,m_{k-1}\}$:
    \begin{equation}
        S_{k-1}(\{m_1,...,m_{k-1}\})>S_{k}(\{n_1,...n_k\})
    \end{equation}
    We take $k=2$ and $n_1=1$, $n_2=N-1$:
    \begin{equation}
        S_2=1+(N-1)^2
    \end{equation}
    \begin{equation}
        \boxed{\text{dim}(Orb)=N^2-(1+(N-1)^2)=2N-2}
    \end{equation}
    \item 
    \begin{multline}
        \{f_{S_{ij}}, f_{S_{kl}}\}=\{\xi_{i\alpha}\eta_{j}^{\alpha},\xi_{k\beta}\eta_{l}^{\beta}\}=\xi_{k\beta}\{\xi_{i\alpha}\eta_{j}^{\alpha},\eta_{l}^{\beta}\}+\{\xi_{i\alpha}\eta_{j}^{\alpha},\xi_{k\beta}\}\eta_{l}^{\beta}=\\=\xi_{k\beta}\xi_{i\alpha}\{\eta_{j}^{\alpha},\eta_{l}^{\beta}\}+\xi_{k\beta}\{\xi_{i\alpha},\eta_{l}^{\beta}\}\eta_{j}^{\alpha}+\xi_{i\alpha}\{\eta_{j}^{\alpha},\xi_{k\beta}\}\eta_{l}^{\beta}+\{\xi_{i\alpha},\xi_{k\beta}\}\eta_{j}^{\alpha}\eta_{l}^{\beta}=\\=\xi_{k\beta}\eta_{j}^{\alpha}\delta_{il}\delta_{\alpha}^{\beta}-\xi_{i\alpha}\eta_{l}^{\beta}\delta_{jk}\delta^{\alpha}_{\beta}=\xi_{k\alpha}\eta_{j}^{\alpha}\delta_{il}-\xi_{i\alpha}\eta_{l}^{\alpha}\delta_{jk}=f_{S_{kj}}\delta_{il}-f_{S_{il}}\delta_{jk}
    \end{multline}
    $S_{ij}$ are elements of the algebra $\mathfrak{gl}(N)$ or an equally likely linear function on $\mathfrak{gl}^*(N)$.
    If we take the differential of this function by definition
    \begin{equation}
        f_{S_{ij}}(\xi + \Delta \xi) = f_{S_{ij}}(\xi) + \langle \Delta \xi, df_{S_{ij}}(\xi)
    \end{equation}
    On the other hand,
    \begin{equation}
        f_{S_{ij}}(\xi + \Delta \xi) = \langle \xi + \Delta \xi, S_{ij} \rangle = \langle \xi, S_{ij} \rangle + \braket{\Delta \xi, S_{ij}}
    \end{equation}
    We obtain \[df_{S_{ij}}(\xi) = S_{ij}\in\mathfrak{gl}(N).\]
    Therefore,
    \begin{equation}
        \{f_{S_{ij}}, f_{S_{kl}}\}(\xi) = \langle \xi, [S_{ij}, S_{kl}] \rangle = \langle \xi, S_{kj}\rangle \delta_{il}-\langle \xi, S_{il} \rangle \delta_{kj}
    \end{equation}
    \begin{equation}
        d(\{f_{S_{ij}}, f_{S_{kl}}\}(\xi))=S_{kj} \delta_{il}-S_{il} \delta_{kj}
    \end{equation}
    \begin{equation}
        \boxed{\{f_{S_{ij}}, f_{S_{kl}}\} = f_{S_{kj}} \delta_{il}-f_{S_{il}} \delta_{kj}}
    \end{equation}
\end{itemize}
\item
\item \textbf{Coadjoint orbits of $SL(2,\mathbb{R})$.}\\
Consider a group $G=SL(2,\mathbb{R})$ defined as
\begin{equation}
    G=\left\{\begin{pmatrix}
        a & b\\
        c & d
    \end{pmatrix}:ad-bc=1, a,b,c,d\in\mathbb{R}\right\}
\end{equation}
and its Lie algebra
\begin{equation}
    \mathfrak{g}=\left\{\begin{pmatrix}
        x & y-z\\
        y+z & -x
    \end{pmatrix}: x,y,z\in\mathbb{R}\right\}
\end{equation}
One can also identify a dual space $\mathfrak{g}^*\cong\mathfrak{g}$ via the pairing $\braket{\phi,X}=\text{Tr}(\phi X)$.
\begin{itemize}
    \item Show that a generic coadjoint orbit of $G$ can be identified with a level set of the function
    \begin{equation}
        f(x,y,z)=x^2+y^2-z^2
    \end{equation}
    \item How many coadjoint orbits of $G$ are contained in the singular level set $f(x, y, z) = 0$?
    \item Show that symplectic forms on coadjoint orbits of $G$ in cylindrical coordinates can be presented in the form
    \begin{equation}
        \omega=dz\wedge d\theta,\quad x=\rho\cos\theta,\quad y=\rho\sin\theta
    \end{equation}
\end{itemize}
\textbf{Solution.}
\begin{itemize}
    \item The basis of the Lie algebra $\mathfrak{sl}(2,\mathbb{R})$ is consists of:
    \begin{equation}
        \sigma_x=\begin{pmatrix}
        0 & 1 \\
        1 & 0
    \end{pmatrix},\;
    \sigma_y=
    \begin{pmatrix}
        0 & 1 \\
        -1 & 0
    \end{pmatrix},\;
    \sigma_z=
    \begin{pmatrix}
        1 & 0 \\
        0 & -1
    \end{pmatrix}
    \end{equation}
    \begin{equation}
        \begin{pmatrix}
            x & y-z \\
            y+z & -x
    \end{pmatrix}=y\sigma_x-z\sigma_y+x\sigma_z
    \end{equation}
    Let $\sigma^*_x$, $\sigma^*_y$, $\sigma^*_z$ is the dual basis of $\mathfrak{sl}^*(2,\mathbb{R})$. Isomorphism $\varphi:\mathfrak{g}\rightarrow\mathfrak{g}^*$ maps $\varphi(\sigma_x)=\sigma^*_x$, $\varphi(\sigma_y)=-\sigma^*_y$, $\varphi(\sigma_z)=\sigma^*_z$.\\
    Consider 
    \begin{equation}
        S=\begin{pmatrix}
            a & b \\
            c & d
        \end{pmatrix} \in SL(2,\mathbb{R})
    \end{equation}
    Every matrix $A \in \mathfrak{sl}(2, \mathbb{R})$ can be reduced to one of the following normal forms ($\lambda>0$):
    \begin{enumerate}
        \item
        \begin{equation}
            \begin{pmatrix}
            x & y-z \\
            y+z & -x
        \end{pmatrix}=
        S \begin{pmatrix}
            0 & \lambda \\
            \lambda & 0
        \end{pmatrix}  S^{-1}= 
        \begin{pmatrix}
            \lambda  (bd-ac) & \lambda(a^2-b^2) \\
            -\lambda(c^2-d^2) & -\lambda(bd-ac) \\
    \end{pmatrix}
    \end{equation}
    \begin{equation}
        \begin{cases}
            x=\lambda(bd-ac),\\
            y-z=\lambda(a^2-b^2),\\
            y+z=-\lambda(c^2-d^2);
        \end{cases}\rightarrow\begin{cases}
            x=\lambda(bd-ac),\\
            y=\frac{\lambda}{2}(a^2-b^2-c^2+d^2),\\
            z=\frac{\lambda}{2}(b^2-a^2+d^2-c^2).
        \end{cases}
    \end{equation}
    \begin{equation}
        f(x,y,z)=x^2+y^2-z^2=\lambda^2(bc-ad)^2=\lambda^2
    \end{equation}
    Coadjoint orbit:
    \begin{equation}
        \mathcal{O}_{\lambda\sigma^*_x}=\{y\sigma^*_x+z\sigma^*_y+x\sigma^*_z|x^2+y^2-z^2=\lambda^2\}
    \end{equation}
    It's a hyperboloid of 2 sheet (elliptic hyperboloid).
    \item
        \begin{equation}
            \begin{pmatrix}
            x & y-z \\
            y+z & -x
        \end{pmatrix}=
        S \begin{pmatrix}
            0 & \lambda \\
            -\lambda & 0
        \end{pmatrix}  S^{-1}= 
        \begin{pmatrix}
            -\lambda(ac+bd) & \lambda(a^2+b^2) \\
            -\lambda(c^2+d^2) & \lambda(ac+bd) \\
        \end{pmatrix}
        \end{equation}
        \begin{equation}
            \begin{cases}
                x=-\lambda(ac+bd),\\
                y-z=\lambda(a^2+b^2)\geq0,\\
                y+z=-\lambda(c^2+d^2)\leq0;
            \end{cases}\rightarrow\begin{cases}
                x=-\lambda(ac+bd),\\
                y=\frac{\lambda}{2}(a^2+b^2-c^2-d^2),\\
                z=-\frac{\lambda}{2}(a^2+b^2+c^2+d^2)\leq0;
            \end{cases}
        \end{equation}
        \begin{equation}
            f(x,y,z)=x^2+y^2-z^2=-\lambda^2(bc-ad)^2=-\lambda^2
        \end{equation}
        Coadjoint orbit:
        \begin{equation}
            \mathcal{O}_{-\lambda\sigma^*_y}=\{y\sigma^*_x+z\sigma^*_y+x\sigma^*_z|x^2+y^2-z^2=\lambda^2|z\leq0\}
        \end{equation}
        It's a lower part of a hyperboloid of 1 sheet.
        \item \begin{equation}
            \begin{pmatrix}
            x & y-z \\
            y+z & -x
        \end{pmatrix}=
        S \begin{pmatrix}
            \lambda & 0 \\
            0 & -\lambda
        \end{pmatrix}  S^{-1}= 
        \begin{pmatrix}
            \lambda(ad+bc) & -2\lambda ab \\
            2\lambda cd & -\lambda(ad+bc) \\
    \end{pmatrix}
    \end{equation}
    \begin{equation}
        \begin{cases}
            x=\lambda(ad+bc),\\
            y-z=-2\lambda ab,\\
            y+z=2\lambda cd;
        \end{cases}\rightarrow\begin{cases}
            x=\lambda(ad+bc),\\
            y=\lambda(cd-ab),\\
            z=\lambda(ab+cd).
        \end{cases}
    \end{equation}
    \begin{equation}
        f(x,y,z)=x^2+y^2-z^2=\lambda^2(bc-ad)^2=\lambda^2
    \end{equation}
    Coadjoint orbit:
    \begin{equation}
        \mathcal{O}_{\lambda\sigma^*_z}=\{y\sigma^*_x+z\sigma^*_y+x\sigma^*_z|x^2+y^2-z^2=\lambda^2\}
    \end{equation}
    It's a hyperboloid of 2 sheet (elliptic hyperboloid).
    \item 
        \begin{equation}
            \begin{pmatrix}
            x & y-z \\
            y+z & -x
        \end{pmatrix}=
        S \begin{pmatrix}
            0 & 0 \\
            \lambda & 0
        \end{pmatrix}  S^{-1}= 
        \begin{pmatrix}
            \lambda bd & -\lambda b^2\\
            \lambda d^2 & -\lambda bd \\
        \end{pmatrix}
        \end{equation}
        \begin{equation}
            \begin{cases}
                x=\lambda bd,\\
                y-z=-\lambda b^2\leq0,\\
                y+z=\lambda d^2\geq0;
            \end{cases}\rightarrow\begin{cases}
                x=\lambda bd,\\
                y=\frac{\lambda}{2}(d^2-b^2),\\
                z=\frac{\lambda}{2}(b^2+d^2)\geq0.
            \end{cases}
        \end{equation}
        \begin{equation}
            f(x,y,z)=x^2+y^2-z^2=0
        \end{equation}
        Coadjoint orbit:
        \begin{equation}
            \{y\sigma^*_x+z\sigma^*_y+x\sigma^*_z|x^2+y^2-z^2=0|z\geq0\}
        \end{equation}
        It's a upper part of a cone.
    \item 
        \begin{equation}
            \begin{pmatrix}
            x & y-z \\
            y+z & -x
        \end{pmatrix}=
        S \begin{pmatrix}
            0 & \lambda\\
            0 & 0
        \end{pmatrix}  S^{-1}= 
        \begin{pmatrix}
            -\lambda ac & \lambda a^2\\
            -\lambda c^2 & \lambda ac\\
        \end{pmatrix}
        \end{equation}
        \begin{equation}
            \begin{cases}
                x=-\lambda ac,\\
                y-z=\lambda a^2\geq0,\\
                y+z=-\lambda c^2\leq0;
            \end{cases}\rightarrow\begin{cases}
                x=-\lambda ac,\\
                y=\frac{\lambda}{2}(a^2-c^2),\\
                z=-\frac{\lambda}{2}(a^2+c^2)\geq0.
            \end{cases}
        \end{equation}
        \begin{equation}
            f(x,y,z)=x^2+y^2-z^2=0
        \end{equation}
        Coadjoint orbit:
        \begin{equation}
            \{y\sigma^*_x+z\sigma^*_y+x\sigma^*_z|x^2+y^2-z^2=0|z\leq0\}
        \end{equation}
        It's a lower part of a cone.
    \item 
        \begin{equation}
            \begin{pmatrix}
            x & y-z \\
            y+z & -x
        \end{pmatrix}=
        S \begin{pmatrix}
            0 & 0\\
            0 & 0
        \end{pmatrix}  S^{-1}= 
        \begin{pmatrix}
            0 & 0\\
            0 & 0\\
        \end{pmatrix}
        \end{equation}
        \begin{equation}
            \begin{cases}
                x=0,\\
                y-z=0,\\
                y+z=0;
            \end{cases}\rightarrow\begin{cases}
                x=0,\\
                y=0,\\
                z=0.
            \end{cases}
        \end{equation}
        \begin{equation}
            f(x,y,z)=x^2+y^2-z^2=0
        \end{equation}
        Coadjoint orbit:
        \begin{equation}
            \{y\sigma^*_x+z\sigma^*_y+x\sigma^*_z|x=y=z=0\}
        \end{equation}
        It's a one point $(0,0,0)$.
\end{enumerate}
\item As seen above, the 3 conjugate orbits $SL(2,\mathbb{R})$ are contained in the singular level set $f(x,y,z)=0.$: an upper and a lower parts of a cone and one point $(0,0,0)$.
\item Konstant-Kirillov form:
\begin{equation}
    \omega=\omega_{\mu\nu}(x)dx^\mu\wedge dx^\nu,\quad\omega_{\mu\nu}=-\omega_{\nu\mu}
\end{equation}
\begin{equation}
    \{f,g\}=\frac{1}{2}\omega_{\mu\nu}\frac{\partial f}{\partial x^\mu}\frac{\partial g}{\partial x^\nu}
\end{equation}
Polar coordinates on hyperboloid:
\begin{equation}
    \begin{cases}
        \rho=\sqrt{x^2+y^2}=\sqrt{z^2+1},\\
        \tan\theta=\frac{y}{x}
    \end{cases}
\end{equation}
Poisson brackets:
\begin{equation}
    \{x_i,x_j\}=\epsilon_{ijk}x_k
\end{equation}
\begin{equation}
    \{z,\tan\theta\}=\left\{z,\frac{y}{x}\right\}=\frac{1}{x}\{z,y\}-\frac{y}{x^2}\{z,x\}=-\frac{1}{x}x-\frac{y}{x^2}y=-1-\tan^2\theta=-\frac{1}{\cos^2\theta}
\end{equation}
\begin{equation}
    \{f,g\}=\left(\frac{\partial f}{\partial z}\frac{\partial g}{\partial\theta}-\frac{\partial g}{\partial z}\frac{\partial f}{\partial\theta}\right)\{z,\theta\}
\end{equation}
\begin{equation}
    \{z,\tan\theta\}=-\frac{1}{\cos^2\theta}\{z,\theta\}\rightarrow\{z,\theta\}=1
\end{equation}
\begin{equation}
    \boxed{\omega=\frac{1}{2}dz\wedge d\theta-\frac{1}{2}d\theta\wedge dz=dz\wedge d\theta}
\end{equation}
\end{itemize}
\end{enumerate}
\section{Hamiltonian reduction, projection method}
\begin{enumerate}
    \item \textbf{Free particle?}\\
    Let $M$ be a four-dimensional phase space $T^*\mathbb{R}$ with the canonical symplectic structure
    \begin{equation}
        \omega=dp_1\wedge dq_1+dp_2\wedge dq_2
    \end{equation}
    Consider a system of a free particle with Hamilton function defined on this phase space
    \begin{equation}
        H=\frac{1}{2}(p_1^2+p_2^2)
    \end{equation}
    \begin{itemize}
        \item Show that the diagonal action of $G = U(1) = SO(2)$ on the planes $(q_1, q_2)$ and $(p_1, p_2)$ is Hamiltonian and find the momentum map $\mu(p, q)$ of this action.
        \item Check that the Hamilton function is invariant with respect to this action and find the corresponding Hamiltonian vector field $v_H$. Find integral curves $(q(t), p(t))$ of $v_H$ with initial conditions $(q(0), p(0))$.
        \item Show that for $\mu(p, q) = l\neq0$ the reduced phase space is $M_l\simeq\mathbb{R}_+\times\mathbb{R}$.
        \item Find the expression for the reduced Hamilton function $H_l$. Check that the projection $r(t)$, $p_r(t)$ of the integral curve found above is indeed an integral curve of the Hamiltonian vector field $v_{H_l}$. What is the mechanical interpretation of this result?
    \end{itemize}
    \textbf{Solution.}
    \begin{itemize}
        \item 
        \begin{equation}
            g=\begin{pmatrix}
                \cos\varphi & \sin\varphi\\
                -\sin\varphi & \cos\varphi
            \end{pmatrix}\in G=SO(2)
        \end{equation}
        \begin{equation}
            g.\begin{pmatrix}
                q_1\\
                q_2\\
                p_1\\
                p_2
            \end{pmatrix}=\begin{pmatrix}
                \cos\varphi & \sin\varphi & 0 & 0\\
                -\sin\varphi & \cos\varphi & 0 & 0\\
                0 & 0 & \cos\varphi & \sin\varphi\\
                0 & 0 & -\sin\varphi & \cos\varphi
            \end{pmatrix}\begin{pmatrix}
                q_1\\
                q_2\\
                p_1\\
                p_2
            \end{pmatrix}=\begin{pmatrix}
                q_1\cos\varphi+q_2\sin\varphi\\
                q_2\cos\varphi-q_1\sin\varphi\\
                p_1\cos\varphi+p_2\sin\varphi\\
                p_2\cos\varphi-p_1\sin\varphi
            \end{pmatrix}
        \end{equation}
        \begin{equation}
            \xi=\begin{pmatrix}
                0 & \varphi\\
                -\varphi & 0
            \end{pmatrix}\in\mathfrak{so}(2)
        \end{equation}
        \begin{equation}
            v_\xi(t)=\frac{d}{dt}(e^{\xi t}.x)|_{t=0}
        \end{equation}
        \begin{equation}
            v_\xi=q_2\frac{\partial}{\partial q_1}-q_1\frac{\partial}{\partial q_2}+p_2\frac{\partial}{\partial p_1}-p_1\frac{\partial}{\partial p_2}
        \end{equation}
        \begin{equation}
            \omega=dp_1\wedge dq_1+dp_2\wedge dq_2
        \end{equation}
        Check that action is weakly Hamiltonian:
        \begin{equation}
            i_{v_\xi}\omega=-q_2dp_1+q_1dp_2+p_2dq_1-p_1dq_2=-d(p_1q_2-p_2q_1)=-dH_\xi
        \end{equation}
        \begin{equation}
            H_\xi=p_1q_2-p_2q_1
        \end{equation}
        Check that the action is Hamiltonian:
        \begin{equation}
            \forall g\in SO(2)\forall\xi\in\mathfrak{so}(2)\hookrightarrow g_*H_\xi=H_{\text{Ad}_g(\xi)}=H_\xi
        \end{equation}
        \begin{multline}
            H_\xi=p_1q_2-p_2q_1\rightarrow(p_1\cos\varphi+p_2\sin\varphi)(q_2\cos\varphi-q_1\sin\varphi)-\\-(p_2\cos\varphi-p_1\sin\varphi)(q_1\cos\varphi+q_2\sin\varphi)=p_1q_2-p_2q_1
        \end{multline}
        Momentium map:
        \begin{equation}
            \boxed{\mu(p,q)=p_1q_2-p_2q_1}
        \end{equation}
        \item
        \begin{equation}
            H=\frac{1}{2}(p_1^2+p_2^2)\rightarrow\frac{1}{2}((p_1\cos\varphi+p_2\sin\varphi)^2+(p_2\cos\varphi-p_1\sin\varphi)^2)=\frac{1}{2}(p_1^2+p_2^2)
        \end{equation}
        Hamiltonian vector field $v_H$:
        \begin{equation}
            i_{v_H}\omega=-dH=-p_1dp_1-p_2dp_2
        \end{equation}
        \begin{equation}
            \boxed{v_H=p_1\frac{\partial}{\partial q_1}+p_2\frac{\partial}{\partial q_2}}
        \end{equation}
        \begin{equation}
            \begin{pmatrix}
                \dot{q}_1\\
                \dot{q}_2\\
                \dot{p}_1\\
                \dot{p}_2
            \end{pmatrix}=v_H\begin{pmatrix}
                q_1\\
                q_2\\
                p_1\\
                p_2
            \end{pmatrix}=\begin{pmatrix}
                p_1\\
                p_2\\
                0\\
                0
            \end{pmatrix}\rightarrow\boxed{\begin{cases}
                q_1(t)=p_1(0)t+q_1(0),\\
                q_2(t)=p_2(0)t+q_2(0),\\
                p_1(t)=p_1(0),\\
                p_2(t)=p_2(0).
            \end{cases}}
        \end{equation}
        \item
        \begin{equation}
            \mu(p, q)=p_1q_2-p_2q_1 = l\neq0
        \end{equation}
        \begin{equation}
            M_l=\mu^{-1}(l)/G_l,\quad G_l=\{g\in G=SO(2)|\text{Ad}^*_gl=l\}=G=SO(2)
        \end{equation}
        Polar coordinates:
        \begin{equation}
            \begin{cases}
                q_1=r\cos\varphi,\\
                q_2=r\sin\varphi
            \end{cases}
        \end{equation}
        \begin{multline}
            p_1dq_1+p_2dq_2=p_1(dr\cos\varphi-r\sin\varphi d\varphi)+p_2(dr\sin\varphi+r\cos\varphi d\varphi)=\\=dr(p_1\cos\varphi+p_2\sin\varphi)+rd\varphi(p_2\cos\varphi-p_1\sin\varphi)=p_rdr+p_\varphi d\varphi
        \end{multline}
        \begin{equation}
            \begin{cases}
                p_r=p_1\cos\varphi+p_2\sin\varphi,\\
                p_\varphi=p_2r\cos\varphi-p_1r\sin\varphi=p_2q_1-p_1q_2=-l
            \end{cases}
        \end{equation}
        \begin{equation}
            \mu^{-1}(l)=\{(q_1,p_1,q_2,p_2|p_1q_2-p_2q_1=l)\}=\{(r,p_r,\varphi,p_\varphi)|p_\varphi=-l\}
        \end{equation}
        \begin{equation}
            \boxed{M_l=\{(r,p_r,p_\varphi)|p_\varphi=-l\}\simeq\mathbb{R}_+\times\mathbb{R}}
        \end{equation}
        \item Find the expression for the reduced Hamilton function $H_l$.
        \begin{equation}
            \begin{cases}
                p_1=p_r\cos\varphi-\frac{p_\varphi\sin\varphi}{r},\\
                p_2=p_r\sin\varphi+\frac{p_\varphi\cos\varphi}{r}
            \end{cases}
        \end{equation}
        \begin{equation}
            H=\frac{p_1^2+p_2^2}{2}=\frac{1}{2}\left(p_r^2+\frac{p_\varphi^2}{r^2}\right)\rightarrow \boxed{H_l=\frac{1}{2}\left(p_r^2+\frac{l^2}{r^2}\right)}
        \end{equation}
        \begin{equation}
            r(t)=\sqrt{q_1^2(t)+q_2^2(t)}=\sqrt{(p_1(0)t+q_1(0))^2+(p_2(0)t+q_2(0))^2}
        \end{equation}
        \begin{equation}
            \boxed{r(t)=\sqrt{2H(0)t^2+2r(0)p_r(0)t+r^2(0)}}
        \end{equation}
        \begin{equation}
            \begin{cases}
                \cos\varphi(t)=\frac{p_1(0)t+q_1(0)}{\sqrt{2H(0)t^2+2r(0)p_r(0)t+r^2(0)}},\\
                \sin\varphi(t)=\frac{p_2(0)t+q_2(0)}{\sqrt{2H(0)t^2+2r(0)p_r(0)t+r^2(0)}}
            \end{cases}
        \end{equation}
        \begin{equation}
            \boxed{p_r(t)=\frac{2H(0)t+r(0)p_r(0)}{\sqrt{2H(0)t^2+2r(0)p_r(0)t+r^2(0)}}}
        \end{equation}
        \begin{multline}
            \omega=dp_1\wedge dq_1+dp_2\wedge dq_2=\\=\left(dp_r\cos\varphi-p_r\sin\varphi d\varphi-\frac{dp_\varphi\sin\varphi}{r}-\frac{p_\varphi\cos\varphi d\varphi}{r}+\frac{p_\varphi\sin\varphi dr}{r^2}\right)\wedge\\\wedge(dr\cos\varphi-r\sin\varphi d\varphi)+\left(dp_r\sin\varphi+p_r\cos\varphi d\varphi+\frac{dp_\varphi\cos\varphi}{r}-\frac{p_\varphi\sin\varphi d\varphi}{r}-\frac{p_\varphi\cos\varphi dr}{r^2}\right)\wedge\\\wedge(dr\sin\varphi+r\cos\varphi d\varphi)=dp_r\wedge dr+dp_\varphi\wedge d\varphi
        \end{multline}
        \begin{equation}
            i_{v_{H_l}}\omega=-dH_l=-p_rdp_r+\frac{l^2}{r^3}dr
        \end{equation}
        \begin{equation}
            \boxed{v_{H_l}=p_r\frac{\partial}{\partial r}+\frac{l^2}{r^3}\frac{\partial}{\partial p_r}+\frac{l}{r^2}\frac{\partial}{\partial\varphi}}
        \end{equation}
        \begin{equation}
            \begin{pmatrix}
                \dot{r}\\
                \dot{\varphi}\\
                \dot{p}_r\\
                \dot{p}_\varphi
            \end{pmatrix}=v_H\begin{pmatrix}
                r\\
                \varphi\\
                p_r\\
                p_\varphi
            \end{pmatrix}=\begin{pmatrix}
                p_r\\
                \frac{l}{r^2}\\
                \frac{l^2}{r^3}\\
                0
            \end{pmatrix}\rightarrow\boxed{\begin{cases}
                r(t)=\sqrt{c_1t^2+c_2t+c_3},\\
                p_r(t)=\frac{c_1t+\frac{c_2}{t}}{\sqrt{c_1t^2+c_2t+c_3}}
            \end{cases}}
        \end{equation}
        As seen, the projection of the integral curve is indeed an intergral curve of the vector field $v_{H_l}$. Physical meaning that when projected, the motion effectively becomes one-dimensional and potential energy $V_\text{eff}=\frac{l^2}{2r^2}$ appears.
    \end{itemize}
    \item \textbf{Geodesic moving.}\\
    Consider a particle with mass moving on the geodesics on a two-dimensional sphere $x_0^2+x_1^2+x_2^2=1$.
    \begin{itemize}
        \item Write explicitly the geodesic equation on the sphere and the generic form of geodesic line.
        \item Consider the projection map
        \begin{equation}
            q=\pi(x)=\arccos x_0
        \end{equation}
        Find the Hamilton function $H(p, q)$ which describes the motion in the system after projection.
        \item Use the form of geodesic line to solve explicitly the equations of motion for the system after projection.
    \end{itemize}
    \textbf{Solution.}
    \begin{itemize}
        \item Lagrangian of a particle with mass moving on a sphere:
        \begin{equation}
            L=\frac{m\dot{\bm{x}}^2}{2}=\frac{m}{2}(\dot{\theta}^2+\sin^2\theta\dot{\varphi}^2)
        \end{equation}
        Euler-Lagrange equations:
        \begin{equation}
            \frac{d}{dt}\frac{\partial L}{\partial\dot{q}}-\frac{\partial L}{\partial q}=0\rightarrow\begin{cases}
                \Ddot{\theta}-\dot{\varphi}^2\sin\theta\cos\theta=0,\\
                \Ddot{\varphi}\sin^2\theta+2\dot{\varphi}\dot{\theta}\sin\theta\cos\theta=0
            \end{cases}
        \end{equation}
        \begin{equation}
            \frac{\partial L}{\partial\varphi}=0\rightarrow p_\varphi=\frac{\partial L}{\partial\dot{\varphi}}=m\sin^2\theta\dot{\varphi}=\text{const}
        \end{equation}
        \begin{equation}
            \dot{\varphi}=\frac{p_\varphi}{m\sin^2\theta}
        \end{equation}
        Legendre transformation:
        \begin{equation}
            H=p_\varphi\dot{\varphi}+p_\theta\dot{\theta}-L=\frac{m}{2}(\dot{\theta}^2+\sin^2\theta\dot{\varphi}^2)=\text{const}
        \end{equation}
        \begin{equation}
            H=\frac{m}{2}\left(\dot{\theta}^2+\frac{p^2_\varphi}{m^2\sin^2\theta}\right)\rightarrow\dot{\theta}=\sqrt{\frac{2H}{m}-\frac{p^2_\varphi}{m^2\sin^2\theta}}
        \end{equation}
        \begin{equation}
            \int\frac{d\theta}{\sqrt{\frac{2H}{m}-\frac{p^2_\varphi}{m^2\sin^2\theta}}}=\int dt=t-t_0
        \end{equation}
        \begin{multline}
            \int\frac{d\theta}{\sqrt{\frac{2H}{m}-\frac{p^2_\varphi}{m^2\sin^2\theta}}}=\pm\int\frac{\sin\theta d\theta}{\sqrt{\frac{2H}{m}\sin^2\theta-\frac{p^2_\varphi}{m^2}}}=\pm\int\frac{d\cos\theta}{\sqrt{\frac{2H}{m}-\frac{p_\varphi^2}{m^2}-\frac{2H}{m}\cos^2\theta}}=\\=\pm m\int\frac{\frac{d\cos\theta}{\sqrt{2mH-p_\varphi^2}}}{\sqrt{1-\frac{2mH}{2mH-p_\varphi^2}\cos^2\theta}}=\pm\sqrt{\frac{m}{2H}}\arcsin\left(\frac{\cos\theta}{\sqrt{1-\frac{p_\varphi^2}{2mH}}}\right)
        \end{multline}
        \begin{equation}
            \boxed{\cos\theta=\pm\sqrt{1-\frac{p_\varphi^2}{2mH}}\sin\left(\sqrt{\frac{2H}{m}}(t-t_0)\right)}
        \end{equation}
        \begin{equation}
            \int d\varphi=\int\frac{p_\varphi dt}{m\left(1-\left(1-\frac{p_\varphi^2}{2mH}\right)\sin^2\left(\sqrt{\frac{2H}{m}}(t-t_0)\right)\right)}=\varphi-\varphi_0
        \end{equation}
        \begin{multline}
            \int\frac{p_\varphi dt}{m\left(1-\left(1-\frac{p_\varphi^2}{2mH}\right)\sin^2\left(\sqrt{\frac{2H}{m}}(t-t_0)\right)\right)}=\int\frac{d\left(\sqrt{\frac{p_\varphi^2}{2mH}}\tan\left(\sqrt{\frac{2H}{m}}(t-t_0)\right)\right)}{1+\frac{p_\varphi^2}{2mH}\tan^2\left(\sqrt{\frac{2H}{m}}(t-t_0)\right)}=\\=\arctan\left(\sqrt{\frac{p_\varphi^2}{2mH}}\tan\left(\sqrt{\frac{2H}{m}}(t-t_0)\right)\right)
        \end{multline}
        \begin{equation}
            \boxed{\varphi=\varphi_0+\arctan\left(\sqrt{\frac{p_\varphi^2}{2mH}}\tan\left(\sqrt{\frac{2H}{m}}(t-t_0)\right)\right)}
        \end{equation}
        Geodesic curves -- circles of large diameter.
        \item 
        \begin{equation}
            x_0=\cos\theta\rightarrow q=\text{arccos}x_0=\theta,\quad p_\theta=p
        \end{equation}
        So, hamiltonian after projection will only have terms with $\theta$:
        \begin{equation}
            \boxed{H(p,q)=\frac{p^2}{2m}+\frac{p^2_\varphi}{2m\sin^2 q}}
        \end{equation}
        \item Parametrization of geodesic line:
        \begin{equation}
            \bm{r}(t)=\bm{r}_0\cos t+\dot{\bm{r}}_0\sin t
        \end{equation}
        \begin{equation}
            |\bm{r}_0|=|\bm{r}(0)|=1,\quad |\dot{\bm{r}}_0|=|\dot{\bm{r}}(0)|=1,\quad\bm{r}_0\cdot\dot{\bm{r}}_0=0
        \end{equation}
        \begin{equation}
            \boxed{\begin{cases}
                q(t)=\cos\theta(t)=z_0\cos t+\dot{z}_0\sin t,\\
                p(t)=p_\theta(t)=\sqrt{2mH-\frac{p_\varphi^2}{\sin^2q(t)}}
            \end{cases}}
        \end{equation}
    \end{itemize}
    \item \textbf{Kepler problem -- 1.}\\
    Consider the standard Kepler problem in a three-dimensional space with the Hamilton function
    \begin{equation}
        H=\frac{p^2}{2}-\frac{\alpha}{r}
    \end{equation}
    where $p^2 = p_1^2 + p_2^2 + p_3^2$, $r = \sqrt{q_1^2+q_2^2+q_3^2}$ and the coupling constant $\alpha>0$. The equations of motion of this system are given by
    \begin{equation}
        \begin{cases}
            \dot{q}_i=p_i,\\
            \dot{p}_i=-\frac{\alpha}{r^3}q_i
        \end{cases}
    \end{equation}
    \begin{itemize}
        \item Find the integrals of motion for the system using Laplace method. Let $I(p,q)$ be an integral of motion, then
        \begin{equation}
            \{H,I(p,q)\}=\sum\limits_{i=1}^3\left(\frac{\partial I}{\partial q_i}p_i-\frac{\partial I}{\partial p_i}\frac{\alpha}{r^3}q_i\right)=0
        \end{equation}
        Expand $I(p,q)$ in homogeneous polynomials in $p_i$ of degree $k\geq0$
        \begin{equation}
            I(p,q)=\sum\limits_{k=0}^\infty I_k(p,q)
        \end{equation}
        Write the system of equations on $I_k$, which follows from the conservation of $I$. Show that this system is finite if one assumes $I$ to be a polynomial in $p_i$ of degree $n$. Solve these finite systems for $n = 1, 2$ and write the corresponding integrals of motion. How many functionally independent conservation laws are obtained in this way?
        \item Consider the angular momentum vector and the Laplace vector
        \begin{equation}
            \bm{l}=[\bm{q}\times\bm{p}],\quad\bm{A}=[\bm{l}\times\bm{p}]+\frac{\alpha}{r}\bm{q}
        \end{equation}
        Show that the components of these vector are integrals of motion. Check that the Poisson brackets between the components correspond to the Lie algebra $\mathfrak{so}(4)$. Namely, check that
        \begin{equation}
            \{l_i,l_j\}=\epsilon_{ijk}l_k,\quad\{l_i,A_j\}=\epsilon_{ijk}A_k,\quad \{A_i,A_j\}=-2H\epsilon_{ijk}l_k
        \end{equation}
        and make an appropriate change of variables to show this Poisson algebra corresponds to $\mathfrak{so}(4)$ Lie algebra. These calculations shows that the Kepler problem has $\mathfrak{so}(4)$ symmetry.
    \end{itemize}
    \textbf{Solution.}
    \begin{itemize}
        \item Expand $I(p,q)$ in homogeneous polynomials in $p_i$ of degree $k\geq0$
        \begin{equation}
            I(p,q)=\sum\limits_{k=0}^\infty I_k(p,q),\quad I_k(\lambda p,q)=\lambda^kI_k(p,q)
        \end{equation}
        $I(p,q)$ is an integral of motion, so
        \begin{equation}
            \{H,I(p,q)\}=\left\{H,\sum\limits_{k=0}^\infty I_k(p,q)\right\}=\sum\limits_{i=1}^3\left(p_i\frac{\partial}{\partial q_i}\sum\limits_{k=0}^\infty I_k-\frac{\alpha q_i}{r^3}\frac{\partial}{\partial p_i}\sum\limits_{k=0}^\infty I_k\right)=0
        \end{equation}
        The system of equations on $I_k$:
        \begin{equation}
        \boxed{\begin{cases}
            \sum\limits_{i=1}^3q_i\frac{\partial I_1}{\partial p_i}=0,\\
            \sum\limits_{i=1}^3\left(p_i\frac{\partial I_{k-1}}{\partial q_i}-\frac{\alpha q_i}{r^3}\frac{\partial I_{k+1}}{\partial p_i}\right)=0,\quad k\geq1
        \end{cases}}
        \end{equation}
        This system is finite if one assumes $I$ to be a polynomial in $p_i$ of degree $n$:
        \begin{equation}
            I(p,q)=\sum\limits_{k=0}^nI_k(p,q)
        \end{equation}
         \begin{equation}
        \boxed{\begin{cases}
            \sum\limits_{i=1}^3q_i\frac{\partial I_1}{\partial p_i}=0,\\
            \sum\limits_{i=1}^3\left(p_i\frac{\partial I_{k-1}}{\partial q_i}-\frac{\alpha q_i}{r^3}\frac{\partial I_{k+1}}{\partial p_i}\right)=0,\quad k\in\{1,...,n-1\},n>1\\
            \sum\limits_{i=1}^3p_i\frac{\partial I_{n-1}}{\partial q_i}=0,\\
            \sum\limits_{i=1}^3p_i\frac{\partial I_n}{\partial q_i}=0.
        \end{cases}}
        \end{equation}
        Consider cases:
        \begin{itemize}
            \item $n=1$.
            \begin{equation}
                \begin{cases}
                    \sum\limits_{i=1}^3q_i\frac{\partial I_1}{\partial p_i}=0,\\
                    \sum\limits_{i=1}^3p_i\frac{\partial I_0}{\partial q_i}=0,\\
                    \sum\limits_{i=1}^3p_i\frac{\partial I_1}{\partial q_i}=0.
                \end{cases}
            \end{equation}
            $I_0=\text{const}$, $I_1=f_1(q)p_1+f_2(q)p_2+f_3(q)p_3$.
            \begin{multline}
                \begin{cases}
                    \sum\limits_{i=1}^3q_if_i(q)=q_1f_1(q)+q_2f_2(q)+q_3f_3(q)=0,\\
                    \sum\limits_{i=1}^3p_i\frac{\partial(f_1(q)p_1+f_2(q)p_2+f_3(q)p_3)}{\partial q_i}=p_1(\frac{\partial f_1}{\partial q_1}p_1+\frac{\partial f_2}{\partial q_1}p_2+\frac{\partial f_3}{\partial q_1}p_3)+\\+p_2(\frac{\partial f_1}{\partial q_2}p_1+\frac{\partial f_2}{\partial q_2}p_2+\frac{\partial f_3}{\partial q_2}p_3)+p_3(\frac{\partial f_1}{\partial q_3}p_1+\frac{\partial f_2}{\partial q_3}p_2+\frac{\partial f_3}{\partial q_3}p_3)=0.
                \end{cases}
            \end{multline}
            From the first equation $f_i(q)=f_{i1}q_1+f_{i2}q_2+f_{i3}q_3$.
            \begin{multline}
                \begin{cases}
                    q_1(f_{11}q_1+f_{12}q_2+f_{13}q_3)+q_2(f_{21}q_1+f_{22}q_2+f_{23}q_3)+\\+q_3(f_{31}q_1+f_{32}q_2+f_{33}q_3)=0,\\
                    p_1(f_{11}p_1+f_{21}p_2+f_{31}p_3)+p_2(f_{12}p_1+f_{22}p_2+f_{32}p_3)+\\+p_3(f_{13}p_1+f_{23}p_2+f_{33}p_3)=0.
                \end{cases}
            \end{multline}
            $f_{11}=f_{22}=f_{33}=0$, $f_{12}+f_{21}=0$, $f_{13}+f_{31}=0$, $f_{23}+f_{32}=0$.
            \begin{equation}
                f_{12}=-f_{21}=c_1,\quad f_{13}=-f_{31}=c_2,\quad f_{23}=-f_{32}=c_3
            \end{equation}
            \begin{equation}
                I_0=\text{const},\quad I_1=c_1(q_2p_1-q_1p_2)+c_2(q_3p_1-q_1p_3)+c_3(q_3p_2-q_2p_3)
            \end{equation}
            \begin{equation}
                \boxed{I=I_0-c_1l_3+c_2l_2-c_3l_1}
            \end{equation}
            We have 3 functionally independent conservation laws.
            \item $n=2$.
            \begin{equation}
                \begin{cases}
                    \sum\limits_{i=1}^3q_i\frac{\partial I_1}{\partial p_i}=0,\\
                    \sum\limits_{i=1}^3\left(p_i\frac{\partial I_0}{\partial q_i}-\frac{\alpha q_i}{r^3}\frac{\partial I_2}{\partial p_i}\right)=0,\\
                    \sum\limits_{i=1}^3p_i\frac{\partial I_1}{\partial q_i}=0,\\
                    \sum\limits_{i=1}^3p_i\frac{\partial I_2}{\partial q_i}=0.
            \end{cases}
            \end{equation}
            Equations on $I_1$ remain the same.
            \begin{equation}
                I_1=-c_1l_3+c_2l_2-c_3l_1
            \end{equation}
            \begin{equation}
                \begin{cases}
                    \sum\limits_{i=1}^3\left(p_i\frac{\partial I_0}{\partial q_i}-\frac{\alpha q_i}{r^3}\frac{\partial I_2}{\partial p_i}\right)=0,\\
                    \sum\limits_{i=1}^3p_i\frac{\partial I_2}{\partial q_i}=0.
            \end{cases}
            \end{equation}
            $I_0=f(q)$, $I_2=\sum\limits_{i,j}a_{ij}(q)p_ip_j$, $a_{ij}=a_{ji}$.
            \begin{multline}
                \sum\limits_{i=1}^3\left(p_i\frac{\partial I_0}{\partial q_i}-\frac{\alpha q_i}{r^3}\frac{\partial I_2}{\partial p_i}\right)=\sum\limits_{i=1}^3\left(p_i\frac{\partial f}{\partial q_i}-\frac{2\alpha q_i}{r^3}\sum\limits_ja_{ij}p_j\right)=\\=\sum\limits_{i=1}^3p_i\left(\frac{\partial f}{\partial q_i}-\frac{2\alpha}{r^3}\sum\limits_ja_{ij}q_j\right)=0
            \end{multline}
            \begin{multline}
                \sum\limits_{i=1}^3p_i\frac{\partial I_2}{\partial q_i}=\sum\limits_{i=1}^3p^3_i\frac{\partial a_{ii}}{\partial q_i}+\sum\limits_{i\neq j}p_i^2p_j\left(\frac{\partial a_{ii}}{\partial q_j}+2\frac{\partial a_{ij}}{\partial q_i}\right)+\\+p_1p_2p_3\left(\frac{\partial a_{12}}{\partial q_3}+\frac{\partial a_{13}}{\partial q_2}+\frac{\partial a_{23}}{\partial q_1}\right)=0
            \end{multline}
            \begin{equation}
                \begin{cases}
                    \frac{\partial f}{\partial q_i}-\frac{2\alpha}{r^3}\sum\limits_ja_{ij}q_j=0,\\
                    \frac{\partial a_{ii}}{\partial q_i}=0,\\
                    \frac{\partial a_{ii}}{\partial q_j}+2\frac{\partial a_{ij}}{\partial q_i}=0,\\
                    \frac{\partial a_{12}}{\partial q_3}+\frac{\partial a_{13}}{\partial q_2}+\frac{\partial a_{23}}{\partial q_1}=0;
                \end{cases}
            \end{equation}
            \begin{equation}
                \frac{\partial f}{\partial q_i}=\frac{2\alpha}{r^3}\sum\limits_ja_{ij}q_j\rightarrow\frac{\partial^2f}{\partial q_j\partial q_i}=\frac{2\alpha}{r^3}\sum\limits_k\frac{\partial a_{ik}}{\partial q_j}q_k+\frac{2\alpha}{r^3}a_{ij}-\frac{6\alpha q_j}{r^5}\sum\limits_ka_{ik}q_k
            \end{equation}
            \begin{equation}
                \frac{\partial^2f}{\partial q_j\partial q_i}-\frac{\partial^2f}{\partial q_i\partial q_j}=\frac{2\alpha}{r^3}\sum\limits_k\left(\frac{\partial a_{ik}}{\partial q_j}-\frac{\partial a_{jk}}{\partial q_i}\right)q_k-\frac{6\alpha}{r^5}\sum\limits_k(a_{ik}q_j-a_{jk}q_i)q_k=0
            \end{equation}
            $a_{ij}$ is homogeneous over $q$:
            \begin{equation}
                a_{ij}(q)=\alpha_{ij}+\sum\limits_{k}\alpha_{ijk}q_k,\quad a_{ijk}=a_{jik}
            \end{equation}
            \begin{multline}
                \frac{2\alpha}{r^3}\sum\limits_k\left(\alpha_{ikj}-\alpha_{jki}\right)q_k-\frac{6\alpha}{r^5}\sum\limits_k(\alpha_{ik}q_j-\alpha_{jk}q_i)q_k-\\-\frac{6\alpha}{r^5}\sum\limits_{k,l}(\alpha_{ikl}q_j-\alpha_{jkl}q_i)q_kq_l=0
            \end{multline}
            \begin{multline}
                \frac{2\alpha}{r^5}\sum\limits_{k,l}\left(\alpha_{ikj}-\alpha_{jki}\right)q_kq_lq_l-\frac{6\alpha}{r^5}\sum\limits_k(\alpha_{ik}q_j-\alpha_{jk}q_i)q_k-\\-\frac{6\alpha}{r^5}\sum\limits_{k,l}(\alpha_{ikl}q_j-\alpha_{jkl}q_i)q_kq_l=0
            \end{multline}
            \begin{equation}
                \alpha_{ij}=a\delta_{ij}   
            \end{equation}
            \begin{equation}
                \begin{cases}
                    \alpha_{ijk}=\alpha_{jik},\\
                    %\alpha_{ikj}=\alpha_{jki}???,\\
                    \alpha_{ikl}=\delta_{ik}a_l\text{ or }\alpha_{ikl}=\delta_{il}a_j,\\
                    \alpha_{iii}=0,\\
                    \alpha_{iij}+2\alpha_{iji}=0,\\
                    \alpha_{123}+\alpha_{132}+\alpha_{231}=0;
                \end{cases}
            \end{equation}
            \begin{equation}
                \begin{cases}
                    \alpha_{ii1}=a_1, \alpha_{ii2}=a_2,\alpha_{ii3}=a_3,\\
                    \alpha_{i1i}=\alpha_{1ii}=-\frac{a_1}{2},\\
                    \alpha_{i2i}=\alpha_{2ii}=-\frac{a_2}{2},\\
                    \alpha_{i3i}=\alpha_{3ii}=-\frac{a_3}{2},\\
                    \alpha_{iii}=\alpha_{ijk}=0.
                \end{cases}
            \end{equation}
            \begin{multline}
                I_2=\sum\limits_{i,j}a_{ij}(q)p_ip_j=(a+a_2q_2+a_3q_3)p_1^2+(a+a_1q_1+a_3q_3)p_2^2+(a+a_1q_1+a_2q_2)p_3^2-\\-(a_2q_1+a_1q_2)p_1p_2-(a_3q_2+a_2q_3)p_2p_3-(a_3q_1+a_1q_3)p_1p_3=\\=a(p_1^2+p_2^2+p_3^2)+a_1(q_1(p_2^2+p_3^2)-p_1(q_2p_2+q_3p_3)+\\+a_2(q_2(p_1^2+p_3^2)-p_2(q_1p_1+q_3p_3))+a_3(q_3(p_1^2+p_2^2)-p_3(q_1p_1+q_2p_2))=\\=ap^2+a_1(q_1p^2-p_1\bm{q}\bm{p})+a_2(q_2p^2-p_2\bm{q}\bm{p})+a_3(q_3p^2-p_3\bm{q}\bm{p})
            \end{multline}
            \begin{equation}
                I_2=ap^2-a_1[\bm{l}\times\bm{p}]_1-a_2[\bm{l}\times\bm{p}]_2-a_3[\bm{l}\times\bm{p}]_3
            \end{equation}
            \begin{equation}
            \begin{cases}
                \frac{\partial f}{\partial q_1}=\frac{2\alpha}{r^3}\sum\limits_ja_{1j}q_j=\frac{\alpha}{r^3}(2aq_1+a_2q_1q_2+a_3q_1q_3-a_1q_2^2-a_1q_3^2),\\
                \frac{\partial f}{\partial q_2}=\frac{2\alpha}{r^3}\sum\limits_ja_{2j}q_j=\frac{\alpha}{r^3}(2aq_2+a_1q_1q_2+a_3q_2q_3-a_2q_1^2-a_2q_3^2),\\
                \frac{\partial f}{\partial q_3}=\frac{2\alpha}{r^3}\sum\limits_ja_{3j}q_j=\frac{\alpha}{r^3}(2aq_3+a_1q_1q_3+a_2q_2q_3-a_3q_1^2-a_3q_2^2);
            \end{cases}
            \end{equation}
            \begin{equation}
            \begin{cases}
                \frac{\partial f}{\partial q_1}=\alpha(\frac{2aq_1}{r^3}+\frac{a_1q_1+a_2q_2+a_3q_3}{r^3}q_1-\frac{a_1}{r}),\\
                \frac{\partial f}{\partial q_2}=\alpha(\frac{2aq_2}{r^3}+\frac{a_1q_1+a_2q_2+a_3q_3}{r^3}q_2-\frac{a_2}{r}),\\
                \frac{\partial f}{\partial q_3}=\alpha(\frac{2aq_3}{r^3}+\frac{a_1q_1+a_2q_2+a_3q_3}{r^3}q_3-\frac{a_3}{r}).
            \end{cases}
            \end{equation}
            \begin{equation}
                I_0=f(q)=-\frac{2a\alpha}{r}-\frac{\alpha}{r}(a_1q_1+a_2q_2+a_3q_3)
            \end{equation}
            \begin{multline}
                I=I_0+I_1+I_2=-\frac{2a\alpha}{r}-\frac{\alpha}{r}(a_1q_1+a_2q_2+a_3q_3)-c_1l_3+c_2l_2-c_3l_1+\\+ap^2-a_1[\bm{l}\times\bm{p}]_1-a_2[\bm{l}\times\bm{p}]_2-a_3[\bm{l}\times\bm{p}]_3=2a\left(\frac{p^2}{2}-\frac{\alpha}{r}\right)-c_1l_3+c_2l_2-c_3l_1-\\-a_1\left([\bm{l}\times\bm{p}]_1+\frac{\alpha}{r}q_1\right)-a_2\left([\bm{l}\times\bm{p}]_2+\frac{\alpha}{r}q_2\right)-a_3\left([\bm{l}\times\bm{p}]_3+\frac{\alpha}{r}q_3\right)
            \end{multline}
            \begin{equation}
                \boxed{I=2aH-c_1l_3+c_2l_2-c_3l_1-a_1A_1-a_2A_2-a_3A_3}
            \end{equation}
            \begin{equation}
                \text{rg}\left(\frac{\partial(H,\bm{l},\bm{A})}{\partial(\bm{q},\bm{p})}\right)=\text{rg}\begin{pmatrix}
                    \frac{\partial H}{\partial q_1} & \frac{\partial H}{\partial q_2} & \frac{\partial H}{\partial q_3} &
                    \frac{\partial H}{\partial p_1} &
                    \frac{\partial H}{\partial p_2} &
                    \frac{\partial H}{\partial p_3}\\
                    \frac{\partial l_1}{\partial q_1} & \frac{\partial l_1}{\partial q_2} & \frac{\partial l_1}{\partial q_3} &
                    \frac{\partial l_1}{\partial p_1} &
                    \frac{\partial l_1}{\partial p_2} &
                    \frac{\partial l_1}{\partial p_3}\\
                    \frac{\partial l_2}{\partial q_1} & \frac{\partial l_2}{\partial q_2} & \frac{\partial l_2}{\partial q_3} &
                    \frac{\partial l_2}{\partial p_1} &
                    \frac{\partial l_2}{\partial p_2} &
                    \frac{\partial l_2}{\partial p_3}\\
                    \frac{\partial l_3}{\partial q_1} & \frac{\partial l_3}{\partial q_2} & \frac{\partial l_3}{\partial q_3} &
                    \frac{\partial l_3}{\partial p_1} &
                    \frac{\partial l_3}{\partial p_2} &
                    \frac{\partial l_3}{\partial p_3}\\
                    \frac{\partial A_1}{\partial q_1} & \frac{\partial A_1}{\partial q_2} & \frac{\partial A_1}{\partial q_3} &
                    \frac{\partial A_1}{\partial p_1} &
                    \frac{\partial A_1}{\partial p_2} &
                    \frac{\partial A_1}{\partial p_3}\\
                    \frac{\partial A_2}{\partial q_1} & \frac{\partial A_2}{\partial q_2} & \frac{\partial A_2}{\partial q_3} &
                    \frac{\partial A_2}{\partial p_1} &
                    \frac{\partial A_2}{\partial p_2} &
                    \frac{\partial A_2}{\partial p_3}\\
                    \frac{\partial A_3}{\partial q_1} & \frac{\partial A_3}{\partial q_2} & \frac{\partial A_3}{\partial q_3} &
                    \frac{\partial A_3}{\partial p_1} &
                    \frac{\partial A_3}{\partial p_2} &
                    \frac{\partial A_3}{\partial p_3}
                \end{pmatrix}=5
            \end{equation}
            We obtain 5 functionally independent conservation laws.
        \end{itemize}
        \item
        \begin{equation}
            l_i=\epsilon_{ijk}q_jp_k,\quad A_i=\epsilon_{ijk}l_jp_k+\frac{\alpha}{r}q_i
        \end{equation}
        Poisson brackets:
        \begin{equation}
            \{q_i,p_j\}=\delta_{ij},\quad\{q_i,q_j\}=\{p_i,p_j\}=0
        \end{equation}
        \begin{itemize}
            \item Check that $\{l_i,l_j\}=\epsilon_{ijk}l_k$:
            \begin{equation}
                \{l_i,q_j\}=\epsilon_{ij'k}\{q_{j'}p_k,q_j\}=-\epsilon_{ij'k}q_{j'}\delta_{kj}=-\epsilon_{ij'j}q_{j'}=\epsilon_{ijk}q_k
            \end{equation}
            \begin{equation}
                \{l_i,p_j\}=\epsilon_{ij'k}\{q_{j'}p_k,p_j\}=\epsilon_{ij'k}p_k\delta_{j'j}=\epsilon_{ijk}p_k
            \end{equation}
            \begin{multline}
                \{l_i,l_j\}=\{l_i,\epsilon_{jkl}q_kp_l\}=\epsilon_{jkl}(\{l_i,q_k\}p_l+q_k\{l_i,p_l\})=\\=\epsilon_{jkl}\epsilon_{ikm}q_mp_l+\epsilon_{jkl}\epsilon_{ilm}q_kp_m=(\delta_{ji}\delta_{lm}-\delta_{jm}\delta_{il})q_mp_l-\\-(\delta_{ji}\delta_{km}-\delta_{jm}\delta_{ik})q_kp_m=\delta_{ij}q_lp_l-q_jp_i-\delta_{ij}q_kp_k+q_ip_j=\\=q_ip_j-q_jp_i
            \end{multline}
            \begin{equation}
                \epsilon_{ijk}l_k=\epsilon_{ijk}\epsilon_{klm}q_lp_m=(\delta_{il}\delta_{jm}-\delta_{im}\delta_{jl})q_lp_m=q_ip_j-q_jp_i
            \end{equation}
            \begin{equation}
                \boxed{\{l_i,l_j\}=\epsilon_{ijk}l_k}
            \end{equation}
            \item Check that $\{l_i,A_j\}=\epsilon_{ijk}A_k$:
            \begin{equation}
                \{q_i,f(q)\}=0,\quad\{p_i,f(q)\}=-\frac{\partial f}{\partial q_i}
            \end{equation}
            \begin{equation}
                \left\{q_i,\frac{\alpha}{r}\right\}=0,\quad\left\{p_i,\frac{\alpha}{r}\right\}=\frac{\alpha q_i}{r^3}
            \end{equation}
            \begin{equation}
                \left\{l_i,\frac{\alpha}{r}\right\}=\epsilon_{ijk}\left\{p_jq_k,\frac{\alpha}{r}\right\}=\epsilon_{ijk}\frac{\alpha}{r^3} q_jq_k=0
            \end{equation}
            \begin{multline}
                \{l_i,A_j\}=\left\{l_i,\epsilon_{jkl}l_kp_l+\frac{\alpha}{r}q_j\right\}=\epsilon_{jkl}(\{l_i,l_k\}p_l+l_k\{l_i,p_l\})+\\+\left\{l_i,\frac{\alpha}{r}\right\}q_j+\frac{\alpha}{r}\{l_i,q_j\}=\epsilon_{jkl}\epsilon_{ikm}l_mp_l+\epsilon_{jkl}l_k\epsilon_{ilm}p_m+\frac{\alpha}{r}\epsilon_{ijk}q_k=\\=(\delta_{ji}\delta_{lm}-\delta_{jm}\delta_{li})l_mp_l-(\delta_{ji}\delta_{km}-\delta_{jm}\delta_{ki})l_kp_m+\frac{\alpha}{r}\epsilon_{ijk}q_k=\\=\delta_{ij}l_lp_l-l_jp_i-\delta_{ij}l_kp_k+l_ip_j+\frac{\alpha}{r}\epsilon_{ijk}q_k=l_ip_j-l_jp_i+\frac{\alpha}{r}\epsilon_{ijk}q_k
            \end{multline}
            \begin{multline}
                \epsilon_{ijk}A_k=\epsilon_{ijk}\epsilon_{klm}l_lp_m+\frac{\alpha}{r}\epsilon_{ijk}q_k=(\delta_{il}\delta_{jm}-\delta_{im}\delta_{lj})l_lp_m+\frac{\alpha}{r}\epsilon_{ijk}q_k=\\=l_ip_j-l_jp_i+\frac{\alpha}{r}\epsilon_{ijk}q_k
            \end{multline}
            \begin{equation}
                \boxed{\{l_i,A_j\}=\epsilon_{ijk}A_k}
            \end{equation}
            \item Check that $\{A_i,A_j\}=-2H\epsilon_{ijk}l_k$:
            \begin{multline}
                \{p_i,A_j\}=\left\{p_i,\epsilon_{jkl}l_kp_l+\frac{\alpha}{r}q_j\right\}=\epsilon_{jkl}\{p_i,l_k\}p_l+\left\{p_i,\frac{\alpha}{r}\right\}q_j+\frac{\alpha}{r}\{p_i,q_j\}=\\=\epsilon_{jkl}\epsilon_{ikm}p_mp_l+\frac{\alpha q_iq_j}{r^3}-\frac{\alpha}{r}\delta_{ij}=(\delta_{ji}\delta_{lm}-\delta_{jm}\delta_{il})p_mp_l+\frac{\alpha q_iq_j}{r^3}-\frac{\alpha}{r}\delta_{ij}=\\=\delta_{ij}p^2-p_ip_j+\frac{\alpha q_iq_j}{r^3}-\frac{\alpha}{r}\delta_{ij}
            \end{multline}
            \begin{multline}
                \{q_i,A_j\}=\left\{q_i,\epsilon_{jkl}l_kp_l+\frac{\alpha}{r}q_j\right\}=\epsilon_{jkl}(\{q_i,l_k\}p_l+l_k\{q_i,p_l\})=\\=\epsilon_{jkl}\epsilon_{ikm}q_mp_l+\epsilon_{jkl}l_k\delta_{il}=(\delta_{ji}\delta_{lm}-\delta_{jm}\delta_{li})q_mp_l+\epsilon_{ijk}l_k=\\=\delta_{ij}q_lp_l-p_iq_j+\epsilon_{ijk}l_k=\delta_{ij}\bm{q}\bm{p}-p_iq_j+\epsilon_{ijk}l_k
            \end{multline}
            \begin{multline}
                \left\{\frac{\alpha}{r},A_j\right\}=\left\{\frac{\alpha}{r},\epsilon_{jkl}l_kp_l+\frac{\alpha}{r}q_j\right\}=\epsilon_{jkl}\left(\left\{\frac{\alpha}{r},l_k\right\}p_l+l_k\left\{\frac{\alpha}{r},p_l\right\}\right)=\\=-\epsilon_{jkl}l_k\frac{\alpha q_l}{r^3}
            \end{multline}
            \begin{multline}
                \{A_i,A_j\}=\left\{\epsilon_{ikl}l_kp_l+\frac{\alpha}{r}q_i,A_j\right\}=\epsilon_{ikl}(\{l_k,A_j\}p_l+l_k\{p_l,A_j\})+\\+\left\{\frac{\alpha}{r},A_j\right\}q_i+\frac{\alpha}{r}\{q_i,A_j\}=\\=\epsilon_{ikl}\left(\epsilon_{kjm}A_mp_l+l_k\left(\delta_{lj}p^2-p_lp_j+\frac{\alpha q_lq_j}{r^3}-\frac{\alpha}{r}\delta_{lj}\right)\right)-\epsilon_{jkl}l_k\frac{\alpha q_lq_i}{r^3}+\\+\frac{\alpha}{r}(\delta_{ij}\bm{q}\bm{p}-p_iq_j+\epsilon_{ijk}l_k)=-(\delta_{ij}\delta_{lm}-\delta_{im}\delta_{lj})A_mp_l-\epsilon_{ijk}l_kp^2-\\-\epsilon_{ikl}l_kp_lp_j+\epsilon_{ikl}\frac{\alpha}{r^3}q_jq_ll_k+2\epsilon_{ijk}l_k\frac{\alpha}{r}-\epsilon_{jkl}\frac{\alpha}{r^3}q_iq_ll_k+\frac{\alpha}{r}q_lp_l\delta_{ij}-\frac{\alpha}{r}p_iq_j=\\=-\delta_{ij}A_lp_l+A_ip_j-2H\epsilon_{ijk}l_k-\epsilon_{ikl}l_kp_lp_j+\frac{\alpha}{r^3}l_kq_l(\epsilon_{ikl}q_j-\epsilon_{jkl}q_i)+\\+\frac{\alpha}{r}q_lp_l\delta_{ij}-\frac{\alpha}{r}p_iq_j=-\delta_{ij}\epsilon_{lkm}l_kp_mp_l-\frac{\alpha}{r}\delta_{ij}q_lp_l+\epsilon_{ikl}l_kp_lp_j+\frac{\alpha}{r}q_ip_j-\\-2H\epsilon_{ijk}l_k-\epsilon_{ikl}l_kp_lp_j+\frac{\alpha}{r^3}l_kq_l(\epsilon_{ikl}q_j-\epsilon_{jkl}q_i)+\frac{\alpha}{r}q_lp_l\delta_{ij}-\frac{\alpha}{r}p_iq_j=\\=-2H\epsilon_{ijk}l_k+\frac{\alpha}{r^3}l_kq_l(\epsilon_{ikl}q_j-\epsilon_{jkl}q_i)
            \end{multline}
            \begin{multline}
                \frac{\alpha}{r^3}l_kq_l(\epsilon_{ikl}q_j-\epsilon_{jkl}q_i)=\frac{\alpha}{r^3}\epsilon_{kmn}q_mp_nq_l(\epsilon_{ikl}q_j-\epsilon_{jkl}q_i)=\\=-\frac{\alpha}{r^3}(\delta_{im}\delta_{ln}-\delta_{in}\delta_{lm})q_mp_nq_lq_j+\frac{\alpha}{r^3}(\delta_{mj}\delta_{nl}-\delta_{ml}\delta_{nj})q_mp_nq_lq_i=\\=-\frac{\alpha}{r^3}q_iq_jp_lq_l+\frac{\alpha}{r^3}p_iq_jq_lq_l+\frac{\alpha}{r^3}q_iq_jp_lq_l-\frac{\alpha}{r^3}q_ip_jq_lq_l=0
            \end{multline}
            \begin{equation}
                \boxed{\{A_i,A_j\}=-2H\epsilon_{ijk}l_k}
            \end{equation}
         \end{itemize}
         Change variables:
         \begin{equation}
             l_i\rightarrow l_i,\quad A_i\rightarrow\frac{u_i}{\sqrt{-2H}}
         \end{equation}
         \begin{equation}
             \boxed{\{l_i,l_j\}=\epsilon_{ijk}l_k,\quad\{l_i,u_j\}=\epsilon_{ijk}u_k,\quad\{u_i,u_j\}=\epsilon_{ijk}l_k}
         \end{equation}
         Poisson algebra corresponds to $\mathfrak{so}(4)$ Lie algebra. Thus, the Kepler problem has $\mathfrak{so}(4)$ symmetry.
    \end{itemize}
\end{enumerate}
\section{Classical $r$-matrix structure}
\begin{enumerate}
    \item \textbf{Classical r-matrix for oscillator.}\\
    Consider a classical one-dimensional harmonic oscillator
    \begin{equation}
        H=\frac{p^2}{2}+\frac{\omega^2q^2}{2}
    \end{equation}
    with the Lax operator from Problem 1, Task 1
    \begin{equation}
        L=\begin{pmatrix}
            p & \omega q\\
            \omega q & -p
        \end{pmatrix}
    \end{equation}
    \begin{itemize}
        \item Find the classical $r$-matrix for this $L$-operator, i.e. a $4\times 4$ matrix $r$ such that:
        \begin{equation}
            \{L_1,L_2\}=[r_{12},L_1]-[r_{21},L_2],
        \end{equation}
        where
        \begin{equation}
            L_1=L\otimes1,\quad L_2=1\otimes L,\quad\{L_1,L_2\}=\sum\limits_{ij,kl}\{L_{ij},L_{kl}\}E_{ij}\otimes E_{kl}
        \end{equation}
        \begin{equation}
            r_{12}=\sum_{ij,kl}r_{ij,kl}E_{ij}\otimes E_{kl},\quad r_{21}=\sum\limits_{ij,kl}r_{ij,kl}E_{kl}\otimes E_{ij}
        \end{equation}
        \item Using the classical $r$-matrix find the matrix $M$, such that
        \begin{equation}
            \dot{L}=[L,M]
        \end{equation}
        Compare the result with the Problem 1 from Task 1.
    \end{itemize}
    \textbf{Solution.}
    \begin{itemize}
        \item 
        \begin{equation}
            L=\begin{pmatrix}
            p & \omega q\\
            \omega q & -p
        \end{pmatrix}=p\sigma_z+\omega q\sigma_x
        \end{equation}
        \begin{equation}
            \{L_1,L_2\}=\{p,\omega q\}\sigma_z\otimes\sigma_x+\{\omega q,p\}\sigma_x\otimes\sigma_z=\omega(\sigma_z\otimes \sigma_x-\sigma_x\otimes\sigma_z)
        \end{equation}
        Suppose $r_{12}=r^{yx}\sigma_y\otimes\sigma_x$, $r_{21}=r^{yx}\sigma_x\otimes\sigma_y$, therefore
        \begin{equation}
            [r_{12},L_1]=r^{yx}p[\sigma_y,\sigma_z]\otimes\sigma_x+r^{yx}\omega q[\sigma_y,\sigma_x]\otimes\sigma_x=2ir^{yx}(p\sigma_x\otimes\sigma_x-\omega q\sigma_z\otimes\sigma_x)
        \end{equation}
        \begin{equation}
            [r_{21},L_2]=2ir^{yx}(p\sigma_x\otimes\sigma_x-\omega q\sigma_x\otimes\sigma_z)
        \end{equation}
        \begin{equation}
            [r_{12},L_1]-[r_{21},L_2]=2i\omega qr^{yx}(\sigma_x\otimes\sigma_z-\sigma_z\otimes\sigma_x)
        \end{equation}
        \begin{equation}
            \{L_1,L_2\}=[r_{12},L_1]-[r_{21},L_2]\rightarrow r^{yx}=\frac{1}{2iq}
        \end{equation}
        \begin{equation}
            \boxed{r_{12}=\frac{1}{2iq}\sigma_y\otimes\sigma_x,\quad r_{21}=\frac{1}{2iq}\sigma_x\otimes\sigma_y}
        \end{equation}
        \item For the given $L$-operator:
        \begin{equation}
            H_k=\frac{1}{k}\text{Tr}L^k\rightarrow\frac{\partial L}{\partial t_k}=[L,M_k],\quad (M_k)_1=\text{Tr}_2(r_{12}L_2^{k-1})
        \end{equation}
        \begin{equation}
            H=H_2=\frac{1}{2}\text{Tr}L^2=\frac{p^2}{2}+\frac{\omega^2q^2}{2}
        \end{equation}
        \begin{multline}
            M=\text{Tr}_2(r_{12}L_2)=\text{Tr}_2\left(\frac{1}{2iq}(\sigma_y\otimes\sigma_x)(1\otimes (p\sigma_z+\omega q\sigma_x))\right)=\\=\text{Tr}_2\left(\frac{1}{2iq}(p\sigma_y\otimes\sigma_x\sigma_z+\omega q\sigma_y\otimes\sigma_x^2)\right)=\frac{1}{2iq}p\sigma_y\text{Tr}(\sigma_x\sigma_z)+\frac{\omega}{2i}\sigma_y\text{Tr}(\bm{1})=\frac{\sigma_y}{i}
        \end{multline}
        \begin{equation}
            \boxed{M=\begin{pmatrix}
                0 & -\omega\\
                \omega & 0
            \end{pmatrix}}
        \end{equation}
    \end{itemize}
    \item \textbf{Spectral parameter.}\\
    Consider a classical Euler top with three different components of the inverse of inertia tensor
    \begin{equation}
        H=\frac{1}{2}\sum\limits_{a=1}^3J_aS^2_a,\quad\{S_a,S_b\}=\sum\limits_c\epsilon_{abc}S_c,\quad J_1\neq J_2\neq J_3\neq J_1
    \end{equation}
    Define the $3\times3$ matrices
    \begin{equation}
        S=\sum\limits_{i,j}S_{ij}E_{ij},\quad S_{ij}=\sum\limits_k\epsilon_{ijk}S_k,\quad\Omega=\sum\limits_{ij}\Omega_{ij}E_{ij},\quad\Omega_{ij}=\sum\limits_k\epsilon_{ijk}J_kS_k
    \end{equation}
    \begin{itemize}
        \item Check that the equations of motion can be presented in the form $\dot{S}=[S,\Omega]$, but this Lax representation is not provide any nontrivial conservation laws (Casimir only).
        \item Let $K$ be a diagonal matrix with elements $K_i = \frac{1}{2}(J^{-1}_j + J^{-1}_k-J^{-1}_i)$ (all indices different). Check that $S = K\Omega + \Omega K$ and that the top has the Lax representation with spectral parameter
        \begin{equation}
            L(z)=S+zK^2,\quad M(z)=\Omega+zK
        \end{equation}
        \item Show that $\text{Tr}L(z)$ and $\text{Tr}L^2(z)$ do not provide nontrivial integrals of motion, but $\text{Tr}L^3(z)$ provides -- its expansion in $z$ contains the Hamilton function $H$.
    \end{itemize}
    \textbf{Solution.}
    \begin{itemize}
        \item Consider the matrix $S$:
        \begin{equation}
            S=\sum\limits_{i,j}S_{ij}E_{ij}=\sum\limits_{i,j,k}\epsilon_{ijk}E_{ij}S_k=\begin{pmatrix}
                0 & S_3 & -S_2\\
                -S_3 & 0 & S_1\\
                S_2 & -S_1 & 0
            \end{pmatrix}
        \end{equation}
        The equations of motion:
        \begin{equation}
            \dot{S}_i=\{H,S_i\}
        \end{equation}
        \begin{equation}
            \{H,S_i\}=\frac{1}{2}\sum\limits_{a=1}^3J_a\{S_a^2,S_i\}=\sum\limits_{a=1}^3J_aS_a\{S_a,S_i\}=-\sum\limits_{a,c}\epsilon_{iac}J_aS_aS_c
        \end{equation}
        \begin{equation}
            \begin{cases}
                \dot{S}_1=(J_3-J_2)S_2S_3,\\
                \dot{S}_2=(J_1-J_3)S_1S_3,\\
                \dot{S}_3=(J_2-J_1)S_1S_2.
            \end{cases}
        \end{equation}
        \begin{equation}
            \dot{S}=\sum\limits_{i,j,k}\epsilon_{ijk}\dot{S}_kE_{ij}=\begin{pmatrix}
                0 & \dot{S}_3 & -\dot{S}_2\\
                -\dot{S}_3 & 0 & \dot{S}_1\\
                \dot{S}_2 & -\dot{S}_1 & 0
            \end{pmatrix}
        \end{equation}
        \begin{equation}
            \dot{S}=\begin{pmatrix}
                0 & (J_2-J_1)S_1S_2 & (J_3-J_1)S_1S_3\\
                (J_1-J_2)S_1S_2 & 0 & (J_3-J_2)S_2S_3\\
                (J_1-J_3)S_1S_3 & (J_2-J_3)S_2S_3 & 0
            \end{pmatrix}
        \end{equation}
        Consider the matrix $\Omega$:
        \begin{equation}
            \Omega=\sum\limits_{i,j}\Omega_{ij}E_{ij}=\sum\limits_{i,j,k}\epsilon_{ijk}E_{ij}J_kS_k=\begin{pmatrix}
                0 & J_3S_3 & -J_2S_2\\
                -J_3S_3 & 0 & J_1S_1\\
                J_2S_2 & -J_1S_1 & 0
            \end{pmatrix}
        \end{equation}
        \begin{equation}
            \boxed{\dot{S}=[S,\Omega]}
        \end{equation}
        Lax representation:
        \begin{equation}
            \dot{L}=[L,M],\quad L=S, M=\Omega
        \end{equation}
        However, this Lax representation os not good:
        \begin{equation}
            \text{Tr}L=0,\quad\text{Tr}L^2=-\sum\limits_aS_a^2=-C,
        \end{equation}
        where $C$ -- Casimir element. $\text{Tr\;} L^k$ -- also functions of $C$. Lax representation is not provide any nontrivial conservation laws. 
        \item Let $K$ be a diagonal matrix with elements $K_{ii} = \frac{1}{2}(J^{-1}_j + J^{-1}_k-J^{-1}_i)$ (all indices different).
        \begin{equation}
            K=\begin{pmatrix}
                \frac{1}{2}(J_2^{-1}+J_3^{-1}-J_1^{-1}) & 0 & 0\\
                0 & \frac{1}{2}(J_1^{-1}+J_3^{-1}-J_2^{-1}) & 0\\
                0 & 0 & \frac{1}{2}(J_1^{-1}+J_2^{-1}-J_3^{-1})
            \end{pmatrix}
        \end{equation}
        \begin{equation}
            K\Omega+\Omega K=\begin{pmatrix}
                0 & S_3 & -S_2\\
                -S_3 & 0 & S_1\\
                S_2 & -S_1 & 0
            \end{pmatrix}=S
        \end{equation}
        \begin{equation}
            \boxed{S=K\Omega+\Omega K}
        \end{equation}
        Check that the top has the Lax representation with spectral parameter
        \begin{equation}
            L(z)=S+zK^2,\quad M(z)=\Omega+zK
        \end{equation}
        \begin{equation}
            \dot{L}(z)=\dot{S}=[S,\Omega]
        \end{equation}
        \begin{multline}
            [L(z),M(z)]=[S,\Omega]+z[S,K]+z[K^2,\Omega]=[S,\Omega]+z[K\Omega+\Omega K,K]+z[K^2,\Omega]=\\=[S,\Omega]+z(K\Omega K+\Omega K^2-K^2\Omega-K\Omega K+K^2\Omega-\Omega K^2)=[S,\Omega]
        \end{multline}
        \begin{equation}
            \boxed{\dot{L}(z)=[L(z),M(z)]}
        \end{equation}
        \item
        \begin{equation}
            L(z)=S+zK^2
        \end{equation}
        \begin{equation}
            L(z)=\begin{pmatrix}
                \frac{z}{4}(J_2^{-1}+J_3^{-1}-J_1^{-1})^2 & S_3 & -S_2\\
                -S_3 & \frac{z}{4}(J_1^{-1}+J_3^{-1}-J_2^{-1})^2 & S_1\\
                S_2 & -S_1 & \frac{z}{4}(J_1^{-1}+J_2^{-1}-J_3^{-1})^2
            \end{pmatrix}
        \end{equation}
        \begin{equation}
            \text{Tr}L(z)=\frac{3z}{4}\left(\frac{1}{J^2_1}+\frac{1}{J^2_2}+\frac{1}{J^2_3}\right)-\frac{z}{2}\left(\frac{1}{J_1J_2}+\frac{1}{J_1J_3}+\frac{1}{J_2J_3}\right)
        \end{equation}
        \begin{multline}
            \text{Tr}L^2(z)=-2(S_1^2+S_2^2+S_3^2)+\frac{z^2}{16}\left[\left(\frac{1}{J_1}+\frac{1}{J_2}-\frac{1}{J_3}\right)^4+\left(-\frac{1}{J_1}-\frac{1}{J_2}+\frac{1}{J_3}\right)^4+\right.\\\left.+\left(\frac{1}{J_1}+\frac{1}{J_2}+\frac{1}{J_3}\right)^4\right]
        \end{multline}
        \begin{multline}
            \text{Tr}L^3(z)=-\frac{3}{2J_1^2J_2^2J_3^2}((J_1^2+J_2^2+J_3^2)(S_1^2+S_2^2+S_3^2)-2J_1J_2J_3(J_1S_1^2+J_2S_2^2+J_3S_3^2))+\\+f(J_1,J_2,J_3)
        \end{multline}
        Expansion of $T(z)$ in $z$ contains the Hamilton function $H$.
    \end{itemize}
    \item \textbf{Exercises with permutation matrices.}\\
    Denote the standard basis in $\text{Mat}_{N\times N}$ as $\{E_{ab}|a,b = 1,...,N\}$, the matrix elements of these matrices are $(E_{ab})_{ij}= \delta_{ai}\delta_{bj}$.
    \begin{itemize}
        \item Consider a permutation operator $P\in\text{Mat}^{\otimes2}_{N\times N}$ defined by its action on two $N$-dimensional vectors
        \begin{equation}
            P(a\otimes b)=b\otimes a
        \end{equation}
        Show that in the standard basis the permutation operator has the form
        \begin{equation}
            P=\sum_{i,j=1}^{N}E_{ij}\otimes E_{ji}
        \end{equation}
        \item Consider permutation operators in the tensor product of $K$ vector spaces, defined as
        \begin{equation}
            P_{ij} (v_1 \otimes ... \otimes v_i \otimes ...\otimes v_j \otimes ... \otimes v_K)=(v_1 \otimes ... \otimes v_j \otimes ... \otimes v_i \otimes ...\otimes v_K)
        \end{equation}
        Write the representation of this operator in $\text{Mat}^{\otimes K}_{N\times N}$ and check the following formulas in this representation (consider all indices $i, j, k$ are distinct)
        \begin{equation}
            P_{ij} P_{ij} = 1,\quad P_{ij}P_{jk} = P_{jk}P_{ik} = P_{ik}P_{ij},\quad P_{ij}P_{ik}P_{jk} = P_{jk}P_{ik}P_{ij}
        \end{equation}
        \item Let $\hbar$ and $\{z_i | i = 1,...,K\}$ be arbitrary constants. Consider matrices
        \begin{equation}
            R_{ij}(z_i,z_j)=\frac{1}{\hbar} +\frac{P_{ij}}{z_i-z_j}
        \end{equation}
        Show that the matrices defined above satisfy the quantum Yang–Baxter equation
        \begin{equation}
            R_{ij}(z_i,z_j)R_{ik}(z_i,z_k)R_{jk}(z_j ,z_k) = R_{jk}(z_j,z_k)R_{ik}(z_i, z_k)R_{ij}(z_i,z_j)
        \end{equation}
        and unitarity condition
        \begin{equation}
            R_{ij}(z_i,z_j)R_{ji}(z_j,z_i)\propto 1
        \end{equation}
        \item Consider an operator $R_{ij} (z_i, z_j )$ satisfying the quantum Yang–Baxter equation as a series in $\hbar$
        \begin{equation}
            R_{ij}(z_i,z_j) = \frac{1}{\hbar} + r_{ij}(z_i,z_j) + \mathcal{O}(\hbar)
        \end{equation}
        Show that the operator $r_{ij}$ then satisfies the classical Yang–Baxter equation
        \begin{equation}
            [r_{ij}(z_i, z_j), r_{ik}(z_i, z_k)] + [r_{ij}(z_i, z_j), r_{jk}(z_j, z_k)] + [r_{ik}(z_i, z_k), r_{jk}(z_j, z_k)] = 0
        \end{equation}
    \end{itemize}
    \textbf{Solution.}
    \begin{itemize}
        \item Consider tensor products:
        \begin{equation}
            a\otimes b=\begin{pmatrix}
                a_1\begin{pmatrix}
                    b_1\\
                    \vdots\\
                    b_n
                \end{pmatrix}\\
                \vdots\\
                a_n\begin{pmatrix}
                    b_1\\
                    \vdots\\
                    b_n
                \end{pmatrix}
            \end{pmatrix},\quad b\otimes a=\begin{pmatrix}
                b_1\begin{pmatrix}
                    a_1\\
                    \vdots\\
                    a_n
                \end{pmatrix}\\
                \vdots\\
                b_n\begin{pmatrix}
                    a_1\\
                    \vdots\\
                    a_n
                \end{pmatrix}
            \end{pmatrix}
        \end{equation}
        Suppose, that $P=\sum\limits_{i,j=1}^{N}E_{ij}\otimes E_{ji}$. Then
        \begin{multline}
            P(a\otimes b)_{kN+l}=\left(\sum_{i,j=1}^NE_{ij} \otimes E_{ji}\right)(a\otimes b)_{kN+l}=\sum_{i,j=1}^N(E_{ij}a)_k(E_{ji}b)_l=\\=\sum\limits_{i,j,p,q}(E_{ij})_{kp}a_p(E_{ji})_{lq}b_q=\sum\limits_{i,j,p,q}\delta_{ik}\delta_{jp}a_p\delta_{jl}\delta_{iq}b_q=a_lb_k=(b\otimes a)_{kN+l}
        \end{multline}
        \begin{equation}
            \boxed{P=\sum\limits_{i,j=1}^{N}E_{ij}\otimes E_{ji}}
        \end{equation}
        \item
        \begin{equation}
            P_{ij} (v_1 \otimes ... \otimes v_i \otimes ...\otimes v_j \otimes ... \otimes v_K)=(v_1 \otimes ... \otimes v_j \otimes ... \otimes v_i \otimes ...\otimes v_K)
        \end{equation}
        \begin{equation}
            \boxed{P_{ij}=\sum_{k,l=1}^{N} \bm{1}\otimes \dots \otimes\bm{1} \otimes \underbrace{E_{kl}}_{i}\otimes \dots \otimes \underbrace{E_{lk}}_{j}\otimes \bm{1}\otimes \dots \otimes\bm{1}}
        \end{equation}
        \begin{multline}
            P^2_{ij}(v_1 \otimes ... \otimes v_i \otimes ...\otimes v_j \otimes ... \otimes v_K)=P_{ij}(v_1 \otimes ... \otimes v_j \otimes ... \otimes v_i \otimes ...\otimes v_K)=\\=v_1 \otimes ... \otimes v_i \otimes ...\otimes v_j \otimes ... \otimes v_K
        \end{multline}
        \begin{equation}
            \boxed{P_{ij}P_{ij}=1}
        \end{equation}
        \begin{multline}
            P_{ij}P_{jk}(v_1 \otimes ... \otimes v_i \otimes ... \otimes v_j \otimes ...\otimes v_k\otimes ... \otimes v_K)=P_{ij}(v_1 \otimes ... \otimes v_i \otimes ... \otimes v_k \otimes ...\otimes v_j\otimes ... \otimes v_K)=\\=(v_1 \otimes ... \otimes v_k \otimes ... \otimes v_i \otimes ...\otimes v_j\otimes ... \otimes v_K)
        \end{multline}
        \begin{multline}
            P_{jk}P_{ik}(v_1 \otimes ... \otimes v_i \otimes ... \otimes v_j \otimes ...\otimes v_k\otimes ... \otimes v_K)=P_{jk}(v_1 \otimes ... \otimes v_k \otimes ... \otimes v_j \otimes ...\otimes v_i\otimes ... \otimes v_K)=\\=(v_1 \otimes ... \otimes v_k \otimes ... \otimes v_i \otimes ...\otimes v_j\otimes ... \otimes v_K)
        \end{multline}
        \begin{multline}
            P_{ik}P_{ij}(v_1 \otimes ... \otimes v_i \otimes ... \otimes v_j \otimes ...\otimes v_k\otimes ... \otimes v_K)=P_{ik}(v_1 \otimes ... \otimes v_j \otimes ... \otimes v_i \otimes ...\otimes v_k\otimes ... \otimes v_K)=\\=(v_1 \otimes ... \otimes v_k \otimes ... \otimes v_i \otimes ...\otimes v_j\otimes ... \otimes v_K)
        \end{multline}
        \begin{equation}
            \boxed{P_{ij}P_{jk}=P_{jk}P_{ik}=P_{ik}P_{ij}}
        \end{equation}
        \begin{multline}
            P_{ij}P_{ik}P_{jk}(v_1 \otimes ... \otimes v_i \otimes ... \otimes v_j \otimes ...\otimes v_k\otimes ... \otimes v_K)=P_{ij}P_{ik}(v_1 \otimes ... \otimes v_i \otimes ... \otimes v_k \otimes ...\otimes v_j\otimes ... \otimes v_K)=\\=P_{ij}(v_1 \otimes ... \otimes v_j \otimes ... \otimes v_k \otimes ...\otimes v_i\otimes ... \otimes v_K)=\\=(v_1 \otimes ... \otimes v_k \otimes ... \otimes v_j \otimes ...\otimes v_i\otimes ... \otimes v_K)
        \end{multline}
        \begin{multline}
            P_{jk}P_{ik}P_{ij}(v_1 \otimes ... \otimes v_i \otimes ... \otimes v_j \otimes ...\otimes v_k\otimes ... \otimes v_K)=P_{jk}P_{ik}(v_1 \otimes ... \otimes v_j \otimes ... \otimes v_i \otimes ...\otimes v_k\otimes ... \otimes v_K)=\\=P_{jk}(v_1 \otimes ... \otimes v_k \otimes ... \otimes v_i \otimes ...\otimes v_j\otimes ... \otimes v_K)=\\=(v_1 \otimes ... \otimes v_k \otimes ... \otimes v_j \otimes ...\otimes v_i\otimes ... \otimes v_K)
        \end{multline}
        \begin{equation}
            \boxed{P_{ij}P_{ik}P_{jk}=P_{jk}P_{ik}P_{ij}}
        \end{equation}
        \item
        \begin{equation}
            R_{ij}(z_i,z_j)=\frac{1}{\hbar} +\frac{P_{ij}}{z_i-z_j}
        \end{equation}
        \begin{multline}
            R_{ij} (z_i, z_j )R_{ik}(z_i, z_k)R_{jk}(z_j , z_k) =\left(\frac{1}{\hbar}+\frac{P_{ij}}{z_i-z_j}\right)\left(\frac{1}{\hbar}+\frac{P_{ik}}{z_i-z_k}\right)\left(\frac{1}{\hbar}+\frac{P_{jk}}{z_j-z_k}\right)=\\=\frac{1}{\hbar^3}+\frac{1}{\hbar^2}\left(\frac{P_{ij}}{z_i-z_j}+\frac{P_{ik}}{z_i-z_k}+\frac{P_{jk}}{z_j-z_k}\right)+\\+\frac{1}{\hbar}\left(\frac{P_{ij}}{z_i-z_j}\frac{P_{ik}}{z_i-z_k}+\frac{P_{ik}}{z_i-z_k}\frac{P_{jk}}{z_j-z_k}+\frac{P_{ij}}{z_i-z_j}\frac{P_{jk}}{z_j-z_k}\right)+\\+\frac{P_{ij}}{z_i-z_j}\frac{P_{ik}}{z_i-z_k}\frac{P_{jk}}{z_j-z_k}
        \end{multline}
        \begin{multline}
            R_{jk} (z_j, z_k )R_{ik}(z_i, z_k)R_{ij}(z_i , z_j)=\left(\frac{1}{\hbar}+\frac{P_{jk}}{z_j-z_k}\right)\left(\frac{1}{\hbar}+\frac{P_{ik}}{z_i-z_k}\right)\left(\frac{1}{\hbar}+\frac{P_{ij}}{z_i-z_j}\right)=\\=\frac{1}{\hbar^3}+\frac{1}{\hbar^2}\left(\frac{P_{jk}}{z_j-z_k}+\frac{P_{ik}}{z_i-z_k}+\frac{P_{ij}}{z_i-z_j}\right)+\\+\frac{1}{\hbar}\left(\frac{P_{jk}}{z_j-z_k}\frac{P_{ik}}{z_i-z_k}+\frac{P_{ik}}{z_i-z_k}\frac{P_{ij}}{z_i-z_j}+\frac{P_{ij}}{z_i-z_j}\frac{P_{jk}}{z_j-z_k}\right)+\\+\frac{P_{jk}}{z_j-z_k}\frac{P_{ik}}{z_i-z_k}\frac{P_{ij}}{z_i-z_j}
        \end{multline}
        \begin{multline}
            R_{ij}(z_i, z_j)R_{ik}(z_i, z_k)R_{jk}(z_j, z_k)-R_{jk} (z_j, z_k )R_{ik}(z_i, z_k)R_{ij}(z_i, z_j)=\\=\frac{1}{\hbar}\left(\frac{P_{ij}P_{ik}-P_{ik}P_{ij}}{(z_i-z_j)(z_i-z_k)}+\frac{P_{ik}P_{jk}-P_{jk}P_{ik}}{(z_i-z_k)(z_j-z_k)}+\frac{P_{ij}P_{jk}-P_{jk}P_{ij}}{(z_i-z_j)(z_j-z_k)}\right)+\\+\frac{P_{ij}P_{ik}P_{jk}-P_{jk}P_{ik}P_{ij}}{(z_j-z_k)(z_i-z_k)(z_i-z_j)}
        \end{multline}
        Using formulas from previous item, we obtain
        \begin{multline}
            R_{ij}(z_i,z_j)R_{ik}(z_i, z_k)R_{jk}(z_j, z_k)-R_{jk} (z_j, z_k )R_{ik}(z_i, z_k)R_{ij}(z_i, z_j)=\\=\frac{[P_{ij},P_{ik}]}{\hbar}\left(\frac{1}{(z_i-z_j)(z_i-z_k)}+\frac{1}{(z_i-z_k)(z_j-z_k)}-\frac{1}{(z_i-z_j)(z_j-z_k)}\right)=0
        \end{multline}
        \begin{equation}
            \boxed{R_{ij}(z_i,z_j)R_{ik}(z_i, z_k)R_{jk}(z_j, z_k)-R_{jk}(z_j, z_k)R_{ik}(z_i, z_k)R_{ij}(z_i, z_j)=0}
        \end{equation}
        So, matrices $R_{ij}$ satisfy the quantum Yang–Baxter equation.
        \begin{multline}
            R_{ij} (z_i, z_j )R_{ji}(z_j , z_i)=\left(\frac{1}{\hbar}+\frac{P_{ij}}{z_i-z_j}\right)\left(\frac{1}{\hbar}+\frac{P_{ji}}{z_j-z_i}\right)=\\=\frac{1}{\hbar^2}+\frac{1}{\hbar}\left(\frac{P_{ij}}{z_i-z_j}+\frac{P_{ji}}{z_j-z_i}\right)+ \frac{P_{ij}}{z_i-z_j}\frac{P_{ji}}{z_j-z_i}=\frac{1}{\hbar^2}-\frac{1}{(z_i-z_j)^2} \propto 1
        \end{multline}
        So, matrices $R_{ij}$ satisfy unitary condition.
        \item
        \begin{equation}
            R_{ij}(z_i,z_j)=\frac{1}{\hbar}+r_{ij}(z_i, z_j)+\mathcal{O}(\hbar)=\frac{1}{\hbar}+r_{ij}(z_i, z_j)+q_{ij}(z_i,z_j)\hbar+\mathcal{O}(\hbar^2)
        \end{equation}
        \begin{multline}
            R_{ij}(z_i,z_j)R_{ik}(z_i,z_k)R_{jk}(z_j,z_k)=\left(\frac{1}{\hbar}+r_{ij}(z_i, z_j)+q_{ij}(z_i,z_j)\hbar+\mathcal{O}(\hbar^2)\right)\times\\\times\left(\frac{1}{\hbar}+r_{ik}(z_i, z_k)+q_{ik}(z_i,z_k)\hbar+\mathcal{O}(\hbar^2)\right)\left(\frac{1}{\hbar}+r_{jk}(z_j,z_k)+q_{jk}(z_j,z_k)\hbar+\mathcal{O}(\hbar^2)\right)=\\=\frac{1}{\hbar^3}+\frac{1}{\hbar^2}(r_{ij}(z_i,z_j)+r_{ik}(z_i,z_k)+r_{jk}(z_j,z_k))+\\+\frac{1}{\hbar}(r_{ij}(z_i,z_j)r_{ik}(z_i,z_k)+r_{ik}(z_i,z_k)r_{jk}(z_j,z_k)+r_{ij}(z_i,z_j)r_{jk}(z_j,z_k)+\\+q_{ij}(z_i,z_j)+q_{ik}(z_i,z_k)+q_{jk}(z_j,z_k))+\mathcal{O}(1)
        \end{multline}
        \begin{multline}
            R_{jk}(z_j, z_k)R_{ik}(z_i, z_k)R_{ij}(z_i, z_j)=\left(\frac{1}{\hbar}+r_{jk}(z_j,z_k)+q_{jk}(z_j,z_k)\hbar+\mathcal{O}(\hbar^2)\right)\times\\\times\left(\frac{1}{\hbar}+r_{ik}(z_i, z_k)+q_{ik}(z_i,z_k)\hbar+\mathcal{O}(\hbar^2)\right)\left(\frac{1}{\hbar}+r_{ij}(z_i, z_j)+q_{ij}(z_i,z_j)\hbar+\mathcal{O}(\hbar^2)\right)=\\=\frac{1}{\hbar^3}+\frac{1}{\hbar^2}\left(r_{ij}(z_i,z_j)+r_{ik}(z_i,z_k)+r_{jk}(z_j,z_k)\right)+\\+\frac{1}{\hbar}(r_{jk}(z_j,z_k)r_{ik}(z_i,z_k)+r_{ik}(z_i,z_k)r_{ij}(z_i,z_j)+r_{jk}(z_j,z_k)r_{ij}(z_i,z_j)+\\+q_{ij}(z_i,z_j)+q_{ik}(z_i,z_k)+q_{jk}(z_j,z_k))+\mathcal{O}(1)
        \end{multline}
        \begin{multline}
            R_{ij}(z_i,z_j)R_{ik}(z_i, z_k)R_{jk}(z_j,z_k)-R_{jk} (z_j, z_k)R_{ik}(z_i,z_k)R_{ij}(z_i,z_j)=\\=\frac{1}{\hbar}([r_{ij}(z_i,z_j),r_{ik}(z_i,z_k)]+[r_{ik}(z_i,z_k),r_{jk}(z_j,z_k)]+\\+[r_{ij}(z_i,z_j),r_{jk}(z_j,z_k)])+\mathcal{O}(1)=0
        \end{multline}
        We obtain
        \begin{equation}
            \boxed{[r_{ij}(z_i, z_j), r_{ik}(z_i, z_k)] + [r_{ij}(z_i, z_j), r_{jk}(z_j, z_k)] + [r_{ik}(z_i, z_k), r_{jk}(z_j, z_k)] = 0}
        \end{equation}
        So, matrices $r_{ij}$ satisfy the classical Yang–Baxter equation.
    \end{itemize}
    \item \textbf{Higher flows.}\\
    Consider the Calogero–Moser system with Lax operator
    \begin{equation}
        L = \sum_{i=1}^n p_iE_{ii} +\sum_{i\neq j}\frac{\nu}{q_i-q_j}E_{ij},
    \end{equation}
    where $\nu$ is a constant and $p_i, q_j$ have the canonical Poisson brackets.
    \begin{itemize}
        \item Compute three first conservation laws using the $L$-operator
        \begin{equation}
            H_1 = \text{Tr\;}L,\quad H_2 = \frac{1}{2}\text{Tr\;}L^2,\quad H_3 = \frac{1}{3}\text{Tr\;}L^3
        \end{equation}
        \item Write the canonical equations of motion for coordinates and momenta in these three cases
        \begin{equation}
            \frac{dp_i}{dt_k}=\{H_k, p_i\},\quad \frac{dq_i}{dt_k}=\{H_k,q_i\}
        \end{equation}
        \item Check that the matrix
        \begin{equation}
            r_{12} = -\sum_{i\neq j}\frac{1}{q_i-q_j}E_{ij}\otimes E_{ji}-\sum_{i\neq j}\frac{1}{q_i-q_j}E_{ii}\otimes E_{ij}
        \end{equation}
        is the classical $r$-matrix for the Calogero–Moser Lax operator $L$.
        \item Compute three $M$-operators, corresponding to the Hamiltonians written above
        \begin{equation}
            \frac{dL}{dt_k}=\{H_k, L\}=[L, M_k]
        \end{equation}
        \item Check explicitly that the second and the third flows commute
        \begin{equation}
            \left[\frac{d}{dt_2}+M_2,\frac{d}{dt_3}+M_3\right]=0
        \end{equation}
    \end{itemize}
    \textbf{Solution.}
    \begin{itemize}
        \item 
        \begin{equation}
            L=\sum_{i=1}^n p_iE_{ii}+\sum_{i\neq j}\frac{\nu}{q_i-q_j}E_{ij}
        \end{equation}
        \begin{equation}
            \boxed{H_1=\text{Tr}L=\sum\limits_ip_i}
        \end{equation}
        \begin{multline}
            L^2=\sum_{i,j=1}^np_ip_jE_{ii}E_{jj}+\sum\limits_{k=1}^n\sum\limits_{i\neq j}\frac{\nu p_k}{q_i-q_j}(E_{kk}E_{ij}+E_{ij}E_{kk})+\\+\sum\limits_{i\neq j}\sum\limits_{k\neq l}\frac{\nu^2}{(q_i-q_j)(q_k-q_l)}E_{ij}E_{kl}
        \end{multline}
        \begin{equation}
            E_{ij}E_{kl}=\delta_{il}\delta_{jk}E_{ii}
        \end{equation}
        \begin{equation}
            L^2=\sum\limits_{i=1}^np_i^2E_{ii}-\sum\limits_{i\neq j}\frac{\nu^2}{(q_i-q_j)^2}E_{ii}
        \end{equation}
        \begin{equation}
            \boxed{H_2=\frac{1}{2}\text{Tr}L^2=\frac{1}{2}\sum\limits_ip_i^2-\frac{\nu^2}{2}\sum\limits_{i\neq j}\frac{1}{(q_i-q_j)^2}}
        \end{equation}
        \begin{multline}
            L^3=\sum_{i,j,k=1}^np_ip_jp_kE_{ii}E_{jj}E_{kk}+\sum\limits_{k,l=1}^n\sum\limits_{i\neq j}\frac{\nu p_kp_l}{q_i-q_j}(E_{kk}E_{ll}E_{ij}+E_{ll}E_{ij}E_{kk}+E_{ij}E_{kk}E_{ll})+\\+\sum\limits_{m=1}^n\sum\limits_{i\neq j}\sum\limits_{k\neq l}\frac{\nu^2p_m}{(q_i-q_j)(q_k-q_l)}(E_{mm}E_{ij}E_{kl}+E_{ij}E_{mm}E_{kl}+E_{ij}E_{kl}E_{mm})+\\+\sum\limits_{i\neq j}\sum\limits_{k\neq l}\sum\limits_{m\neq p}\frac{\nu^3}{(q_i-q_j)(q_k-q_l)(q_m-q_p)}E_{ij}E_{kl}E_{mp}=\\=\sum_{i=1}^np^3_iE_{ii}-3\nu^2\sum\limits_{i\neq j}\frac{p_i}{(q_i-q_j)^2}E_{ii}+\nu^3\sum\limits_{i\neq j\neq l}\frac{\nu^3}{(q_i-q_j)(q_j-q_l)(q_l-q_i)}E_{ii}
        \end{multline}
        \begin{equation}
            \boxed{H_3=\frac{1}{3}\text{Tr}L^3=\frac{1}{3}\sum\limits_ip_i^3-\nu^2\sum\limits_{i\neq j}\frac{p_i}{(q_i-q_j)^2}}
        \end{equation}
        \item 
        \begin{equation}
            \frac{dp_i}{dt_k}=\{H_k, p_i\},\quad \frac{dq_i}{dt_k}=\{H_k,q_i\}
        \end{equation}
        Consider cases:
        \begin{itemize}
            \item $k=1$.
            \begin{equation}
                \begin{cases}
                    \frac{dp_i}{dt_1}=\{H_1,p_i\}=\{\sum\limits_jp_j,p_i\},\\
                    \frac{dq_i}{dt_1}=\{H_1,q_i\}=\{\sum\limits_jp_j,q_i\};
                \end{cases}
            \end{equation}
            \begin{equation}
                \boxed{\begin{cases}
                    \frac{dp_i}{dt_1}=0,\\
                    \frac{dq_i}{dt_1}=1.
                \end{cases}}
            \end{equation}
            \item $k=2$.
            \begin{equation}
                \begin{cases}
                    \frac{dp_i}{dt_2}=\{H_2,p_i\}=\{\sum_{j}\limits\frac{p_j^2}{2}-\frac{\nu^2}{2}\sum_{j\neq k}\limits\frac{1}{(q_j-q_k)^2},p_i\},\\
                    \frac{dq_i}{dt_2}=\{H_2,q_i\}=\{\sum_{j}\limits\frac{p_j^2}{2}-\frac{\nu^2}{2}\sum_{j\neq k}\limits\frac{1}{(q_j-q_k)^2},q_i\};
               \end{cases}
            \end{equation} 
             \begin{equation}
                \boxed{\begin{cases}
                    \frac{dp_i}{dt_1}=-2\nu^2\sum_{j\neq i}\limits\frac{1}{(q_i-q_j)^3},\\
                    \frac{dq_i}{dt_1}=p_i.
                \end{cases}}
            \end{equation}
            \item $k=3$.
            \begin{equation}
                \begin{cases}
                    \frac{dp_i}{dt_3}=\{H_3,p_i\}=\{\sum_{k}\limits\frac{p_k^3}{3}
                    -\nu^2\sum_{k\neq j}\limits\frac{p_k}{(q_k-q_j)^2},p_i\},\\
                    \frac{dq_i}{dt_3}=\{H_3,q_i\}=\{\sum_{k}\limits\frac{p_k^3}{3}
                    -\nu^2\sum_{k\neq j}\limits\frac{p_k}{(q_k-q_j)^2},q_i\};
                \end{cases}
            \end{equation}
            \begin{equation}
                \boxed{\begin{cases}
                    \frac{dp_i}{dt_3}=-2\nu^2 \sum_{k\neq i}\limits\frac{p_i+p_k}{(q_i-q_k)^3},\\
                    \frac{dq_i}{dt_3}=p_i-\nu^2 \sum_{k\neq i}\limits\frac{1}{(q_i-q_k)^2}.
                \end{cases}}
            \end{equation}
        \end{itemize}
    \end{itemize}
\end{enumerate}
\section{Integrable systems related to Lie algebras}
\begin{enumerate}
    \item \textbf{Weyl group.}\\
    Let $\mathfrak{g}$ be a simple finite–dimensional complex Lie algebra and $R$ its root system. Weyl group $W$ is generated by all the reflections with respect to all roots
    \begin{equation}
        w_\alpha(\beta)=\beta-\frac{2\braket{\alpha,\beta}}{\braket{\alpha,\alpha}}\alpha
    \end{equation}
    \begin{itemize}
        \item Prove that Weyl group $W$ is finite.
        \item Consider a simple Lie algebra $\mathfrak{g} = \mathfrak{sl}(n)$, prove that the Weyl group of $\mathfrak{sl}(n)$ is isomorphic to the symmetric group $S_{n-1}$.
    \end{itemize}
    An abstract crystallographic root system $\Delta$ is a collection of the following data:
    \begin{itemize}
        \item[(a)] A finite-dimensional Euclidean space $E$ and a finite set of its vectors $\Delta$, which span the whole space $E$.
        \item[(b)] The only scalar multiples of a root $\alpha\in\Delta$ are $\alpha$ itself and $-\alpha\in\Delta$.
        \item[(c)] For any two roots $\alpha,\beta\in\Delta$ it follows that $w_\alpha(\beta)$ belongs to the root system $\Delta$.
        \item[(d)] For any two roots $\alpha,\beta\in\Delta$, the number $\frac{2\braket{\alpha,\beta}}{\braket{\alpha,\alpha}}$ is integer.
    \end{itemize}
    \begin{itemize}
        \item Define a coroot by the formula $\alpha^\vee=\frac{2\alpha}{\braket{\alpha,\alpha}}$ and denote the set of coroots by $\Delta^\vee$. Prove that $\Delta^\vee$ is again a root system.
        \item Describe the dual root system $\Delta^\vee$ for the case of a root system of $\mathfrak{g}=\mathfrak{sl}(n)$.
        \item Prove that Weyl groups of $\Delta$ and $\Delta^\vee$ are isomorphic.
    \end{itemize}
    \textbf{Solution.}
    \begin{itemize}
        \item
        \begin{equation}
            \text{ad}_he_\alpha=\alpha(h)e_\alpha,\quad \alpha(h)=(H_\alpha,h)
        \end{equation}
        Let $\mathfrak{h}$ be a Cartan subalgebra of $\mathfrak{g}$, $\{h_\alpha\}$ is a basis in $\mathfrak{h}$. Prove that $W$ keeps the root system.\\
        Finite-dimensional representations of $\mathfrak{sl}_2(\mathbb{C})$ ($[e,f]=h$, $[h,f]=-2f$, $[h,e]=2e$):
        \begin{equation}
            V=\underset{n\in\mathbb{Z}}{\oplus}V(n)
        \end{equation}
        For $V(n)$:
        \begin{equation}
            ev^k=(n+1-k)v^{k-1},\quad fv^k=(k+1)v^{k+1},\quad hv^k=(n-2k)v^k
        \end{equation}
        \begin{equation}
            ev^0=0,\quad fv^n=0
        \end{equation}
        If $\alpha$ is root, then $-\alpha$ is also a root. Let be $\mathfrak{g}=\mathfrak{h}\oplus\underset{\alpha\in\Delta}{\oplus}\mathfrak{g}_\alpha$ and $e_\alpha\in\mathfrak{g}_\alpha$, $f_\alpha\in\mathfrak{g}_{-\alpha}$, $h_\alpha\in\mathfrak{h}$:
        \begin{equation}
            (e_\alpha,f_\alpha)=\frac{2}{\braket{\alpha,\alpha}},\quad h_\alpha=\frac{2H_\alpha}{\braket{\alpha,\alpha}}
        \end{equation}
        \begin{equation}
            ([e_\alpha,f_\alpha],h)=[h,(e_\alpha,f_\alpha)]=\alpha(h)(e_\alpha,f_\alpha)=(H_\alpha,h)(e_\alpha,f_\alpha)
        \end{equation}
        \begin{equation}
            [e_\alpha,f_\alpha]=H_\alpha(e_\alpha,f_\alpha)=h_\alpha
        \end{equation}
        \begin{equation}
            [h_\alpha,f_\alpha]=-\alpha(h_\alpha)f_\alpha=-(H_\alpha,h_\alpha)f_\alpha=-2f_\alpha
        \end{equation}
        \begin{equation}
            [h_\alpha,e_\alpha]=\alpha(h_\alpha)e_\alpha=(H_\alpha,h_\alpha)e_\alpha=2e_\alpha
        \end{equation}
        So, $\mathfrak{g}$ is a representation of $\mathfrak{sl}_2(\mathbb{C})$. Let be $e_\beta\in\mathfrak{g}_\beta$, then
        \begin{equation}
            \text{ad}_{h_\alpha}e_\beta=\beta(h_\alpha)e_\beta=(h_\alpha,H_\beta)e_\beta=\frac{2\braket{\alpha,\beta}}{\braket{\alpha,\alpha}}e_\beta,\quad\frac{2\braket{\alpha,\beta}}{\braket{\alpha,\alpha}}\in\mathbb{Z}
        \end{equation}
        \begin{equation}
            \text{ad}_{f_\alpha}e_\beta=[f_\alpha,e_\beta]\subset\mathfrak{g}_{\beta-\alpha}\rightarrow\text{ad}^n_{f_\alpha}e_\beta=[f_\alpha,...[f_\alpha,e_\beta]]\subset\mathfrak{g}_{\beta-\alpha n}
        \end{equation}
        Let be $n=\frac{2\braket{\alpha,\beta}}{\braket{\alpha,\alpha}}$. If $\alpha,\beta\in\Delta$, then $w_\alpha(\beta)=\beta-\frac{2\braket{\alpha,\beta}}{\braket{\alpha,\alpha}}\alpha\in\Delta$. So, $W$ keeps the root system.\\
        $\Delta$ is a finite set, so $W$ is a finite group.
        \item Prove that the Weyl group of $\mathfrak{sl}(n)$ is isomorphic to the symmetric group $S_n$.
        \begin{equation}
            S_n=\{\sigma_1,\sigma_2,...,\sigma_{n-1}|\sigma_i^2=1,(\sigma_i\sigma_{i+1})^3=e,[\sigma_i,\sigma_j]=0,|i-j|>1\}
        \end{equation}
        Let be $\beta=\lambda_k-\lambda_l\in\Delta$. Consider the comparison $\sigma_i$ with $w_{\alpha_i}(\beta)$, $\alpha_i=\lambda_i-\lambda_{i+1}$ -- simple root.
        \begin{equation}
            \braket{\lambda_i,\lambda_j}=\delta_{ij}\rightarrow\braket{\alpha_i,\alpha_i}=2,\quad\braket{\alpha_i,\alpha_j}=-1,\quad|i-j|=1
        \end{equation}
        \begin{multline}
            w^2_{\alpha_i}(\beta)=w_{\alpha_i}(\beta-\braket{\alpha_i,\beta}\alpha_i)=\beta-\braket{\alpha_i,\beta}\alpha_i-\braket{\alpha_i,\beta}\alpha_i+\braket{\alpha_i,\beta}\braket{\alpha_i,\alpha_i}\alpha_i=\\=\beta-\braket{\alpha_i,\beta}\alpha_i-\braket{\alpha_i,\beta}\alpha_i+2\braket{\alpha_i,\beta}\alpha_i=\beta
        \end{multline}
        Let be $|i-j|=1$, than
        \begin{multline}
            w_{\alpha_i}w_{\alpha_j}w_{\alpha_i}(\beta)=w_{\alpha_i}w_{\alpha_j}(\beta-\braket{\alpha_i,\beta}\alpha_i)=\\=w_{\alpha_i}(\beta-\braket{\alpha_i,\beta}\alpha_i-\braket{\alpha_j,\beta}\alpha_j+\braket{\alpha_i,\beta}\braket{\alpha_j,\alpha_i}\alpha_j)=\\=w_{\alpha_i}(\beta-\braket{\alpha_i,\beta}\alpha_i-\braket{\alpha_j,\beta}\alpha_j-\braket{\alpha_i,\beta}\alpha_j)=\\=\beta-\braket{\alpha_i,\beta}\alpha_i-\braket{\alpha_j,\beta}\alpha_j-\braket{\alpha_i,\beta}\alpha_j-\braket{\alpha_i,\beta}\alpha_i+2\braket{\alpha_i,\beta}\alpha_i-\\-\braket{\alpha_j,\beta}\alpha_i-\braket{\alpha_i,\beta}\alpha_i-\braket{\alpha_i,\beta}\alpha_i=\\=\beta-\braket{\alpha_j,\beta}\alpha_j-\braket{\alpha_i,\beta}\alpha_j-\braket{\alpha_j,\beta}\alpha_i-\braket{\alpha_i,\beta}\alpha_i-\braket{\alpha_i,\beta}\alpha_i=\\=w_{\alpha_j}w_{\alpha_i}w_{\alpha_j}(\beta)
        \end{multline}
        \begin{equation}
            (w_{\alpha_i}w_{\alpha_{i+1}})^3(\beta)=\beta
        \end{equation}
        Let be $|i-j|>1$, than
        \begin{multline}
            w_{\alpha_i}w_{\alpha_j}(\beta)=w_{\alpha_i}(\beta-\braket{\alpha_j,\beta}\alpha_j)=\beta-\braket{\alpha_j,\beta}\alpha_j-\braket{\alpha_i,\beta}\alpha_i+\braket{\alpha_j,\beta}\braket{\alpha_i,\alpha_j}\alpha_i=\\=\beta-\braket{\alpha_j,\beta}\alpha_j-\braket{\alpha_i,\beta}\alpha_i=w_{\alpha_j}w_{\alpha_i}(\beta)
        \end{multline}
        \begin{equation}
            (w_{\alpha_i}w_{\alpha_j})^2(\beta)=w_{\alpha_i}w_{\alpha_j}w_{\alpha_j}w_{\alpha_i}(\beta)=\beta
        \end{equation}
        \begin{equation}
            \boxed{W\simeq S_n}
        \end{equation}
        \item
        \begin{equation}
            \alpha^\vee=\frac{2\alpha}{\braket{\alpha,\alpha}}\subset\Delta^\vee
        \end{equation}
        Prove that $\Delta^\vee$ is a system of roots.
        \begin{itemize}
            \item[(a)] $\Delta^\vee$ spans the space $E$.
            \item[(b)] If $\alpha^\vee\in\Delta^\vee$, then $-\alpha^\vee\in\Delta^\vee$.
            \item[(c)] Let $\alpha^\vee,\beta^\vee\in\Delta^\vee$, then
            \begin{multline}
                w_{\alpha^\vee}(\beta^\vee)=\beta^\vee-\frac{2\braket{\alpha^\vee,\beta^\vee}}{\braket{\alpha^\vee,\alpha^\vee}}\alpha^\vee=\frac{2\beta}{\braket{\beta,\beta}}-\frac{2\braket{\alpha,\beta}\braket{\alpha,\alpha}^2}{\braket{\alpha,\alpha}\braket{\beta,\beta}\braket{\alpha,\alpha}}\frac{2\alpha}{\braket{\alpha,\alpha}}=\\=\frac{2\beta}{\braket{\beta,\beta}}-\frac{2\braket{\alpha,\beta}}{\braket{\beta,\beta}}\frac{2\alpha}{\braket{\alpha,\alpha}}=\frac{2\beta}{\braket{\beta,\beta}}-\frac{4\braket{\alpha,\beta}}{\braket{\alpha,\alpha}\braket{\beta,\beta}}\alpha
            \end{multline}
            \begin{equation}
                (w_\alpha(\beta))^\vee=\frac{2\left(\beta-\frac{2\braket{\alpha,\beta}}{\braket{\alpha,\alpha}}\alpha\right)}{\braket{\beta-\frac{2\braket{\alpha,\beta}}{\braket{\alpha,\alpha}}\alpha,\beta-\frac{2\braket{\alpha,\beta}}{\braket{\alpha,\alpha}}\alpha}}
            \end{equation}
            \begin{equation}
                \left<\beta-\frac{2\braket{\alpha,\beta}}{\braket{\alpha,\alpha}}\alpha,\beta-\frac{2\braket{\alpha,\beta}}{\braket{\alpha,\alpha}}\alpha\right>=\braket{\beta,\beta}-\frac{4\braket{\alpha,\beta}^2}{\braket{\alpha,\alpha}}+\frac{4\braket{\alpha,\beta}^2}{\braket{\alpha,\alpha}}=\braket{\beta,\beta}
            \end{equation}
            \begin{equation}
                (w_\alpha(\beta))^\vee=\frac{2\beta}{\braket{\beta,\beta}}-\frac{4\braket{\alpha,\beta}}{\braket{\alpha,\alpha}\braket{\beta,\beta}}\alpha
            \end{equation}
            \begin{equation}
                w_{\alpha^\vee}(\beta^\vee)=(w_\alpha(\beta))^\vee\subset\Delta^\vee
            \end{equation}
            \item[(d)]
            \begin{equation}
                \frac{2\braket{\alpha^\vee,\beta^\vee}}{\braket{\alpha^\vee,\alpha^\vee}}=\frac{2\braket{\alpha,\beta}\braket{\alpha,\alpha}^2}{\braket{\alpha,\alpha}\braket{\beta,\beta}\braket{\alpha,\alpha}}=\frac{2\braket{\alpha,\beta}}{\braket{\alpha,\alpha}\braket{\beta,\beta}}\in\mathbb{Z}
            \end{equation}
        \end{itemize}
        \item A root system of $\mathfrak{g}=\mathfrak{sl}(n)$:
        \begin{equation}
            \Delta=\{\lambda_i-\lambda_j|1\leq i,j\leq n, i\neq j\}
        \end{equation}
        \begin{equation}
            (\lambda_i-\lambda_j)^\vee=\frac{2(\lambda_i-\lambda_j)}{\braket{\lambda_i-\lambda_j,\lambda_i-\lambda_j}}=\lambda_i-\lambda_j
        \end{equation}
        \begin{equation}
            \boxed{\Delta^\vee=\Delta}
        \end{equation}
        \item Prove that Weyl groups of $\Delta$ and $\Delta^\vee$ are isomorphic.
        \begin{equation}
            \varphi:(\beta\rightarrow w_\alpha(\beta))\rightarrow(\beta^\vee\rightarrow w_{\alpha^\vee}(\beta^\vee)=(w_\alpha(\beta))^\vee)
        \end{equation}
        $\varphi$ is a homomorphism, since
        \begin{equation}
            (w_{\alpha_2}(w_{\alpha_1}(\beta)))^\vee=w_{\alpha^\vee_2}(w_{\alpha_1}(\beta))^\vee=w_{\alpha^\vee_2}(w_{\alpha^\vee_1}(\beta^\vee))
        \end{equation}
        Since $\varphi$ is a bijection, then $\varphi$ is isomorphism.
    \end{itemize}
    \item \textbf{$\mathfrak{g}_2$ Lie algebra.}\\
    Let $V$ be a three–dimensional complex vector space. Let $\mathfrak{sl}(V)$ be traceless matrices on $V$ and consider the following vector space $\mathfrak{g}_2=V^*\oplus\mathfrak{sl}(V)\oplus V$ with an antisymmetric bracket on it
    \begin{equation}
        [A,B]:=\begin{cases}
            AB-BA,\quad\quad\quad\quad\;\;\; A,B\in\mathfrak{sl}(V)\\
            A(B),\quad\quad\quad\quad\quad\quad\;\;\; A\in\mathfrak{sl}(V), B\in V\\
            -A(B),\quad\quad\quad\quad\quad\quad A\in\mathfrak{sl}(V), B\in V^*\\
            A\otimes B-\frac{1}{3}B(A)\cdot1,\quad A\in V,B\in V^*
        \end{cases}
    \end{equation}
    \begin{itemize}
        \item Describe the bracket as a matrix commutator with the help of $7\times7$ matrices.
        \item Prove that the bracket satisfies all the Lie algebra axioms, thus making $\mathfrak{g}_2$ a Lie algebra.
        \item Check that diagonal matrices in $\mathfrak{sl}(V)$ form the Cartan subalgebra of $\mathfrak{g}_2$, and describe the root system of $\mathfrak{g}_2$. Which lengths do the simple roots have? Which angles are between the simple roots?
        \item Is $\mathfrak{g}_2$ a semisimple Lie algebra? Is it simple?
        \item Write down a Cartan matrix and draw the Dynkin diagram for $\mathfrak{g}_2$ algebra.
    \end{itemize}
    \textbf{Solution.}
    \begin{itemize}
        \item 
        \item 
        \item Diagonal matrices in $\mathfrak{sl}(V)$:
        \begin{equation}
            h=\begin{pmatrix}
                h_1 & 0 & 0\\
                0 & h_2 & 0\\
                0 & 0 & h_3
            \end{pmatrix},\quad h_1+h_2+h_3=0
        \end{equation}
        All matrices $h$ commutate, so subalgebra of this matrices is abelian. Let be $A\in\mathfrak{sl}(V)$, then
        \begin{equation}
            A=\begin{pmatrix}
                a_{11} & a_{12} & a_{13}\\
                a_{21} & a_{22} & a_{23}\\
                a_{31} & a_{32} & a_{33}
            \end{pmatrix},\quad a_{11}+a_{22}+a_{33}=0
        \end{equation}
        \begin{equation}
            \text{ad}_hA=[h,A]=\begin{pmatrix}
                0 & a_{12}(h_1-h_2) & a_{33}(h_1-h_3)\\
                a_{21}(h_2-h_1) & 0 & a_{23}(h_2-h_3)\\
                a_{31}(h_3-h_1) & a_{32}(h_3-h_2) & 0
            \end{pmatrix}
        \end{equation}
        \begin{equation}
            \text{ad}_he_\alpha=\alpha(h)e_\alpha
        \end{equation}
        For $e_\alpha=E_{ij}:\alpha(h)=h_i-h_j$. Let be $\lambda_i\in\mathfrak{h}^*:$
        \begin{equation}
            \lambda_i(h)=h_i\rightarrow \alpha=\lambda_i-\lambda_j
        \end{equation}
        The space of roots of $\mathfrak{sl}(V)$:
        \begin{equation}
            \Delta=\{\pm(\lambda_1-\lambda_2),\pm(\lambda_1-\lambda_3)=\pm(2\lambda_1+\lambda_2),\pm(\lambda_2-\lambda_3)=\pm(\lambda_1+2\lambda_2)\}
        \end{equation}
        For $e_\alpha\in\left\{\begin{pmatrix}
            1\\
            0\\
            0
        \end{pmatrix},\begin{pmatrix}
            0\\
            1\\
            0
        \end{pmatrix},\begin{pmatrix}
            0\\
            0\\
            1
        \end{pmatrix}\right\}\subset V$:
        \begin{equation}
            \text{ad}_he_\alpha=h(e_\alpha)=\alpha(h)e_\alpha\rightarrow \alpha(h)=h_i\rightarrow\alpha=\lambda_i
        \end{equation}
        For $e_\alpha\in\left\{\begin{pmatrix}
            1 & 0 & 0
        \end{pmatrix},\begin{pmatrix}
            0 & 1 & 0
        \end{pmatrix},\begin{pmatrix}
            0 & 0 & 1
        \end{pmatrix}\right\}\subset V^*$:
        \begin{equation}
            \text{ad}_he_\alpha=-h(e_\alpha)=\alpha(h)e_\alpha\rightarrow \alpha(h)=-h_i\rightarrow\alpha=-\lambda_i
        \end{equation}
        The space of roots of $V\oplus V^*$:
        \begin{equation}
            \Delta=\{\pm\lambda_1,\pm\lambda_2,\pm\lambda_3=\mp(\lambda_1+\lambda_2)\}
        \end{equation}
        Thus, matrices $h$ form Cartan subalgebra ($\text{ad}_h$ is diagonalizable for all $h\in\mathfrak{h}$).\\
        Simple roots:
        \begin{equation}
            \boxed{\prod=\{\lambda_1-\lambda_2,\lambda_2\}}
        \end{equation}
        Positive roots:
        \begin{equation}
            \Delta_+=\{\lambda_1,\lambda_2,\lambda_1+\lambda_2,\lambda_1-\lambda_2,2\lambda_1+\lambda_2,\lambda_1+2\lambda_2\}
        \end{equation}
        Negative roots:
        \begin{equation}
            \Delta_-=\{-\lambda_1,-\lambda_2,-\lambda_1-\lambda_2,\lambda_2-\lambda_1,-2\lambda_1-\lambda_2,-\lambda_1-2\lambda_2\}
        \end{equation}
        A map $\mathfrak{h}\rightarrow\mathfrak{h}^*$:
        \begin{equation}
            \lambda_i(h)=(h_{\lambda_i},h)=\text{Tr}(\text{ad}_{h_i}\text{ad}_h)
        \end{equation}
        Nondegenerate bilinear form on $\mathfrak{h}^*$:
        \begin{equation}
            \braket{\lambda_i,\lambda_j}=(h_i,h_j)=\text{Tr}(\text{ad}_{h_i}\text{ad}_{h_j})
        \end{equation}
        %For $\mathfrak{sl}(n)$:
        %\begin{equation}
            %\lambda_i(h)=6\text{Tr}(h_ih)=6((h_i)_{11}h_1+(h_i)_{22}h_2+((h_i)_{11}+(h_i)_{22})(h_1+h_2))=h_i
        %\end{equation}
        %\begin{equation}
            %\lambda_1(h)=6((h_1)_{11}h_1+(h_1)_{22}h_2+((h_1)_{11}+(h_1)_{22})(h_1+h_2))=h_1
        %\end{equation}
        %\begin{equation}
            %\begin{cases}
                %2(h_1)_{11}+(h_{1})_{22}=\frac{1}{6},\\
                %2(h_1)_{22}+(h_{1})_{11}=0;
            %\end{cases}\rightarrow (h_1)_{11}=
        %\end{equation}
        \begin{equation}
            \lambda_i(h)=\text{Tr}(h_ih)=((h_i)_{11}h_1+(h_i)_{22}h_2+((h_i)_{11}+(h_i)_{22})(h_1+h_2))=h_i
        \end{equation}
        \begin{equation}
            \lambda_1(h)=((h_1)_{11}h_1+(h_1)_{22}h_2+((h_1)_{11}+(h_1)_{22})(h_1+h_2))=h_1
        \end{equation}
        \begin{equation}
            \begin{cases}
                2(h_1)_{11}+(h_{1})_{22}=1,\\
                2(h_1)_{22}+(h_{1})_{11}=0;
            \end{cases}\rightarrow (h_1)_{11}=\frac{2}{3}, (h_1)_{22}=-\frac{1}{3}
        \end{equation}
        \begin{equation}
            h_1=\begin{pmatrix}
                \frac{2}{3} & 0 & 0\\
                0 & -\frac{1}{3} & 0\\
                0 & 0 & -\frac{1}{3}
            \end{pmatrix}
        \end{equation}
        \begin{equation}
            \lambda_2(h)=((h_2)_{11}h_1+(h_2)_{22}h_2+((h_2)_{11}+(h_2)_{22})(h_1+h_2))=h_2
        \end{equation}
        \begin{equation}
            \begin{cases}
                2(h_2)_{11}+(h_2)_{22}=0,\\
                2(h_2)_{22}+(h_2)_{11}=1;
            \end{cases}\rightarrow (h_2)_{11}=-\frac{1}{3}, (h_2)_{22}=\frac{2}{3}
        \end{equation}
        \begin{equation}
            h_2=\begin{pmatrix}
                -\frac{1}{3} & 0 & 0\\
                0 & \frac{2}{3} & 0\\
                0 & 0 & -\frac{1}{3}
            \end{pmatrix}
        \end{equation}
        \begin{equation}
            \braket{\lambda_i,\lambda_j}=\text{Tr}(h_ih_j)
        \end{equation}
        \begin{equation}
            \braket{\lambda_1,\lambda_1}=\braket{\lambda_2,\lambda_2}=\frac{1+4+1}{9}=\frac{2}{3},\quad\braket{\lambda_1,\lambda_2}=\frac{-2-2+1}{9}=-\frac{1}{3}
        \end{equation}
        \begin{equation}
            \braket{\lambda_1-\lambda_2,\lambda_1-\lambda_2}=\braket{\lambda_1,\lambda_1}+\braket{\lambda_2,\lambda_2}-2\braket{\lambda_1,\lambda_2}=\frac{4}{3}+\frac{2}{3}=2
        \end{equation}
        Length of $\lambda_1-\lambda_2$ is $\sqrt{2}$ and length of $\lambda_2$ is $\sqrt{\frac{2}{3}}$.
        \begin{equation}
            \braket{\lambda_1-\lambda_2,\lambda_2}=-\frac{1}{3}-\frac{2}{3}=-1
        \end{equation}
        Angle between simple roots:
        \begin{equation}
            \cos\alpha=\frac{\braket{\lambda_1-\lambda_2,\lambda_2}}{\sqrt{\braket{\lambda_1-\lambda_2,\lambda_1-\lambda_2}\braket{\lambda_2,\lambda_2}}}=\frac{-1}{\sqrt{2}\sqrt{\frac{2}{3}}}=-\frac{\sqrt{3}}{2}\rightarrow\boxed{\alpha=\frac{5\pi}{6}}
        \end{equation}
        \item Prove, that group $\mathfrak{g}_2$ is simple. Suppose the contrary, let he algebra have a nontrivial ideal $I$:
        \begin{equation}
            i=v+A+\Tilde{v}\in I,\quad v\in V, A\in\mathfrak{sl}(V),\tilde{v}\in V^*
        \end{equation}
        $\forall v,u\in V\hookrightarrow\exists A\in\mathfrak{sl}(V):[A,v]=u$, so
        \begin{equation}
            V\oplus V^*\subset I
        \end{equation}
        $\forall A\in \mathfrak{sl}(V)\hookrightarrow\exists v,u\in V:[v,u]=A$, so
        \begin{equation}
            \mathfrak{g}_2=V^*\oplus\mathfrak{sl}(V)\oplus V=I
        \end{equation}
        Thus, $\mathfrak{g}_2$ is simple. So, $\mathfrak{g}_2$ is semisimple.
        \item Cartan matrix:
        \begin{equation}
            a_{ij}=\frac{2\braket{\alpha_j,\alpha_i}}{\braket{\alpha_i,\alpha_i}}
        \end{equation}
        \begin{equation}
            a_{11}=a_{22}=2,\quad a_{12}=\frac{-2}{\frac{2}{3}}=-3,\quad a_{21}=\frac{-2}{2}=-1
        \end{equation}
        \begin{equation}
            \boxed{A=\begin{pmatrix}
                2 & -3\\
                -1 & 2
            \end{pmatrix}}
        \end{equation}
        Dynkin diagram: \tble{G/2}
    \end{itemize}
    \item \textbf{Dynkin diagrams and Cartan matrices.}\\
    Cartan matrix is a matrix with elements
    \begin{equation}
        a_{ij}=\frac{2\braket{\alpha_j,\alpha_i}}{\braket{\alpha_i,\alpha_i}},
    \end{equation}
    where $\prod=\{\alpha_1,...,\alpha_r\}$ is a collection of simple roots of a root system. Dynkin diagram is a finite graph with vertices representing simple roots and two vertices are connected by 0 edges of roots are orthogonal, 1 if the angle between roots is $\frac{2\pi}{3}$, 2 if the angle is $\frac{3\pi}{4}$, 3 if the angle is $\frac{5\pi}{6}$. Additionally, we orient those edges connecting the simple roots of different lengths from the long one to the short one.
    \begin{itemize}
        \item Prove that for a complex semisimple Lie algebra the Cartan matrix is nondegenerate.
        \item Prove that Cartan matrix a symmetrisable positively defined matrix, i.e. it can be written as a product of diagonal matrix with positive elements and a symmetric positively defined matrix.
        \item Describe all Cartan matrices of rank 2 and draw the corresponding Dynkin diagrams on the plane. Find lengths of all roots and angles between all roots. Name the obtained root systems.
        \item Describe explicitly the Weyl groups for the root systems constructed above for the Cartan matrices of rank 2.
        \item Prove that Dynkin diagram without multiple edges, i.e. two vertices are connected by either 0 or 1 edge, can't have cycles and vertices with degree $\geq 4$.
    \end{itemize}
    \textbf{Solution.}
    \begin{itemize}
        \item Cartan matrix:
        \begin{equation}
            A=(a_{ij}),\quad a_{ij}=\frac{2\braket{\alpha_j,\alpha_i}}{\braket{\alpha_i,\alpha_i}}
        \end{equation}
        By multiplying the $i$ line of Cartan matrix $A$ by the positive number $\frac{\braket{\alpha_i,\alpha_i}}{2}$, it becomes the matrix $A=\braket{\alpha_j,\alpha_i}$ which has positive determinant because the simple roots span a Euclidean space.
        \item 
        \begin{equation}
            A=DS,\quad D_{ij}=\frac{\delta_{ij}}{2}\braket{\alpha_i,\alpha_i},\quad S_{ij}=\braket{\alpha_j,\alpha_i}
        \end{equation}
        where $D$ -- diagonal matrix with positive elements $\frac{\braket{\alpha_i,\alpha_i}}{2}>0$ and $S$ -- symmetric Gram matrix $\braket{\alpha_i,\alpha_j}$. Gram matrix is nondegenerate.
        \item
        \begin{equation}
            a_{ii}=\frac{2\braket{\alpha_i,\alpha_i}}{\braket{\alpha_i,\alpha_i}}=2,\quad a_{ij}\leq0,\quad i\neq j
        \end{equation}
        If $a_{ij}=0$, then $a_{ji}=0$, because in this case $\braket{\alpha_i,\alpha_j}=0$.\\
        Cartan matrix of rank 2:
        \begin{equation}
            A=\begin{pmatrix}
                2 & a\\
                b & 2
            \end{pmatrix}\rightarrow \det A=4-ab>0
        \end{equation}
        Consider possible cases:
        \begin{itemize}
            \item $a=b=0$.\\
            Cartan matrix:
            \begin{equation}
                A=\begin{pmatrix}
                    2 & 0\\
                    0 & 2
                \end{pmatrix}\rightarrow\braket{\alpha_1,\alpha_1}=c_1^2,\quad\braket{\alpha_2,\alpha_2}=c_2^2,\quad\braket{\alpha_1,\alpha_2}=0
            \end{equation}
            Length of roots:
            \begin{equation}
                \boxed{|\alpha_1|=\sqrt{\braket{\alpha_1,\alpha_1}}=c_1,\quad|\alpha_2|=\sqrt{\braket{\alpha_2,\alpha_2}}=c_2}
            \end{equation}
            Angle between roots:
            \begin{equation}
                \boxed{\cos\alpha=\frac{\braket{\alpha_1,\alpha_2}}{\sqrt{\braket{\alpha_1,\alpha_1}\braket{\alpha_2,\alpha_2}}}=0\rightarrow\alpha=\frac{\pi}{2}}
            \end{equation}
            Dynkin diagram for $A_1\times A_1\simeq D_2\leftarrow\mathfrak{so}(4)$: \tble{A/1}\tble{A/1}
            \item $a=b=-1$.\\
            Cartan matrix:
            \begin{equation}
                A=\begin{pmatrix}
                    2 & -1\\
                    -1 & 2
                \end{pmatrix}\rightarrow\braket{\alpha_1,\alpha_1}=\braket{\alpha_2,\alpha_2}=c^2,\quad\braket{\alpha_1,\alpha_2}=-\frac{c^2}{2}
            \end{equation}
            Length of roots:
            \begin{equation}
                \boxed{|\alpha_1|=\sqrt{\braket{\alpha_1,\alpha_1}}=|\alpha_2|=\sqrt{\braket{\alpha_2,\alpha_2}}=c}
            \end{equation}
            Angle between roots:
            \begin{equation}
                \boxed{\cos\alpha=\frac{\braket{\alpha_1,\alpha_2}}{\sqrt{\braket{\alpha_1,\alpha_1}\braket{\alpha_2,\alpha_2}}}=-\frac{1}{2}\rightarrow\alpha=\frac{2\pi}{3}}
            \end{equation}
            Dynkin diagram for $A_2\leftarrow\mathfrak{sl}(3)$: \tble{A/2}
            \item $a=-1, b=-2$.\\
            Cartan matrix:
            \begin{equation}
                A=\begin{pmatrix}
                    2 & -1\\
                    -2 & 2
                \end{pmatrix}\rightarrow\braket{\alpha_1,\alpha_1}=2c^2,\quad\braket{\alpha_2,\alpha_2}=c^2,\quad\braket{\alpha_1,\alpha_2}=-c^2
            \end{equation}
            Length of roots:
            \begin{equation}
                \boxed{|\alpha_1|=\sqrt{\braket{\alpha_1,\alpha_1}}=\sqrt{2}c,\quad|\alpha_2|=\sqrt{\braket{\alpha_2,\alpha_2}}=c}
            \end{equation}
            Angle between roots:
            \begin{equation}
                \boxed{\cos\alpha=\frac{\braket{\alpha_1,\alpha_2}}{\sqrt{\braket{\alpha_1,\alpha_1}\braket{\alpha_2,\alpha_2}}}=-\frac{1}{\sqrt{2}}\rightarrow\alpha=\frac{3\pi}{4}}
            \end{equation}
            Dynkin diagram for $B_2\leftarrow\mathfrak{so}(5)$: \tble{B/2}
            \item $a=-1, b=-3$.\\
            Cartan matrix:
            \begin{equation}
                A=\begin{pmatrix}
                    2 & -1\\
                    -3 & 2
                \end{pmatrix}\rightarrow\braket{\alpha_1,\alpha_1}=3c^2,\quad\braket{\alpha_2,\alpha_2}=c^2,\quad\braket{\alpha_1,\alpha_2}=-\frac{3}{2}c^2
            \end{equation}
            Length of roots:
            \begin{equation}
                \boxed{|\alpha_1|=\sqrt{\braket{\alpha_1,\alpha_1}}=\sqrt{3}c,\quad|\alpha_2|=\sqrt{\braket{\alpha_2,\alpha_2}}=c}
            \end{equation}
            Angle between roots:
            \begin{equation}
                \boxed{\cos\alpha=\frac{\braket{\alpha_1,\alpha_2}}{\sqrt{\braket{\alpha_1,\alpha_1}\braket{\alpha_2,\alpha_2}}}=-\frac{\sqrt{3}}{2}\rightarrow\alpha=\frac{5\pi}{6}}
            \end{equation}
            Dynkin diagram for $G_2\leftarrow\mathfrak{g}_2$: \tble{G/2}
        \end{itemize}
        \item Dihedral group:
        \begin{equation}
            \text{Dih}_n=\{r,s|r^n=s^2=(sr)^n=e\}
        \end{equation}
        Let be
        \begin{equation}
            \alpha_1=\begin{pmatrix}
                1\\
                0
            \end{pmatrix},\quad \alpha_2=\begin{pmatrix}
                0\\
                1
            \end{pmatrix}
        \end{equation}
        Weyl group $W$ is generated by all the reflections with respect to all roots
        \begin{equation}
            w_{\alpha_i}(\alpha_j)=\alpha_j-\frac{2\braket{\alpha_i,\alpha_j}}{\braket{\alpha_i,\alpha_i}}\alpha_i
        \end{equation}
        \begin{equation}
            w_{\alpha_i}(\alpha_i)=-\alpha_i
        \end{equation}
        Consider cases:
        \begin{itemize}
            \item 
            \begin{equation}
                \braket{\alpha_1,\alpha_1}=c_1^2,\quad\braket{\alpha_2,\alpha_2}=c_2^2,\quad\braket{\alpha_1,\alpha_2}=0
            \end{equation}
            \begin{equation}
                w_{\alpha_1}(\alpha_2)=\alpha_2,\quad w_{\alpha_2}(\alpha_1)=\alpha_1
            \end{equation}
            \begin{equation}
                w_{\alpha_1}=\begin{pmatrix}
                    -1 & 0\\
                    0 & 1
                \end{pmatrix},\quad w_{\alpha_2}=\begin{pmatrix}
                    1 & 0\\
                    0 & -1
                \end{pmatrix}
            \end{equation}
            \begin{equation}
                w^2_{\alpha_1}=w^2_{\alpha_2}=(w_{\alpha_1}w_{\alpha_2})^2=\begin{pmatrix}
                    1 & 0\\
                    0 & 1
                \end{pmatrix}
            \end{equation}
            \begin{equation}
                \boxed{W(D_2)=\text{Dih}_2}
            \end{equation}
            \item
            \begin{equation}
                \braket{\alpha_1,\alpha_1}=\braket{\alpha_2,\alpha_2}=c^2,\quad\braket{\alpha_1,\alpha_2}=-\frac{c^2}{2}
            \end{equation}
            \begin{equation}
                w_{\alpha_1}(\alpha_2)=\alpha_2+\alpha_1,\quad w_{\alpha_2}(\alpha_1)=\alpha_1+\alpha_2
            \end{equation}
            \begin{equation}
                w_{\alpha_1}=\begin{pmatrix}
                    -1 & 1\\
                    0 & 1
                \end{pmatrix},\quad w_{\alpha_2}=\begin{pmatrix}
                    1 & 0\\
                    1 & -1
                \end{pmatrix}
            \end{equation}
            \begin{equation}
                w^2_{\alpha_1}=w^2_{\alpha_2}=(w_{\alpha_1}w_{\alpha_2})^3=\begin{pmatrix}
                    1 & 0\\
                    0 & 1
                \end{pmatrix}
            \end{equation}
            \begin{equation}
                \boxed{W(A_2)=\text{Dih}_3}
            \end{equation}
            \item
            \begin{equation}
                \braket{\alpha_1,\alpha_1}=2c^2,\quad\braket{\alpha_2,\alpha_2}=c^2,\quad\braket{\alpha_1,\alpha_2}=-c^2
            \end{equation}
            \begin{equation}
                w_{\alpha_1}(\alpha_2)=\alpha_2+\alpha_1,\quad w_{\alpha_2}(\alpha_1)=\alpha_1+2\alpha_2
            \end{equation}
            \begin{equation}
                w_{\alpha_1}=\begin{pmatrix}
                    -1 & 1\\
                    0 & 1
                \end{pmatrix},\quad w_{\alpha_2}=\begin{pmatrix}
                    1 & 0\\
                    2 & -1
                \end{pmatrix}
            \end{equation}
            \begin{equation}
                w^2_{\alpha_1}=w^2_{\alpha_2}=(w_{\alpha_1}w_{\alpha_2})^4=\begin{pmatrix}
                    1 & 0\\
                    0 & 1
                \end{pmatrix}
            \end{equation}
            \begin{equation}
                \boxed{W(B_2)=\text{Dih}_4}
            \end{equation}
            \item 
            \begin{equation}
                \braket{\alpha_1,\alpha_1}=3c^2,\quad\braket{\alpha_2,\alpha_2}=c^2,\quad\braket{\alpha_1,\alpha_2}=-\frac{3}{2}c^2
            \end{equation}
            \begin{equation}
                w_{\alpha_1}(\alpha_2)=\alpha_2+\alpha_1,\quad w_{\alpha_2}(\alpha_1)=\alpha_1+3\alpha_2
            \end{equation}
            \begin{equation}
                w_{\alpha_1}=\begin{pmatrix}
                    -1 & 1\\
                    0 & 1
                \end{pmatrix},\quad w_{\alpha_2}=\begin{pmatrix}
                    1 & 0\\
                    3 & -1
                \end{pmatrix}
            \end{equation}
            \begin{equation}
                w^2_{\alpha_1}=w^2_{\alpha_2}=(w_{\alpha_1}w_{\alpha_2})^6=\begin{pmatrix}
                    1 & 0\\
                    0 & 1
                \end{pmatrix}
            \end{equation}
            \begin{equation}
                \boxed{W(G_2)=\text{Dih}_6}
            \end{equation}
        \end{itemize}
        \item
    \end{itemize}
    \item
    \item
    \item
    \item
    \item \textbf{Matrix models, orthogonal polynomials and tau-functions.}\\
    Consider a Hermitian one-matrix model with partition function
    \begin{equation}
        Z_N(t)=c_N\int_{\mathcal{H}_N}DMe^{-\text{tr}V(M)},\quad V(M)=\sum\limits_{k=0}^\infty t_kM^k
    \end{equation}
    where $c_N$ is a factor depending only on the size of matrices, which will be fixed further, $\mathcal{H}_N =\{M\in\text{Mat}_N(\mathbb{C})|M = M^\dagger\}$ is the space of $N \times N$ Hermitian matrices and the integration measure is the standard invariant Haar measure on $\mathcal{H}_N$ given by
    \begin{equation}
        DM=\prod\limits_{i=1}^NdM_{ii}\prod\limits_{1\leq i<j\leq N}d\text{Re}M_{ij}d\text{Im}M_{ij}
    \end{equation}
    \begin{itemize}
        \item It is known that every Hermitian matrix $M$ can be diagonalized via the unitary transformation, i.e. $M = U^\dagger\Lambda U$ for the unitary matrix $U$ and the diagonal matrix $\Lambda$ which contains the eigenvalues of $M$: $\Lambda = \text{diag}(\lambda_1,...,\lambda_N)$.\\
        Show that the defined integration measure $DM$ can be presented in the form
        \begin{equation}
            DM=\frac{\mu_{U(N)}}{\mu_{U(1)^N}}\prod\limits_{i=1}^Nd\lambda_i\prod\limits_{1\leq i<j\leq N}(\lambda_i-\lambda_j)^2
        \end{equation}
        \textit{Hint}: use the correspondence between measure and norm, for example, one can use the expression for the norm in the spherical coordinates $ds^2 = dr^2 + r^2d\theta^2 + r^2\sin^2\theta d\varphi^2$ to get the integration measure $dr\cdot rd\theta\cdot r\sin\theta d\varphi$.\\
        Thus, the partition function for $c_N=\frac{1}{N!}\frac{\text{Vol}(U(1))^N}{\text{Vol}(U(N))}$ is rewritten as
        \begin{equation}
            Z_N(t)=\frac{1}{N!}\int_{\mathbb{R}^N}\prod\limits_{i=1}^N(d\lambda_ie^{-V(\lambda_i)})\prod\limits_{1\leq i<j\leq N}(\lambda_i-\lambda_j)^2
        \end{equation}
        \item Consider a family of polynomials $\{\pi_0(x),\pi_1(x),...,\pi_{N-1}(x)\}$, such that
        \begin{itemize}
            \item[(a)] $\text{deg}\;\pi_k=k$,
            \item[(b)] The leading coefficients equal to 1: $\pi_k=\sum\limits_{l\leq k}\gamma_{kl}x^l$, $\gamma_{kk}=1$.
            \item[(c)] $\braket{\pi_k(x),\pi_l(x)}=e^{q_k(t)}\delta_{kl}$, where $q_k(t)$ are some functions of the parameters of the potential $V(x)$, and the scalar product is defined as $\braket{f(x),g(x)}=\int_{\mathbb{R}}f(x)g(x)e^{-V(x)}dx$.
        \end{itemize}
        Show that the partition function $Z$ can be rewritten in the form
        \begin{equation}
            Z_N(t)=\frac{1}{N!}\prod\limits_{i=1}^N\int_{\mathbb{R}}d\lambda_ie^{-V(\lambda_i)}\underset{1\leq j,k\leq N}{\det}(\pi_{j-1}(\lambda_k))\underset{1\leq l,m\leq N}{\det}(\pi_{l-1}(\lambda_m))=\prod\limits_{k=0}^{N-1}e^{q_k(t)}
        \end{equation}
        \item Compute the scalar products $\braket{x\pi_k(x),\pi_l(x)}$ for $l<k$ and show that the orthogonal polynomials $\pi_k(x)$ satisfy the three-term identity
        \begin{equation}
            x\pi_k(x)=\pi_{k+1}(x)-p_k(t)\pi_k(x)+R_k(t)\pi_{k-1}(x)
        \end{equation}
        for some coefficients $p_k(t)$, $R_k(t)$ (not depending on $x$, only on parameters $t_1$, $t_2$, ...) of the potential. Show that $R_k(t)=e^{q_k(t)-q_{k-1}(t)}$.
        \item Compute the derivative of the scalar product $\braket{\pi_k,\pi_k}$ with respect to $t_1$ and use the properties of the orthogonal polynomials to show that $\frac{\partial q_k(t)}{\partial t_1}=p_k(t)$.
        \item Compute the derivative of the scalar product $\braket{\pi_k,\pi_l}$ for $k\neq l$ with respect to $t_1$ and find the expression for the derivative $\frac{\partial\pi_k(x)}{\partial t_1}$ in terms of $p_i$, $q_i$, $\pi_i$.
    \end{itemize}
    \textbf{Solution.}
    \begin{itemize}
        \item There is a natural volume form on each finite-dimensional inner-product space of dimension $n$. Each symmetric positively defined $g\in\text{Mat}_n(\mathbb{R})$ defines an inner-product and metric on $\mathbb{R}^n$:
        \begin{equation}
            \braket{x,y}_g=\sum\limits_{j,k=1}^ng_{jk}x_jy_k,\quad ds^2=\sum\limits_{j,k=1}^ng_{jk}dx_jdx_k
        \end{equation}
        The associated $n$-dimensional volume form is
        \begin{equation}
            Dx=\sqrt{\det g}dx_1...dx_n
        \end{equation}
        The space of Hermitian matrices $\mathcal{H}_N$ is a vector-space of real dimension $n=N^2$, as may be seen by the isomorphism $\mathcal{H}_N\rightarrow\mathbb{R}^n$:
        \begin{equation}
            M\rightarrow\xi=(M_{11},...,M_{NN},\text{Re}M_{12},...,\text{Re}M_{N-1,N},\text{Im}M_{12},...,\text{Im}M_{N-1,N})
        \end{equation}
        The Hilbert-Schmidt inner product on $\mathcal{H}_N$ is
        \begin{equation}
            \mathcal{H}_N\times\mathcal{H}_N\rightarrow\mathbb{C},\quad(M,N)\rightarrow\text{Tr}(M^\dagger N)
        \end{equation}
        The associated infinitesimal length element is
        \begin{equation}\label{eq7}
            ds^2=\text{Tr}(dM^2)=\sum\limits_{i=1}^NdM_{ii}^2+2\sum\limits_{1\leq i<j\leq N}d\text{Re}M_{ij}^2+d\text{Im}M_{ij}^2
        \end{equation}
        Thus, in the coordinates $\xi$, the metric is an $N^2\times N^2$ diagonal matrix whose first $N$ entries are 1 and all other entries are $2$, so
        \begin{equation}
            \det g=2^{N(N-1)}
        \end{equation}
        \begin{equation}\label{eq8}
            DM=2^{\frac{N(N-1)}{2}}\prod\limits_{i=1}^NdM_{ii}\prod\limits_{1\leq i<j\leq N}d\text{Re}M_{ij}d\text{Im}M_{ij}
        \end{equation}
        The unitary group, $U(N)$ is the group of linear isometries of $\mathbb{C}^N$ equipped with the standard inner-product $\braket{x,y}=x^\dagger y$. Thus, $U(n)$ is equivalent to the group of matrices $U\in\text{Mat}_N(\mathbb{C})$ such that $U^\dagger U = I$. The inner-product (\ref{eq7}) and volume form (\ref{eq8}) are invariant under the transformation $M\rightarrow UMU^\dagger$.\\
        The Lie algebra $\mathfrak{u}(N)$
        \begin{equation}
            \mathfrak{u}(N)=T_IU(N)=\{A\in\text{Mat}_N(\mathbb{C})|A=-A^\dagger\}
        \end{equation}
        \begin{equation}
            T_UU(N)=\{UA,A\in\mathfrak{u}(N)\}
        \end{equation}
        For $A,\tilde{A}\in\mathfrak{u}(n)$, we define their inner product $\text{Tr}(A^\dagger\Tilde{A})=-\text{Tr}(A\Tilde{A})$. This inner-product is natural, because it is invariant under left application of $U(N)$. That is, for two vector $UA,U\Tilde{A}\in T_UU(N)$ we find $\text{Tr}((UA)^\dagger U\Tilde{A})=\text{Tr}(A^\dagger\Tilde{A})$. The associated volume form on $U(n)$ is called \textit{Haar measure}. It is unique, upto a normalizing factor, and we write
        \begin{equation}
            D\Tilde{U}=2^{\frac{N(N-1)}{2}}\prod\limits_{i=1}^NdA_{ii}\prod\limits_{1\leq i<j\leq N}d\text{Re}A_{ij}d\text{Im}A_{ij}
        \end{equation}
        However, when viewing diagonalization $M=U\Lambda U^\dagger$ as a change of variables on $\mathcal{H}_N$, it is necessary to quotient out the following degeneracy: $\forall\theta=(\theta_1,...,\theta_N)\in\mathbb{R}^N$, the diagonal matrix $D = \text{diag}(e^{i\theta_1},...,e^{i\theta_N})$ is unitary and $M = U\Lambda U^\dagger\Leftrightarrow M = UD\Lambda D^\dagger U^\dagger$. Thus, for $\mathcal{H}_N$, the measure $D\Tilde{U}$ must be replaced a measure on $U(N)/\mathbb{R}^N$.
        \begin{equation}
            dM=dU\Lambda U^\dagger+Ud\Lambda U^\dagger+U\Lambda dU^\dagger
        \end{equation}
        \begin{equation}
            UU^\dagger=I\rightarrow dUU^\dagger+UdU^\dagger=0\rightarrow dU^\dagger=-U^{-1}dUU^\dagger=-U^\dagger dUU^\dagger
        \end{equation}
        If $U=I$, then
        \begin{equation}
            dU^\dagger=-dU
        \end{equation}
        \begin{equation}
            dM=dU\Lambda U^\dagger+Ud\Lambda U^\dagger-U\Lambda U^\dagger dUU^\dagger
        \end{equation}
        \begin{equation}
            dM=U(d\Lambda+[U^\dagger dU,\Lambda])U^\dagger
        \end{equation}
        \begin{equation}
            (U^\dagger dU)^\dagger=dU^\dagger U=-U^\dagger dUU^\dagger U=-U^\dagger dU
        \end{equation}
        Matrix $U^\dagger dU$ is antihermitian.\\
        Thus, the volume form on the quotient $U(N)/\mathbb{R}^N$ is locally equivalent to a volume form on the subspace of anti-Hermitian matrices consisting of matrices with zero diagonal:
        \begin{equation}
            DU=2^{\frac{N(N-1)}{2}}\prod\limits_{1\leq i<j\leq N}d\text{Re}A_{ij}d\text{Im}A_{ij}
        \end{equation}
        Let be $A$:
        \begin{equation}
            U^\dagger dU=dA
        \end{equation}
        \begin{multline}
            \text{Tr}(dM)^2=\text{Tr}(dM)^\dagger dM=\text{Tr}U(d\Lambda+[U^\dagger dU,\Lambda])^\dagger U^\dagger U(d\Lambda+[U^\dagger dU,\Lambda])U^\dagger=\\=\text{Tr}d\Lambda^2+2\text{Tr}d\Lambda[dA,\Lambda]+\text{Tr}[dA,\Lambda]^\dagger[dA,\Lambda]=\text{Tr}d\Lambda^2+\text{Tr}[dA,\Lambda]^\dagger[dA,\Lambda]
        \end{multline}
        \begin{equation}
            dA=d\text{Re}A+id\text{Im}A
        \end{equation}
        \begin{multline}
            \text{Tr}[dA,\Lambda]^\dagger[dA,\Lambda]=\text{Tr}(d\text{Re}A)\Lambda(d\text{Re}A)\Lambda+\text{Tr}\Lambda(d\text{Re}A)\Lambda(d\text{Re}A)-\\-\text{Tr}\Lambda(d\text{Re}A)^2\Lambda-\text{Tr}(d\text{Re}A)\Lambda^2(d\text{Re}A)+\\+\text{Tr}(d\text{Im}A)\Lambda(d\text{Im}A)\Lambda+\text{Tr}\Lambda(d\text{Im}A)\Lambda(d\text{Im}A)-\\-\text{Tr}\Lambda(d\text{Im}A)^2\Lambda-\text{Tr}(d\text{Im}A)\Lambda^2(d\text{Im}A)=\\=2\sum\limits_{i<j}(\lambda_i-\lambda_j)^2d\text{Re}A^2_{ij}+2\sum\limits_{i<j}(\lambda_i-\lambda_j)^2d\text{Im}A^2_{ij}
        \end{multline}
        \begin{equation}
            ds^2=\text{Tr}(dM)^2=\sum\limits_{i=1}^Nd\lambda_i^2+2\sum\limits_{i<j}(\lambda_i-\lambda_j)^2d\text{Re}A^2_{ij}+2\sum\limits_{i<j}(\lambda_i-\lambda_j)^2d\text{Im}A^2_{ij}
        \end{equation}
        \begin{equation}
            DM=2^{\frac{N(N-1)}{2}}\prod\limits_{i=1}^Nd\lambda_i\prod\limits_{1\leq i<j\leq N}(\lambda_i-\lambda_j)^2d\text{Re}A_{ij}d\text{Im}A_{ij}=\prod\limits_{1\leq i<j\leq N}(\lambda_i-\lambda_j)^2D\Lambda DU
        \end{equation}
        \begin{equation}
            DM=\frac{\mu_{U(N)}}{\mu_{U(1)^N}}\prod\limits_{i=1}^Nd\lambda_i\prod\limits_{1\leq i<j\leq N}(\lambda_i-\lambda_j)^2
        \end{equation}
        Thus, the partition function for $c_N=\frac{1}{N!}\frac{\text{Vol}(U(1))^N}{\text{Vol}(U(N))}$ is rewritten as
        \begin{equation}
            \boxed{Z_N(t)=\frac{1}{N!}\int_{\mathbb{R}^N}\prod\limits_{i=1}^N(d\lambda_ie^{-V(\lambda_i)})\prod\limits_{1\leq i<j\leq N}(\lambda_i-\lambda_j)^2}
        \end{equation}
        \item Vandermonde determinant:
        \begin{equation}
            \Delta(\Lambda)=\prod\limits_{1\leq i<j\leq N}(\lambda_i-\lambda_j)^2=\det\begin{pmatrix}
                1 & 1 & \hdots & 1\\
                \lambda_1 & \lambda_2 & \hdots & \lambda_N\\
                \vdots & \vdots & \ddots & \vdots\\
                \lambda^{N-1}_1 & \lambda^{N-1}_2 & \hdots & \lambda^{N-1}_N
            \end{pmatrix}
        \end{equation}
        \begin{equation}
            \pi_k(x)=1+\sum\limits_{1\leq l\leq k}\gamma_{kl}x^l
        \end{equation}
        By elementary column operations on the Vandermonde determinant:
        \begin{multline}
            \prod\limits_{1\leq i<j\leq N}(\lambda_i-\lambda_j)=\det\begin{pmatrix}
                \pi_0(\lambda_1) & \pi_0(\lambda_2) & \hdots & \pi_0(\lambda_N)\\
                \pi_1(\lambda_1) & \pi_1(\lambda_2) & \hdots & \pi_1(\lambda_N)\\
                \vdots & \vdots & \ddots & \vdots\\
                \pi_{N-1}(\lambda_1) & \pi_{N-1}(\lambda_2) & \hdots & \pi_{N-1}(\lambda_N)
            \end{pmatrix}=\\=\underset{1\leq j,k\leq N}{\det}(\pi_{j-1}(\lambda_k))
        \end{multline}
        \begin{equation}
            Z_N(t)=\frac{1}{N!}\prod\limits_{i=1}^N\int_{\mathbb{R}}d\lambda_ie^{-V(\lambda_i)}\underset{1\leq j,k\leq N}{\det}(\pi_{j-1}(\lambda_k))\underset{1\leq l,m\leq N}{\det}(\pi_{l-1}(\lambda_m))
        \end{equation}
        \begin{equation}
            \underset{1\leq j,k\leq N}{\det}(\pi_{j-1}(\lambda_k))=\sum\limits_{\sigma\in S_N}\text{sgn}(\sigma)\prod\limits_{j=1}^N\pi_{\sigma_j-1}(\lambda_j)
        \end{equation}
        \begin{equation}
            \underset{1\leq j,k\leq N}{\det}(\pi_{j-1}(\lambda_k))\underset{1\leq l,m\leq N}{\det}(\pi_{l-1}(\lambda_m))=\sum\limits_{\sigma\in S_N}\text{sgn}(\sigma)\text{sgn}(\tau)\prod\limits_{j=1}^N\pi_{\sigma_j-1}(\lambda_j)\pi_{\tau_j-1}(\lambda_j)
        \end{equation}
        \begin{multline}
            Z_N(t)=\frac{1}{N!}\int_{\mathbb{R}^N}\prod\limits_{i=1}^Nd\lambda_ie^{-V(\lambda_i)}\sum\limits_{\sigma,\tau\in S_N}\text{sgn}(\sigma)\text{sgn}(\tau)\prod\limits_{j=1}^N\pi_{\sigma_j-1}(\lambda_j)\pi_{\tau_j-1}(\lambda_j)=\\=\frac{1}{N!}\sum\limits_{\sigma,\tau\in S_N}\text{sgn}(\sigma)\text{sgn}(\tau)\int_{\mathbb{R}^N}\prod\limits_{i=1}^Nd\lambda_ie^{-V(\lambda_i)}\prod\limits_{j=1}^N\pi_{\sigma_j-1}(\lambda_j)\pi_{\tau_j-1}(\lambda_j)=\\=\frac{1}{N!}\sum\limits_{\sigma,\tau\in S_N}\text{sgn}(\sigma)\text{sgn}(\tau)\prod\limits_{j=1}^Ne^{q_{\sigma_j-1}(t)}\delta_{\sigma_j-1,\tau_j-1}=\frac{1}{N!}\sum\limits_{\sigma\in S_N}\prod\limits_{j=1}^Ne^{q_{\sigma_j-1}(t)}
        \end{multline}
        \begin{equation}
            \boxed{Z_N(t)=\prod\limits_{k=0}^{N-1}e^{q_k(t)}}
        \end{equation}
        \item Since $x\pi_k(x)$ is a polynomial of degree $k+1$ it can be expressed as a linear combination of $\pi_j(x)$:
        \begin{equation}
            x\pi_k(x)=\sum\limits_{j=0}^{k+1}c_{j,k}\pi_j(x)
        \end{equation}
        Since $\pi_j(x)=x^j+...$, we have $c_{k+1,k}=1$.
        \begin{equation}
            \braket{x\pi_k(x),\pi_l(x)}=\sum\limits_{j=0}^{k+1}c_{j,k}\braket{\pi_j(x),\pi_l(x)}=\sum\limits_{j=0}^{k+1}c_{j,k}e^{q_j(t)}\delta_{jl}=c_{l,k}e^{q_l(t)}
        \end{equation}
        \begin{equation}
            c_{j,k}=e^{-q_j(t)}\braket{x\pi_k(x),\pi_j(x)}
        \end{equation}
        For $j\in\{0,...,k-2\}$
        \begin{equation}
            \braket{x\pi_k(x),\pi_j(x)}=\braket{\pi_k(x),x\pi_j(x)}=0
        \end{equation}
        since $x\pi_j$ lies in the span of $\{\pi_0,...,\pi_{k-1}\}$. Thus, $c_{j,k} = 0$ for $j=0,...,k-2$ and we find
        \begin{equation}
            x\pi_k(x)=\pi_{k+1}(x)+c_{k,k}\pi_k(x)+c_{k-1,k}\pi_{k-1}(x)
        \end{equation}
        Let be $p_k(t)=-c_{k,k}, R_k(t)=c_{k-1,k}$, then we obtain three-term identity
        \begin{equation}
            \boxed{x\pi_k(x)=\pi_{k+1}(x)-p_k(t)\pi_k(x)+R_k\pi_{k-1}(x)}
        \end{equation}
        \begin{equation}
            R_k(t)=c_{k-1,k}=e^{-q_{k-1}(t)}\braket{x\pi_k(x),\pi_{k-1}(x)}
        \end{equation}
        \begin{equation}
            \braket{x\pi_k(x),\pi_{k-1}(x)}=\braket{\pi_k(x),x\pi_{k-1}(x)}=\braket{\pi_k(x),\pi_k(x)}=e^{q_k(t)}
        \end{equation}
        \begin{equation}
            \boxed{R_k(t)=e^{q_k(t)-q_{k-1}(t)}}
        \end{equation}
        \item
        \begin{equation}
            \braket{\pi_k(x),\pi_k(x)}=\int_{\mathbb{R}}\pi^2_k(x)e^{-V(x)}dx=e^{q_k(t)}
        \end{equation}
        \begin{multline}
            \frac{\partial}{\partial t_1}\braket{\pi_k(x),\pi_k(x)}=-\int_{\mathbb{R}}x\pi^2_k(x)e^{-V(x)}dx=-\braket{x\pi_k(x),\pi_k(x)}=-e^{q_k(t)}c_{k,k}=\\=e^{q_k(t)}p_k(t)
        \end{multline}
        \begin{equation}
            \frac{\partial}{\partial t_1}\braket{\pi_k(x),\pi_k(x)}=e^{q_k(t)}\frac{\partial q_k(t)}{\partial t_1}
        \end{equation}
        \begin{equation}
            \boxed{\frac{\partial q_k(t)}{\partial t_1}=p_k(t)}
        \end{equation}
        \item 
        \begin{equation}
            \braket{\pi_k(x),\pi_l(x)}=\int_{\mathbb{R}}\pi_k(x)\pi_l(x)e^{-V(x)}dx=0
        \end{equation}
        \begin{multline}
            \frac{\partial}{\partial t_1}\braket{\pi_k(x),\pi_l(x)}=-\int_{\mathbb{R}}x\pi_k(x)\pi_l(x)e^{-V(x)}dx=-\braket{x\pi_k(x),\pi_l(x)}=-c_{l,k}e^{q_l(t)}
        \end{multline}
    \end{itemize}
\end{enumerate}
\section{Integrable systems related to infinite-dimensional Lie algebras}
\begin{enumerate}
    \item \textbf{Pseudodifferential operators.}\\
    Consider a ring of pseudodifferential operators with elements of the standard form
    \begin{equation}\label{eq6}
        \sum\limits_{k=0}^\infty c_k(x)\partial^{N-k}=c_0\partial^N+c_1\partial^{N-1}+...,
    \end{equation}
    where $c_k(x)$ are functions of one variable $x$, $\partial$ is a derivative with respect to $x$, which has the standard commutation rules with functions: $\partial f(x) = f(x)\partial + f'(x)$. $\partial^{-1}$ is a formal inverse of $\partial$, such that $\partial\partial^{-1}=\partial^{-1}\partial=1$.
    \begin{itemize}
        \item Find the explicit expression for the commutation rule of $\partial^{-1}$ with an arbitrary function, namely, rewrite the product $\partial^{-1}f(x)$ in the standard form (\ref{eq6}).
        \item Consider a pseudodifferential operator $Q$ with the property $L = Q^2 =\partial^2+u(x)$. Write down the first five nontrivial coefficients $a_0$, $a_1$, $a_2$, $a_3$ and $a_4$ in the expansion of this operator $Q=\partial+\sum\limits_{k\geq0}a_k\partial^{-k}$.
        \item Write down the expressions for operators $M_3 = (Q^3)_+$ and $M_5 = (Q^5)_+$, where $()_+$ denotes the positive part of the pseudodifferential operator (all $\partial^k$, $k<0$ terms set to zero)
        \begin{equation}
            \left(\sum\limits_{k=0}^\infty c_k(x)\partial^{N-k}\right)_+=\sum\limits_{k=0}^Nc_k(x)\partial^{N-x}
        \end{equation}
        Show that the equation $\frac{\partial L}{\partial t_3} = [M_3, L]$ is equivalent to the KdV equation, and the equation $\frac{\partial L}{\partial t_5} = [M_5, L]$ can be considered as one of the higher flows of KdV hierarchy (i.e. it commutes with $t_3$ flow).
    \end{itemize}
    \textbf{Solution.}
    \begin{itemize}
        \item Commutation rule $\partial$ with functions:
        \begin{equation}
            \partial f(x)=f(x)\partial+f'(x)
        \end{equation}
        \begin{equation}
            f(x)\partial^{-1}=\partial^{-1}\partial(f(x)\partial^{-1})=\partial^{-1}(f'(x)\partial^{-1})+\partial^{-1}f(x)
        \end{equation}
        \begin{equation}
            \partial^{-1}f(x)=f(x)\partial^{-1}-\partial^{-1}(f'(x)\partial^{-1})=f(x)\partial^{-1}-f'(x)\partial^{-2}+\partial^{-1}(f''(x)\partial^{-1})
        \end{equation}
        \begin{equation}
            \boxed{\partial^{-1}f(x)=\sum\limits_{k=1}^\infty(-1)^{k-1}f^{(k-1)}(x)\partial^{-k}}
        \end{equation}
        \item
        \begin{equation}
            Q=\partial+\sum\limits_{k\geq0}a_k\partial^{-k}
        \end{equation}
        \begin{equation}
            L=Q^2=\left(\partial+\sum\limits_{k\geq0}a_k\partial^{-k}\right)\left(\partial+\sum\limits_{k\geq0}a_k\partial^{-k}\right)=\partial^2+u(x)
        \end{equation}
        In the product of the sums we will leave the terms only up to $\partial^{-3}$.
        \begin{multline}
            \partial(\partial+a_0+a_1\partial^{-1}+a_2\partial^{-2}+a_3\partial^{-3}+a_4\partial^{-4})=\\=\partial^2+a'_0+a'_1\partial^{-1}+a_1+a'_2\partial^{-2}+a_2\partial^{-1}+a_3'\partial^{-3}+a_3\partial^{-2}+a_4\partial^{-3}+...=\\=\partial^2+a'_0+a_1+(a'_1+a_2)\partial^{-1}+(a'_2+a_3)\partial^{-2}+(a'_3+a_4)\partial^{-3}+...
        \end{multline}
        \begin{equation}
            a_0(\partial+a_0+a_1\partial^{-1}+a_2\partial^{-2}+a_3\partial^{-3})=a_0\partial+a_0^2+a_0a_1\partial^{-1}+a_0a_2\partial^{-2}+a_0a_3\partial^{-3}
        \end{equation}
        \begin{multline}
            a_1\partial^{-1}(\partial+a_0+a_1\partial^{-1}+a_2\partial^{-2})=\\=a_1+a_1(a_0\partial^{-1}-a'_0\partial^{-2}+a''_0\partial^{-3}-...)+a_1(a_1\partial^{-2}-a'_1\partial^{-3}+...)+a_1a_2\partial^{-3}+...=\\=a_1+a_0a_1\partial^{-1}+a_1(a_1-a'_0)\partial^{-2}+a_1(a''_0-a'_1+a_2)\partial^{-3}+...
        \end{multline}
        \begin{multline}
            a_2\partial^{-2}(\partial+a_0+a_1\partial^{-1})=a_2\partial^{-1}+a_2\partial^{-1}(a_0\partial^{-1}-a'_0\partial^{-2}+...)+a_2\partial^{-1}a_1\partial^{-2}-...=\\=a_2\partial^{-1}+a_2(a_0\partial^{-2}-a'_0\partial^{-3})-a_2a'_0\partial^{-3}+a_1a_2\partial^{-3}-...=\\=a_2\partial^{-1}+a_0a_2\partial^{-2}+a_2(a_1-2a'_0)\partial^{-3}+...
        \end{multline}
        \begin{equation}
            a_3\partial^{-3}(\partial+a_0)=a_3\partial^{-2}+a_3\partial^{-2}(a_0\partial^{-1}-...)=a_3\partial^{-2}+a_0a_3\partial^{-3}+...
        \end{equation}
        \begin{equation}
            a_4\partial^{-4}\partial=a_4\partial^{-3}
        \end{equation}
        \begin{multline}
            \partial^2+a_0\partial+(a'_0+a_0^2+2a_1)+(2a_0a_1+a'_1+2a_2)\partial^{-1}+\\+(2a_0a_2-a'_0a_1+a^2_1+a'_2+2a_3)\partial^{-2}+\\+(a''_0a_1-2a'_0a_2+2a_0a_3-a_1a'_1+2a_1a_2+a'_3+2a_4)\partial^{-3}=\partial^2+u(x)
        \end{multline}
        \begin{equation}
            \begin{cases}
                a_0=0,\\
                2a_1=u(x),\\
                a'_1+2a_2=0,\\
                a_1^2+a'_2+2a_3=0,\\
                -a_1a'_1+2a_1a_2+a'_3+2a_4=0
            \end{cases}\rightarrow
            \begin{cases}
                a_0=0,\\
                a_1=\frac{u(x)}{2},\\
                a_2=-\frac{u'(x)}{4},\\
                a_3=\frac{u''(x)-u^2(x)}{8},\\
                a_4=\frac{6u(x)u'(x)-u'''(x)}{16}
            \end{cases}
        \end{equation}
        \begin{equation}
            \boxed{Q=\partial+\frac{u(x)}{2}\partial^{-1}-\frac{u'(x)}{4}\partial^{-2}+\frac{u''(x)-u^2(x)}{8}\partial^{-3}+\frac{6u(x)u'(x)-u'''(x)}{16}\partial^{-4}}
        \end{equation}
        \item 
        \begin{equation}
            M_3 = (Q^3)_+=(LQ)_+
        \end{equation}
        \begin{equation}
            M_3=\left((\partial^2+u(x))\left(\partial+\frac{u(x)}{2}\partial^{-1}-\frac{u'(x)}{4}\partial^{-2}\right)\right)_+
        \end{equation}
        \begin{equation}
            \boxed{M_3=\partial^3+\frac{3u(x)}{2}\partial+\frac{3u'(x)}{4}}
        \end{equation}
        \begin{equation}
            \frac{\partial L}{\partial t_3}=\partial_3\partial^2+\frac{\partial u(x)}{\partial t_3}
        \end{equation}
        \begin{multline}
            [M_3,L]=\left[\partial^3+\frac{3u(x)}{2}\partial+\frac{3u'(x)}{4},\partial^2+u(x)\right]=u'''(x)-u(x)\partial^3+\frac{3u(x)}{2}\partial^3-\\-\frac{3u''(x)}{2}\partial-3u'(x)\partial^2-\frac{3u(x)}{2}\partial^3+\frac{3u(x)u'(x)}{2}-\frac{3u^2(x)}{2}\partial+\\+\frac{3u'(x)}{4}\partial^2-\frac{3u'''(x)}{4}=-u(x)\partial^3-\frac{9}{4}u'(x)\partial^2-\frac{3}{2}(u''(x)+u^2(x))\partial+\\+\frac{1}{4}(u'''(x)+6u(x)u'(x))
        \end{multline}
        \begin{equation}
            \frac{\partial L}{\partial t_3}=M_3\rightarrow\boxed{\frac{\partial u(x)}{\partial t_3}=\frac{1}{4}(u'''(x)+6u(x)u'(x))}
        \end{equation}
        \begin{equation}
            M_5 = (Q^5)_+=(L^2Q)_+
        \end{equation}
        \begin{multline}
            L^2=(\partial^2+u(x))(\partial^2+u(x))=\partial^4+u(x)\partial^2+u''(x)+2u'(x)\partial+u(x)\partial^2+u^2(x)=\\=\partial^4+2u(x)\partial^2+2u'(x)\partial+u''(x)+u^2(x)
        \end{multline}
        \begin{multline}
            M_5=\left((\partial^4+2u(x)\partial^2+2u'(x)\partial+u''(x)+u^2(x))\times\right.\\\left.\times\left(\partial+\frac{u(x)}{2}\partial^{-1}-\frac{u'(x)}{4}\partial^{-2}+\frac{u''(x)-u^2(x)}{8}\partial^{-3}+\frac{6u(x)u'(x)-u'''(x)}{16}\partial^{-4}\right)\right)_+=\\=\partial^5+2u'''(x)+3u''(x)\partial+2u'(x)\partial^2+\frac{u(x)}{2}\partial^3-\frac{3u'''(x)}{2}-u''(x)\partial-\frac{u'(x)}{4}\partial^2+\\+\frac{u'''(x)-2u(x)u'(x)}{2}+\frac{u''(x)-u^2(x)}{8}\partial+\frac{6u(x)u'(x)-u'''(x)}{16}+\\+2u(x)\partial^3+2u(x)u'(x)+u^2(x)\partial-\frac{u(x)u'(x)}{2}+2u'(x)\partial^2+u'(x)u(x)+\\+(u''(x)+u^2(x))\partial
        \end{multline}
        \begin{equation*}
            \boxed{M_5=\partial^5+\frac{5u(x)}{2}\partial^3+\frac{15u'(x)}{4}\partial^2+\frac{25u''(x)+15u^2(x)}{8}\partial+\frac{15}{8}\left(\frac{u'''(x)}{2}+u'(x)u(x)\right)}
        \end{equation*}
        \begin{equation}
            \frac{\partial L}{\partial t_5}=\partial_5\partial^2+\frac{\partial u(x)}{\partial t_5}
        \end{equation}
        \begin{multline*}
            [M_5,L]=\left[\partial^5+\frac{5u(x)}{2}\partial^3+\frac{15u'(x)}{4}\partial^2,\partial^2+u(x)\right]+\\+\left[\frac{25u''(x)+15u^2(x)}{8}\partial+\frac{15}{8}\left(\frac{u'''(x)}{2}+u'(x)u(x)\right),\partial^2+u(x)\right]=\\=u^{(5)}(x)-u(x)\partial^5-5u'(x)\partial^4-\frac{5u''(x)}{2}\partial^3+\frac{5u(x)u'''(x)}{2}-\frac{5u^2(x)}{2}\partial^3-\\-\frac{15u'''(x)}{4}\partial^2-\frac{15u''(x)}{2}\partial+\frac{15u'(x)u''(x)}{4}-\frac{15u'(x)u(x)}{4}\partial^2-\\-\frac{25u'''(x)+30u(x)u'(x)}{8}\partial^2-\frac{25u^{(4)}(x)+30(u'(x))^2+30u(x)u''(x)}{8}\partial+\\+\frac{25u''(x)+15u^2(x)}{8}u'(x)-\frac{25u''(x)+15u^2(x)}{8}u(x)\partial+\\+\frac{15}{8}\left(\frac{u'''(x)}{2}+u'(x)u(x)\right)\partial^2-\frac{15}{8}\left(\frac{u^{(5)}(x)}{2}+u(x)u'''(x)+3u'(x)u''(x)\right)
        \end{multline*}
        \begin{equation}
            \boxed{\frac{\partial u(x)}{\partial t_5}=\frac{1}{16}(u^{(5)}(x)+10u(x)u'''(x)+20u'(x)u''(x)+30u^2(x)u'(x))}
        \end{equation}
    \end{itemize}
    \item \textbf{Bihamiltonian structure.}\\
    Two Poisson brackets structures $\{\cdot,\cdot\}_1$ and $\{\cdot,\cdot\}_2$ are compatible if any linear combination of them $\lambda_1\{\cdot,\cdot\}_1+\lambda_2\{\cdot,\cdot\}_2$ also has the Poisson brackets structure (i.e. Jacobi identity is satisfied).\\
    Define two Poisson brackets for KdV hierarchy: let $u(x) = \sum\limits_nu_nx^{-n-2}$ be a series in $x$, with the dynamical variables $u_n$ as the coefficients, and delta-function is defined as $\delta(x-y)=\sum\limits_nx^ny^{-n-1}$
    \begin{equation}
        \{u(x),u(y)\}_1=-\delta'(x-y)
    \end{equation}
    \begin{equation}
        \{u(x),u(y)\}_2=-2u(x)\delta'(x-y)-u'(x)\delta(x-y)-\delta'''(x-y)
    \end{equation}
    \begin{itemize}
        \item Rewrite the brackets $\{u(x),u(y)\}_1$ and $\{u(x),u(y)\}_2$ as Poisson brackets on $u_k$ elements.
        \item Show that the Poisson structures $\{u(x),u(y)\}_1$ and $\{u(x),u(y)\}_2$ are compatible.
        \item Consider a linear combination $\{\cdot,\cdot\}_\lambda=\{\cdot,\cdot\}_1-\lambda\{\cdot,\cdot\}_2$. Let $H_\lambda=\sum\limits_k\lambda^kH_k$ be a central element for these brackets $\{H_\lambda,f\}_\lambda= 0$, $\forall f$. Show that $\{H_k, f\}_1=\{H_{k-1},f\}_2$ and the coefficients $H_k$ are in involution with respect to the first and the second Poisson brackets $\{H_k, H_l\}_1 = \{H_k, H_l\}_2 = 0$.
        \item Consider several first Hamiltonians in the KdV hierarchy
        \begin{equation}
            H_0=\int u(x)dx,\quad H_1=\int u^2(x)dx,\quad H_2=\int(u^3(x)-u'(x)^2)dx
        \end{equation}
        Check that they are in involution and check explicitly that
        \begin{equation}
            \frac{\partial u(x)}{\partial t_1}=\{H_1,u(x)\}_1=\{H_0,u(x)\}_2,\quad\frac{\partial u(x)}{\partial t_3}=\{H_1,u(x)\}_2=\{H_2,u(x)\}_1
        \end{equation}
    \end{itemize}
    \textbf{Solution.}
    \begin{itemize}
        \item
        \begin{equation}
            u(x) = \sum\limits_nu_nx^{-n-2}
        \end{equation}
        Rewrite the bracket $\{u(x),u(y)\}_1$:
        \begin{multline}
            \{u(x),u(y)\}_1=\left\{\sum_n u_n x^{-n-2},\sum_mu_my^{-m-2}\right\}_1=\sum_{n,m} x^{-n-2}y^{-m-2} \left\{u_n,u_m\right\}_1=\\=\sum_{l,m}x^{l-1}y^{-m-2}\left\{u_{-l-1},u_m\right\}_1
        \end{multline}
        \begin{equation}
            \delta(x-y)=\sum\limits_nx^ny^{-n-1}\rightarrow\delta'(x-y)=\sum\limits_nnx^{n-1}y^{-n-1}
        \end{equation}
        \begin{equation}
            \{u(x),u(y)\}_1=-\delta'(x-y)\rightarrow\boxed{\left\{u_{n} , u_m \right\}_1=(n+1) \delta_{n+m+2,0}}
        \end{equation}
        Rewrite the bracket $\{u(x),u(y)\}_2$:
        \begin{multline}
            \{u(x),u(y)\}_2=\left\{\sum_n u_n x^{-n-2},\sum_m u_m y^{-m-2}\right\}_2=\sum_{n,m} x^{-n-2}y^{-m-2}\left\{u_n,u_m\right\}_2=\\=\sum_{l,m}x^{l-1}y^{-m-2}\left\{u_{-l-1},u_m\right\}_2
        \end{multline}
        \begin{multline}
            2u(x)\delta'(x-y)+u'(x)\delta(x-y)+\delta'''(x-y)=2\sum_nu_nx^{-n-2}\sum_m mx^{m-1}y^{-m-1}+\\+\sum_n(-n-2)u_nx^{-n-3}\sum_mx^{m}y^{-m-1}+\sum_nn(n-1)(n-2)x^{n-3}y^{-n-1}=\\=\sum_{n,m}(n-2m+2)u_nx^{m-n-3}y^{-m-1}-\sum_{m,n}m(m-1)(m-2)x^{-n-2}y^{-m-1}\delta_{m+n,1}
        \end{multline}
        \begin{equation}
            \{u(x),u(y)\}_2=-2u(x)\delta'(x-y)-u'(x)\delta(x-y)-\delta'''(x-y)
        \end{equation}
        \begin{equation}
            \boxed{\left\{u_n,u_m\right\}_2=u_{n+m}(n-m)+(n^3-n)\delta_{n+m,0}}
        \end{equation}
        \item  Show that the Poisson structures $\{u(x),u(y)\}_1$ and $\{u(x),u(y)\}_2$ are compatible:
        \begin{multline}
            \{u_k,\{u_l,u_m\}\}+\{u_m,\{u_k,u_l\}\}+\{u_l,\{u_m,u_k\}\}=\{u_k,\lambda_1\{u_l,u_m\}_1+\lambda_2\{u_l,u_m\}_2\}+\\+\{u_m,\lambda_1\{u_k,u_l\}_1+\lambda_2\{u_k,u_l\}_2\}+\{u_l,\lambda_1\{u_m,u_k\}_1+\lambda_2\{u_m,u_k\}_2\}=\\=\lambda_1\{u_k,\lambda_1\{u_l,u_m\}_1+\lambda_2\{u_l,u_m\}_2\}_1+\lambda_2\{u_k,\lambda_1\{u_l,u_m\}_1+\lambda_2\{u_l,u_m\}_2\}_2+\\+\lambda_1\{u_m,\lambda_1\{u_k,u_l\}_1+\lambda_2\{u_k,u_l\}_2\}_1+\lambda_2\{u_m,\lambda_1\{u_k,u_l\}_1+\lambda_2\{u_k,u_l\}_2\}_2+\\+\lambda_1\{u_l,\lambda_1\{u_m,u_k\}_1+\lambda_2\{u_m,u_k\}_2\}_1+\lambda_2\{u_l,\lambda_1\{u_m,u_k\}_1+\lambda_2\{u_m,u_k\}_2\}_2=\\=\lambda_1^2(\{u_k,\{u_l,u_m\}_1\}_1+\{u_m,\{u_k,u_l\}_1\}_1+\{u_l,\{u_m,u_k\}_1\}_1)+\\+\lambda_2^2(\{u_k,\{u_l,u_m\}_2\}_2+\{u_m,\{u_k,u_l\}_2\}_2+\{u_l,\{u_m,u_k\}_2\}_2)+\\+\lambda_1\lambda_2(\{u_k,\{u_l,u_m\}_2\}_1+\{u_k,\{u_l,u_m\}_1\}_2+\{u_m,\{u_k,u_l\}_2\}_1+\\+\{u_m,\{u_k,u_l\}_1\}_2+\{u_l,\{u_m,u_k\}_2\}_1+\{u_l,\{u_m,u_k\}_1\}_2)=\\=\lambda_1\lambda_2(\{u_k,u_{l+m}(l-m)+(l^3-l)\delta_{l+m,0}\}_1+\{u_k,(l+1)\delta_{l+m+2,0}\}_2+\\+\{u_m,u_{k+l}(k-l)+(k^3-k)\delta_{k+l,0}\}_1+\{u_m,(k+1)\delta_{k+l+2,0}\}_2+\\+\{u_l,u_{m+k}(m-k)+(m^3-m)\delta_{m+k,0}\}_1+\{u_l,(m+1)\delta_{m+k+2,0}\}_2)=\\=\lambda_1\lambda_2((l-m)\{u_k,u_{l+m}\}_1+(k-l)\{u_m,u_{k+l}\}_1+(m-k)\{u_l,u_{m+k}\}_1)=\\=\lambda_1\lambda_2\delta_{k+l+m+2,0}((l-m)(k+1)+(k-l)(m+1)+(m-k)(l+1))=0
        \end{multline}
        \begin{equation}
            \boxed{\{u_k,\{u_l,u_m\}\}+\{u_m,\{u_k,u_l\}\}+\{u_l,\{u_m,u_k\}\}=0}
        \end{equation}
        \item
        \begin{equation}
            \{\cdot, \cdot\}_\lambda=\{\cdot, \cdot\}_1-\lambda\{\cdot, \cdot\}_2
        \end{equation}
        \begin{equation}
            H_\lambda=\sum\limits_k\lambda^kH_k
        \end{equation}
        $H_\lambda$ is a central element:
        \begin{equation}
            \{H_\lambda,f\}_\lambda= 0,\;\forall f
        \end{equation}
        \begin{multline}
            \{H_\lambda,f\}_\lambda=\left\{\sum_{k}\lambda^k H_k ,f\right\}_\lambda=\left\{\sum_{k}\lambda^k H_k ,f\right\}_1-\lambda\left\{\sum_{k}\lambda^k H_k ,f\right\}_2=\\=\sum_{k}\lambda^k\left\{ H_k ,f\right\}_1-\sum_{k}\lambda^{k+1}\left\{ H_k ,f\right\}_2=\\=\{H_0,f\}_1+\sum_{k}\lambda^k(\left\{ H_k ,f\right\}_1-\left\{ H_{k-1} ,f\right\}_2)
        \end{multline}
        \begin{equation}
            \boxed{\{H_0,f\}_1=0,\quad\{H_k,f\}_1=\left\{H_{k-1} ,f\right\}_2}
        \end{equation}
        \begin{equation}
            \{H_0,f\}_1=0\rightarrow\forall k\hookrightarrow\{H_0,H_k\}_1=0
        \end{equation}
        \begin{equation}
            \{H_0,H_k\}_1=-\{H_k,H_0\}_1=-\{H_{k-1},H_0\}_2=\{H_0,H_{k-1}\}_2=0
        \end{equation}
        \begin{equation}
            \{H_1,H_k\}_1=\{H_0,H_k\}_2=0\rightarrow...
        \end{equation}
        \begin{equation}
            \boxed{\{H_k,H_l\}_1=\{H_k,H_l\}_2=0}
        \end{equation}
        \item First Hamiltonians in the KdV hierarchy:
        \begin{equation}
            H_0=\int u(x)dx,\quad H_1=\int (u(x))^2dx,\quad H_2=\int((u(x))^3-(u'(x))^2)dx
        \end{equation}
        \begin{multline}
            \{H_0,H_1\}_1=\left\{\int u(x)dx,\int(u(y))^2dy\right\}_1=\int\int dxdy\left\{u(x),(u(y))^2\right\}_1=\\=-2\int\int dxdyu(y)\delta'(x-y)=-2\int dxu'(x)=-2u(x)|_0^{2\pi}=0
        \end{multline}
        \begin{multline}
            \{H_0,H_1\}_2=\left\{\int u(x)dx,\int (u(y))^2dy\right\}_2=\int\int dxdy\left\{u(x),(u(y))^2\right\}_2=\\=-2\int\int dxdyu(y)(2u(x)\delta'(x-y)+u'(x)\delta(x-y)+\delta'''(x-y))=\\=-2\int dx(2u(x)u'(x)+u'(x)u(x)+u'''(x))=3(u(x))^2|_0^{2\pi}+2u''(x)|_0^{2\pi}=0
        \end{multline}
        \begin{multline}
            \{H_0,H_2\}_1=\left\{\int u(x)dx,\int((u(y))^3-(u'(y))^2)dy\right\}_1=\int\int dxdy\left\{u(x),(u(y))^3\right\}_1-\\-\int\int dxdy\left\{u(x),(u'(y))^2\right\}_1=-\int\int dxdy3u^2(y)\delta'(x-y)+\\+\int\int dxdy2u'(y)\partial_y\delta'(x-y)=-6\int dxu(x)u'(x)-2\int dxu'''(x)=\\=-3u^2(x)|_0^{2\pi}-2u''(x)|_0^{2\pi}=0
        \end{multline}
        \begin{multline}
            \{H_0,H_2\}_2=\left\{\int u(x)dx,\int((u(y))^3-(u'(y))^2)dy\right\}_2=\int\int dxdy\left\{u(x),(u(y))^3\right\}_2-\\-\int\int dxdy\left\{u(x),(u'(y))^2\right\}_2=-\int\int dxdy3(u(y))^2(2u(x)\delta'(x-y)+u'(x)\delta(x-y)+\delta'''(x-y))+\\+\int\int dxdy2u'(y)\partial_y(2u(x)\delta'(x-y)+u'(x)\delta(x-y)+\delta'''(x-y))=\\=-12\int dx(u(x))^2u'(x)-3\int dxu^2(x)u'(x)-3\int dx(2u(x)u'''(x)+6u'(x)u''(x))-\\-2\int dxu(x)u'''(x)-2\int dxu'(x)u''(x)-2\int dxu^{(4)}(x)=0
        \end{multline}
        Check that
        \begin{equation}
            \frac{\partial u(x)}{\partial t_1}=\{H_1,u(x)\}_1=\{H_0,u(x)\}_2,\quad\frac{\partial u(x)}{\partial t_3}=\{H_1,u(x)\}_2=\{H_2,u(x)\}_1
        \end{equation}
        \begin{multline}
            \left\{H_1, u(x)\right\}_1=\left\{\int dy(u(y))^2, u(x)\right\}_1=\int dy2u(y)\left\{u(y),u(x)\right\}_1=\\=\int dy2u(y)\delta'(x-y)=2u'(x)
        \end{multline}
        \begin{multline}
            \left\{H_0, u(x)\right\}_2=\left\{\int dyu(y), u(x)\right\}_2=\int dy\left\{u(y),u(x)\right\}_2=\\=\int dy(2u(x)\delta'(x-y)-u'(x)\delta(x-y)-\delta'''(x-y))=2u'(x)
        \end{multline}
        \begin{equation}
            t_1=\frac{x}{2}\rightarrow\frac{\partial u(x)}{\partial t_1}=\{H_1,u(x)\}_1=\{H_0,u(x)\}_2
        \end{equation}
        \begin{multline}
            \left\{H_1, u(x)\right\}_2=\left\{\int dy(u(y))^2,u(x) \right\}_2=\int dy2u(y)\left\{u(y),u(x)\right\}_2=\\=\int dy2u(y)(2u(x)\delta'(x-y)-u'(x)\delta(x-y)-\delta'''(x-y))=\\=4u(x)u'(x)-2u(x)u^{\prime}(x)-2 u'''(x)=2u(x)u'(x)-2u'''(x)
        \end{multline}
        \begin{multline}
            \left\{H_2, u(x)\right\}_1=\{\int(dy(u(y))^3-(u'(y))^2),u(x)\}_1=3\int dyu^2(y)\{u(y),u(x)\}_1-\\-2\int dyu'(y)\{u'(y),u(x)\}_1=\int dy (3u^2(y)\delta'(x-y)-2u'(y)\partial_y\delta'(x-y)))=\\=6u(x)u'(x)-2u'''(x)
        \end{multline}
    \end{itemize}
    \item \textbf{Virasoro algebra as a central extension.}\\
    Consider the Witt Lie algebra with generators $L_n$, $n\in\mathbb{Z}$ and Lie brackets
    \begin{equation}
        [L_n,L_m]=(n-m)L_{n+m}
    \end{equation}
    \begin{itemize}
        \item Check that differential operators $L_n=-x^{n+1}\partial_x$ form the representation of this Lie algebra.
        \item Show that $\omega(L_n,L_m)=(n^3-n)\delta_{n+m,0}$ is a Lie algebra 2-cocycle and it can be used to centrally extend the Witt algebra to define Virasoro algebra
        \begin{equation}
            [L_n,L_m]=(n-m)L_{n+m}+\frac{n^3-n}{12}\delta_{n+m,0}c,\quad[c,L_n]=0
        \end{equation}
        Show that this central extension is unique up to a multiplication on the arbitrary constant.
        \item Construct any nontrivial representation of the Virasoro algebra (for example for the central charge equal to one).
    \end{itemize}
    \textbf{Solution.}
    \begin{itemize}
        \item Differential operators:
        \begin{equation}
            L_n=-x^{n+1}\partial_x
        \end{equation}
        \begin{multline}
            [L_n,L_m]=[-x^{n+1}\partial_x,-x^{m+1}\partial_x]=x^{n+1}\partial_x( x^{m+1}\partial_x)-x^{m+1}\partial_x(x^{n+1}\partial_x)=\\=x^{n+1}((m+1)x^m \partial_x +x^{m+1}\partial^2_x)-x^{m+1}((n+1)x^{n}\partial_x+x^{n+1}\partial^2_x)=\\=x^{n+1}(m+1)x^m \partial_x - x^{m+1}(n+1)x^{n}\partial_x=-(n- m)x^{n+m+1}\partial_x=(n-m)L_{n+m}
        \end{multline}
        \item
        \begin{equation}
            [L_m,L_n]=(m-n)L_{m+n}+\lambda_{m,n}
        \end{equation}
        \begin{equation}
            [L_n,L_m]=-[L_m,L_n]
        \end{equation}
    \begin{equation}
        (n-m)L_{n+m}+\lambda_{n,m}=-(m-n)L_{m+n}+\lambda_{m,n}\rightarrow\lambda_{n,m}=-\lambda_{m,n}
    \end{equation}
    Move the generators
    \begin{equation}
        L_n\rightarrow L_n+q_n
    \end{equation}
    \begin{equation}
        [L_m,L_n]=(m-n)L_{m+n}+\lambda_{m,n}
    \end{equation}
    \begin{equation}
        [L_m+q_m,L_n+q_n]=(m-n)(L_{m+n}+q_{m+n})+\lambda_{m,n}
    \end{equation}
    \begin{equation}
        \lambda_{m,n}\rightarrow\lambda_{m,n}+(m-n)q_{m+n}
    \end{equation}
    Choose $q_m=-\frac{1}{m}\lambda_{m,0}$ for $m\neq0$ and $q_0=-\frac{1}{2}\lambda_{1,-1}$. Then
    \begin{equation}
        \lambda_{m,0}\rightarrow\lambda_{m,0}+mq_m=0\quad\forall m\neq0
    \end{equation}
    \begin{equation}
        \lambda_{1,-1}\rightarrow\lambda_{1,-1}+2q_0=0
    \end{equation}
    \begin{equation}
        [L_m,L_0]=mL_m,\quad[L_1,L_{-1}]=2L_0
    \end{equation}
    \begin{equation}
        [[L_m,L_n],L_0]=[(m-n)L_{n+m}+\lambda_{m,n},L_0]=(m-n)((m+n)L_{m+n}+\lambda_{m+n,0})
    \end{equation}
    \begin{equation}
        [[L_n,L_0],L_m]=[nL_n+\lambda_{n,0},L_m]=n((n-m)L_{n+m}+\lambda_{n,m})
    \end{equation}
    \begin{equation}
        [[L_0,L_m],L_n]=[-mL_m+\lambda_{0,m},L_n]=-m((m-n)L_{m+n}+\lambda_{m,n})
    \end{equation}
    Consider Jacobi identity:
    \begin{multline}
        [[L_m,L_n],L_0]+[[L_n,L_0],L_m]+[[L_0,L_m],L_n]=(m-n)\lambda_{m+n,0}+n\lambda_{n,m}-m\lambda_{m,n}=\\=(m+n)\lambda_{n,m}=0
    \end{multline}
    In case $m\neq-n$ we have $\lambda_{n,m}=0$. Therefore, the only non-vanishing central extensions are $\lambda_{n,-n}$ for $|n|\geq2$.
    \begin{equation}
        \lambda_{n,m}=\lambda(n)\delta_{m+n,0}
    \end{equation}
    \begin{equation}
        [[L_{-n+1},L_n],L_{-1}]=[(-2n+1)L_1+\lambda_{-n+1,n},L_{-1}]=(-2n+1)(2L_0+\lambda_{1,-1})
    \end{equation}
    \begin{equation}
        [[L_n,L_{-1}],L_{-n+1}]=[(n+1)L_{n-1}+\lambda_{n,-1},L_{-n+1}]=(n+1)(2(n-1)L_0+\lambda_{n-1,1-n})
    \end{equation}
    \begin{equation}
        [[L_{-1},L_{-n+1}],L_n]=[(n-2)L_{-n}+\lambda_{-1,-n+1},L_n]=(n-2)((-2n)L_0+\lambda_{-n,n})
    \end{equation}
    Consider Jacobi identity:
    \begin{multline}
        [[L_{-n+1},L_n],L_{-1}]+[[L_n,L_{-1}],L_{-n+1}]+[[L_{-1},L_{-n+1}],L_n]=(-2n+1)\lambda_{1,-1}+(n+1)\lambda_{n-1,1-n}+\\+(n-2)\lambda_{-n,n}=(n+1)\lambda_{n-1,1-n}-(n-2)\lambda_{n,-n}=0
    \end{multline}
    We obtain recurrent identity:
    \begin{equation}
        \lambda_{n,-n}=\frac{n+1}{n-2}\lambda_{n-1,1-n}=...=C^{n+1}_3\lambda_{2,-2}=\frac{(n+1)n(n-1)}{6}\lambda_{2,-2}
    \end{equation}
    We choose $\lambda_{2,-2}=\frac{c}{2}$.
    \begin{equation}
        \lambda_{m,n}=\frac{c}{12}(m^3-m)\delta_{m+n,0}
    \end{equation}
    We obtain Virasoro algebra:
    \begin{equation}
        \boxed{[L_m,L_n]=(m-n)L_{m+n}+\frac{c}{12}(m^3-m)\delta_{m+n,0}}
    \end{equation}
    \item A highest weight representation of the Virasoro algebra is a representation generated by a primary state:
    \begin{equation}
        L_0\Phi_\Delta=\Delta\Phi_\Delta,\quad L_n\Phi_\Delta=0,\quad n>0
    \end{equation}
    $\Delta$ is called the conformal dimension of $\Phi_\Delta$. A highest weight representation is spanned by eigenstates of $L_0$. The eigenvalues take the form $\Delta+n$, where the integer $n\geq0$ is called the level of the corresponding eigenstate:
    \begin{equation}
        L_0L_{-n}\Phi_\Delta=(\Delta+n)L_{-n}\Phi_\Delta
    \end{equation}
    More precisely, a highest weight representation is spanned by $L_0$-eigenstates of the type $L_{-n_1}L_{-n_2}\cdots L_{-n_k}\Phi_\Delta$ with $0<n_1\leq n_2\leq \cdots n_k$ and $k\geq0$, whose levels are $N=\sum\limits_{{i=1}}^{k}n_{i}$. Any state whose level is not zero is called a descendant state of $\Phi_\Delta$.
    \end{itemize}
\end{enumerate}
\end{document}
